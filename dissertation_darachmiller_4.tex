\chapter{Miscellaneous experiments}

%
%
%
\section{Changes in poly-adenylated transcripts per cell upon changes
in growth rates}
%
%
%

Part of this work in a pre-print.

%
%
%
\subsection{Introduction}
%
%
%

The study of microbial physiology is a long standing area of
investigation, and changes in physiological composition is still the
subject of quantitative and conceptual advances 
\parencite{henrici1928
malooe
waldron?
carter?
nitrogen upshift stuff?
more warner?
scott2010
hwarecent stuff
}.

In recent years, the study of changes in abundance of specific mRNA
factors in the budding yeast has characterized a phenomenon in which
approximately one quarter of the yeast transcriptome scales with
growth rate
\parencite{ Brauer 2008, Regenberg, Castrillo,  Airoldi2009
airoldi2016}
This phenomenon is characterized at the level of molecular species,
and thus can be compared to changes that occur in response to
stressors, summarized as the Environmental Stress Response 
\parencite{gasch2000genomic}.
Interestingly, overlap?
In addition, a shared signature of knockout mutants, commonly used to
probe gene function, is that associated with changes in cell-cycle
progression distribution due to growth rate changes
\parencite{
cell cycle kemmern thing}.
Thus the appreciation of the systematic physiological changes that
occur in response to genetic perturbations holds light to many
biological problems, even if only to identify the domineering and 
confounding factor of growth-associated physiological changes.

We know that the total RNA content of a cell changes upon changes in
growth rates (Kief or whatever, benjy's figure, warner changes).
We know that specific mRNA, a small component of the cell's total RNA
also change in relative abundance. A less-characterized question is if
the whole mRNA transcriptome changes, if this is relevant in changing
the concentration of mRNA species, or if these relative changes have a
significant regulatory role (as opposed to a regime of saturated
translation machinery \parencite{shah2013??}.
Transcriptomic measurements are usually normalized to the relative,
and is based on the assumption that the total transcriptome does not
change in abundance.
However, we now know of cases of where this assumption is violated.
\parencite{myc example}.

Spike-in normalized RNA sequencing can estimate absolute mRNA
abundance per cell, but has been criticized before for
being "too noisy" and instead computational methods of "removing
unwanted variation" were used \parencite{}. 
Led by Rodoniki Athansidou, our group pursued a more thorough approach
to this design by normalizing RNA sequencing data using the ERCC
spike-in set \parencite{ERCC group}, using preliminary sequencing runs
to first determine the appropriate amount of spike-ins necessary for
accurate sequencing. Then, yeast were grown in systematically varied
nutrient limitations of growth, then RNA sequencing using a known
quantity of the exogenous spike-ins was used to normalize the
measurements to absolute mRNA per cell.

I sought to complement this work by orthogonally estimating the size 
of the whole yeast transcriptome. To do this, I adapted the screening 
strategy of \cite{amberg1992???} to work with flow cytometry. 
Essentially, this utilized a poly-deoxythymidine oligo singly labeled
with a fluorophore. This was hybridized in with the fixed and
permeabilized yeast cell, and the resulting fluorescence after washing
is taking to be a proxy for the number of hybridized poly-dT probes,
presumably hybridized to a poly-adenosine sequence, and thus mRNA.

Another motivation of this was to serve as a fixation-digestion
control for methods involving single-gene mRNA FISH.
We had patterns of mRNA FISH hybridization signal that appeared
bimodal \autoref{gap1Delete}. 
This could be a technical issue of incomplete permeabilization due to
over-fixing, or a biological phenomenon.
To distinguish the two would take two-color FISH, with a positive
control \parencite{regenbergtwocolor???}.
Since nitrogen-limitation causes a severe restriction of the total
transcriptome content, we don't have an obvious pick for a uniformly
expressed positive control. 
However, most of the mRNA should be poly-adenylated, so FISH against
that sequence should be present in all cells, and in high-copy.
While I did not integrate this into the single gene mRNA FISH as an
internal control, I did use it to optimize fixation/permeabilization 
conditions.

%
%
%
\subsection{The assay design}
%
%
%

The assay uses a similar fixation permeabilization methods as the
single-gene mRNA FISH assay, then an overnight hybridization using
dextran sulfate against a poly-dT probe, the flow cytometry.

The yeast cells are sampled via vacuum filtration onto nylon
filters, then the filters are quickly flash-frozen in liquid nitrogen.
These are resuspended in 0.75x PBS buffered 4\% PFA (from ampules from
EMS), the cells vortexed off the filter, then the filter discarded.
The cell suspension in the fixative is incubated for hours at RT to
complete fixation, with the assumption that rapid fixation halts
RNA metabolism in the cell and long-term fixation stabilizes the fixed
components into a configuration that can survive digestion and 
permeabilization. The fix is critically quenched using 2.5M glycine,
then collected by centrifugation and washed with PBS. The cells are
digested for one hour at 37C using lyticase and beta-mercaptoethanol
in 1.2M sorbitol buffered
by potassium phosphate at about 7.4 pH, with 20mM vanadyl
ribonucleoside complex to inhibit RNAses. 
This is washed and further permeabilized with 70\% ethanol overnight,
then is resuspened using hybridization buffer
(10\% dextran sulfate w/v, 2x SSC final, 100ug/ml ecoli tRNA, 
250mg/ml salmon sperm DNA) plus 100nM of a (dT)50+V oligo 5'-labeled
with with Alexa 488, as ordered from IDT. 
This is incubated for 14+ hours on a 37C roller drum, then washed with
2x SSC several times before resuspending in PBS and flowing through an
Accurri flow cytometer.
Poly(A) content signal was determined by the signal area on the
514/20nm detector.

To test this procedure, I used RNAseA treated cells as a negative
control.
\autoref{fig:ypdnlim} shows the RNAseA-treated controls for two
samples, where in the treatment abrogates the signal for the vast
majority of the cells in the sample.

To optimize this design, I varied formamide from 0\% to 50\%, and
probe concentrations from 10nM to 1$\mu$M. I found that 100nM and 0\%
formamide saturated the signal of YPD-grown cells without largely
increasing the signal on the RNAseA-treated cells.
This assay takes approximately 8 hours of work spread over 3 days.
More detailed protocol is maintained by the Gresham laboratory.

%%%%% PUT A CONTROL FIG HERE.

%
%
%
\subsection{Nutrient limitation and transcriptome size}
%
%
%

Yeast growing in
YPD complete a division approximately every 1.5 hours (0.45 specific
growth rate), while 
proline-limited media (NLimPro) only supports division approximately
every 4.5 hours (.15 specific growth rate). 
Using this poly-dT FISH method, we see differences in the total 
poly-adenylated mRNA signal between the different media conditions. 
The distributions are broad, but the distributions are significantly
different (KS test and Wilcoxon, p-value < $2.2 \times 10^{-16}$).
We know that fast growing cells (YPD) have more RNA per cell, so it
appears that part of this difference is contributed by a global
scaling of the mRNA as well.
The fold-changes in the mean and median of the YPD-grown cells versus
the proline-limited-grown cells were 3.34 and 3.68, respectively.

\afig{
  \includegraphics[width=.7\textwidth]{img/polya_ypdnlim_controls.png}
  }{
  Wild-type yeast grown in proline-limited media (left) or YPD rich
  media (right) were assayed in exponential growth for poly-A content.
  Included are RNAsed controls treated with RNAseA, to show negative
  samples.
  The plot is cropped from 0 to $10^5$ arbitrary units of polyA
  signal to show the center of the distributions.
  The distributions from different media have different means by 
  KS or Wilcoxon tests, with unreasonably small p-values.
  \label{fig:ypdnlim}
  }

To investigate the dynamics of changes in poly-A abundance between 
different growth conditions, I grew cells in proline-limited media
overnight to reach a steady-state of growth, then collected samples
during a nitrogen-upshift.
I assayed the poly-A content of the cells using the above assay.
I found that the total poly-A content took about two hours to increase
to the new steady-state of a larger transcriptome, a similar timescale
as the changes in cell size and lag in population growth rate
(\autoref{fig:figure1a}).
The final steady-state differences were of a fold-change of 2.16 and
1.93 for the mean and median poly-A content, consistent with the
change between a specific growth rate of 0.15 and 0.35 (lower than the
0.45 rate in YPD). 

\afig{
  \includegraphics[width=\textwidth]{img/polya_upshift.png}
  }{
  Wild-type yeast were grown in proline-limited media, then glutamine
  was added at time 0 minutes. Samples were assayed for polyA content
  using the poly-dT assay.
  \label{fig:upshift}
  }

Previously, others in the lab has used ERCC-normalized RNA sequencing
to assay the absolute abundance of mRNA in yeast grown at
systematically varied growth rates (0.12,0.2,0.3 specific growth rate)
in chemostats. In a repeat experiment of this, I took samples from
chemostats limited by nitrogen or carbon at these growth rates, and
processed them to assay the distribution of poly-A content of the 
cells. \autoref{fig:nikis} shows the distributions and the
relationship between the distribution means and the estimates from
SPARQ (the spike-in normalized RNAseq method).

\afig{
  \includegraphics[width=.49\textwidth]{img/polya_niki_box.png}
  \includegraphics[width=.49\textwidth]{img/polya_niki_summary.png}
  }{
  (Left) PolyA content was estimated for cultures grown in two nutrient
  limitations at three different dilution (growth) rates.
  (Right) Comparing these measurements to SPARQ (the spike-in 
  normalized RNAseq method) shows that two methods are well 
  correlated (Pearson's r=0.95, \texttt{cor.test} p-value = 0.003684),
  although the poly-dT method remains uncalibrated.
  \label{fig:nikis}
  }

\subsection{Conclusion and future directions}

This assay appears to detect changes in the scaling of the yeast
transcriptome between different growth rates. As a flow cytometry 
assay this has the potential to be used as a marker for
high-throughput investigations of the genetics of transcriptome size
changes (or regulation), using methods as described in
\autoref{subsection:bff}. 

However, more work remains to use the assay to reliably inform on
these changes without incorporating an orthogonal measure.
Changes in polyA tail length could hypothetically affect
hybridization, and a distribution shifting such that more of the
functional mRNA have a tail length less than minimum tail length
requisite for hybridization would produce a similar graded effect.
Hybridization of this probe to synthetic mRNA cross-linked to a nylon
substrate would allow quantitative testing of this in similar
conditions as the hybridization occurs, provided a method for
manufacturing accurately generated poly-A tail lengths exists.

This method offers a conveniently high-throughput assay for total
transcriptome size, and as such is one more tool that microbial
physiologists can use to probe the functional changes that occur as
organisms systemically adapt to their environments and growth
programs.

\section{Screening for genes important for remodeling physiology for
growth}

With changing physiology in response to growth rate changes, many
molecular and functional phenotypes change. One of these is the
resistance to stress. 
It has been long known that slow growing cells are more resistant to
stressors \parencite{??}.
Yeast appears to have adapted to its ecological niche by adopting a
boom/bust, feast or famine (yeast or famine) approach to quickly
growing during favorable conditions to the expense of stress
resistance. 
Resistance to stress seems to offer "cross-protection", and the
anti-correlation of growth rate and stress resistance suggests that
the two processes might be opposed in objectives and mechanisms to
achieve these objectives.

I am interested in understanding how a yeast cell remodels its
cellular physiology and transcriptional program to effect a rapid
growth rate. 
One approach is to study the changes that the cell effects, and infer
that since the most logical response to a stress is to express this
adaptation, then the gene expression increasing upon a stress is
adaptive. This has been shown not to be the case for heatshocks, as
the genes whose expression increases do not overlap well with the
genes important for resistance \parencite{lee li botstein}.
The later genetic measurement is possible to do in high-throughput, as
the yeast community has access to a yeast deletion collection and
high-throughput means of assaying genetic effects on the quantitative
phenotype of continued existence. 

Upon consultation with advisors \autoref{section:acknow}, and given
the opportunity to work with a talented young scientist named Stephen
Nyarko, we decided to pursue this question by using the correlated
phenotypes of growth and susceptibility to stress.

The logic is that if we are interested in isolating mutants that are
defective in increasing their growth rate upon a nitrogen upshift, and
an increase in growth is associated with a susceptibility to stress,
then a somewhat-lethal stress should enrich for mutants defective in
susceptibility to the stress --- ie defective in rapidly increasing 
growth rate.

Thus we devised a screen, wherein a pooled yeast deletion collection
was grown in conditions of nitrogen-limitation (proline-only, 4.5 hour
doubling time). This was subject to an upshift of adding glutamine.
Samples were taking before or after the glutamine upshift,
heatshocked, then outgrown to enrich for living mutants.
These libraries were sequenced using an amplicon-sequencing procedure
to quantify the mutants in the resulting library.

%%% resultz?


\section{Apparent cell-cycle halt upon nitrogen-upshift}

Apparent cell-cycle halt upon
glutamine addition Previous work on cell cycle halt, basically just
alberghina, PKA and CLN1 This phenomenon has been previously seen.
Upshift ecoli, they get bigger.  This is thought to be because the
critical cell size threshold has been reset by growth signalling
pathways to a larger size. However, this result might argue that
instead it could be regulated by CLN1 transcript abundance.
Alberghina’s demonstrated that depends on Swi4?p, so there you go


Experiment, results Conclusion ( anything that didn’t get into chapter
3 )

