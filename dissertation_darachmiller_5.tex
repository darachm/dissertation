\chapter{Conclusion}

The nitrogen-upshift in budding yeast is a model system for studying
how diverse mechanisms of regulation converge on remodeling cellular
physiology and reprogramming the molecular specifics to 
carry out the feasting (or yeasting) of the \textit{Saccharomyces
cerevisiae} feast-and-famine life cycle.

\section{Trans}

\section{Possible mechanisms of \textit{GAP1} clearance}

The potential role of Scd6p and eIF4G2 in
regulating mRNA in different environmental conditions Scd6p
interaction with eIF4G1 has been mainly studied, but here we see a
similar phenotype with eIF4G2 (eIF4G1 was not tested). These are
homologs with similar function, although eIF4G1 is expressed higher
during “normal” laboratory growth conditions, ie rich media. eIF4G2
has been shown to be more highly expressed and localized to
processing-bodies during conditions of nutrient limitations 
(Brengues and Parker 2007)
, thus this initiation factor may help to specify
proper storage of certain mRNA during conditions of low amino-acid,
and thus low charged tRNA, availability to help ensure translation
with low resource availability. 

Similarities between different
phenomena of destabilization Hrp1p binding sites are enriched in the
set of transcripts destabilized during a nitrogen upshift 
(González et al. 2000) (Kebaara et al. 2003) (Kessler et al. 1997) (Guisbert et al.
2005) 
The translation and degradation of mRNA are intimately coupled
processes which compete for the same substrate, the mRNA. Perhaps this
is playing a role. But this could also be very indirect.  Degradation
is mediated by connections between 3’ and 5’, and affected by the
process of translation What if mRNA degradation simply depends on the
Lsm1-7 complex interacting with the cap? This is facilitated by
deadenylation and antagonized by translation. One idea could be that
translation initiation and elongation just wiggles the 3’ away from
the 5’, a sort of ribosome-dependent 3’ extrusion ( referencing the
cohesin-dependent loop extrusion model ).  This would sorta explain
Zenklusen’s latest, that cycloheximide preserves distance from 5’ to
3’, and puromycin doesn’t.  

An efficient method for estimating mRNA
abundance in barcoded mutant pools 


Much of what we know about mRNA
degradation has been worked out using genetics. While appealing in its
ease of use and functional consequences, the use of genetic
perturbations also comes with the caveat. Perturbing an exquisitely
homeostatic system, like a living cell, can result in indirect effects
as systems of regulatory feedback percolate the signal through the
network. Thus in this work, the use of knockout mutants is the use of
an indeterminately perturbed system. Additionally, the laborious
creation and maintenance of mutants introduces potential artifacts of
suppressor mutations and additional mutations in the mutant background
(Markowitz et al. 2017). Technologies have been recently developed to
scale heterologous repression of expression to genome-wide scales
(CRISPRi (Smith et al. 2016) and are currently being developed to
deliver programmed single residue editing at thousands of thousands of
sites (personal communication). With the ability to induce programmed
mutagenesis, with  replicates, provides for each experiment a pooled
de novo mutant library. The pooled assay concept demonstrated here is
easily composable with these approaches, and should be the next step.
Critical improvements to this method include a better design of the
gates to improve estimates of mutants with largely divergent
phenotypes. A suboptimal design was used here because the samples were
collected and sorted before the library preparation procedure was
improved to the point where low-input was a feasible option.  The
scalable efficiency of the pooled approach could be used to limit the
search space of more precise but resource-intensive automated assays
(Worley et al. 2015), and thus this hybrid approach could be used to
efficiently increase throughput of measured transcript dynamics. In
particular, the explicit measurement of degradation instead of
dynamics by use of transcriptional inhibition could be used on a
case-by-case basis (Worley et al. 2015), so long as the inhibition was
optimized to offer a relevant measurement (Pelechano and Pérez-Ortín
2008).

\section{Hypotheses}

\subsection{Possible mechanisms of \textit{GAP1} clearance}

More or less ribsomes could affect stability.

Scd6

eIF4G2, sloppy initiation, Nterminal extensions, out of frame stuff
with the Celiak2017 paper.

\subsection{The role of pH in sensing nutrient upshifts}

Glucose upshifts result in a rapid but transient reduction in pH
that is not mediated by xyz
\parencite{kresnowati2008quantitative}
Hesso etc have shown that a similar thing could be happening.
Shifting pH can have strong biophysical effects, including
the aggregation of the poly-A binding protein pab1
Riback 2017
It could also be Gcn2 mediated (huesso)

