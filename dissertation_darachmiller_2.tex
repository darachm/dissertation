\chapter{Modeling transcript dynamics upon a nitrogen upshift}
\label{chapter:two} 

This chapter was published as part of the article 
\textit{"Steady-state and dynamic gene expression programs in 
Saccharomyces cerevisiae in response to variation in 
environmental nitrogen”} in \textit{Molecular Biology of the Cell}
(vol. 27 no. 8 1383-1396. April 15, 2016.
\url{doi.org/10.1091/mbc.E14-05-1013}).

Authorship of this article was: Edoardo M. Airoldi, 
\textbf{Darach Miller},
Rodoniki Athanasiadou, Nathan Brandt, Farah Abdul-Rahman, Benjamin
Neymotin, Tatsu Hashimoto, Tayebeh Bahmani, and David Gresham. 

Below is reprinted the abstract, then excerpts of the introduction,
results, and conclusion to which I contributed. 
The text has been edited for clarity. Supplemental tables,
figures, and files are available on the MBoC article website (
\url{doi.org/10.1091/mbc.E14-05-1013} ).  

\section{Abstract} 

Cell growth rate is
regulated in response to the abundance and molecular form of essential
nutrients. In \textit{Saccharomyces cerevisiae} (budding yeast), the molecular
form of environmental nitrogen is a major determinant of cell growth
rate, supporting growth rates that vary at least threefold.
Transcriptional control of nitrogen use is mediated in large part by
nitrogen catabolite repression (NCR), which results in the repression
of specific transcripts in the presence of a preferred nitrogen source
that supports a fast growth rate, such as glutamine, that are
otherwise expressed in the presence of a nonpreferred nitrogen source,
such as proline, which supports a slower growth rate. Differential
expression of the NCR regulon and additional nitrogen-responsive genes
results in >500 transcripts that are differentially expressed in cells
growing in the presence of different nitrogen sources in batch
cultures. Here we find that in growth rate-controlled cultures using
nitrogen-limited chemostats, gene expression programs are strikingly
similar regardless of nitrogen source. NCR expression is derepressed
in all nitrogen-limiting chemostat conditions regardless of nitrogen
source, and in these conditions, only 34 transcripts exhibit nitrogen
source-specific differential gene expression. Addition of either the
preferred nitrogen source, glutamine, or the nonpreferred nitrogen
source, proline, to cells growing in nitrogen-limited chemostats
results in rapid, dose-dependent repression of the NCR regulon. Using
a novel means of computational normalization to compare global gene
expression programs in steady-state and dynamic conditions, we find
evidence that the addition of nitrogen to nitrogen-limited cells
results in the transient overproduction of transcripts required for
protein translation. Simultaneously, we find that that accelerated
mRNA degradation underlies the rapid clearing of a subset of
transcripts, which is most pronounced for the highly expressed
NCR-regulated permease genes \textit{GAP1}, \textit{MEP2}, 
\textit{DAL5}, \textit{PUT4}, and \textit{DIP5}. Our
results reveal novel aspects of nitrogen-regulated gene expression and
highlight the need for a quantitative approach to study how the cell
coordinates protein translation and nitrogen assimilation to optimize
cell growth in different environments.  

\section{Introduction}

The rate at which budding yeast cells grow is sensitive to the
molecular form of nitrogen in the environment. Yeast cells are able to
use and discriminate between different nitrogen sources 
\parencite{cooper1982nitrogen,magasanik2002nitrogen}. 
When a variety of nitrogen sources are
available, a yeast cell will preferentially transport and metabolize
particular nitrogen-containing compounds by decreasing levels of
transcripts and proteins required for use of nonpreferred nitrogen
sources 
\parencite{cooper1982nitrogen,magasanik2002nitrogen}. 
A study of yeast
cells growing in the presence of different individual nitrogen sources
provided a genome-wide view of nitrogen-regulated gene expression and
suggested that >500 genes are differentially expressed as a function
of environmental nitrogen source 
\parencite{godard2007effect}. On the basis
of differential gene expression, promoter sequence elements, and
published literature, \cite{godard2007effect} assigned membership of many
of these transcripts to five regulons that are responsive to
environmental nitrogen: the nitrogen catabolite repression A (NCR-A)
regulon, which includes bona fide NCR targets; the potential NCR
target (NCR-P) regulon; the general amino acid control (GAAC) regulon;
the unfolded protein response (UPR) regulon; and the \textit{SSY1}-PTR3-SSY5
(SPS) regulon.

Transcriptional control of the NCR regulon (i.e., both NCR-A and NCR-P
regulons) is mediated by the transcription factors 
\textit{GLN3}, \textit{GAT1}, \textit{DAL8}0,
and \textit{GZF3}, which bind to the 5′-GATAA-3′ consensus sequence in target
gene promoter regions 
\parencite{cooper1982nitrogen,magasanik2002nitrogen}. 
Whereas \textit{DAL8}0 and \textit{GZF3} act as repressors of 
NCR transcription, \textit{GLN3}
and \textit{GAT1} activate the transcription of NCR genes in a nitrogen
source-dependent manner. The evolutionarily conserved TOR complex 1
(TORC1) is believed to be an upstream regulator of NCR expression, as
it promotes the nuclear exclusion of \textit{GLN3} by physical association with
\textit{URE2} in a phosphorylation-dependent manner
\parencite{beck1999tor}.

%To study the effect on mRNA expression of environmental
%nitrogen source variation in nitrogen-limited, growth rate-controlled
%conditions, we studied cells growing in chemostats using six different
%nitrogen sources at four different dilution rates. We show that
%differential expression of the NCR (Nitrogen Catabolite Repression)
%and SPS (SSY1-PTR3-SSY5) regulons is primarily a function of growth in
%a nitrogen-limited environment, with the molecular form of nitrogen
%having minimal effect on differential gene expression when cells are
%limited for nitrogen. By contrast, the GAAC (General Amino Acid
%Control) and UPR (Unfolded Protein Response) regulons do not respond
%specifically to nitrogen limitation compared with other
%nutrient-limited conditions. 
To study the dynamics of
nitrogen-responsive gene expression, we performed transient
perturbation experiments in which different quantities and sources of
nitrogen were added to cells growing in nitrogen-limited chemostats.
The addition of either the preferred nitrogen source, glutamine, or
the nonpreferred nitrogen source, proline, to cells growing in
nitrogen-limited conditions results in rapid repression of the NCR
regulon in a dose-dependent manner. Surprisingly, a sudden increase in
environmental nitrogen does not correspond to a detectable increase in
biomass production or cell number, consistent with a time delay
between activation of the transcriptional growth program and its
manifestation in an increased rate of cell growth. To compare global
gene expression in dynamic conditions with mRNA expression in
steady-state conditions, we used computational estimation of
instantaneous growth rate from gene expression profiles 
\parencite{brauer2008coordination,airoldi2009predicting} 
and defined gene expression responses to
growth rate in both steady-state and dynamic conditions using linear
regression. We find that the response of transcripts required for
protein translation (RP and RiBi) in cells provided with an increase
in nitrogen exceeds the response to growth rate in cells growing in
steady-state conditions consistent with a transient overproduction of
RP and RiBi transcripts. Finally, we show that accelerated degradation
of some NCR transcripts underlies gene expression remodeling in
response to sudden relief from nitrogen limitation, indicating the
activity of a posttranscriptional mechanism controlling
nitrogen-responsive gene expression.  

\section{Results} 

To obtain a high-resolution view of mRNA abundance
changes during the first 10 min after addition of nitrogen, when
changes in gene expression are maximal, we repeated the
pulse experiments (addition of nitrogen source to yeast grown in
steady-state nitrogen-limited chemostat cultures) and assayed global
gene expression at 1-2 min intervals after the addition of 40 $\mu$M
glutamine or 80 $\mu$M proline. We observed a rapid increase in expression
of the RiBi and RP regulons in response to a pulse of glutamine, with
a concomitant rapid decrease in expression of the NCR-A and NCR-P
regulons. Consistent with our initial observation, we observed a
similar response to a pulse of proline. 

\subsection{Accelerated degradation of
mRNAs contributes to remodeling of the transcriptome }

The majority of
NCR transcripts are strongly repressed in response to a nitrogen
pulse. If gene expression is repressed at the promoters of these genes
and mRNA synthesis ceases, the decrease in mRNA abundance is expected
to be a function of the degradation rate of the corresponding mRNA.
Using our high-density time-series data, we estimated the rate of
change in abundance for all transcripts, assuming a first-order
exponential degradation model (Materials and Methods; Supplemental
Table S7), which is the standard method for estimating mRNA
degradation rates. We found that in response to a glutamine pulse, 269
genes fit a first-order exponential decay model (FDR < 0.05;
Supplemental Table S4), whereas 458 transcripts fit a first-order
exponential decay model in response to the proline pulse (Supplemental
Table S4).  

\newpage

\afig{
  \includegraphics[width=\textwidth,bb=0 0 310 300]{img/airoldi2016_F6.large.jpg}
  }{
  \textcolor{white}{\label{fig:airoldi2016f6}}
  }{
  Accelerated mRNA degradation contributes to gene expression 
  remodeling
  }

\begin{framed}
\noindent
\autoref{fig:airoldi2016f6} ---
Accelerated mRNA degradation contributes
to gene expression remodeling. Upon addition of glutamine to
NCR-derepressed cells, a subset of transcripts degrade more rapidly
than their steady-state degradation rate both (A) in cells grown in
ammonia-limited chemostats and (B) in cells growing in proline media
in batch cultures. All points are genes that fit a model of
exponential decrease in abundance (FDR < 0.05). Orange points are NCR
genes that show significant accelerated degradation, blue points are
NCR genes that are not significant, green points are non-NCR genes
that show significantly accelerated degradation, and gray points are
genes that are neither accelerated nor NCR. The dashed line denotes
equal degradation rates in both conditions (i.e., slope equal to 1).
Names of nitrogen transporter genes are displayed. We measured the
transient changes in the degradation rates of (C) \textit{GAP1} and (D) \textit{DIP5}
mRNA using a pulse-chase experiment. Cells were grown for 24 h in the
presence of 4-thiouracil, which was chased at t = 0 min by the
addition of excess uracil. At t = 13 min, we added either glutamine in
water (orange) or equal volume of water (blue). We extracted and
quantified the abundance of 4-thiouracil-labeled mRNA relative to a
thiolated external spike-in using qPCR. We found significant
acceleration of degradation for both \textit{GAP1} and \textit{DIP5} mRNAs (p < 0.001).
Points are the mean of triplicate qPCR measurements, error bars are
the propagated SD of transcript and spike-in measurements, and dotted
lines are the log-linear model fit.  
\end{framed}

We compared the half-lives of rapidly degraded transcripts
after the glutamine pulse with half-life estimates in steady-state
conditions determined using RATE-seq 
\parencite{neymotin2014determination}. We found
that some transcripts decay significantly faster than expected,
suggesting that their degradation rate is accelerated in response to
the glutamine pulse (\autoref{fig:airoldi2016f6}A). 
Batch culture growth in proline also
results in derepression of the NCR regulon 
\parencite{godard2007effect}. To
test whether accelerated mRNA decay is specifically a response to the
nitrogen-limited conditions of a chemostat, we added a pulse of
glutamine to cells growing in batch cultures containing proline as a
sole nitrogen source and measured genome-wide gene expression
(Supplemental Table S7). The half-lives of transcripts that exhibit an
exponential decrease is similar in chemostat and batch cultures
(Supplemental Figure S7B), and many of the same transcripts show
evidence of accelerated degradation rates in batch cultures 
(\autoref{fig:airoldi2016f6}B
and Supplemental Table S4). Strikingly, the five nitrogen permease
genes \textit{GAP1}, \textit{DIP5}, \textit{MEP2}, 
\textit{PUT4}, and \textit{DAL5} are the most rapidly cleared
mRNAs in both the chemostat and batch culture experiments.  

To verify
that the addition of glutamine stimulates accelerated degradation of
specific NCR transcripts, we performed pulse-chase experiments using
the metabolic label 4-thiouracil (4-tU). After several generations of
batch culture growth in proline medium in the presence of 4-tU to
allow complete labeling of mRNAs, we added unlabeled uracil to the
culture. We allowed the chase to occur for 13 min and then added
either glutamine or water (mock) to the cells. We purified labeled
transcripts and analyzed \textit{GAP1} and \textit{DIP5} 
mRNAs using quantitative PCR
(qPCR) and normalization to external spike-ins. Consistent with our
genome-wide assay, the addition of glutamine results in a clear
accelerated degradation of both \textit{GAP1} mRNA 
(\autoref{fig:airoldi2016f6}C)
and \textit{DIP5} mRNA (\autoref{fig:airoldi2016f6}D), 
confirming that the transition from NCR-derepressed to
NCR-repressed conditions results in the accelerated degradation of
some transcripts.  

\section{Discussion} 

Some mRNAs are rapidly degraded when cells
transition from NCR-activating to NCR-repressing conditions in both
chemostats and batch culture. Comparison with mRNA degradation rates
suggests that the degradation of some of these transcripts is
accelerated. Using in vivo metabolic labeling with 4-tU, we provide
additional evidence that the addition of glutamine to nitrogen-limited
cells accelerates the degradation of specific transcripts. A previous
study of the transcriptional response to glucose addition in
carbon-limited chemostats suggested a role for accelerated degradation
of mRNAs 
\parencite{kresnowati2006transcriptome}
, and there is increasing evidence
that mRNA stability plays an important role in regulating gene
expression programs 
\parencite{puig2005coordinated,bennett2008metabolic,baumgartner2011antagonistic}. 
Consistent with a posttranscriptional
mechanism underlying the rapid clearing of some NCR transcripts,
previous work showed that \textit{GAP1} mRNA transiently decreases in abundance
during a nitrogen up-shift in the absence of \textit{URE2} 
\parencite{ter1998repression}, 
which is required for NCR repression by sequestering \textit{GLN3} in
the cytoplasm. Several studies have shown that TORC1 can affect
transcript stability 
\parencite{albig2001target,munchel2011dynamic}.
Our results suggest that posttranscriptional regulation of mRNA
stability may play an important role in remodeling gene expression in
response to changes in environmental nitrogen. Transient stabilization
of the RP and RiBi regulons also could contribute to their rapid
increase in expression 
\parencite{yin2003glucose}. Defining the role of
regulated changes in mRNA stability in dynamic conditions is an
important area for further study. 

What is the underlying rationale
for rapid induction of RP/RiBi transcripts occurring in parallel with
accelerated degradation of NCR transcripts? We propose that
accelerated degradation of NCR transcripts may allow for reallocation
of ribosomes to transcripts required for growth and proliferation
\parencite{kief1981coordinate,lee2011dynamic}. Our observations are
consistent with a model in which TORC1 orchestrates the balance
between transcripts required for protein production and transcripts
required for the acquisition and assimilation of nitrogen. When
nitrogen is abundant, TORC1 activates the expression of the RP and
RiBi regulons while actively repressing the NCR-A and NCR-P regulons.
Conversely, when nitrogen levels are in growth-limiting
concentrations, TORC1 activity decreases, leading to reduced
activation of the RP and RiBi regulons and derepression of the NCR-A
and NCR-P regulons. In NCR-derepressing conditions, NCR transcripts,
including \textit{GAP1}, \textit{MEP2}, and \textit{PUT4}, 
are the most abundant transcripts
(Supplemental Table S5). When a cell encounters a sudden increase in
environmental nitrogen, some highly expressed transcripts may be
targeted for accelerated degradation to increase the pool of free
ribosomes facilitating rapid translation of newly transcribed RiBi and
RP transcripts, thereby accelerating physiological remodeling of the
cell for rapid growth.  

\section{Materials and methods}

\subsubsection{Strains and culturing conditions}

We used the prototrophic haploid strain FY4 (MATa), which
is isogenic to the S288c reference strain, for all experiments. We
used minimal defined media for all experiments, using a common base
medium for nitrogen limitation, as described previously 
\parencite{brauer2008coordination,boer2010growth}. The appropriate concentrations of
allantoin, glutamine, glutamate, urea, ammonium sulfate, proline, and
arginine were added from 100 mM stock. Batch culture experiments were
performed in 30$^{\circ}$C shaking incubators using 100-ml cultures. Continuous
culturing in chemostats using Sixfors bioreactors (Infors, Laurel, MD)
was performed as described 
\parencite{brauer2008coordination,boer2010growth}
using a 300-ml working volume. Culture parameters were determined
using either a Klett colorimeter or a Coulter counter after
sonication. For perturbation studies, a single bolus of proline,
glutamine, or a mix of both was added to the chemostat to a final
concentration of 80 or 800 $\mu$M nitrogen.  

\subsubsection{RNA analysis} 

Cell samples for
mRNA analysis were preserved by rapid filtration and quick freezing
using liquid nitrogen. We isolated total RNA using hot acid-phenol
extraction and subsequently purified RNA samples using RNeasy columns.
We performed gene expression profiling using Agilent (Santa Clara, CA)
60-mer DNA microarrays and Cy3 and Cy5 incorporation as previously
described 
\parencite{brauer2008coordination}. We used a common reference obtained
from a sample growing in an ammonium sulfate-limited chemostat at a
dilution rate of 0.12 hours$^1$ for all hybridization experiments and
hybridized labeled cRNA to Agilent Yeast DNA microarrays for 20 h at
65$^{\circ}$C. We washed arrays and scanned microarrays using an Agilent
two-color scanner and extracted hybridization signals using Agilent
Feature Extractor Software. Supplemental Table S6 gives the entire
data set of processed log$_2$ ratios.  

\subsubsection{Pulse chase} 

Cells were grown in
600 ml of minimal medium containing 800 $\mu$M proline, 500 $\mu$M uracil, and
500 $\mu$M 4-thiouracil at 30$^{\circ}$C for 24 h. The culture was divided into two
300-ml cultures, and uracil was added to a final concentration of 2mM. 
We acquired 20-ml samples after the chase using rapid filtration
and flash freezing in liquid nitrogen. At 13 min after starting the
chase, we added either glutamine to a final concentration of 400 $\mu$M or
an equal volume of water and acquired additional samples.  

After RNA
extraction, samples were mixed with an in vitro-transcribed thiolated
spike-in (\textit{BAC1}200) at a ratio of 1 ng of spike-in to 25 $\mu$g of total
RNA and reacted with EZ-Link HPDP-Biotin (ThermoFisher Scientific,
Waltham, MA) at 2 mg/ml for 200 min. Reactions were cleaned up by
centrifugation and ethanol precipitation and then conjugated with 180
$\mu$l of streptavidin magnetic beads (M0253L; NEB, Ipswich, MA). Labeled
RNA was eluted using 5\% $\beta$-mercaptoethanol.  

Samples were reverse
transcribed with Moloney murine leukemia virus reverse transcriptase
(NEB) and random hexamer priming. We performed qPCR in technical
triplicate on a LightCycler 480 (Roche, Branchburg, NJ) using the
following primers: \\[1em]
\begin{tabular}{l | p{15em} p{15em}}
%  \toprule
%  mRNA & Forward & Reverse \\                                                
%  \midrule
  \textit{GAP1} & 5'-\texttt{ACGGTATCAAGGGTTTGCCAAG}-3' &
    5'-\texttt{GCATAAATGGCAGAGTTAC}-3' \\
  \textit{DIP5} & 5'-\texttt{TGGCGTACATGAATGTGTCTTCA}-3' &
    5'-\texttt{GGTGATCCAACTCAAGATTC}-3' \\
  BAC1200 & 5'-\texttt{CTGGACGACTTCGACTACGG}-3' & 
    5'-\texttt{ATCAGCCTTTCCTTTCGTCA}-3' \\
\end{tabular} \\[1em]
$C_p$ values
were calculated for each sample and the spike-in and log-linear
regression performed using the ratio of either \textit{GAP1} mRNA or \textit{DIP5} mRNA
to the spike-in in R.  

\subsubsection{mRNA decay estimation}

We estimated rates of
mRNA decay for all transcripts using high-temporal resolution data. We
used ratios ($y_t$) of hybridization intensities for each transcript
obtained from two-color DNA microarrays co-hybridized with a common
reference. Data were normalized to the initial data point ($y_0$) and
then log-transformed. We modeled the degradation rate $k_deg$ of each
gene:
$$ ln\left(\frac{y_t}{y_0}\right)=k_{\text{deg}}\times t$$
where $t$ is the sampling time in minutes. Transcript
half-lives were computed as $\frac{ln(2)}{k_{deg}}$. 
Accelerated degradation was
assessed by fitting the model 
$$ ln\left(\frac{y_t}{y_0}\right)=(k_{\text{transient
deg}}+k_{\text{steady-state deg}})\times t$$
where $k_{steady-state deg}$ 
is the specific degradation rate for transcript $i$ as reported in 
\cite{neymotin2014determination}.
For all linear modeling, we assessed statistical
significance of coefficients using a t-statistic and determined
empirical p-values by permuting data for each gene 1000 times. The
false discovery rate was determined using the \texttt{qvalue} package in R.
Data availability DNA microarray data are available through gene
expression omnibus (GEO) GSE57293.

