
\chapter{measuring destabilization and looking for regulators}

This
has been submitted for publication, as Miller, Brandt, Gresham 2018.
“BarSeq after FACS after FISH identifies mRNA decapping factors
associated with the swift repression of GAP1 mRNA upon a nitrogen
upshift” Darach Miller, Nathan Brandt, and David Gresham 

\section{Abstract}

Cellular responses to changing environmental conditions frequently
involve rapid reprogramming of the transcriptome, in part by
modulating mRNA synthesis. Altered mRNA degradation rates can
accelerate this mRNA regulation by acting to clear or stabilize extant
transcripts. Understanding the extent and mechanisms of
post-transcriptional regulation in different dynamic conditions will
allow us to determine if this regulatory mechanism is adaptive and
how. Budding yeast respond to an improvement in nitrogen-availability
by triggering a transcriptional reprogramming that functions to
upregulate ribosome biogenesis and repress alternative nitrogen-source
catabolism. Here, we measured mRNA stability across a nitrogen upshift
and found that 78 mRNA are subject to destabilization. This set is
enriched for factors of the Nitrogen Catabolite Repression (NCR)
regulon, secondary-active transporters, and factors of carbon
metabolism, suggesting that mRNA destabilization upon the upshift is
an uncharacterized mechanism contributing to NCR and other functional
changes. To explore the molecular basis of destabilization we focused
on a gene subject to NCR-control and a 3-fold increase in degradation
rate, GAP1. We developed a method to screen for the trans genetic
factors of this swift mRNA repression, combining branched-DNA mRNA
FISH, fluorescence-activated cell sorting, and low-input multiplexed
barcode sequencing to estimate GAP1 mRNA abundance for mutants in a
pooled prototrophic haploid yeast deletion collection. By using the
phenotype of defective GAP1 mRNA dynamics during the nitrogen upshift,
we identify that the Lsm1-7/Pat1 complex plays a role in GAP1
repression and that modulators of decapping activity also perturb the
GAP1 dynamics in combination with elements in the 5’ UTR. These
results identify candidates that may be responsible for transducing
the signal of a nitrogen upshift and suggests that different
translational status may be associated with altered stability upon the
nitrogen upshift.  

\section{Introduction}

Regulation of mRNA abundance in
response to environmental signals operates in concert with additional
layers of gene expression regulation, and is a centrally important
aspect of cellular responses to perturbations. Both synthesis and
degradation rates of mRNA determine the steady-state abundance of a
particular mRNA and modulate the kinetics with which regulation occurs
(Pérez-Ortín et al. 2013). In budding yeast, the rate of mRNA
degradation is affected by changes in many factors including
nutritional conditions  (Munchel et al. 2011; Jona et al. 2000),
cellular growth rate (García-Martínez et al. 2016), and environmental
stresses (Canadell et al. 2015). Regulated changes in mRNA degradation
rates during C. elegans oogenesis (West et al. 2017) and the early
early development of Drosophila (Alonso 2012) indicate that mRNA
degradation rate regulation fulfills an important mechanistic role in
gene expression regulation in diverse systems.  

Environmental shifts
trigger rapid reprogramming of the budding yeast transcriptome in
response to various stresses and nutrient additions (Gasch et al.
2000; Conway et al. 2012). mRNA degradation rate changes have been
characterized to play a role in responses to heat-shock, osmotic
stress, pH increases, and oxidative stress, sharing a similar program
of destabilization of mRNA coding for ribosomal-biogenesis gene
products and stabilization of stress-responsive mRNA (Canadell et al.
2015). Simultaneous increases in both synthesis and degradation rates
of some of these mRNA are suggested to serve to return the
transcriptome quickly to a new steady-state after effecting a
transient pulse of regulation (Shalem et al. 2008). Relieving nutrient
limitation with a glucose upshift has been shown to mediate both
stabilization of mRNA in the ribosomal protein subunit regulon (Yin et
al. 2003) and destabilization of gluconeogenic transcripts (de la Cruz
et al. 2002; Mercado et al. 1994). Destabilization of transcripts is
expected to reduce the corresponding protein products, but
transcriptional repression can have a delayed effect on reducing
protein levels compared to up-regulated genes (Lee et al. 2011). This
suggests that accelerated mRNA degradation may serve a different role.
Others have suggested that degradation can help recycle nucleotides
(Kresnowati et al. 2006) or that reprogramming the transcriptome would
help to reallocate the extant translational capacity of the cell to
enact a growth-optimal program (Kief and Warner 1981; Giordano et al.
2016; Shachrai et al. 2010). Identifying the genetic factors
responsible for the degradation would allow us to test if the
destabilization is adaptive, and if so to make progress in
understanding the mechanistic basis of this phenomenon. 

 Yeast cells
metabolize a wide variety of nitrogen sources, but preferentially
assimilate and metabolize specific nitrogen compounds. This is in part
controlled through transcriptional regulation known as “nitrogen
catabolite repression” (NCR)  (Magasanik and Kaiser 2002) of  mRNA
encoding a variety of transporters, metabolic enzymes, and regulatory
factors. Transcriptional synthesis repression is relieved in the
absence of a preferred nitrogen source or in the presence of
growth-limiting concentrations (in the low uM range) of any nitrogen
source, including preferred nitrogen sources (Airoldi et al. 2016;
Godard et al. 2007).  Expression of NCR target genes is mediated by
two activating GATA factors, Gln3p and Gat1p, and two additional
repressing GATA factors, Dal80p and Gzf3p that function to modulate
NCR expression levels. GAT1, GZF3, and DAL80 promoters contain GATAA
binding sites and thus transcriptional regulation of NCR targets
entails self-regulatory and cross-regulatory loops. When supplied with
a preferred nitrogen source such as glutamine, the NCR-activating
transcription factors Gat1p and Gln3p are excluded from the nucleus by
multiple mechanisms (Tate and Cooper 2013; Tate et al. 2017), but the
activity of products of genes subject to NCR are sometimes also
controlled after transcription occurs (Cooper and Sumrada 1983). For
example the General Amino-acid Permease Gap1p is known to be
post-translationally regulated (Stanbrough and Magasanik 1995), and is
rapidly inactivated upon a nitrogen upshift with adaptive consequences
(Merhi and Andre 2012; Risinger et al. 2006). Recently, we have
identified an additional level of regulation of NCR transcripts: cells
growing in NCR de-repressing conditions accelerated the degradation
rate of GAP1 and DIP5 mRNAs upon addition of glutamine (Airoldi et al.
2016). Thus, mRNA degradation control could be an uncharacterized
mechanism of immediately effecting NCR and other mRNA regulation upon
a glutamine upshift.  


Multiple pathways mediate the degradation of
mRNAs. The main pathway of mRNA degradation is deadenylation and
decapping prior to 5’ to 3’ exonucleolytic degradation by Xrn1p;
however, transcripts are also degraded 3’ to 5’ via the exosome, or
via activation of cotranslational quality control mechanisms (Parker
2012). Deadenylation of mRNAs by the Ccr4-Not complex allows the mRNA
to be bound by the Lsm1-7p/Pat1p complex, a heptameric ring comprising
the SM-like proteins Lsm2-7 and the cytoplasmic-specifying Lsm1
(Tharun et al. 2000; Sharif and Conti 2013), which then recruits
factors for decapping of the mRNA by Dcp2, followed by degradation by
Xrn1p. This pathway is rate-limited by the recruitment of the
decapping enzyme (Coller and Parker 2004), therefore Lsm1-7p, Pat1p,
and associated factors play a key role in regulating mRNA
degradation(Nissan et al. 2010). Regulation of this and other pathways
of degradation can alter the stability of individual mRNAs. In the
canonical example of Puf3p, this RNA-binding protein (RBP) recognizes
a unique sequence cis-element in 3’ UTRs (Olivas and Parker 2000) and
the effect of this association on mRNA degradation rates varies
depending on Puf3p phosphorylation status (Miller et al. 2014).
Upstream open reading frames (uORFs) can also trigger degradation of a
specific mRNA in cis via activation of mRNA quality control pathways
(Hinnebusch 2005). Recent studies indicate that transcript properties
associated with rates of translation affect mRNA degradation (Presnyak
et al. 2015; Neymotin et al. 2016) and previously an elegant mechanism
of competition between the decapping enzymes and translation
initiation processes has been described (????), demonstrating the
connection between mRNA translation and degradation. In addition to
RNA cis-elements, promoters have been shown to mark certain
RNA-protein (RNP) complexes to specify their post-transcriptional
regulation (Trcek et al. 2011; Braun et al. 2015; Haimovich et al.
2013; Mercado et al. 1994). This effector mechanisms may be controlled
by a variety of different signalling pathways including Snf1 (Young et
al. 2012; Braun et al. 2014) or TORC1 (Talarek et al. 2010). Thus, the
regulation of mRNA degradation rates entails numerous mechanisms that
collectively the tune stability of mRNAs in response to signalling
pathways.  

Here, we studied the global regulation of mRNA degradation
rates upon a nitrogen upshift using 4-thiouracil (4tU) labeling and
RNAseq. We find that a set of 78 mRNAs show clear evidence for
accelerated mRNA degradation, including many NCR transcripts as well
as components of carbon metabolism. To identify the mechanism
underlying accelerated mRNA degradation of a specific transcript,
GAP1, we developed a high-throughput genetic screen using single
molecule mRNA FISH (smFISH) as a marker for fluorescent activated cell
sorting (FACS), using barcode-sequencing (Bar-seq) to quantify DNA
barcodes in each bin, and modeling to estimate mRNA abundance for each
DNA barcode in the pool. We applied this to the barcoded yeast
deletion collection to screen for effects of each gene deletion on the
abundance of GAP1 mRNA in NCR de-repressing conditions of growth on
proline and the rapid repression of GAP1 mRNA 10 minutes after the
addition of glutamine, a repressive nitrogen source. We find that the
Lsm1-7p/Pat1p complex and decapping modifiers affect both GAP1 mRNA
steady-state expression and the accelerated degradation of GAP1 mRNA
upon a nitrogen upshift. This work expands our understanding of mRNA
stability regulation in remodeling the transcriptome during a relief
from growth-limitation and demonstrates a generalizable approach to
the study of genetic determinants of mRNA dynamics.  

\section{Results}
Transcriptional reprogramming precedes physiological remodeling
Cellular responses to environmental signals entail coordinated changes
in both gene expression and cellular physiology.  Previously, we
studied the steady-state and dynamic responses of Saccharomyces
cerevisiae (budding yeast) to environmental nitrogen (Airoldi et al.
2016), and found that the transcriptome is rapidly reprogrammed
following a nitrogen upshift in either a chemostat or batch culture.
To study physiological changes in response to an upshift, we measured
cell size and cell proliferation rates. A prototrophic haploid lab
strain (FY4, isogenic to S288c) grows with a 4.5 hour doubling time in
minimal media containing proline as a sole nitrogen source (Fig 1a).
Upon addition of 400uM glutamine the cells undergo a 2-hour lag period
during which growth rate remains unchanged (Fig 1a), but  the average
cell size continuously increases (21\% increase in mean volume)  (Fig
1b). The culture adopts a 2.1 hour doubling time following a lag of
about 2 hours as previously described (Carter et al. 1978), suggesting
that extensive physiological remodeling is required for faster growth
in response to a nitrogen upshift. By contrast, global gene expression
changes within three minutes of the upshift (Airoldi et al. 2016).
Thus, transcriptome remodeling precedes physiological remodeling and
initiation of a new rapid-growth state following a nitrogen upshift.
  

Figure 1. Kinetics of physiological and transcriptome remodeling
during a nitrogen upshift.  A nitrogen upshift was performed by adding
400uM glutamine to a culture of yeast cells growing in minimal media
containing 800uM proline as a sole nitrogen source. a) Culture density
before and after the upshift. Dotted lines denote linear regression of
the natural log of cell density against time before the upshift and 2
hours after the upshift. b) Cell size changes during the upshift.
Dotted lines denote the mean cell diameter before the upshift and  2
hours after the upshift. c) PCA analysis of global mRNA expression in
steady-state chemostats and during an upshift (Airoldi et al. 2016).
Steady-state nitrogen-limited chemostat cultures maintained at
different growth rates (colored circles) primarily vary along
principal component 2.  Expression following a nitrogen-upshift in
either a chemostat (squares) or batch culture (triangles) show similar
trajectories and primarily vary along principal component 1.  Fit grey
lines  illustrate the major trajectory of variation for the
steady-state and upshift experiments.  To evaluate concordance in
transcriptome remodeling between the two nitrogen upshift conditions,
and the extent to which they reflect changes in gene expression
observed during steady-state growth, we performed  principal component
analysis of global gene expression (Figure 1c). The first two
principal components, which account for almost half of the total
variation, clearly separate steady-state and nitrogen upshift
cultures.  Systematic changes in growth rate using steady-state
chemostat cultures primarily result in separation of gene expression
states along the second principal component.  By contrast, gene
expression states following a nitrogen upshift in both chemostat and
batch cultures follow similar trajectories, along the first principal
component.  This suggests that although a nitrogen upshift results in
a gene expression state that reflects an increase in cell growth rate
(Airoldi et al. 2016), the transcriptome is remodeled through a
distinct state.  In longer time-series experiments performed in
upshift experiments in chemostats, the gene expression trajectory
returns to the initial steady-state condition as the excess nitrogen
is depleted by consumption and dilution (Supplementary Figure 1).  To
investigate the functional reprogramming that distinguishes gene
expression in the upshift and steady-state conditions, we computed the
correlation of each transcript with the loadings on these first two
principal components and performed gene-set enrichment analysis
(Supplementary Table 1). Both steady-state and dynamic gene
expression trajectories increase with principal component 2, but they
diverge along principal component 2. Transcripts that are positively
correlated with principal component 1 and principal component 2 are
enriched for functions including ribosome biogenesis, nucleolus, and
tRNA processing and transcripts that are negatively correlated are
enriched for mRNAs associated with the vacuole, cell cortex, and
carbohydrate metabolism.  Transcripts that are positively correlated
with principal component 1 are enriched for  mitochondrial translation
and structural constituents of the ribosome, and negatively correlated
transcripts are enriched for endocytosis, plasma membrane,
sporulation, cell wall biogenesis, and responses stress responses.
This suggests that although the dynamic reprogramming of gene
expression in response to a nitrogen upshift reflects changes in
cellular processes that are also observed during systematic changes in
that steady-state growth rates, a nitrogen upshift results in a
distinct set of gene expression changes that likely reflect a distinct
physiological process of adapting to the new condition. Importantly,
the reprogramming is common to both the chemostat and the batch
upshift (Figure 1c). As batch cultures are a technically simpler
experimental system, we performed all subsequent experiments using
batch culture nitrogen upshifts.  Global analysis of mRNA stability
changes during the upshift Previously, we found that GAP1 and DIP5
mRNAs are destabilized in response to a nitrogen upshift (Airoldi et
al. 2016). We sought to determine if mRNA destabilization is specific
to NCR transporter mRNAs. Therefore, we combined 4-thiouracil (4tU)
labeling and RNA-seq using a label-chase design (Neymotin et al. 2014)
(Munchel et al. 2011). As 4tU labeling requires nucleotide transport,
which may be altered upon a nitrogen-upshift (Hein et al. 1995), we
designed experiments such that the chase was initiated prior to the
nitrogen upshift and analyzed data to quantify the change in stability
of each transcript upon the upshift (Figure 2a). We normalized data
using thiolated spike-ins by fitting a log-linear model to spike-in
counts across time (Supplementary Materials), which reduced noise and
increased our power to detect stability changes (Supplementary Table
2, Supplementary Table 3, Supplementary Table 4).  All model fits can
be explored using a Shiny application (see below).
  

Figure 2. Global mRNA stability change following a nitrogen upshift.
a) 4tU-labeled mRNA from each gene was measured over time, before and
after the addition of glutamine (nitrogen-upshift) or water (mock) at
the vertical dotted line. Linear regression models were fit to the
data with a pre-shift slope (solid line) and a post-shift change in
slope (dashed line). HTA1 is not destabilized, whereas mRNAs encoding
NCR-regulated transporters or pyruvate and trehalose metabolism
enzymes are destabilized. b) comparison between the pre-shift mRNA
degradation rate (y-axis) and the upshift mRNA degradation rate
(x-axis)  c)  comparison between changes in mRNA expression following
upshift (Airoldi et al. 2016) (y-axis) and the upshift mRNA
degradation rate (x-axis). Both plots share the same x-axis
Transcripts that are significantly destabilized are colored red, and
shown with red rug-marks in the marginal histogram for the x-axis at
top, and for each y-axis on the right . c)


To detect changes in mRNA degradation rates, we modeled
log-transformed data using linear regression and quantified the
significance of slope changes corrected for multiple hypothesis
testing ( Supplementary Table 5). Of 4,859 mRNAs that we measured we
identified 94 that increased in degradation rate and 38 that decreased
(FDR < 0.01 Storey et al. 2015). We used a model of nucleotide
labeling kinetics to assess the effect of an incomplete chase on our
experimental design ( Supplementary Materials), and found that
complete transcriptional inhibition alone could result in an apparent
17\% increase in the degradation rate (for a typical transcript by
assuming a median degradation rate prior to the upshift). Accordingly,
since transcript synthesis is bounded by zero, we choose to exclude
any apparent stabilization effects and focus only on destabilization
of at least a doubling (100\% increase) of degradation rates between
pre-shift and post-shift. This left 78 mRNA significantly destabilized
upon a nitrogen upshift. Given that many of these destabilized
transcripts are below the median degradation rate, we expect that the
possible contribution of synthesis rate changes to the measurement are
well below the median 17\% effect of instantaneous repression. This
conservative cutoff of at least a doubling in decay rates is a
tradeoff that appropriately restricts our calls of significant
destabilization to indicating a rapidly triggered destabilization of
the mRNA.  The median pre-shift half-life was 6.89 minutes and the
median half-life following the upshift was 6.4 minutes (Table 1)
suggesting that there is not a large global change in mRNA stability.
Indeed, the vast majority of transcripts (4,796 of 4,859) do not show
individual evidence for stability changes upon addition of glutamine
(e.g.  HTA1, Figure 2a). Transcripts that are destabilized following
the upshift tend to have slower decay rates in steady-state conditions
(Figure 2b). Global stability estimates are considerably lower than
previous estimates in rich medium (Munchel et al. 2011) (Neymotin et
al. 2014), which may reflect the different nutrient conditions or
methods used in our study. For 78 transcripts, we detect significant
destabilization upon the glutamine-upshift. These transcripts have a
median half-life of 9.46 minutes before the upshift and a median
half-life of 3.02 minutes following the upshift (a median 3.06-fold
increase in degradation rates). For example, mRNAs for the NCR
transporters GAP1, DAL5, and MEP2 (blue label, Figure 2a), the
pyruvate metabolism enzymes PYK2 and PYC1 (orange label), and
trehalose synthase subunits TPS1 and TPS2 (yellow label) are
destabilized upon the nitrogen upshift. Data and models for each gene
are available for inspection as an GUI interactive shiny object tool
(see http://shiny.bio.nyu.edu/users/dhm267/ or Supplementary Material
for instructions).  Table 1. Summaries of mRNA stability measurements
for  total and destabilized mRNA


	Median pre-shift
	

	Median post-shift
	

	Median change
	

	Specific degradation rate half-life (minutes) Specific degradation
rate half-life (minutes) Specific degradation rate All transcripts
0.100 6.92 0.110 6.32 0.00865 - Destabilized set 0.0732 9.46 0.229
3.02 0.158 - No change 0.101 6.89 0.108 6.40 0.00728 
	

To determine the specificity of mRNA destabilization we tested for
functional enrichment among the set of 78 most strongly destabilized
mRNAs. Destabilized transcripts are strongly enriched for NCR
transcripts (16 of  78, p < 10-11. Approximately half of the
destabilized transcripts are also annotated as “ESR-up” genes
(Supplementary Figure X), on the basis of  their rapid induction
during the environmental stress response (Gasch et al. 2000). The 78
genes that show accelerated mRNA degradation are also enriched (FDR <
0.05) for GO terms and KEGG pathways (Supplementary Table 7)
including glycolysis/gluconeogenesis (6), carbohydrate metabolic
process (24), trehalose-phosphatase activity (3), glycogen metabolic
process (7), pyruvate metabolic process (6), exopeptidase activity
(5), allantoin metabolic process (3), and secondary active
transmembrane transporter activity (8). This analysis shows that the
destabilization of mRNA upon a nitrogen upshift is a regulatory
mechanism enriched for, but not restricted to, NCR-regulated mRNA.
Destabilization also targets transcripts required for secondary-active
transporters and specific aspects of carbon metabolism, namely
glycogen, trehalose, and pyruvate metabolism.  To investigate the
extent to which mRNA stability changes contribute to transcriptome
reprogramming, we compared upshift degradation rates to mRNA
expression changes following the  upshift (as reported in (Airoldi et
al. 2016), Figure 2c). Changes in mRNA degradation rates and
expression change rates are anti-correlated (R2= -0.376), consistent
with stability changes impacting gene expression dynamics. However,
they are not entirely co-incident, as some destabilized transcripts do
not exhibit decreases in abundance (red points in Figure 2c and
Supplementary Figure 4), showing that expression changes do not
explain all changes in measured degradation rates. This analysis shows
that  increases in degradation rates play a large role in the rapid
reprogramming of the transcriptome upon a glutamine upshift, but that
in some cases cases they are counteracted by increases in mRNA
synthesis rates resulting in no significant change in abundance
(Shalem et al. 2008).  Functional coordination of mRNA stability
changes suggests  a possible role for cis-element regulation. We
analyzed UTRs and CDS for the presence of  sequence motifs or
enrichment for known motifs of RBPs. We tested for sequence and
structural motifs using several software tools (Materials and
Methods), and two definitions of UTR sequences (using a fixed 200bp
window upstream or downstream of the CDS or using the most distal
isoform ends defined by TIF-seq) (Pelechano et al. 2014). 3’ UTRs of
destabilized transcripts were depleted of Puf3p binding sites,
suggesting that this factor is not involved and regulates transcripts
distinct from this set (Supplementary Figure X). 5’ UTRs are enriched
for a GGGG motif, which we may be explained by convergence between
mRNA stability changes and transcript synthesis rate control as a
result of  Msn2/4, consistent with the overlap of membership with ESR
“up” genes (Supplementary Figure X). 5’ UTRs were enriched for
binding motifs reported for Hrp1p suggesting a role for this
nuclear-cytoplasmic shuttling  factor in affecting mRNA stability in
the cytosol (Supplementary Figure X). We also compared the CDS length
and fraction of optimal codons (Khong et al. 2017) in each feature and
found that on average, destabilized mRNAs are longer and contain more
optimal codons (Supplementary Figure X). Together, this analysis
suggests that the mechanism of destabilization may be related to cis
elements in the 5’ UTR and potentially to the translational status of
the mRNA.
  

Figure 3. Quantifying the kinetics of accelerated GAP1 mRNA
degradation using BFF.  a) Kinetics of accelerated GAP1 mRNA
degradation . A nitrogen-limited (batch proline) culture of yeast was
upshifted (glutamine addition), and GAP1 mRNA quantified using RT-qPCR
relative to an external spike-in mRNA standard. The dashed line
indicates a log-linear regression model fit to points after 2 minutes.
b) Flow cytometry of wild-type yeast in nitrogen-limited conditions
and following an upshift. The vertical grey lines indicate FACS gate
boundaries used for cell sorting. c) Representative cells  from each
bin sorted from wildtype cells. d) Quantification of the microscopy
data. Each black dot represents a single cell  The mean number of foci
per cell in  each bin  is displayed as a red point.  Developing a
genome-wide screen for trans-factors of GAP1 mRNA repression We sought
to identify trans-factors mediating accelerated mRNA degradation in
response to a nitrogen upshift. We selected GAP1 mRNA as
representative of transcript destabilization, as it is abundant in
nitrogen-limiting conditions and is rapidly cleared upon addition of
glutamine  (3.24-fold increase in degradation rate) (Figure 3a)
(Supplementary Table 5). Previous approaches to high-throughput
genetics of mRNA abundance have used either protein expression
reporters (Neklesa and Davis 2009) or extensive automation of nucleic
acid purification and qPCR (Worley et al. 2015). However, for our
purposes, we required a method that was suitable for direct
measurement of GAP1 mRNA to report on the dynamics of accelerated mRNA
degradation and amenable to replicated measurements across multiple
timepoints, using available facilities. Therefore, we developed a
single molecule mRNA FISH (smFISH) assay to quantify native GAP1
transcripts in single-cells, facilitating efficient screening of the
prototrophic yeast deletion collection (VanderSluis et al. 2014) using
fluorescence activated cell sorting (FACS) to separate mutants on the
basis of mRNA abundance. The abundance of each strain can then be
estimated using BARseq, a method for amplicon sequencing of the
15-22bp strain barcodes that identify each knockout (Smith et al.
2009; Robinson et al. 2013; Giaever and Nislow 2014). We employed this
approach, which is a variant of the Sort-seq experimental approach
(Peterman and Levine 2016), to estimate transcript abundance for each
mutant in the pool at two timepoints: before the upshift which GAP1
mRNA is fully induced (pre-shift) and 10 minutes after the upshift
(post-shift). This approach facilitates identification of mutants with
defects in mRNA regulation at both the transcriptional and
post-transcriptional level without altering GAP1 mRNA cis-elements
that may affect its regulation. Moreover, our design enables
identification of factors that regulate both the steady-state
abundance of GAP1 mRNA and its accelerated degradation following an
upshift.  Development of our screen required that GAP1 mRNA could be
accurately quantified in single cells using flow cytometry. We found
that  individually labeled probes tiled across GAP1 mRNA (Zenklusen et
al. 2008; Raj et al. 2008) were insufficiently bright to assay GAP1
mRNA expression using flow cytometry (data not shown), likely due to
the small cell size of nitrogen-limited cells and the low transcript
numbers in yeast cells compared to mammalian cells (Klemm et al.
2014). Therefore, we used branched DNA probes (Quantigene), which
serve to amplify the smFISH signal (Hanley et al. 2013). We developed
a fixation and permeabilization protocol (Supplementary Methods )
that enabled detection of the  distribution of  GAP1 mRNA in
steady-state nitrogen-limited conditions and its repression following
the  upshift (Figure 3b). In control experiments, we found that the
signal is eliminated in a GAP1 deletion or by omitting the  targeting
probe that confers specificity (Supplementary Figure 5, Figure 3b).
To validate flow cytometry  quantification we sorted a sample of cells
into quartiles and counted the number of fluorescent foci per cell
using microscopy (Figure 3c) . We found that increased flow cytometry
signal is associated with an increase in the number of foci in the
cells (Figure 3d, R2 = 0.607, p-value < 10-11 ).


SortSeq requires barcode sequencing to quantify the abundance of
genotypes in pooled assays. Previous methods using FACS to fractionate
the yeast deletion collection have used outgrowth of the sorted
fractions to generate sufficient material for PCR and sequencing using
Barseq (Sliva et al. 2016).  However, the fixation of cells precludes
outgrowth. We found that below approximately 106 templates, the
barcode PCR reaction using universal primers produced primer dimers
that outcompete the barcode PCR product. Therefore, we re-designed the
PCR reaction  (Robinson et al. 2013; Smith et al. 2009) to be robust
for very low sample inputs (Supplementary Materials). Our protocol
incorporates a 6-bp unique molecular identifier (UMI) into the first
round of extension to help correct for spurious quantification noise
from PCR, and uses 3’ phosphorylated DNA oligonucleotides and a
strand-displacing polymerase to block primer dimer formation and off
target amplification. We developed a custom bioinformatics pipeline to
take advantage of this new amplicon design, using pairwise alignment
for per-read quality-filtering and compatibility with variable barcode
length, and using the degenerate UMI barcodes to help account for PCR
duplicates.  (methods).  We refer to our experimental approach as BFF
(Bar-seq after FACS after FISH).  We used BFF to estimate GAP1 mRNA
abundance for every mutant in the haploid prototrophic deletion
collection (VanderSluis et al. 2014) in nitrogen-limiting conditions
and 10 minutes following the upshift. We performed biological
triplicates of the upshift and sampling, then processed the samples
independently through GAP1 mRNA FISH and FACS sorting into quartiles
of GAP1 mRNA expression (Figure 4a). These heterogeneous samples of
mutants were quantified using our above amplicon-sequencing protocol
in technical triplicate for every bin of sorted cells. We found that
UMIs approached saturation at a slower rate per input sample size than
expected for random sampling, consistent with a PCR amplification bias
towards repeated measurements ( Supplementary Figure 6) and adopted
the correction from (Fu et al. 2011) to account for possible UMI
collisions. After filtering counts (Supplementary Materials), we
calculated a pseudo-events metric that approximates a metric of mutant
cells sorted into each bin, and is the proportion of counts per mutant
multiplied by a scaling factor of how much of the library was sorted
into each bin. Using principal components analysis of these
pseudo-events, we find that the samples are separated primarily by
FACS bin within each condition and biological replicates are clustered
indicating that our approach reproducibly captures the variation of
GAP1 mRNA flow cytometry signal across the library   ( Supplementary
Figure 7).
  

Figure 4. Validation of BFF estimates of GAP1 mRNA  abundance.  a)
Flow cytometry analysis of GAP1 mRNA abundance in the prototrophic
deletion collection before and after the upshift.  The x-axis reports
expression in arbitrary units (a.u.) of fluorescence, vertical gray
lines denote the FACS gate boundaries, biological replicates are
indicated by color. b) Measurements for individual genes before and
following the upshift.  Black dashed lines indicate the log-normal
model fits.  Colors and axes as in a. c) Distribution of  mean
expression level for each mutant derived from maximum likelihood
fitting of a log-normal to pseudo-events for each mutant. The model
fits are drawn as black points connected by dashed lines as the
proportion of the fit model for that strain in each bin. cd) The Mean
GAP1 mRNA expression levels for individual mutants before and after
the upshift,   The distribution of all expression values for the pool
of mutants is shown as a reflected histogram for reference.
Estimating GAP1 mRNA abundance for individual  mutants We quantified
the distribution of GAP1 mRNA for each mutant by modeling
pseudoevents in each quartile, converted to proportions as a
log-normal distribution using likelihood maximization methods  (Figure
4b). From model fits we estimated the mean expression value for each
mutant and found that the distribution of means within each replicate
were similarly distributed (Figure 4c) and reflected the overall
distribution of flow cytometry signal (Figure 4a). To estimate GAP1
mRNA per strain, we used all replicate measurement to perform model
fitting and filtered models for sufficient measurements  (at least two
of three replicates in at least three of the four bins). We generated
expression distribution estimates for 3,230 strains, and used the mean
of each distribution as the estimate of GAP1 mRNA abundance for each
strain (Supplementary Table 13).  To validate our approach we
examined strains for which we expected to have a specific phenotype
and compared their mean expression level to the distribution of
expression for the entire population (Figure 4d). We find that the
wildtype genotype (his3KO, which is complemented by the spHis5 gene as
part of library construction) has an expression level that is
centrally located in the distribution both before and following the
upshift. The gap1KO genotype is a negative control, as no GAP1 mRNA
should be produced in the genotype, and we estimate that it is at the
extreme low end of the distribution before and following the upshift.
dal80KO, gzf3KO, and ure2KO are direct or indirect transcriptional
repressors of NCR, and we find that their GAP1 mRNA expression is
defective in repression with the dal80KO genotype having a stronger
defect in GAP1 repression. Counter-intuitively, a gat1KO, a
transcriptional activator of GAP1, appears to have higher steady-state
expression of GAP1 mRNA, but increased expression of GAP1 mRNA in a
gat1KO mutant has previously been reported (Scherens et al. 2006) and
is thought to result from the complex interplay of NCR transcription
factors on their own expression levels. Thus, analysis of a small
number of genotypes assayed using BFF validates our approach to pooled
analysis of mutant expression of GAP1 mRNA in steady-state conditions
and following an upshift.  To identify cellular processes that
regulate GAP1 mRNA abundance, we used gene-set enrichment analysis
(Supplementary Table 14). Mutants that exhibit high GAP1 mRNA
expression are enriched (FDR < 0.05) for sulfate assimilation.
Following the upshift we find mutants that maintain high GAP1 mRNA
expression are enrichment for negative regulation of gluconeogenesis
(Supplementary Figure X) and the Lsm1-7p/Pat1p complex (Figure 5a).
Mutants in the TORC1 signalling pathway were not enriched; however, we
find that a tco89KO mutant has  greatly increased GAP1 mRNA expression
before and after the shift (Supplementary Figure X), consistent with
the repressive role of TORC1 on the NCR regulon.To compare the
expression between the two timepoints for each mutant, we first
compared the difference between the log of the estimated means for
each strain by regressing the post-shift mean expression against the
pre-shift mean expression for each genotype (Supplementary Figure 8).
We used the residuals for each strain to identify mutants that clear
GAP1 mRNA with slower kinetics than expected on the basis of
steady-state expression.  We find that the Lsm1-7p/Pat1p complex is
strongly enriched for slower than expected GAP1 mRNA clearance
(Supplementary Table 14). Specifically the lsm1KO, lsm6KO, and pat1KO
strains are strongly impaired in the repression of GAP1 mRNA (Figure
5a) .
  

Figure 5. Disrupting the Lsm1-7p/Pat1p complex impairs accelerated
degradation of  GAP1 mRNA.  a) Distribution of fit GAP1 mRNA means for
mutants in the pool. Indicated by colored points and lines are the
means for labeled knockouts. b-d), GAP1 mRNA is quantified relative to
HTA1 mRNA before or 10 minutes after a glutamine upshift, in
biological triplicates. Lines are a log-linear regression fit.


To confirm the role of the Lsm1-7p/Pat1p  complex in clearing GAP1
mRNA during the nitrogen upshift we measured the rate of GAP1 mRNA
repression using qPCR for GAP1 mRNA normalized to the housekeeping
gene HTA1, which is not subject to destabilization upon the upshift
(Figure 2a). We also tested mutants that were not detected using BFF,
or were not suitable for modeling due to only being detected in the
highest GAP1 bins (e.g. xrn1KO Supplementary Figure X). We find that
the main 5’-3’ exonuclease xrn1KO and deadenylase complex (ccr4KO and
pop2KO) are impaired in GAP1 repression as expected for mutants in
this key mRNA degradation pathway (Supplementary Figure X, Figure
5b). We confirmed that the accelerated degradation of GAP1 mRNA is
impaired in a lsm1KO and lsm6KO (Figure 5c), but that this effect is
very slight (~30\% decrease in specific degradation rate, t-test p-val
< 0.03) which we address further in the discussion. We also tested
scd6KO and edc3KO, two modifiers of the decapping or processing-body
assembly functions associated with this complex, and found two
distinct phenotypes.  edc3KO has similar expression before the shift,
but is cleared much more slowly. scd6KO has a greatly reduced GAP1 BFF
signal and GAP1/HTA1 signal before the shift, failing to induce GAP1
mRNA to wild-type levels, but is still consistently slower in
repression of this residual amount. We also tested a tif4632KO, a
mutant in the eIF4G complex that is known to interact with Scd6p, and
found a similar phenotype. The phenotype from a deletion of an
initiation factor subunit suggested that perhaps the phenotype may be
specified in translation rate control on the mRNA, so we deleted
~100-150bp downstream of the stop codon (approximate 3’ UTR) or
upstream of the start codon (approximate 5’ UTR), and found that while
the 3’ UTR deletion did not have an effect the two 5’ UTR deletions
exhibited the same phenotype of reduced GAP1 mRNA before the upshift
and reduced rate of transcript clearance after (Figure 5d). These
measurements point towards altered mRNP composition of the
Lsm1-7p/Pat1p complex and associated decapping factors are associated
with defects in GAP1 mRNA repression upon a nitrogen upshift, and
importantly that the phenotype of the scd6KO, tif4632KO, or 5’ UTR
deletions preceed the addition of glutamine and suggest that these may
alter the priming of GAP1 to be susceptible to a destabilization
event.  As these factors are associated with processing-bodies, we
tested if microscopically-observable p-body dynamics may co-occur with
destabilization using Dcp2-GFP. We do not observe qualitative changes
in Dcp2-GFP distribution (data not shown, raw images in supplementary)
indicating that any differences in RNA and RNP interactions do not
result in a microscopically visible phenotype of processing-body foci,
as seen in other stresses. This is consistent with previous
investigations of amino-acid limitation stress (Hoyle et al. 2007) and
suggests that the defects in GAP1 mRNA clearance in the Lsm1-7p/Pat1p
complex and associated factors likely result from roles in decapping
or modulating translation of GAP1 mRNA.  Discussion The control of
mRNA stability allows cells to effect rapid transcriptome
reprogramming, especially in the clearance of un-necessary
transcripts. We refined our methods to facilitate a high-throughput
examination of the post-transcriptional regulation of the yeast
transcriptome in a growth up-shift environment. 4-thiouracil metabolic
labeling has been used many times to track mRNA stability in budding
yeast (Miller et al. 2011; Neymotin et al. 2014; Munchel et al. 2011)
and a pulse-chase design has been used to track stability of mRNA
during dynamic conditions (Braun et al. 2015; Munchel et al. 2011),
but we believe our modifications here will allow for improved
quantification of extant transcripts and interpretation of
measurements through explicit modeling of labelling dynamics to
account for some of the limitations (Pérez-Ortín et al. 2013). We
refer to this as a label-chase design because we are not tracking a
cohort of mRNA with a pulse-chase, but rather the whole steady-state
transcriptome. Additionally, continuing development of the label
purification biochemistry and incorporation of explicit per-transcript
efficiency terms will improve these measurements further (Chan et al.
2017).  We found that the yeast transcriptome is on average less
stable during nitrogen limitation. Given that this is the case in
comparison to label-incorporation and label-chase experimental
designs, we do not believe this is an artifact of the particular
label-chase design here.  This is different than what has been
observed upon entry into starvation of carbon-sources, so this may
suggest that the different nutrient limitation or degree of limitation
may play a role. Using these transcript stability estimates, we
compare them to transcriptome abundance dynamics during a nitrogen
upshift and find that although they have the expected relationship
more often than not ( anticorrelation, $R^2$=-0.376 ), we again see
that increases in degradation do not always result in rapid
repression. This contradiction has been observed for transcripts
up-regulated in stress conditions, and has been proposed as a
mechanism to effect a rapid rebalancing of the transcriptome after a
transient phase of reprogramming (Shalem et al. 2008). Thus it appears
that a similar phenomena occurs during the relief of stress. The
significant overlap of the genes destabilized during this upshift with
members of the ESR “up” regulon (Gasch et al. 2000) suggests that mRNA
stability is one mechanism that rapidly regulate these transcripts
during either improvement or worsening of environmental conditions.
The NCR regulon is rapidly repressed upon this nitrogen upshift, and
our measurements indicate that post-transcriptional regulation of
sixteen NCR transcripts accelerates this repression. However, this is
not limited to simple NCR but also targets transcripts functionally
enriched in carbon metabolism pathways, particularly pyruvate
metabolism. Our BFF assay found that mutants in negative regulation of
gluconeogenesis pathways were enriched in high GAP1 after the shift,
suggesting that disruptions to central carbon metabolism may have more
of a role in priming the metabolic state of the cell for rapid
clearance of GAP1.  Naively we had expected that the destabilizing
effect was mediated by an RBP binding the 3’ UTR of these
de-stabilized transcripts. Cis-element analysis eliminated known RBPs
from explaining the destabilization, and in particular demonstrated
that  Puf3p motifs de-enriched from the  destabilized set. These
destabilized transcripts were surprisingly enriched for a binding
motif of Hrp1p in the 5’ UTR. This essential component of mRNA
cleavage for polyadenylation in the nucleus  has been shown to shuttle
to the cytoplasm and bind to amino-acid metabolism mRNAs (Kim Guisbert
2005) and been shown to interact genetically to mediate
nonsense-mediated decay (NMD) of a PGK1 mRNA harboring an premature
stop-codonin the CDS (González et al. 2000) or a cis-element spanning
the 5’ UTR and first 92 coding bp of PPR1 mRNA (Kebaara 2003).
Elements in the 5’ UTR have also been demonstrated to destabilize GAL1
mRNA (Baumgartner et al. 2011) and SDH2 mRNA upon glucose addition,
perhaps due to the competition between translation initiation and
decapping mechanisms (de la Cruz et al. 2002). Interestingly, both
GAP1 and SDH2 share the feature of a second start codon downstream of
the canonical start (Neymotin et al. 2016). This, in light of recent
analyses further highlighting the contribution of translation dynamics
to mRNA stability (Cheng et al. 2017), suggests to us that the
mechanism of degradation may be modulated through dynamic changes in
translation initiation or elongation that trigger decapping of GAP1
and other mRNA. Additionally, promoter-mediated mRNA stability has
been demonstrated several times before in yeast, and in the
destabilization of mRNA upon a glucose upshift (Braun et al. 2015), so
these factors may be interacting with these sequences in the nucleus
to mark the mRNA for destabilization.Previous work from our group
mutated the start codon of GAP1 and found that this mutation reduced
steady-state mRNA abundances (Neymotin et al. 2016), so perhaps the
reduction of GAP1 mRNA in limiting conditions may reflect an altered
ribosome loading status. Future work interrogating this possible
interaction of translational activity with mRNA stability during
dynamic conditions could inform our understanding of the relationship
between the two in steady-state conditions.  To find the factors
driving this, we developed a trans-factor screen using mRNA FISH,
FACS, and sequencing. We believe this is the first time direct mRNA
abundance has been estimated using a SortSeq approach, although
sorting on indirect markers or using mRNA to enrich as subpopulation
has been demonstrated before (Klemm et al. 2014; Hanley et al. 2013).
This approach could be used with other barcoding mutagenesis
technologies, like transposon-sequencing or Cas9 mediated
perturbations, to systematically probe for the genetic basis of
transcript dynamic phenotypes. Additionally, the use of branched-DNA
mRNA FISH using the Affymetrix Quantigene technology (now Invitrogen
PrimeFlow RNA) or other methods (Rouhanifard et al. 2017) would allow
for mRNA estimation without requiring genetic manipulation, which
makes it suitable for use on natural variants in an extreme QTL
mapping application. While the cell wall of yeast makes optimization
crucial to this assay, future development of hybridization protocols
may improve accuracy and make the assay more robust (Richter et al.
2017; Wadsworth et al. 2017). The molecular methods designed should
permit accurate quantification of yeast deletion collection libraries
post-FACS sort with fixed samples, expanding the possibilities of
markers to fixed-cell flow cytometry assays.  We identify that mutants
in the Lsm1-7p/Pat1p complex have elevated GAP1 mRNA before and after
the shift, and have a defect in repressing GAP1 relative to HTA1 mRNA
upon the glutamine upshift. By qPCR normalized to a housekeeping gene
we confirm the expectation that this is not just a phenotype specific
to NCR mRNA, but that there is no steady-state phenotypic difference
in the relative quantification. There is a slight decrease in
repression rate. Our initial estimate using the BFF methodology may be
confounded, as technical reasons steered us towards using a quantile
binning strategy instead of using more bin resolution in the tails of
the distribution. While this gave us good design for estimating
deviance from the wild-type expression, this hindered our ability to
outright measure dynamics unless the phenotype was similar to or
crossed the wild-type value in one of the conditions (ex: scd6KO or
edc3KO). Given that the GAP1 mRNA is destabilized during this
transition we suspect that these core mRNA degradation factors are
directly involved.  Because factors associated with the Lsm1-7p/Pat1p
complex are also involved in processing-body formation we looked for
processing-body dynamics during the nitrogen upshift, but did not see
qualitative changes in the granularity of Dcp2-GFP signal. However, it
has been proposed that pre-existing mRNPs seed the formation of
processing-bodies, thus the phenotype may not be microscopic and would
require molecular assays to eliminate this possibility. Interestingly,
during cross-comparisons with recent datasets exploring mRNA
localization to RNP condensates we found that the set of destabilized
transcripts are on average longer in CDS and have an increased
codon-optimality, two factors that have been recently explored in
connection with stress-granule localization in yeast (Khong et al.
2017). Perhaps this phenotype is manifest at the molecular method, but
this possibility would require different molecular techniques to test
and exclude.  This study explores another example of mRNA
destabilization control in budding yeast. Future work contrasting this
process with the similar process in glucose upshifts may share
similarities or differences that would greatly inform our
understanding of mRNA stability specification. We also develop and
demonstrate a method to estimate mRNA abundance for every knock-out
mutant, in a high-throughput pooled approach. The methods here allow
for the use of fixed-cell flow-cytometry assays in pooled Sort-seq
assays on yeast, and would be useful to inform the development of
similar assays in other systems. Development of this approach to
estimating mRNA abundance on pooled mutants would enable the
combination of transcriptomics as a high-dimensional marker of
cellular signalling pathways with the use of transcript markers to
explore the genetics of these pathways.  Supplementary issues
Pulse-chase modeling ( I basically want to reprint the stuff in the
supplement here ) BFF rationale, methodology, and future directions (
I basically want to reprint the stuff in the supplement here, then
waste paper speculating )

