\chapter{Measuring the extent of mRNA destabilization and looking 
for genetic factors of \textit{GAP1} repression}

BFF identifies mRNA decapping factors
important for GAP1 mRNA dynamics upon a nitrogen upshift

This chapter is currently submitted to a journal, seeking publication
as an article. It is also posted on \textit{biorxiv}, titled:
\textit{"Global analysis of gene expression dynamics identifies factors
required for accelerated mRNA degradation"}.
Authorship of this article is: Darach Miller, Nathan Brandt, 
and David Gresham.

The \textit{biorxiv} version is at \url{doi.org/10.1101/254920}).
The below is adapted for this format, although arranged such
that supplementary figures are associated with each primary figure.
Supplemental tables will be available when published, are
obtainable from the OSF repository linked with this work
(\url{https://osf.io/7ybsh/files/}), and is re-producible
using the \texttt{Makefile} in the git repository distributed
with the paper
(\url{http://github.com/darachm/millerBrandtGresham2018}).

%The methods here allow
%for the use of fixed-cell flow-cytometry assays in pooled Sort-seq
%assays on yeast, and would be useful to inform the development of
%similar assays in other systems. Development of this approach to
%estimating mRNA abundance on pooled mutants would enable the
%combination of transcriptomics as a high-dimensional marker of
%cellular signalling pathways with the use of transcript markers to
%explore the genetics of these pathways.  Supplementary issues
%Pulse-chase modeling ( I basically want to reprint the stuff in the
%supplement here ) BFF rationale, methodology, and future directions (
%I basically want to reprint the stuff in the supplement here, then
%waste paper speculating )

\section{Abstract}

Cellular responses to changing environments frequently
involve rapid reprogramming of the transcriptome.
Regulated changes in mRNA degradation rates can
accelerate reprogramming by clearing or stabilizing extant transcripts. 
%Budding yeast respond to an improvement in
%nitrogen-availability by triggering a transcriptional reprogramming
%that functions to upregulate ribosome biogenesis and repress
%alternative nitrogen-source catabolism. 
Here, we measured mRNA stability using 4-thiouracil labeling
in the budding yeast \textit{Saccharomyces cerevisiae}
during a nitrogen upshift and found that 78 mRNAs are subject
to destabilization. These transcripts include Nitrogen
Catabolite Repression (NCR) and carbon metabolism mRNAs,
suggesting that mRNA destabilization is a mechanism 
for targeted reprogramming.
To explore the molecular basis of
destabilization we implemented a SortSeq approach to
screen using the pooled deletion collection library
for \textit{trans} factors that mediate rapid \textit{GAP1}
mRNA repression.
We combined low-input multiplexed \underline{B}arcode sequencing 
with branched-DNA single-molecule mRNA \underline{F}ISH and 
\underline{F}luorescence-activated cell sorting (\underline{BFF})
to identify that the Lsm1-7p/Pat1p complex and general mRNA
decay machinery are important for \textit{GAP1} mRNA clearance.
We also find that the decapping modulator \textit{SCD6}, translation
factor eIF4G2, and the 5' UTR of \textit{GAP1}
are important for this repression, 
suggesting that translational control may impact the 
post-transcriptional fate of mRNAs in response to 
environmental changes.

\section{Introduction}

Regulated changes in mRNA abundance are a primary cellular response
to external stimuli.
Both the rate of synthesis and the rate of degradation determine the
steady-state abundance of a particular mRNA and the kinetics
with which abundance changes occur
\citep{Hargrove1989,Perez-Ortin2013}. 
Changes in mRNA degradation rates fulfill an important 
mechanistic role in diverse systems, including 
development \citep{Alonso2012,West2017} and disease
\citep{Aghib1990}.
In budding yeast, the rate of
mRNA degradation is affected by environmental stresses
\citep{Canadell2015}, cellular growth rate
\citep{Garcia-Martinez2016}, as well as by improvements in 
nutrient conditions \citep{Scheffler1998}.

Environmental shifts trigger rapid reprogramming of the budding yeast
transcriptome in response to stresses and nutritional
changes \citep{Gasch2000,Conway2012}. mRNA degradation rate changes
have been shown to play a role in responses to heat-shock, osmotic
stress, pH increases, and oxidative stress
\citep{Castells2011,Romero2009,Canadell2015,Molina2008}. 
In response to these
diverse stresses destabilization of mRNAs encoding 
ribosomal-biogenesis gene products, and  
stress-induced mRNA occurs \citep{Canadell2015}. 
Simultaneous increases in both synthesis and
degradation rates of some  mRNAs may serve to speed the return to a
steady-state following a transient pulse of regulation
\citep{Shalem2008}. Addition of glucose to carbon-limited cells 
results in both stabilization of 
ribosomal protein mRNAs \citep{Yin2003} and destabilization
of gluconeogenic transcripts \citep{De_la_Cruz2002,Mercado1994}.
Destabilization of transcripts can
have a delayed effect on reducing protein levels compared to
up-regulated genes \citep{Lee2011}. This suggests that accelerated
mRNA degradation may serve additional purposes. For example, clearance
of specific mRNAs could increase nucleotide pools
\citep{Kresnowati2006} or facilitate reallocation of
translational capacity 
\citep{Kief1981,Giordano2016,Shachrai2010}. 
%Identifying the genetic
%factors responsible for the accelerated mRNA degradation would allow
%us to test if regulated destabilization of specific transcripts is
%adaptive.

Yeast cells metabolize a wide variety of nitrogen sources, but
preferentially assimilate and metabolize specific nitrogen compounds.
Transcriptional regulation, known as
“nitrogen catabolite repression” (NCR) \citep{Magasanik2002},
controls the expression of mRNAs
encoding transporters, metabolic enzymes, and regulatory
factors required for utilization of alternative nitrogen sources. 
NCR-regulated transcripts are expressed in the
absence of a readily metabolized (preferred) nitrogen sources or in
the presence of growth-limiting concentrations (in the low $\mu$M range)
of any nitrogen source \citep{Godard2007,Airoldi2016}. Regulation
of NCR targets is mediated by two activating GATA
transcription factors, Gln3p and Gat1p, and two repressing
GATA factors, Dal80p and Gzf3p. \textit{GAT1}, \textit{GZF3}, and
\textit{DAL80} promoters
contain GATAA motifs, and thus transcriptional regulation of NCR
targets entails self-regulatory and cross-regulatory loops. When
supplied with a preferred nitrogen source such as glutamine, the
NCR-activating transcription factors Gat1p and Gln3p are excluded from
the nucleus by TORC1-dependent and -independent mechanisms
\citep{Beck1999,Tate2013,Tate2017} and NCR transcripts are strongly
repressed. The activity of some NCR gene
products is also controlled by post-translational mechanisms
\citep{Cooper1983} such as the General Amino-acid Permease
(Gap1p) which is rapidly inactivated upon a nitrogen 
upshift via ubiquitination
\citep{Stanbrough1995,Risinger2006,Merhi2012}. Recently, we have
identified an additional level of regulation of NCR transcripts: cells
growing in NCR de-repressing conditions accelerate the degradation
of \textit{GAP1} %and \textit{DIP5} mRNAs
mRNA upon addition of glutamine
\citep{Airoldi2016}. Thus, mRNA degradation rate regulation may be an
additional mechanism for clearing NCR-regulated transcripts upon 
improvements in environmental nitrogen availability.

Multiple pathways mediate the degradation of mRNAs. The main pathway
of mRNA degradation occurs by deadenylation and decapping
prior to 5’ to 3’ exonucleolytic degradation by Xrn1p; however,
transcripts are also degraded 3’ to 5’ via the exosome, or via
activation of co-translational quality control mechanisms
\citep{Parker2012}. Deadenylation of mRNAs by the Ccr4-Not complex
allows the mRNA to be bound at the 3' end by the 
Lsm1-7p/Pat1p complex, a heptameric
ring comprising the SM-like proteins Lsm2-7p and the
cytoplasmic-specific Lsm1p \citep{Tharun2000,Sharif2013}, which then
recruits factors for decapping by Dcp2p. 
Recruitment of the decapping enzyme \citep{Coller2004} is the 
rate-limiting step for canonical 5'-3' degradation.
Therefore Lsm1-7p, Pat1p,
and associated factors play a key role \citep{Nissan2010}. 

Regulation of mRNA degradation pathways can alter the stability of
specific mRNAs. For example, the RNA-binding protein (RBP) Puf3p
recognizes a \textit{cis}-element in 3’ UTRs \citep{Olivas2000} 
and affects mRNA degradation rates depending on
Puf3p phosphorylation status \citep{lee2015}. 
%Transcript properties
%also associated with translation dynamics affect mRNA degradation, at
%the level of elongation \citep{Sweet2012,Presnyak2015,Neymotin2016} or
%competition between the decapping enzymes and translation initiation
%\citep{Schwartz2000}. 
In addition to \textit{cis}-elements within the transcirpt, 
promoters have
been shown to mark certain RNA-protein (RNP) complexes to specify
their post-transcriptional regulation
\citep{Mercado1994,Haimovich2013,Trcek2011,Braun2015}. These
mechanisms may be controlled by a variety of different signalling
pathways including Snf1 \citep{Young2012,Braun2014}, PKA
\citep{Ramachandran2011}, Phk1/2 \citep{Luo2011}, and TORC1
\citep{Talarek2010}. Thus, regulated changes in  mRNA degradation
rates entails numerous mechanisms that collectively tune stability of
mRNAs in response to the activity of signalling pathways. 

Here, we studied the global regulation of mRNA degradation rates upon
improvment in environmental nitrogen using 4-thiouracil (4tU) 
label-chase and RNAseq.
We found that a set of 78 mRNAs are subject to accelerated mRNA
degradation, including many NCR transcripts as well as mRNAs
encoding components of
carbon metabolism. To identify the mechanism underlying accelerated
mRNA degradation we designed a high-throughput genetic screen using 
\underline{B}arcode-sequencing of a pooled library which was
fractionated using \underline{F}luorescence-activated cell 
sorting of single molecule mRNA \underline{F}ISH signal (BFF). 
We screened the barcoded
yeast deletion collection to test the effect of each gene deletion
on the abundance of \textit{GAP1} mRNA in NCR de-repressing 
conditions and its clearance following the 
addition of glutamine. We
find that the Lsm1-7p/Pat1p complex and decapping modifiers affect
both \textit{GAP1} mRNA steady-state expression and its 
accelerated degradation.
This work expands our
understanding of mRNA stability regulation in remodeling the
transcriptome during a relief from growth-limitation and demonstrates
a generalizable approach to the study of genetic determinants of mRNA
dynamics.

\section{Results}

\subsection{Transcriptional reprogramming precedes physiological remodeling}

Cellular responses to environmental signals entail coordinated changes
in both gene expression and cellular physiology.  Previously, we
studied the steady-state and dynamic responses of 
\textit{Saccharomyces cerevisiae} 
(budding yeast) to environmental nitrogen
\citep{Airoldi2016}, and found that the transcriptome is rapidly
reprogrammed following a single pulsed addition of glutamine to
nitrogen-limited cells in either a chemostat or
batch culture. To study physiological changes in response to a
nitrogen upshift, we measured growth rates of a population of 
cells. A prototrophic haploid lab strain 
(FY4, isogenic to S288c) grows with a
4.5 hour doubling time in batch culture in minimal media 
containing proline as a sole
nitrogen source (\FIG{figure1}a). Upon addition of 400$\mu$M glutamine
the cells undergo a 2-hour lag period during which no change in
population growth rate is detected, but the average cell size
continuously increases ($\sim$21\% increase in mean volume 
\FIG{figure1}b). Following the lag, the population adopts a 2.1 
hour doubling time.
%This lag in population growth rate upon an upshift has been 
%described before \citep{Carter1978}.
By contrast, global gene expression changes are detected
within three minutes of the upshift \citep{Airoldi2016}. 
Thus, transcriptome remodeling precedes
physiological remodeling in response to a nitrogen upshift.

\begin{figure}[h!]
\includegraphics[width=\textwidth]{output/Figure1.png}
\caption{
  \textbf{Dynamics of physiological and transcriptome remodeling
during a nitrogen upshift.}
  \textbf{a)} 
  400$\mu$M glutamine was added to a
  culture of yeast cells growing in minimal media containing 800$\mu$M
  proline as a sole nitrogen source. Measurements
  of culture density across the upshift are plotted. 
  Dotted lines denote linear regression of the
  natural log of cell density against time before the upshift and 
  after the 2 hour lag. \textbf{b)} Average cell size.
  Dotted lines denote the mean cell diameter before the upshift
  and after the 2 hour lag. 
  \textbf{c)} PCA analysis of global
  mRNA expression in steady-state chemostats and following an upshift
  \citep{Airoldi2016}. Steady-state nitrogen-limited chemostat
  cultures maintained at different growth rates (colored circles)
  primarily vary along principal component 2. Expression following a
  nitrogen-upshift in either a chemostat (squares) or batch culture
  (triangles) show similar trajectories and primarily vary along
  principal component 1. Grey lines depict the major trajectory
  of variation for the steady-state and upshift experiments.}
\label{fig:figure1}
%\label{figsupp:f1s1}
%  \figsupp{The increase in average cell size upon nitrogen addition is robust to choice of summary statistic. The same data presented in Figure 1 a) and b) was analyzed for different summaries of the average cell size after addition of glutamine to cells limited for growth on proline.}
%    {\includegraphics[width=\textwidth]{output/Figure1_S_coulterCounterUpshiftOtherStats.png}}
}
\end{figure}
\begin{figure}[h!]
\caption{
\figsupp[Long-term transcriptome dynamics upon nitrogen upshift.]{
The coarse long-term transcriptome dynamics of a glutamine upshift. 
Principal components analysis (SVD) of microarray data from 
\cite{Airoldi2016}. 
Colored points are from steady-state chemostats grown in
limitation for various nitrogen sources, at different growth rates.
Time-series experiments are show in grey points, connected by lines,
and line-type is the type of upshift (in batch or in chemostat).
}{\includegraphics[width=\textwidth]{output/Figure1_S_longTermPCA.png}}
\label{figsupp:longTermPCA}
}
\end{figure}
\begin{figure}[h!]
\caption{
\figsupp{Gene set enrichment analysis of loadings on principal components one and two.}
  {Figure1\_Table\_GSEofGOtermsAgainstPCcorrelation.csv}
\label{figsupp:microarrayPCAgsea}. 
\end{figure}

To evaluate concordance in transcriptome remodeling between chemostat
and batch nitrogen upshifts, and the extent to which they reflect
changes in gene expression observed during systematic steady-state 
changes in growth rates using chemostats, we
performed principal component analysis of global gene expression
(\FIG{figure1}c). The first two principal components, which
account for almost half of the total variation, clearly separate
steady-state and nitrogen upshift cultures.  Systematic changes in
growth rate primarily results in
separation of gene expression states along the second principal
component, whereas upshift experiments vary along the first 
principal component.  This suggests that
although a nitrogen upshift results in a gene expression state 
reflecting increased cell growth rates \citep{Airoldi2016}, the
transcriptome is remodeled through a distinct state. 
In upshift experiments in
chemostats, the gene expression trajectory ultimately returns to 
the initial steady-state condition as excess nitrogen is 
depleted by consumption and dilution 
(\FIGSUPP[figure1]{longTermPCA}). 

To investigate the functional basis of gene expression programs
in the upshift and steady-state conditions, we computed the
correlation of each transcript with the loadings on these first two
principal components and performed gene-set enrichment analysis
(\FIGSUPP[figure1]{microarrayPCAgsea}). 
Component 1 is positively correlated with functions like
mRNA processing, transcription from RNA polymerases (I,II,and III),
and chromatin organization, and negatively correlated with
cytoskeleton organization,
vesicle organization, membrane fusion, and cellular respiration.
Both steady-state and upshift gene expression
trajectories increase with principal component 2, but they diverge
along principal component 1. Components 1 and 2 are 
strongly enriched for terms including ribosome biogenesis, 
nucleolus, and
tRNA processing, and negatively correlated with
vacuole, cell cortex, and carbohydrate metabolism terms. 
Together, this analysis suggests that both upshift and
increased steady-state growth rates share upregulation of
ribosome-associated components, but the reprogramming
preceeding the upshift in growth reflects an immediate focus on 
gene expression machinery instead of structural cellular components.
Importantly,
dynamic reprogramming is similar in both the chemostat and batch
upshift (\FIG{figure1}c). As batch cultures are a technically
simpler experimental system, we performed all subsequent experiments
using batch culture nitrogen upshifts. 

\subsection{Global analysis of mRNA stability changes during the
nitrogen upshift}

Previously, we found that \textit{GAP1} and \textit{DIP5} mRNAs 
are destabilized in
response to a nitrogen upshift \citep{Airoldi2016}. We sought to
determine if mRNA destabilization is specific to NCR transporter
mRNAs by measuring global mRNA stability across the upshift
using 4-thiouracil (4tU) labeling and RNA-seq 
\citep{Neymotin2014,Munchel2011}.
As 4tU labeling requires nucleotide transport, which may be altered
upon a nitrogen-upshift \citep{Hein1995}, we designed experiments such
that following complete 4tU labeling and metabolism to nucleotides 
the chase was initiated prior to addition of glutamine or water (mock).
We normalized data using \textit{in vitro} synthesized thiolated 
spike-ins by fitting a log-linear model to spike-in counts
across time (\FIGSUPP[figure2]{writeup2}), which reduced noise and increased
our power to detect stability changes (
\FIGSUPP[figure2]{dme211raw},
\FIGSUPP[figure2]{dme211filterDirect},
\FIGSUPP[figure2]{dme211filterModel}).
Data and models for each transcript can be visualized in browser
using a Shiny appplication (see
\url{http://shiny.bio.nyu.edu/users/dhm267/} or \autoref{app:shiny} ). 

We modeled the
log-transformed normalized signal for each mRNA using linear
regression (\FIGSUPP[figure2]{dme211resultsModel}).
Of 4,859 mRNAs measured we identified 94 that increased in 
degradation rate and 38 that decreased (FDR < 0.01, using \cite{Storey2003}). 
We generated a model of nucleotide
labeling kinetics to assess the effect of an incomplete label 
chase on our experimental design ( \FIGSUPP[figure2]{writeup2} ),
 and found that complete transcriptional inhibition alone could 
theoretically result in a 17\% increase in the apparent 
degradation rate. In order to eliminate the possibility that
rapid synthesis changes could affect our estimates,
we only considered destabilization of at least a
doubling (100\% increase) of apparent degradation rates between 
pre-upshift and post-upshift.
This conservative cutoff 
left 78 mRNA that are significantly destabilized 
upon a nitrogen upshift. 


The vast majority of transcripts (4,781 of 4,859) do not show
individual evidence for stability changes upon addition of glutamine
(e.g. \textit{HTA1}, \FIG{figure2}a). 
The median pre-upshift half-life is 6.89 minutes and the median
half-life following the upshift is 6.4 minutes (\autoref{tab:table1})
suggesting that there is not a global change in mRNA stability.
Global stability estimates are
considerably lower than previous estimates in rich medium
\citep{Munchel2011,Neymotin2014,Miller2011}, which may reflect the
different nutrient conditions used in our study. 
The 78 transcripts significantly destabilized upon the 
glutamine-upshift include
mRNAs encoding NCR transporters \textit{GAP1}, \textit{DAL5}, and
\textit{MEP2} (blue label, \FIG{figure2}a), the pyruvate metabolism enzymes
\textit{PYK2} and \textit{PYC1} (orange label), and trehalose synthase
subunits \textit{TPS1} and
\textit{TPS2} (yellow label).
Destabilized mRNA tend to be more stable before the upshift (\FIG{figure2}b),
(median half-life of 9.46 minutes) and exhibit 
a median 3.06-fold increase in degradation rates (median half-life of
3.02 minutes following the upshift). 
%Plots of data and models for each gene are available in the Shiny application
%(\autoref{app:shiny}).

\input{../output/Table1.tex}

%\begin{table}[h!]
%\caption{\label{tab:table1} Summary of mRNA stability, median values}
%% Use "S" column identifier to align on decimal point 
%\begin{tabular}{p{8em} | l l l l | l l}
%\toprule
%& \multicolumn{2}{c}{Pre-upshift} & \multicolumn{2}{c}{Post-upshift} &
%Change in & Fold-change\\
% & specific rate & half-life & specific rate & half-life & specific
%rate & specific rate\\
% & (min$^{-1}$) & (min) & (min$^{-1}$) & (min) & (min$^{-1}$) & \\
%\midrule
%\raggedright All transcripts & 0.100 & 6.92 & 0.110 & 6.32 & 0.00865 & 1.08 \\
%\raggedright Destabilized (n=78) & 0.0732 & 9.46 & 0.229 & 3.02 & 0.158 & 3.06 \\
%\raggedright Unchanged (n=4781) & 0.101 & 6.89 & 0.108 & 6.40 & 0.00728 & 1.07 \\
%\bottomrule
%\end{tabular}
%\end{table}

We tested for
functional enrichment among the set of 78 destabilized
mRNAs and found that they are strongly enriched for NCR
transcripts (16 of 78, p < $10^{-11}$). Almost half of the
destabilized transcripts are annotated as “ESR-up” genes
(\FIGSUPP[figure2]{comparisonESR}), on the basis of  their rapid induction
during the environmental stress response \citep{Gasch2000}. These 78
destabilized mRNA are enriched (FDR < 0.05) for GO terms and KEGG 
pathways (\FIGSUPP[figure2]{dme211goAndKegg}) including
glycolysis/gluconeogenesis (6 genes), 
carbohydrate metabolic process (24),
trehalose-phosphatase activity (3), 
pyruvate metabolic process (6), 
and secondary active transmembrane transport
(8, a subset of the enriched 11 ion transmembrane transport genes).
%We also see destabilization of \textit{PYK2} and \textit{HXK1},
%both of which are isozymes expressed highly in poor nutrient conditions.
Thus destabilized mRNA upon a nitrogen upshift regulates, 
but is not restricted to, NCR-regulated mRNA and reflects broader
metabolic changes in the cell. 


\begin{figure}[H]
\includegraphics[width=\textwidth]{output/Figure2.png}
\caption{
  \textbf{Global mRNA stability change following a nitrogen upshift.}
  \textbf{a)} 4tU-labeled mRNA from each gene was measured over time, before and
  after the addition (vertical dotted line) of glutamine 
  (nitrogen-upshift) or water (mock). Linear regression models were 
  fit to the data with a rate before the upshift (solid line) 
  and a rate after glutamine addition (dashed line). 
  \textit{HTA1} is not significantly destabilized, 
  whereas mRNAs encoding NCR-regulated transporters or 
  pyruvate and trehalose metabolism enzymes are significantly destabilized. 
  \textbf{b)} Comparison between the pre-upshift mRNA
  degradation rate (y-axis) and the post-upshift mRNA degradation rate
  (x-axis). Positive values result from noise on the slope estimate.
  \textbf{c)} Comparison between changes in mRNA expression following
  upshift (Airoldi et al. 2016) (y-axis) and the post-upshift
  degradation rate (x-axis). Both plots share the same x-axis.
  Transcripts that are significantly destabilized are colored red, and
  shown with red rug-marks in the marginal histograms.}
\label{fig:figure2}
}
\end{figure}
\begin{figure}[h!]
\caption{
  \figsupp[Enrichment of Hrp1p motif in 5' UTRs of destabilized
transcripts.]{
Sequences of destabilized and unaltered mRNAs were analyzed for
RBP binding motif enrichment
using the AME program in MEME, then significant hits were confirmed by using a
logistic model predicting destabilization based on motif content per sequence
length. Hrp1p is significantly ( p<0.0001 ) enriched in
the 5' UTRs of destabilized transcripts. For this plot, motif matches were
counted using the GRanges package for the 5' UTRs, 3' UTRs,
and coding sequence of transcripts using the largest isoforms detected in
\cite{Pelechano2014}.}{\includegraphics[width=\textwidth]
    {output/Figure2_S_averageMotifsPerSection.png}}
\label{figsupp:hrp1}
}
\end{figure}
\begin{figure}[h!]
\caption{
  \figsupp[Comparison between rates of mRNA abundance change
\citep{Airoldi2016} and stability measured in this study.]{
Comparisons of rates from this study with mRNA abundance change rates
from \cite{Airoldi2016}. Pre-upshift decay rates (top) don't explain the
abundance change. Decay rate refers to the rate of change, thus is
the negative of the degradation rate.
The degradation rate changes (difference between pre
and post upshift) and the post-upshift rates (bottom) are anti-correlated
with the abundance changes.}{\includegraphics[width=\textwidth]
    {output/Figure2_S_globalComparisons.png}}
\label{figsupp:kkdComparison}
}
\end{figure}
\begin{figure}[h!]
\caption{
  \figsupp[Many of the destabilized mRNA are members of the ESR-up
regulon \citep{Gasch2000}.]{
Comparisons of degradation rates from this study with mRNA abundance change rates
from \cite{Airoldi2016}. Destabilized transcripts are colored based on
their membership in the ESR gene set, as described in the supplement 
of \cite{Brauer2008}. 
Many of the destabilized set are ESR "up" genes, as they
are increase in expression in response to stresses.}{\includegraphics[width=\textwidth]
    {output/Figure2_S_comparisonToESR.png}}
\label{figsupp:comparisonESR}
}
\end{figure}
\begin{figure}[h!]
\caption{
%  \figsupp[short cap]{Histograms of all fit rates, un-filtered for significance or effect size. Top
%histogram shows the 'solid-line' rate, the pre-upshift rate. The middle is the
%'dotted-line' rate, the post-upshift rate. The bottom is the actual change in
%rates between the two, so a larger change is a destabilization, and a negative
%change is an apparent stabilization. Color of the rugmark (red) denotes
%singificantly destabilized transcripts.}{\includegraphics[width=\textwidth]
%    {output/Figure2_S_histogram_raw.png}}
%\label{figsupp:f2s4}
  \figsupp[Abundance changes versus stability, plotted for only the destabilized transcripts.]{
Scatter plot of significantly destabilized transcripts. For each, the x-axis is
the fit rate of degradation rate post-upshift. On the y-axis is the mRNA abundance
(expression) change rate \citep{Airoldi2016} after the upshift.
These values were modeled to normalized
sequencing signal (x-axis) and normalized microarray ratio (y-axis). The dashed
line is a 1:1 line of equality. }{\includegraphics[width=\textwidth]
    {output/Figure2_S_justDestabilizedDecayvsDynamics.png}}
\label{figsupp:comparisonDestabilized}
}
\end{figure}
\begin{figure}[h!]
\caption{
  \figsupp[Six examples of individual mRNA whose regulation is more
complex than a homo-directional destabilization and synthesis
repression.]{
For several examples of the slowest decreasing (in the microarray fits)
transcripts, we plot the microarray (abundance) and sequencing (decaying labeled
abundance) data normalized to be on the same relative y-axis scale (subtracted
t\_0 y-intercepts of fits).
%Black is the microarrays, so abundance. Blue is the
%water-upshifted samples. Red is the glutamine-upshifted samples. This shows how
%transcripts can be destabilized (blue slope changing to red slope) without being
%greatly repressed (black line is not of same slope as red line), showing that
%synthesis and degradation can contradict each other for some transcripts in this system.
Destabilization does not necessarily result in a rapid clearance
of the mRNA.
}{\includegraphics[width=\textwidth]{output/Figure2_S_sixExamplesOfDestabilizationWithoutRepression.png}}
\label{figsupp:compareSix}
}
\end{figure}
\begin{figure}[h!]
\caption{
  \figsupp[The destabilized set is longer and has a higher frequency
of optimal codons than the rest of the transcriptome.]{
Comparisons of destabilized mRNAs with the rest of the transcriptome.
\textbf{a)} 
Destabilized transcripts tend to have longer CDS lengths ( p-value < 2e-5
by Wilcoxon rank sum test ). 
\textbf{b)} 
On average, the destabilized transcripts have more optimal codons
than the rest of the transcriptome ( p-value < 2e-8 Wilcoxon rank
sum test).
The fraction of optimal codons per feature
was obtained from the supplement of \cite{Khong2017} using definitions
from \cite{Presnyak2015}. 
}{\includegraphics[width=\textwidth]
    {output/Figure2_S_lengthAndCodons.png}}
\label{figsupp:lengthAndCodons}
}
\end{figure}
\begin{figure}[h!]
\caption{
  \figsupp{Supplementary file with experimental rationale, details,
and protocol for the label-chase experiment.}{Figure2\_Supplementary\_Writeup.pdf}
\label{figsupp:writeup2}
}
\end{figure}
\begin{figure}[h!]
\caption{
  \figsupp{Raw counts of labeled mRNA quantified by RNAseq in label-chase experiment.}
    {output/Figure2\_Table\_RawCountsTableForPulseChase.csv}
    \label{figsupp:dme211raw}
  \figsupp{Filtered label-chase RNAseq data for modeling, normalized directly within sample.}
    {output/Figure2\_Table\_PulseChaseDataNormalizedDirectAndFiltered.csv}
    \label{figsupp:dme211filterDirect}
  \figsupp{Filtered label-chase RNAseq data for modeling, normalized by modeling across samples.} 
    {output/Figure2\_Table\_PulseChaseDataNormalizedByModel.csv}
    \label{figsupp:dme211filterModel}
  \figsupp{Degradation rate modeling results, from data normalized within samples.}
    {output/Figure2\_Table\_PulseChaseModelingResultTable\_DirectNormalization.csv}
    \label{figsupp:dme211resultsDirect}
  \figsupp{Degradation rate modeling results, from data normalized across samples.}
    {output/Figure2\_Table\_PulseChaseModelingResultTable\_ModelNormalization.csv}
    \label{figsupp:dme211resultsModel}
  \figsupp{Enriched GO and KEGG terms within the set of mRNA destabilized upon a nitrogen upshift, across sample normalization.}
    {output/Figure2\_Table\_AcceleratedDegradationTranscripts\_EnrichedGOandKEGGterms.csv}
    \label{figsupp:dme211goAndKegg}
\end{figure}

To investigate the extent to which mRNA stability changes contribute
to transcriptome reprogramming, we compared degradation rates
to abundance changes following the upshift 
(\cite{Airoldi2016}, \FIG{figure2}c). Changes in mRNA degradation rates
and expression change rates are anti-correlated (Pearson's $r$ = -0.598,
p-value < $10^{-15}$, \FIGSUPP[figure2]{kkdComparison}),
consistent with stability changes impacting gene expression dynamics.
However, they are not entirely co-incident, as some destabilized
transcripts do not exhibit decreases in abundance (red points in
\FIG{figure2}c, \FIGSUPP[figure2]{comparisonDestabilized},
and \FIGSUPP[figure2]{compareSix}).
This analysis shows that increases in degradation rates play a 
significant role
in the rapid reprogramming of the transcriptome upon a glutamine
upshift, but that in some cases cases they are counteracted by
increases in mRNA synthesis rates \citep{Shalem2008,Canadell2015}. 

Functional coordination of mRNA stability changes suggests  a possible
role for \textit{cis}-element regulation. We analyzed UTRs and coding
sequence for enrichment of new motifs or known RNA binding protein
(RBP) motifs.
3’ UTRs of destabilized transcripts are
depleted of Puf3p binding sites, and we found no enriched sequence
motif in the 3' UTRs.
5’ UTRs are enriched for a GGGG motif, which
may be explained by convergence between mRNA stability changes and
transcriptional control by Msn2/4 on the ESR “up” genes
(\FIGSUPP[figure2]{comparisonESR}, \cite{Gasch2000,Canadell2015}). 
5’ UTRs are also enriched for binding motifs reported for Hrp1p 
(\FIGSUPP[figure2]{hrp1}),
a canonical member of the nuclear cleavage factor I complex \citep{Chen1998}.
However, this protein has been shown to shuttle to the cytoplasm
and where it may play a regulatory role
\citep{Kessler1997,Kebaara2003,Guisbert2005}.
On average,
destabilized mRNAs are longer and contain more optimal codons
(\FIGSUPP[figure2]{lengthAndCodons}, \cite{Khong2017}). 
Together, this analysis suggests that the
mechanism of destabilization may act through cis elements in the 5’
UTR and or biased codon usage.


\subsection{A genome-wide screen for \textit{trans}-factors regulating \textit{GAP1} mRNA repression}

We sought to identify \textit{trans}-factors mediating accelerated mRNA
degradation in response to a nitrogen upshift. We selected \textit{GAP1} 
as representative of transcript destabilization, as it is abundant in
nitrogen-limiting conditions and is rapidly cleared upon addition of
glutamine  (3.24-fold increase in degradation rate, \FIG{figure3}a,
\FIGSUPP[figure2]{dme211resultsModel}). Previous approaches to high-throughput
genetics of transcriptional activity have used protein expression
reporters \citep{Neklesa2009,Sliva2016} or automation of qPCR 
\citep{Worley2015}. However, for our
purposes, we required direct measurement of \textit{GAP1} mRNA 
changes on a rapid timescale.
Therefore, we applied single molecule fluorescent \textit{in situ}
hybridization (smFISH) to quantify 
native \textit{GAP1} transcripts in yeast cells in the pooled
prototrophic yeast deletion collection \citep{Vandersluis2014}.
Using fluorescence activated cell sorting (FACS) and Barseq
\citep{Smith2009,Robinson2013,Giaever2014},
we aimed to quantify and model the distribution of \textit{GAP1} mRNA
in each mutant \citep{Kinney2010,Peterman2016}.

%Development of our screen required that we could detect and
%sort cells using \textit{GAP1} mRNA signal. 
We found that
individually labeled probes tiled across \textit{GAP1} mRNA
\citep{Raj2008} were insufficiently bright for
\textit{GAP1} mRNA quantification using flow cytometry (data not shown),
likely due to the small cell size of nitrogen-limited cells and the
low transcript numbers in yeast cells compared to mammalian cells
\citep{Klemm2014}. Therefore, we used branched DNA probes
(Quantigene), which serve to amplify the FISH signal
\citep{Hanley2013}. We developed a fixation and permeabilization
protocol (\FIGSUPP[figure4]{writeup4}) that enabled detection of the
distribution of  \textit{GAP1} mRNA in steady-state nitrogen-limited conditions
and its repression following the  upshift (\FIG{figure3}b). In control
experiments, we found that the signal is eliminated in a \textit{GAP1} deletion
or by omitting the targeting probe% that confers specificity
(\FIG{figure3}b and \FIGSUPP[figure3]{gap1Delete}). To validate
sorting, we sorted a sample of cells into quartiles and used
microscopy to count fluorescent foci per cell
(\FIG{figure3}c) .
We found that increased flow cytometry signal is associated with an
increase in the number of foci in the cells (\FIG{figure3}d, $R^2$ = 0.607,
p < $10^{-11}$ ). 

\begin{figure}[h!]
\includegraphics[width=\textwidth]{output/Figure3.png}
\caption{\textbf{\textit{GAP1} mRNA dynamics measured by flow cytometry.}
  \textbf{a)} 
  GAP1 mRNA following upshift measured using RT-qPCR, relative
  to an external spike-in mRNA standard. The dashed line is fit
  to points after 2 minutes. 
  \textbf{b)} Flow
  cytometry of wild-type yeast in nitrogen-limited conditions and
  following an upshift. The vertical grey lines indicate FACS gate
  boundaries used for cell sorting. 
  \textbf{c)} Representative cells from each bin sorted from the
  experiment in panel b. 
  \textbf{d)} Quantification of microscopy data. 
  Each black dot represents a single cell. 
  The mean number of foci per cell in each bin is displayed as a red point.}
\label{fig:figure3}
}
\end{figure}
\begin{figure}[h!]
\caption{
  \figsupp[\textit{GAP1} delete or omission of the targeting probe removes
signal of \textit{GAP1} FISH.]{
  \textit{GAP1} delete or omission of the targeting probe removes signal of GAP1 FISH.
Wild-type or \textit{GAP1}$\Delta$ cells were grown in proline-media, which induces expression
of \textit{GAP1}. As seen in the positive control, there is heterogeneity in the
induction. This is likely due to technical issues, namely fixation
time. 
%\textit{gap1}$\Delta$
%cells grow at a similar rate (no growth curve, but dilutions grow at around
%the same rate as the wild-type), and completely lack any induction signal. We
%conclude that any \textit{GAP1} FISH signal is coming from GAP1 mRNA expression.
}{\includegraphics[width=\textwidth]
    {output/Figure3_S_gap1deleteControl.png}}
\label{figsupp:gap1Delete}
}
\end{figure}
\begin{figure}[h!]
\caption{
\end{figure}

Previous SortSeq studies of
the yeast deletion collection have used outgrowth 
to generate sufficient material for 
Barseq \citep{Sliva2016}. However, formaldehyde fixation precludes
outgrowth. We found that below approximately $10^6$ templates, the
Barseq reaction produces primer dimers
that outcompete the intended PCR product (\FIGSUPP[figure4]{writeup4}). 
Therefore, we re-designed the
PCR reaction \citep{Robinson2013,Smith2009} to be robust for
very low sample inputs (\FIGSUPP[figure4]{writeup4}). Our protocol
incorporates a 6-bp unique molecular identifier (UMI) into the first
round of extension to identify PCR duplicates, 
and uses 3’-phosphorylated oligonucleotides and a
strand-displacing polymerase (Vent exo-) to block primer dimer formation and 
off-target amplification. 
%We developed a bioinformatics pipeline 
%using pairwise alignment
%for per-read quality-filtering and compatibility with variable barcode
%length, and using the degenerate UMI barcodes to help account for PCR
%duplicates. 
%UMIs to identify duplicates.
Because strain barcodes are of variable lengths, 
we developed a bioinformatic pipeline to extract barcodes and UMIs 
using pairwise alignment to invariant flanking sequences.
Based on \textit{in silico} benchmarks, this
approach was robust to systematic and simulated random errors 
that can confound analysis of the yeast deletion barcodes 
(\autoref{app:codeanddata}, \FIGSUPP[figure4]{writeup4}). 

We refer to this experimental approach as BFF (Barseq after FACS after FISH). 
We used BFF to estimate \textit{GAP1} mRNA abundance for every mutant in the
haploid prototrophic deletion collection \citep{Vandersluis2014} in
nitrogen-limiting conditions and 10 minutes following the upshift. 
This approach facilitates identification of mutants with
defects in mRNA regulation at both the transcriptional and
post-transcriptional level without altering \textit{GAP1} mRNA 
\textit{cis}-elements that may affect its regulation. 
Moreover, this design enables identification of factors that 
regulate both the steady-state abundance of \textit{GAP1} mRNA and 
its transcriptional repression following an upshift.
We analyzed the deletion pool in biological triplicate
(\FIG{figure4}a). We found that UMIs 
approached saturation at a slower rate than expected for random sampling,
consistent with PCR amplification bias 
(\FIGSUPP[figure4]{rarefaction}), and therefore we adopted the 
correction of \cite{Fu2011}. After
filtering, we calculated a
pseudo-events metric that approximates the number of each mutant sorted
into each bin. 
Principal components analysis shows that the samples are 
separated primarily by FACS bin within each
condition and biological replicates are clustered indicating that our
approach reproducibly captures the variation of  \textit{GAP1} mRNA flow
cytometry signal across the library (\FIGSUPP[figure4]{pca}). 
 
\subsection{Estimating \textit{GAP1} mRNA abundance for individual mutants}

We estimated the distribution of \textit{GAP1} mRNA for each mutant by
modeling pseudo-events in each quartile as a
log-normal distribution using likelihood maximization  
(\FIG{figure4}b). 
From model fits we estimated the mean expression value for each
mutant and found that the distribution of means estimated for
3,230 strains (\FIGSUPP[figure4]{dme209pooledFits}, \FIG{figure4}c) 
recapitulates the overall
distribution of flow cytometry signal (\FIG{figure4}a). 
%Specifically, replicate A had a consistently lower estimate of
%\textit{GAP1} FISH fluoresence in both flow cytometry and modeling.
%Replicate C had fewer mutants sorted (\FIGSUPP[figure4]{writeup4}), 
%reflected in the wider distribution of estimated means.
%To estimate \textit{GAP1}
%mRNA per strain, we used all replicate measurement to perform model
%fitting and filtered models for sufficient measurements  (at least two
%of three replicates in at least three of the four bins). We generated
%expression distribution estimates for 3,230 strains, and used the mean
%of each distribution as the estimate of \textit{GAP1} mRNA abundance for each
%strain (\FIGSUPP[figure4]{dme209pooledFits}). %added after commented out
To validate our approach we first examined
strains for which we expected to have a specific phenotype and
compared their mean expression level to the distribution of expression
for the entire population (\FIG{figure4}d). We found that the wildtype
genotype (\textit{his3}$\Delta$, complemented by the spHis5 in
library construction) has an expression level that is centrally
located in the distribution both before and following the upshift. The
\textit{gap1}$\Delta$ genotype is a negative control and 
we estimate that it is at the extreme
low end of the distribution before and following the upshift. 
\textit{dal80}$\Delta$ is a direct transcriptional repressor
of NCR transcripts %like \textit{GAP1}, 
and we found that this is defective in
repression of \textit{GAP1} before and after the upshift. 
Counter-intuitively, deletion of \textit{GAT1}, a transcriptional activator
of \textit{GAP1}, appears to have higher steady-state expression of
\textit{GAP1} mRNA.
However, increased expression of \textit{GAP1} mRNA in a
\textit{gat1}$\Delta$ background has
previously been reported \citep{Scherens2006} and is thought to
result from the complex interplay of NCR transcription factors on
their own expression levels. 
Data and models for each mutant strain can be visualized in browser
using a Shiny appplication (see
\url{http://shiny.bio.nyu.edu/users/dhm267/} or \autoref{app:shiny} ). 

\begin{figure}[h!]
\includegraphics[width=\linewidth]{output/Figure4.png}
\caption{\textbf{BFF estimates of \textit{GAP1} mRNA abundance.}
  \textbf{a)} Flow cytometry analysis of \textit{GAP1} mRNA 
  abundance in the prototrophic
  deletion collection before and after the upshift. The vertical gray
  lines denote FACS gates. Biological replicates are
  indicated by color. 
  \textbf{b)} Measurements for individual genes before and
  after the upshift. Black dashed lines indicate maximum-likelihood 
  fits of a log-normal to pseudo-events for each mutant. 
  Colors and axes as in panel a. 
  \textbf{c)} Distribution of mean modeled GAP1 mRNA levels
  for each mutant.
  \textbf{d)} The mean \textit{GAP1} mRNA expression levels for 
  individual mutants before and after
  the upshift are shown as points connected by a line, colored
  according to the type of gene. 
  For reference, the background violin plot shows the distribution 
  of all 3,230 mutants fit.}
\label{fig:figure4}
}
\end{figure}
\begin{figure}[h!]
\caption{
  \figsupp[Strain barcodes show no length bias, but do show a slight
but complex relationship between counts and GC-content.]{
Distribution of proportion of counts by different counts of G or C, or by
length. We see a complex relationship between relative counts and
GC content, but no relation with length. We conclude that, as
with any PCR-based sequencing assay, there exists a bias 
associated with their GC content.}{\includegraphics[width=\textwidth]
    {output/Figure4_S_GClengthBiasBarcodes.png}}
\label{figsupp:effectOfGCbias}
}
\end{figure}
\begin{figure}[h!]
\caption{
  \figsupp[Principal components analysis of the abundance estimates for samples.]{
Principal components analysis of the abundance estimates for samples. Each
color is a type of sample, from low to high gates (with black denoting the
input samples before sort). Technical replicates are connected by dashed lines,
biological replicates are each letter A B or C. At top, the first two prinicpal components
show the separation of gates by signal intensity, and reflects that the lower
gates on the upshifted samples were very close (blue and red samples on far right
panel), within the distribution of the negative population. This is consistent
with their tight sampling of the "GAP1-off" population, as seen in
\FIG{figure4}a.}{\includegraphics[width=\textwidth]
    {output/Figure4_S_PCAonFilteredQCdData.png}}
\label{figsupp:pca}
}
\end{figure}
\begin{figure}[h!]
\caption{
  \figsupp[Rarefaction curve of total UMI counts against unique UMI counts.]{
Rarefaction curve of UMI saturation.
The solid-line curve denotes the theoretical expectation of total 
observations per UMI in a sample (x-axis) and the number of 
unique UMIs (y-axis). This curve shows how we
expect UMI-collisions to depress the number of unique UMIs. Each 
point is from real data, with these
two numbers tabulated for each combination of a sample and strain barcode. We
see that these largely follow the curve of saturation of UMI-collisions, but
that it falls well below the expectation of independent UMI-collision, thus
we believe that there is an additional contribution of PCR-amplification noise
(PCR duplicates). 
}{\includegraphics[width=\textwidth]
    {output/Figure4_S_umiSaturationCurve.png}}
\label{figsupp:rarefaction}
}
\end{figure}
\begin{figure}[h!]
\caption{
  \figsupp[\textit{tco89}$\Delta$ and \textit{xrn1}$\Delta$ show
defects in \textit{GAP1} mRNA regulation in the BFF assay.]{
Data and fits for several mutants. \textit{xrn1}$\Delta$ mutant (left) is lowly abundant in
the library and is only observed in the highest bin of \textit{GAP1} signal, consistent
with the role of Xrn1p as a global exonuclease. 
\textit{tco89}$\Delta$ is the only detected member that would abrogate TORC1 activity.
This mutant (right) has elevated \textit{GAP1} mRNA before and after the upshift,
consistent with the role of TORC1 in repressing the NCR regulon. 
}{\includegraphics[width=\textwidth]
    {output/Figure4_S_poorlyQuantifiedStrains.png}}
\label{figsupp:tco89}
}
\end{figure}
\begin{figure}[h!]
\caption{
%  \figsupp[A similar rarefaction plot, but using \texttt{umi_tools}
%error correction on the UMIs.]{
%A similar rarefaction plot, but for the output of umi\_tools. The x-axis is total
%counts after umi\_tools error-correction. We see that it saturates very quickly
%around 200 unique UMIs. I think this is because a barcode with many counts from
%a PCR duplicate will greedily scavange UMIs from it's local neighborhood. On
%this basis, we ignore the umi\_tools deduplication.}{\includegraphics[width=\textwidth]
%    {output/Figure4_S_totalInputReadsVsUMItoolsCounts.png}}
%\label{figsupp:rarefactionUMItools}
  \figsupp{Supplementary file with experimental rationale, details, 
and protocol for the BFF experiment.}{Figure4\_Supplementary\_Writeup.pdf}
\label{figsupp:writeup4}
}
\end{figure}
\begin{figure}[h!]
\caption{
  \figsupp{Raw counts of strain barcode quantification within each bin
in the BFF experiment, and gate settings for the observations.}{output/Figure4\_Table\_BFFcountsAndGateSettingsFACS.csv}
    \label{figsupp:dme209rawCountsGates}
  \figsupp{BFF data filtered for modeling.}{output/Figure4\_Table\_BFFmodelingData.csv}
    \label{figsupp:dme209modelData}
  \figsupp{The parameters of all models fit to the BFF data.}{output/Figure4\_Table\_BFFallFitModels.csv}
    \label{figsupp:dme209allFits}
  \figsupp{All 3230 models used for identifying strains with defective
\textit{GAP1} dynamics.}{output/Figure4\_Table\_BFFfilteredPooledModels.csv}
    \label{figsupp:dme209pooledFits}
  \figsupp{Gene-set enrichment analysis results using \textit{GAP1}
estimates.}{output/Figure4\_Table\_GSEanalysisOfBFFresults.csv}
    \label{figsupp:dme209gsea}
\end{figure}

To identify new cellular processes that regulate \textit{GAP1} mRNA abundance, we
used gene-set enrichment analysis (\FIGSUPP[figure4]{dme209gsea}).
Following the upshift we found mutants that
maintain high \textit{GAP1} mRNA expression are enriched for negative
regulation of gluconeogenesis (\FIGSUPP[figure5]{gluco}) and the
Lsm1-7p/Pat1p complex (\FIG{figure5}a). Mutants in the TORC1 signalling
pathway were not enriched; 
however, we found that a \textit{tco89}$\Delta$ mutant has
greatly increased \textit{GAP1} mRNA expression before and after the upshift
(\FIGSUPP[figure5]{tco89}), consistent with the repressive role of TORC1
on the NCR regulon.
To compare expression before and after the upshift for each mutant,
we regressed the post-upshift mean expression against the pre-upshift 
mean expression for each genotype (\FIGSUPP[figure5]{prePredictPost}). 
We used the residuals for each
strain to identify mutants that clear \textit{GAP1} mRNA with kinetics slower
than expected by this model.
We found that the Lsm1-7p/Pat1p complex is again strongly 
enriched for slower than
expected \textit{GAP1} mRNA clearance (\FIGSUPP[figure4]{dme209pooledFits}). 
Specifically
the \textit{lsm1}$\Delta$, \textit{lsm6}$\Delta$, and 
\textit{pat1}$\Delta$ strains are highly elevated in \textit{GAP1}
expression before the upshift and strongly impaired in the 
repression of \textit{GAP1} mRNA after the upshift (\FIG{figure5}a). 

As these factors are associated with processing-body dynamics, 
we tested if microscopically-observable processing-bodies form or
disassociate during the upshift, using microscopy of Dcp2-GFP. 
We did not observe qualitative changes
in Dcp2-GFP distribution (\FIGSUPP[figure5]{pbodyScope}),
and thus the upshift does not
result in a microscopically visible changes in processing-body foci
as seen in other stresses. This is consistent with previous
investigations of amino-acid limitation stress \citep{Hoyle2007} and
suggests that the defects in \textit{GAP1} mRNA clearance likely 
result from their roles in decapping or associated processes.

\begin{figure}[h!]
\includegraphics[width=\linewidth]{output/Figure5.png}
\caption{\textbf{Disrupting the Lsm1-7p/Pat1p complex impairs
  clearance of  \textit{GAP1} mRNA.}
  \textbf{a)} In the background is the distribution of 
  fit \textit{GAP1} mRNA mean expression levels for all mutants
  in the pool. Indicated by colored points and lines are the means for
  individual knockout strains, as labeled.
  \textbf{b-e)}, \textit{GAP1} mRNA relative to
  \textit{HTA1} mRNA before and 10 minutes after a glutamine upshift, 
  in biological triplicates. Lines are a log-linear regression fit. 
  Points are dodged horizontally for clarity, but timepoints for
  modeling and for drawn lines are 0 and 10 minutes exactly.
  Wild-type is FY4.
  \textbf{b)} \textit{xrn1}$\Delta$, \textit{ccr4}$\Delta$,
  \textit{pop2}$\Delta$ are all slowed in clearance (p-values < 0.004).
  \textbf{c)} \textit{lsm1}$\Delta$ and \textit{lsm6}$\Delta$ are 
  slowed in clearance (p-values < 0.0132 and 0.0299, respectively).
  \textbf{d)} \textit{edc3}$\Delta$ is slowed in clearance 
  (p-value < $10^{-4}$).
  \textit{scd6}$\Delta$ and \textit{tif4632}$\Delta$ are slowed in
  clearance (p-values < $10^{-5}$) and have lower levels of expression
  before the upshift (p-values < 0.003).
  \textbf{e)} A deletion of 150bp 3' of \textit{GAP1} stop codon has
  no significant effect, but a deletion of 100bp 5' of the start
  codon has slower clearance 
  (p-value < $10^{-4}$) and lower level of expression before the upshift
  (p-value < 0.0015).}
\label{fig:figure5}
}
\end{figure}
\begin{figure}[h!]
\caption{
%  \figsupp[Testing several known mutants with the
%\textit{GAP1}/\textit{HTA1} qPCR assay.]{
%Testing several known mutants with the \textit{GAP1}/HTA1 qPCR assay. The three indicated
%strains were grown in nitrogen limitation, and swift repression of \textit{GAP1} was
%triggered with the addition of glutamine. We a defect in the clearance of \textit{GAP1}
%relative to HTA1 for all mutants \textit{xrn1}$\Delta$, \textit{ccr4}$\Delta$, and \textit{pop2}$\Delta$, as expected given
%their central roles in mRNA turnover. We see a slight but significant increase
%in \textit{GAP1}/\textit{HTA1} ratios at steady-state in the
%\textit{xrn1}$\Delta$ mutant. All three mutants are
%significantly slowed in their clearance of \textit{GAP1}/HTA1 signal,
%by ANCOVA, p < 0.05 .}{\includegraphics[width=\textwidth]
%    {output/Figure5_S_ccr4pop2xrn1.png}}
%\label{figsupp:xrn1}
  \figsupp[An independently generated \textit{GAP1} 5' deletion.]{
During strain construction, a deletion of 152bp 5' of the start
codon was also generated. We tested \textit{GAP1} dynamics in this
strain as well, and found that it shares the same phenotype as a
100bp deletion. Methods are the same as in \FIG{figure5}e,
both 5' deletes are slowed in clearance, ANCOVA p < 0.05 .}
{\includegraphics[width=\textwidth]
    {output/Figure5_S_bothutr.png}}
\label{figsupp:bothutr}
}
\end{figure}
\begin{figure}[h!]
\caption{
  \figsupp[Processing-body dynamics are not associated with the
nitrogen upshift, by Dcp2p-GFP microscopy.]{
A strain harboring a copy of Dcp2p-GFP expressed from a plasmid
was 
grown in conditions of exponential phase in YPD or 10 minutes of
starvation in water (first row). This is a common condition known to
result in the formation of processing-body foci of Dcp2-GFP.
We do not see either formation or dissolution of Dcp2-GFP foci during
the nitrogen upshift.}{\includegraphics[width=\textwidth]
    {output/Figure5_S_pbodyMicroscopy.png}}
\label{figsupp:pbodyScope}
}
\end{figure}
\begin{figure}[h!]
\caption{
  \figsupp[Knock-out mutants of negative regulators of 
gluconeogenesis are associated with higher \textit{GAP1} expresion 
after the upshift.]{
Knock-out mutants of negative regulators of gluconeogenesis are associated with
higher estimated \textit{GAP1} mean after the upshift, by GSEA analysis
of GO-terms (p-value < 0.05).}{\includegraphics[width=\textwidth]
    {output/Figure5_S_negGluconeogenesis.png}}
\label{figsupp:gluco}
}
\end{figure}
\begin{figure}[h!]
\caption{
  \figsupp[Knock-out mutants of genes involved in sulfate assimilation
are associated with higher estimated \textit{GAP1} mean after the upshift.]{
Knock-out mutants of involved in sulfate assimilation are associated with
higher estimated \textit{GAP1} mean before the upshift, by GSEA analysis
of GO-terms (p-value < 0.05).}{\includegraphics[width=\textwidth]
    {output/Figure5_S_sulfateAssimilation.png}}
\label{figsupp:sulfate}
}
\end{figure}
\begin{figure}[h!]
\caption{
  \figsupp[The relationship between the estimated mean before the 
shift and after the upshift.]{
The relationship between the estimated mean before the upshift and after the
upshift. Scatter plot of the estimated means, with marginal histograms along
top and right. Red vertical line on top histogram is a cut-off of
\textit{GAP1} mRNA induction for this analysis,
and is the mean of the fit to wild-type minus the standard deviation of that
distribution. The red linear regression line is fit to all points above this
threshold, in which expression was detected before the upshift.
}{\includegraphics[width=\textwidth]
    {output/Figure4_S_PreShiftPredictingPostShiftLM.png}}
\label{figsupp:prePredictPost}
}
\end{figure}
\begin{figure}[h!]
\caption{
\end{figure}

To confirm the role of the Lsm1-7p/Pat1p  complex in clearing \textit{GAP1}
mRNA during the nitrogen upshift we measured \textit{GAP1} mRNA
repression using qPCR normalized to
\textit{HTA1}, which is not subject to destabilization upon the upshift
(\FIG{figure2}a). We also tested mutants that were not detected using BFF,
or were only detected in the highest \textit{GAP1} bin and therefore
not suitable for modeling
(e.g. \textit{xrn1}$\Delta$ \FIGSUPP[figure5]{tco89}). 
Using this assay we found that the main 5’-3’ 
exonuclease \textit{xrn1}$\Delta$ 
and mRNA deadenylase complex (\textit{ccr4}$\Delta$ and
\textit{pop2}$\Delta$) are impaired in \textit{GAP1} repression 
(\FIG{figure5}b).
We found that qPCR confirms results from BFF.
We confirmed that the accelerated degradation of \textit{GAP1} mRNA is impaired
in \textit{lsm1}$\Delta$ and \textit{lsm6}$\Delta$ 
(\FIG{figure5}c). 
We also tested
\textit{scd6}$\Delta$ and \textit{edc3}$\Delta$, two modifiers of the
decapping or processing-body
assembly functions associated with this complex, and found two
distinct phenotypes (\FIG{figure5}d). \textit{edc3}$\Delta$ has similar expression 
as wild-type before the upshift, but is cleared much more slowly.
\textit{scd6}$\Delta$ has a greatly reduced \textit{GAP1} expression
before the upshift but is impaired in \textit{GAP1} clearance. 
\textit{tif4632}$\Delta$, a deletion of the eIF4G2
known to interact with Scd6p \citep{Rajyaguru2012}, 
has a similar phenotype. 

Identification of an initiation factor subunit with defects in
\textit{GAP1} mRNA clearance suggests that translation control may
impact stability changes. Therefore we deleted the 5' UTR
and 3' UTR of \textit{GAP1}. 
%100bp and 152bp upstream of the start codon (approximate 5’ UTR) or the
%100bp downstream of the stop codon (approximate 3’ UTR), 
Whereas the 3’ UTR deletion does not have an effect the 5’ UTR deletion
exhibit the phenotype of reduced \textit{GAP1} mRNA before the upshift
and a reduced rate of transcript clearance following the upshift
(\FIG{figure5}e). 
We observed a similar phenotype with a different deletion of 152bp upstream
of the \textit{GAP1} start codon (\FIGSUPP[figure5]{bothutr}). 
%These measurements suggest altered mRNP composition of the
%Lsm1-7p/Pat1p complex and associated decapping factors are associated
%with defects in \textit{GAP1} mRNA expression dynamics upon a 
%nitrogen upshift. Importantly the phenotype of the \textit{scd6}$\Delta$, 
%\textit{tif4632}$\Delta$, or 5’ sequence deletions preceed the 
%addition of glutamine, suggesting that the observed destabilization
%of \textit{GAP1} may be the halt of a stabilization effect, perhaps
%due to changes in translational status of \textit{GAP1}.
This indicates that \textit{cis}-elements responsible for the
rapid clearance of \textit{GAP1} are unlikely to be located in the
3' UTR, and instead may be exerting an effect at the 5' end of the
mRNA.

\section{Discussion}

Regulated changes in mRNA stability allows cells to rapidly reprogram
gene expression, clearing extant transcripts that are no longer
required and potentially reallocating translational capacity.
%Despite progress in understanding the pathways that mediate
%mRNA degradation, the functional role of mRNA degradation and the
%factors that control regulated changes in mRNA stability remain poorly
%understood. 
Pioneering work in budding yeast has shown that mRNA
stability changes facilitate gene expression remodeling in response to
changes in nutrient availability including changes in carbon sources
\citep{Scheffler1998} and iron starvation \citep{Puig2005}. 
Here, we characterized genome-wide changes
in mRNA stability in response to changes in nitrogen availability and
identified factors that mediate the rapid repression of the
destabilized mRNA, \textit{GAP1}. Our study extends our previous work
characterizing the dynamics of transcriptome changes using chemostat
cultures \citep{Airoldi2016} and shows that accelerated mRNA
degradation targets a specific subset of the transcriptome in response
to changes in nitrogen availability. We developed a novel approach to
identify regulators of mRNA abundance using pooled mutant screens and
find that modulators of decapping activity, and core degradation
factors, are required for accelerated degradation of 
\textit{GAP1} mRNA. 
 
Measuring the stability of the transcriptome requires the ability to
separate pre-existing and newly synthesized transcripts. We modified
existing methods to measure 
post-transcriptional regulation of the yeast transcriptome in a
nitrogen upshift using 4-thiouracil labeling
\citep{Miller2011,Neymotin2014,Munchel2011}. These
modifications entailed improved normalization and quantification of
extant transcripts and explicit modeling of labelling dynamics to
account for some of the inherent limitations of metabolic labeling
approaches \citep{Perez-Ortin2013}. Continued development of
fractionation biochemistry \citep{Duffy2015}  and incorporation of
explicit per-transcript efficiency terms will improve these
methods further \citep{Chan2017}.

Our experiments show that the process of physiological and gene
expression remodeling occur on very different timescales in response
to a nitrogen upshift. Cellular physiology is remodeled over the
course of two hours to achieve a new growth rate.
By contrast, transcriptome remodeling occurs rapidly and through
states that are distinct
from increases in steady-state growth rates. 
%Interestingly,
%we found that the yeast transcriptome is on average less stable in
%both nitrogen poor conditions (6.89 min) and following the upshift
%(6.40 min) compared to rich media conditions reported in other studies
%using metabolic labeling
%\citep{Munchel2011,Neymotin2014,Miller2011}. This relative
%reduction in mRNA stability could be an adaptation to potentially
%limiting ribonucleotides, but further work exploring differences in
%mRNA degradation rates during growth limited by different nutrients is
%required to test this concept \citep{Garcia-Martinez2016}.
%Stability changes upon the nitrogen upshift generally exhibited the
%expected relationship with rates of abundance change (
%anti-correlation, $R^2=$-0.376 ); however, we found multiple cases in
%which increased mRNA degradation rates did  not result in rapid
%decreases in mRNA abundance. This has been observed for transcripts
%up-regulated in stress conditions, and has been proposed as a
%mechanism to effect a rebalancing of the transcriptome after a
%transient phase of reprogramming \citep{Shalem2008}. Importantly,
%the changes in mRNA stability that we detect are nearly coincident
%with the environmental perturbation suggesting that a signal is sensed
%and the effect propagated to impact post-transcriptional regulation
%with rapid kinetics.
Previous studies have shown that transcriptional activation of the NCR
regulon is rapidly repressed upon a nitrogen upshift
\citep{Airoldi2016}. Our
results indicate that accelerated degradation of 
%at least 16 of the 77 probable 
many NCR transcripts \citep{Godard2007} contributes to this
repression. 
A three-fold increase in
the degradation rate of \textit{GAP1} mRNA provides an additional layer of
repressive control. Importantly, our results show that accelerated
degradation is not limited to NCR transcripts but also targets
transcripts enriched in carbon metabolism pathways, particularly
pyruvate metabolism. Conversely, we also detect an apparent reduction in the 
degradation rate for some transcripts 
%enriched in ribosome biogenesis mRNAs (for example \textit{NSR1}) as well as 
including \textit{MAE1}. \textit{MAE1} encodes
an enzyme responsible for the conversion of malate to pyruvate, and
combined with the accelerated degradation of \textit{PYK2} mRNA 
may reflect an adaptive shunt of carbon skeletons from glutamine 
to alanine via the TCA cycle \citep{Boles1998}. 
%Due to the limitations of
%labeling approaches \citep{Perez-Ortin2013} we cannot conclude 
%here that these transcripts are indeed stabilized, 
%however we can conclude that they are strongly upregulated. 
Recently, \cite{Tesniere2017}
described destabilization  of carbon metabolism mRNAs after repletion
of nitrogen following 16 hours of starvation. We do
not detect destabilization of \textit{PGK1} mRNA and note that
the basal half-life of 6.2 minutes estimated in our study is similar
to the accelerated rate reported by \cite{Tesniere2017}.

%To identify the factors that underlie accelerated mRNA degradation, we
%developed a global \textit{trans}-factor screen using mRNA FISH, FACS, and
%sequencing. 
%BFF identified mutants in the Lsm1-7p/Pat1p
%complex as having elevated \textit{GAP1} mRNA levels both before and after the
%upshift.
%Given that the \textit{GAP1} mRNA is destabilized during this
%transition we suspect that these core mRNA degradation factors are
%directly involved. 
%Because factors associated with the Lsm1-7p/Pat1p
%complex are also involved in processing-body formation we looked for
%processing-body dynamics during the nitrogen upshift, but did not see
%qualitative changes in Dcp2-GFP distribution (raw data available in
%supplement). However, it has been proposed that pre-existing mRNPs
%seed the formation of processing-bodies \citep{Lui2014}, thus the
%phenotype may require assays at a finer spatial scale to eliminate
%this possibility \citep{Rao2017}. Interestingly, during
%cross-comparisons with a recent dataset exploring mRNA localization to
%RNP condensates \citep{Khong2017} we found that the set of
%destabilized transcripts in the label-chase experiment are on average
%longer in CDS and have an increased codon-optimality, two factors that
%were shown to be associated with differences in stress-granule
%localization of mRNA \citep{Khong2017}. 

%Regulated changes in mRNA stability can be mediated by RBP binding the
%3’ UTR of specific transcripts.  However, cis element analysis are
%inconsistent with a role for known RBPs in the observed
%destabilization. In particular, Puf3p motifs are de-enriched from the
%destabilized set. We failed to detect new 3’ UTR sequence motifs that
%are enriched in destabilized transcripts, but these 
Destabilized
transcripts are enriched for a binding motif of Hrp1p in
the 5’ UTR. This essential component of mRNA cleavage for
poly-adenylation in the nucleus has been shown to shuttle to the
cytoplasm and bind to amino-acid metabolism mRNAs
\citep{Guisbert2005} and been shown to interact genetically to
mediate nonsense-mediated decay (NMD) of a \textit{PGK1} mRNA harboring a
premature stop-codon \citep{Gonzalez2000} or a \textit{cis}-element spanning
the 5’ UTR and first 92 coding bp of \textit{PPR1} mRNA \citep{Kebaara2003}.
A potential role for these Hrp1p sites warrants further investigation. 

BFF identified mutants in the Lsm1-7p/Pat1p
complex as having elevated \textit{GAP1} mRNA levels both before and after the
upshift. This is expected given their central role in mRNA 
metabolism, and experiments using \textit{GAP1} normalized to
\textit{HTA1} demonstrate that the effect before the upshift is
likely a global effect (\FIG{figure5}c). 
However, these mutants still have a
significant defect in clearance of \textit{GAP1},
and the assay demonstrates that associated decapping factors 
\textit{EDC} and \textit{SCD6} have specific effects (\FIG{figure5}d).
Given that the \textit{GAP1} mRNA is destabilized during this
transition we suspect that these mRNA degradation factors are
directly involved. 
While we found that the \textit{edc3}$\Delta$ mutant has defects in
clearance of \textit{GAP1}, we also 
found that \textit{scd6}$\Delta$,
%mutant shares a phenotype of reduced \textit{GAP1} mRNA
%expression during nitrogen limitation and reduced rate of \textit{GAP1} mRNA
%clearance with a 
\textit{tif4632}$\Delta$, and deletion of the 5' UTR
of \textit{GAP1} impairs clearance (\FIG{figure5}e). 
This deletion does not include the TATA box (ending at -179) or
GATAA sites (nearest at -237) responsible for NCR GATA-factor
regulation of \textit{GAP1} \citep{Stanbrough1996}.
This suggests that interactions of
these factors with \textit{cis}-elements in the 5’ UTR might be responsible for
stabilizing \textit{GAP1} mRNA during limitation, although the 
truncation of the 5' sequence may be enough to inhibit translation 
initiation by virtue of the shorter length \citep{Arribere2013}.
Elements in the 5’ UTR have
also been demonstrated to modulate \textit{GAL1} mRNA stability
\citep{Baumgartner2011} and destabilize \textit{SDH2} mRNA upon glucose
addition, perhaps due to the competition between translation
initiation and decapping mechanisms \citep{De_la_Cruz2002}.
Interestingly, both \textit{GAP1} and \textit{SDH2} 
share the feature of a second start
codon downstream of the canonical start \citep{Neymotin2016} and
we have previously found that mutation of
the start codon of \textit{GAP1} results in lower
steady-state mRNA abundances \citep{Neymotin2016}.
This
%in light of recent analyses further highlighting the contribution of
%translation dynamics to mRNA stability 
%\citep{Presnyak2015,Neymotin2016,Cheng2017}, 
suggests a mechanism of degradation through dynamic changes in 
translation initiation that triggers decapping of \textit{GAP1} 
and other mRNA. 
%However, the deletion of \textit{SCD6} would be
%expected to promote translation of mRNA on the basis of it’s measured
%repressive activity in cell extracts, suggesting that if Scd6p does play a role
%that it may be specified by some condition-specific modulation of its
%activity \citep{Rajyaguru2012,Poornima2016}. 
Future work interrogating
this possible interaction of translational status and mRNA
stability during dynamic conditions could also expand our understanding of
the relationship between these two processes.

To our knowledge, this is the first time mRNA abundance has
been directly estimated using a SortSeq approach, although 
%sorting on indirect markers or 
using mRNA FISH and FACS to enrich subpopulations of cells has been
previously reported \citep{Klemm2014,Hanley2013,Sliva2016}. This
approach could be used with other barcoding mutagenesis technologies,
like transposon-sequencing or Cas9 mediated perturbations, to
systematically test the genetic basis of transcript dynamics.
%phenotypes. A strategy combining this technology with transcriptomics
%as a high-dimensional marker could accelerate unbiased investigation
%of cellular signalling pathways \citep{Gapp2016}. Additionally, 
The use of branched-DNA mRNA FISH, or other methods
\citep{Rouhanifard2017}, allows for mRNA abundance estimation without
requiring genetic manipulation which makes it suitable for a variety
of applications such as extreme QTL mapping. 
%While the cell wall
%of yeast makes optimization crucial to this assay, future development
%of hybridization protocols may improve accuracy and make the assay
%more robust \citep{Richter2017,Wadsworth2017}. 
Furthermore, our methods for library construction should permit accurate
quantification of pooled barcode libraries with small inputs, 
expanding the possibilities for flow cytometry markers to fixed-cell assays.

Why is \textit{GAP1} subject to multiple layers of gene product repression upon
a nitrogen upshift, at the level of transcript synthesis, degradation,
protein maturation, and post-translational inactivation? Given the
strong fitness cost of inappropriate activity \citep{Risinger2006},
this overlap could ensure mechanistic redundancy for robust repression in
the face of phenotypic or genotypic variation. Alternatively, it could
reflect a systematic need to free ribonucleotides or
translational capacity, or result from some as yet uncharacterized
process.
%, or could simply be an effect of some unrelated function. 
%While this question remains open,
%we have made progress towards this goal by identifying 
%factors required for its accelerated degradation.
%of decapping associated with the Lsm1-7p/Pat1p complex play a role. 
Future work aimed at determining the adaptive basis of accelerated
mRNA degradation will serve to illuminate the functional role of
post-transcriptional gene expression regulation.
%dissecting thisb
%mechanism and contrasting the dynamic process of mRNA destabilization
%during other growth transitions would greatly inform our understanding
%of mRNA stability specification at steady-state, possibly in light of
%the relationship between translation and stability of mRNAs. 

\section{Methods and Materials}

\subsection{Availability of data and analysis scripts}

Computer scripts used for all analyses are available as a git repository
on GitHub
(\texttt{\url{https://github.com/darachm/millerBrandtGresham2018}})
and data is available as zip archives on the Open Science
Framework (\texttt{\url{https://osf.io/hn357/}}).
Instructions for obtaining, unpacking, and using these are in
\autoref{app:codeanddata}.

We did not make this \LaTeX class, it is a modified version of one
provided by a journal.

\subsection{Media and upshifts of media}

Nitrogen-limited media (abbreviated as "Nlim") is a minimal media
supplemented with various salts, metals, minerals, vitamins, and
2\% glucose, as previously described \citep{Airoldi2016,Brauer2008}. 
For proline limitation, 
Nlim base media was made with 800$\mu$M L-proline as the sole
nitrogen source (NLim-Pro).
YPD media was made using standard recipes \citep{Amberg2005}.
All growth was at 30$^{\circ}$C, in an air-incubated 200rpm shaker  
using baffled flasks with foil caps, or roller drums for 
overnight cultures in test tubes.
For glutamine upshift experiments, 
400$\mu$M L-glutamine was added from a 100mM stock solution dissolved 
in MilliQ double-deionized water and filter sterilized.
All upshift experiments were
performed at a cell density between 1 and 5 million cells per mL,
in media where saturation is approximately 30 million cells per mL. 
For all experiments, 
a colony was picked from a YPD plate and grown in a 5mL NLimPro 
pre-culture overnight at 30$^{\circ}$C, then innoculated into
the experimental culture from mid-exponential phase.

\subsection{Strains}

See \TABLE{strainsTable} for details.  
The wild-type strain used is FY4, a S288C derivative. 
The pooled deletion collection is as published in 
\cite{Vandersluis2014}.
For all experiments with single strains, colonies were struck 
from a -80$^{\circ}$C frozen stock onto YPD (or YPD+G418 for
deletion strains) to isolate single colonies.
For pooled experiments we inoculated directly into NLim media
from aliquots of frozen glycerol stocks.

Strains with deletions 5' of the start codon and 3' of the stop
codon were generated by the "delitto-perfeto" 
method \citep{Storici2006}, 
by inserting the pCORE-Kp53 casette
at either the 5' or 3' end of the coding sequence, then transforming
with a short oligo product spanning the deletion junction and
counter-selecting against the casette with Gal induction of p53 
from within the cassette.
These strains were generated and confirmed by Sanger sequencing,
and traces are available in directory \texttt{data/qPCRfollowup/} 
within the data zip archive (\autoref{app:codeanddata}).

\subsection{Measurement of growth during upshift}

A single colony of FY4 was inoculated in 5mL NLimPro 
media and grown to exponential phase, then back diluted in NLimPro media
in a baffled flask. 
Samples were collected into an eppendorf, sonicated,
diluted in isoton solution, and analyzed with a Coulter Counter Z2
(Beckman Coulter).
%within approximately 5 minutes 
%of collection from flask. Output files were read and processed by a 
%script (in the git repository) to yield counts per diameter bin, 
%and minimum size particle 
%gating was used with blank controls to exclude measurement noise.
%<!--We are only presenting one replicate, as similar results  
%have been obtained many times by our lab and 
%we chose to generate a rapidly sampled time course to best represent 
%the upshift. Seriously, people have seen this since Slator 1918.
%-->

\subsection{Re-analysis of microarray data} 

%Supplemental files from \cite{Airoldi2016} were 
%downloaded, read
%into \texttt{localc} (an open-source spreadsheet software), 
%a small Excel-generated auto-correction error was 
%fixed ("Oct-1" -> "OCT1"), and the file saved as a CSV. Microarray 
%intensity ratios were processed with 
Gene expression data \citep{Airoldi2016} were analyzed using
\texttt{pcaMethods} to perform a SVD PCA on scaled data. 

\subsection{qPCR}

Each strain was grown from single colonies.
Samples were collected before, during the first ten minutes of
the nitrogen upshift (\FIG{figure3}),
or at ten minutes after the upshift (\FIG{figure3}).
For the experiments described in \FIG{figure5}, all work
was done in biological replicates.
Each 10mL sample was collected by vacuum onto a 25mm nylon filter
and frozen in an eppendorf in liquid nitrogen.
RNA was extracted by adding 400$\mu$L of TES buffer
(10mM Tris (7.5pH), 10mM EDTA, 0.5\% SDS)
and 400$\mu$L of acid-phenol, vortexing vigorously and incubating at 
65$^{\circ}$C for an hour with vortexing every 20 minutes. 
For \FIG{figure3} only, at the beginning of this extraction incubation
we added 10$\mu$L of a 0.1ng/$\mu$L in-vitro synthesized spike-in 
mRNA BAC1200 (as generated
for the label-chase RNAseq (\FIGSUPP[figure2]{writeup2}), 
but without 4-thiouridine). 
All samples were separated by centrifugation and extracted again 
with chloroform on a 2mL phase-lock gel tube (5Prime \#2302830). 
After ethanol precipitation of the aqueous layer, 
RNA was treated DNAse RQ1 (Promega M610A) according to manufacturer
instructions, then the reaction heat-killed at 65$^{\circ}$C for 
10 minutes after adding a mix of 1:1 0.5M EDTA and RQ1 stop-solution.
The resulting RNA was
cleaned with a phenol-chloroform extraction and ethanol 
precipitated.
All samples were hybridized with RT primers by incubating the mixture at 80$^{\circ}$ for 
5 minutes then on ice for 5 minutes.
For \FIG{figure3} 2$\mu$g RNA was primed with 2.08ng/$\mu$L
random hexamers (Invitrogen 51709) and 
2.5mM total dNTPs (Promega U1511),
while for \FIG{figure5} 1$\mu$g RNA was primed with 
5.6mM Oligo(dT)18 primers (Fermentas FERSO132) and
0.56mM total dNTPs (Promega U1511).
These mixtures were combined with 1/10th 10x M-MulvRT buffer (NEB M0253L), 
1/20th volume RNAse-OUT (Invitrogen 51535), and 1/20th volume M-MulvRT (NEB M0253L). 
A negative control with no reverse-transcriptase enzyme was also prepared
and analyzed in the qPCR reaction.
The reaction proceeded for 1 hour at 42$^{\circ}$C, 
then was heat-killed at 90$^{\circ}$C
before diluting 1/8 with hyclone water (GE SH30538). 
This dilution was used as direct
template in 10$\mu$L reactions with SybrGreen I Roche qPCR master-mix
(Roche 04 707 516 001) for measurement on a Roche Lightcycler 480. 
For \FIG{figure3}, we used primers 
DGO230,DGO232 to quantify \textit{GAP1} and 
DGO605,DGO606 to quantify the synthetic spike-in BAC1200.
For \FIG{figure5}, we used primers
DGO229, DGO231 to quantify \textit{GAP1} and
DGO233, DGO236 to quantify \textit{HTA1}.  
See \TABLE{primerTable} for sequence.
These were run on a Roche480 Lightcycler, 
with a max-second derivative estimate
of the cycles-threshold (the $C_p$ value output by analysis) used 
for analysis by scripts included in the git repo 
(\autoref{app:codeanddata}).
Linear regression of the log-transformed values was used to quantify
the dynamics and assess significance of changes in expression
levels or rates of change.

%\subsection{qPCR of absolute \textit{GAP1} dynamics during upshift}
%
%We grew FY4 and sampled before or during the first ten minutes of
%a glutamine upshift.
%Each sample
%by sereological pipette was 10mL onto a 25 millimeter vacuum filter 
%paper (PALL NylaFlo), 
%then this filter paper was placed into an eppendorf, 
%frozen in liquid nitrogen, and stored in -80$^{\circ}$C. 
%RNA was extracted by adding 400$\mu$L of TES buffer
%(10mM Tris (7.5pH), 10mM EDTA, 0.5\% SDS)
%and 400$\mu$L of acid-phenol, vortexing vigorously and incubating at 
%65$^{\circ}$C for an hour with vortexing every 20 minutes. 
%At the beginning of this extraction incubation, 
%we added 10$\mu$L of a 0.1ng/$\mu$L in-vitro synthesized spike-in 
%mRNA BAC1200 (as generated
%for the label-chase RNAseq (\FIGSUPP[figure2]{writeup2}), 
%but without 4-thiouridine). 
%This was separated by centrifugation, then extracted again with
%chloroform on a 2mL phase-lock gel tube (5Prime \#2302830). 
%After ethanol precipitation of the aqueous layer, 
%RNA was treated with 10.6$\mu$L of DNAse RQ1 
%(Promega M610A) in a 44$\mu$L volume with RQ1 buffer,
%then the reaction heat-killed at 65$^{\circ}$C for 10 minutes
%after adding a mix of 1:1 0.5M EDTA and RQ1 stop-solution.
%The resulting RNA was
%cleaned with a phenol-chloroform extraction and ethanol 
%precipitated, then 2$\mu$g were mixed with 
%50ng/$\mu$L random hexamer oligonucleotides (Invitrogen 51709) 
%and 3mM promega dNTPs (Promega U1511)
%in a 20$\mu$L volume. This was incubated at 80$^{\circ}$C for 5 minutes, 
%then iced for 5 minutes to hybridize, then 12$\mu$L of this reaction
%was used for a positive reaction
%with the addition of 1/10th 10x M-MulvRT buffer (NEB M0253L), 
%1$\mu$L RNAse-OUT (Invitrogen 51535), and 1$\mu$L M-MulvRT (NEB M0253L). 
%A negative control with no reverse-transcriptase enzyme was also
%prepared.
%The reaction proceeded for 1 hour at 42$^{\circ}$C, 
%then was heat-killed at 90$^{\circ}$C
%before diluting 1/8 with hyclone water (GE SH30538). 
%This dilution was used as direct
%template in 10$\mu$L reactions with SybrGreen I Roche qPCR master-mix
%(Roche 04 707 516 001) for measurement on a Roche Lightcycler 480. 
%For this reaction, we used primers 
%DGO230,DGO232 to quantify \textit{GAP1} and DGO605,DGO606 to quantify
%the synthetic spike-in BAC1200 (see \TABLE{primerTable}).
%These were run on a Roche480 Lightcycler, with a max-second derivative estimate
%of the cycles-threshold (the $C_p$ value output by analysis) used 
%for analysis by scripts included in the git repo 
%(\autoref{app:codeanddata}).
%
%\subsection{qPCR of relative \textit{GAP1} for individual mutants}
%
%In biological triplicate, each tested strain was grown and sampled before 
%or 10 minutes after a glutamine upshift.
%Each sample
%sereological pipette was 10mL onto a 25 millimeter vacuum filter paper
%(PALL NylaFlo), then this filter paper was placed into an eppendorf, 
%frozen in liquid nitrogen, and stored in -80$^{\circ}$C. 
%RNA was extracted by adding
%400$\mu$L of TES buffer (10mM Tris (7.5pH), 10mM EDTA, 0.5\% SDS) and 
%400$\mu$L of acid-phenol, vortexing vigorously and incubating 
%at 65$^{\circ}$C for an
%hour with vortexing every 20 minutes. This was separated by
%centrifugation, then extracted again with chloroform on a 2mL
%phase-lock gel tube (5Prime \#2302830). 
%After ethanol precipitation of the aqueous layer,  
%RNA was treated with 3$\mu$L of DNAse RQ1 (Promega
%M610A) in a 40$\mu$L volume with RQ1 buffer,  then the reaction
%heat-killed at 65$^{\circ}$C for 10 minutes after adding a mix of 
%1:1 0.5M EDTA and RQ1 stop-solution. The resulting RNA was cleaned with a
%phenol-chloroform extraction and ethanol  precipitated, then 1$\mu$g was
%mixed 1$\mu$L of 100 mM Oligo(dT)18 primers (Fermentas FERSO132) and
%1$\mu$L 10 mM dNTPs Promega (PR U1511) in a 18$\mu$L volume. 
%This was incubated
%at 80$^{\circ}$C  for 5 minutes, then iced for 5 minutes to 
%hybridize, then 15$\mu$L of this reaction was used for a positive 
%reaction with the addition
%of 1/10th volume 10x M-MulvRT buffer (NEB M0253L), 
%1$\mu$L RNAse-OUT (Invitrogen 51535), 1$\mu$L M-MulvRT (NEB M0253L),
%and 1$\mu$L hyclone water (GE SH30538).  
%A negative control with no reverse-transcriptase enzyme was
%also prepared. The reaction proceeded for 1 hour at 42$^{\circ}$C, then was
%heat-killed at 90$^{\circ}$C before  diluting 1/8 with hyclone water (GE
%SH30538). This dilution was used as direct template in 10$\mu$L
%reactions with SybrGreen I Roche qPCR master-mix (Roche 04 707 516
%001) for measurement on  a Roche Lightcycler 480. For this reaction,
%we used primers DGO229, DGO231 to quantify \textit{GAP1} and
%DGO233, DGO236 to quantify \textit{HTA1} (see \TABLE{primerTable} ). 
%These were run on a Roche480 Lightcycler,
%with a max-second derivative estimate of the cycles-threshold (the
%$C_p$
%value output by analysis) used. For analysis of
%\textit{GAP1}/\textit{HTA1}
%dynamics, scripts available in the git repo were used to perform
%linear regression of the log-transformed values to assess significance
%and quantify effects.

\subsection{Microscopy of Dcp2-GFP}

To look for processing-body dynamics in response to
a nitrogen upshift, we used strain DGY525, which is FY3
containing plasmid pRP1315 (gift from Roy Parker).
Samples were collected before and following a nitrogen upshift,
from exponential growth in YPD, or 10 minutes after resuspending
YPD-grown cells in DI water.
All samples were collected by centrifugation at 10,000g for 30 seconds, 
aspirating most supernatant, then centrifugation for 20 seconds
and aspirating all media. Each pellet was 
immediately resuspended in 4\% PFA 
(diluted from EMS 16\% PFA ampule RT15710) 
with 1x PBS ( NaCl 8g/L, KCl 0.2g/L, Na$_2$HPO$_4$ 1.42g/L, 
KH$_2$PO$_4$ 0.24g/L) for 4, 10, 12, 19, or 25
minutes on bench, then spun at 10,000g for 1 minute, aspirated, 
then washed once and resuspended with 1x PBS. 
Samples were kept on ice, then put onto a coverslip
for imaging on a DeltaVision scope. Raw images available in the
microscopy zip archive (\autoref{app:codeanddata}).

\subsection{4tU label-chase and RNA sequencing}

The methods and analysis are detailed in \FIGSUPP[figure2]{writeup2},
including design rationale, protocols, and manufacturer information,
and all data and code are available according to instructions in
\autoref{app:codeanddata}.

FY4 was grown in nitrogen-limitation conditions overnight 
with a 50$\mu$M:50$\mu$M mixture of 4-thiouracil:uracil.
This culture was split, then 4mM uracil was added to chase the
4-thiouracil label (a 41-fold excess of uracil).
30mL samples of the culture were taken by filtration onto 25mm
nylon filters and flash-frozen in eppendorfs. 
After letting the chase proceed for 12.5 minutes, 
we added glutamine from 
a 100mM stock (dissolved in water) to a final concentration 
of 400$\mu$M to one flask, or an equal amounts of water to the 
control flask. 
Samples were extracted using a hot acid-phenol method,
with equal volume of synthetic spike-ins  added to each RNA 
extraction reaction.
4tU-containing spike-ins (polyadenylated coding sequences from
\textit{B. subtilus} and \textit{C. elegans}) were synthesized
\textit{in-vitro} as previously described \citep{Neymotin2014}. 
RNA was reacted with 
MTSEA-biotin to conjugate biotin to the 4-thiouracil-containing
RNA, then purified using streptavidin beads.
Fractionated RNA was depleted of rRNA using a RiboZero kit.
RNA samples were converted into Illumina sequencing libraries using
a strand-specific (UNG) protocol, ligating adapters that contain
UMI's \citep{Hong2017}.
These libraries were pooled and sequenced by the NYU Genomics
Core sequencing facility on an Illumina NextSeq.
Following base-calling and sample demultiplexing by NYU GenCore,
the sequencing reads were trimmed using 
\texttt{cutadapt} \citep{Martin2011}
aligned using \texttt{tophat2} \citep{Kim2013} to a reference genome
that included the yeast reference genome (assembly R64) and
spike-ins, filtered for mapping-quality and length using
\texttt{samtools} \citep{Li2009}, deduplicated 
with \texttt{umi\_tools} \citep{Smith2017}
and feature counting was performed using 
\texttt{htseq-count} \citep{Anders2015}.
Feature counts for yeast mRNAs were normalized to synthetic spike-ins, 
using the fitted values from a log-linear model of spike-in abundance
increase (see Results, and \FIGSUPP[figure2]{writeup2}).
The rate of mRNA degradation and changes in this rate
was quantified to assume an exponential model 
\FIGSUPP[figure2]{writeup2}, fit as a linear model to log 
transformed data.
Significant changes in mRNA degradation rates were quantified from
the coefficient method of the linear model using a cut-off of a FDR 
\citep{Storey2015} less than 0.01 and a doubling in degradation
rate (based on modeling detailed in \FIGSUPP[figure2]{writeup2}).

\subsection{Label-chase RNA sequencing \textit{cis} element analysis}

To detect if \textit{de novo} or known \textit{cis} elements were 
associated with destabilization upon a nitrogen upshift,
we used 
DECOD \citep{Huggins2011}, FIRE \citep{Elemento2007},
TEISER \citep{Goodarzi2012}, and the \#ATS pipeline \citep{Li2010}. 
We also scanned for association with
RBP binding sites from the CISBP-RNA database
\citep{Ray2013} using AME from the MEME suite
\citep{McLeay2010}. 
Final plots in the supplement were made using motif scans with 
GRanges \citep{Lawrence2013}.
Analysis was done using coding sequence and 
four different definitions of untranslated regions 
(200bp upstream of the start codon or downstream of the stop codon, 
the largest detected isoform in TIF-seq data \citep{Pelechano2014},
or the most distal detected gPAR-CliP sites in exponential-phase 
or nitrogen-limited growth \citep{Freeberg2013}).

\subsection{Barseq after FACS after mRNA FISH (BFF)}

The methods and analysis are detailed in \FIGSUPP[figure2]{writeup2},
including design rationale, protocols, and manufacturer information.

An aliquot of the prototrophic deletion collection
\citep{Vandersluis2014} was thawed and diluted, with 
approximately 78 million cells added to 500mL of NLimPro media 
in a 1L baffled flask. This was shaken at 30$^{\circ}$C overnight, 
then split into three flasks (A, B, and C). 
After three hours (at mid-exponential)
we collected samples of 30mL culture filtered onto a 25mm filter and
flash-frozen in an eppendorf in liquid nitrogen. 
We sampled in steady-state growth (pre-upshift) and  
10.5 minutes after adding 400$\mu$M glutamine (post-upshift).
Samples of the pool were fixed with formaldehyde
(4\% PFA diluted in PBS from 10mL aliquot, 
buffered, 2 hours room-temperature) and digested
with lyticase (in BufferB with VRC 37$^{\circ}$ 1 hour), 
\citep{Mcisaac2013}, and permeabilized with ethanol at
4$^{\circ}$ overnight.
Samples were processed with a Affymetrix Quantigene Flow RNA kit 
(product code 15710) designed to target
for \textit{GAP1} mRNA and labelled with Alexa 647.
This hybridization was done using a modified
version of the manufacturer's protocol (Appendix
\FIGSUPP[figure4]{writeup4}, including a DAPI staining step.
Samples were sonicated, then run through a BD FACSAria II.
Cells were gated for singlets and DAPI content 
(estimated 1N or more), then sorted based on emission area from a
660/20nm filter with a 633nm laser activation
into four gates within each timepoint, across replicates.
These were sorted using PBS sheath fluid at room-temperature, into
poly-propylene FACS tubes, then stored at -20$^{\circ}$C.
For each gate, cells were collected via centrifugation and genomic
DNA extracted by NaCl reverse-crosslinking at 65$^{\circ}$,
inspired by \cite{Klemm2014}, with
subsequent proteinase K and RNase A digestions.
Genomic DNA was split into three reactions to amplify in a 
modified barseq protocol (\FIGSUPP[figure4]{writeup4}).
See the supplementary write-up \FIGSUPP[figure4]{writeup4} for 
detailed protocols, rationale, and a discussion of dimers.
Barseq libraries were submitted to the NYU Genomics Core 
for sequencing on a 1x75bp run on a Illumina NextSeq.

\subsection{Analysis of BFF sequencing results}

We devised a pipeline to quantify barcodes using the UMI sequence 
incorporated in the first round of
amplicon priming, and benchmarked on \textit{in silico}
simulated datasets \FIGSUPP[figure4]{writeup4}.
Briefly, raw FASTQ files are processed with SLAPCHOP
(\url{https://github.com/darachm/slapchop}) 
which uses pair-wise alignment
\citep{Cock2009} to filter, extract UMIs from 
variable positions, and extract barcodes into different fields.
We demultiplex using a perl script, and align paritioned
strain barcodes to a reference barcode index \cite{Smith2009}
using \texttt{bwa mem} \cite{Li2013}. Barcodes are counted,
then we used the UMI's with the label-collision correction of 
\cite{Fu2011} to quantify the proportion of each mutant in the
sample. These relative counts are used the FACS data
(the sorted events per bin) to estimate the distribution 
of each mutant across the four gates in each timepoint.
We filtered for strains detected in at least three bins,
and fit a log-normal distribution using \texttt{mle} in R
\cite{Team2000}. The mean of this distribution
was used as the expression value of \textit{GAP1} in plots and
GSEA analysis using \texttt{clusterProfiler} \citep{Yu2012}.

\section{Acknowledgments}

We would like to acknowledge the funding source of 
NIH grant 5R01GM107466.
We would also like to thank
Andreas Hochwagen and Viji Subramanian for microscope access, 
Ken Birnbaum for helpful conversations and equipment usage, 
Michi Pedraza,
Andres Mansisidor, and Matt Paul for helpful reading and comments,
Evelina Tutucci for demonstrating FISH,
staff at 
eBioscience/Affymetrix/ThermoFisher for support with
the Quantigene/FlowRNA probe set, 
the Cold Spring Harbor Yeast Course,
the NYU Genomics Core facility for sequencing, flow cytometry,
and support,
and past and present members of the Gresham and Vogel labs for 
discussions and support.

%\nocite{*} % This command displays all refs in the bib file
\bibliographystyle{scripts/vancouver-elife.bst}
\bibliography{scripts/miller2018_references}

\newpage

\begin{table}%[bt]
\caption{Yeast strains used in this study}
% Use "S" column identifier to align on decimal point 
\label{tab:strainsTable}
\begin{tabular}{l l p{.6\textwidth}}
\toprule
Strain ID & Short description & Details \\
\midrule
DGY1 & FY4 & Isogenic to S288C, prototrophic, MATa \\
- & Deletion collection pool & Haploid (MATa) prototrophic deletion
collection as described in the publication of \cite{Vandersluis2014}\\
DGY410 &xrn1$\Delta$::KanMX &   ygl173c$\Delta$::KanMX from the prototrophic deletion collection \\
DGY564 &ccr4$\Delta$::KanMX &   yal021c$\Delta$::KanMX from the prototrophic deletion collection \\
DGY565 &pop2$\Delta$::KanMX &   ynr052c$\Delta$::KanMX from the prototrophic deletion collection \\
DGY547 &lsm1$\Delta$::KanMX &   yjl124c$\Delta$::KanMX from the prototrophic deletion collection \\
DGY571 &lsm6$\Delta$::KanMX &   ydr378c$\Delta$::KanMX from the prototrophic deletion collection \\
DGY545 &pat1$\Delta$::KanMX &   ycr077c$\Delta$::KanMX from the prototrophic deletion collection \\
DGY554 &edc3$\Delta$::KanMX &   yel015w$\Delta$::KanMX from the prototrophic deletion collection \\
DGY552 &scd6$\Delta$::KanMX &   ypr129w$\Delta$::KanMX from the prototrophic deletion collection \\
DGY611 &tif4632$\Delta$::KanMX &   ygl049c$\Delta$::KanMX from the prototrophic deletion collection \\
DGY539 & \textit{GAP1} 5' UTR delete & confirmed by Sanger sequencing to have 152bp deleted 5' of the start codon \\
DGY576 & \textit{GAP1} 5' UTR delete & confirmed by Sanger sequencing to have 100bp deleted 5' of the start codon \\
DGY577 & \textit{GAP1} 3' UTR delete & confirmed by Sanger sequencing to have 150bp deleted 3' of the stop codon \\
DGY525 & FY3 + pRP1315 & FY3, a ura- auxotroph (ura3-52), transformed with pRP1315 (URA3 marker, expressing a Dcp2-GFP fusion) \\
\bottomrule
\end{tabular}
\end{table}

\begin{table}%[bt]
\caption{Primers used in this study}
\label{tab:primerTable}
% Use "S" column identifier to align on decimal point 
\begin{tabular}{p{.1\textwidth} l p{.35\textwidth}}
\toprule
ID & Sequence & Description \\
\midrule
DGO230 & \scriptsize\tt ACGGTATCAAGGGTTTGCCAAG & Figure 3 qPCR \textit{GAP1} reverse \\
DGO232 & \scriptsize\tt GCATAAATGGCAGAGTTAC & Figure 3 qPCR \textit{GAP1} forward \\
DGO229 & \scriptsize\tt CTCTACGGATTCACTGGCAGCA & Figure 5 qPCR \textit{GAP1} reverse \\
DGO231 & \scriptsize\tt TTTGTTCTGTCTTCGTCAC & Figure 5 qPCR \textit{GAP1} forward \\
DGO236 & \scriptsize\tt TTACCCAATAGCTTGTTCAATT & qPCR HTA1 forward  \\
DGO233 & \scriptsize\tt GCTGGTAATGCTGCTAGGGATA & qPCR HTA1 reverse  \\
DGO605 & \scriptsize\tt CTGGACGACTTCGACTACGG & qPCR 1200 spike-in forward \\
DGO606 & \scriptsize\tt ATCAGCCTTTCCTTTCGTCA & qPCR 1200 spike-in reverse \\
DGO1562 & \scriptsize\tt
GTCTGAACTCCAGTCACATCNCNCNCNTNCNGTCGACCTGCAGCGTA & Degenerate first round primer \\
DGO1588 & \scriptsize\tt CCATTGGTGAGCAGCGAAGGATTTGGTGGA/3Phos/ & First round blocker oligo \\
DGO1589 & \scriptsize\tt AGAAAAAGCAGCGTAGATGTAGAAGCAAGA/3Phos/ & First round blocker oligo \\
DGO1567 & \scriptsize\tt GATGTCCACGAGGTCTCT & Second round outside primer \\
DGO1576 & \scriptsize\tt CGTACGCTGCAGGTCGAC/3Phos/ & Second round blocker oligo \\
DGO1519 & \scriptsize\tt CAAGCAGAAGACGGCATACGAGATGTCTGAACTCCAGTCAC & Second and third round inside primer and P7 adapter \\
Forward index primer & \scriptsize\tt
ACGCTCTTCCGATCTXXXXXGTCCACGAGGTCTCT & Multiplexing primer, where 
XXXXX is one of 120 different barcodes (see below).
\autoref{indexbarcodes}.\\
DGO276 & \scriptsize\tt AATGATACGGCGACCACCGAGATCTACACTCTTTCCCTACACGACGCTCTTCCGATCT & Illumina P5 adapter incorporation primer \\
DGO366 & \scriptsize\tt AATGATACGGCGACCACCGAGATCTACAC & RNAseq Illumina library amplification, forward \\
DGO367 & \scriptsize\tt CAAGCAGAAGACGGCATACGAGAT & RNAseq Illumina library amplifcation, reverse \\
\bottomrule
\end{tabular}
\medskip \\
All primers were synthesized by Integrated DNA Technologies (IDT).
\texttt{N} refers to a standard degenerate position.
\label{indexbarcodes}
Barseq multiplexing barcode sequences and index numbers available in 
the file 
\texttt{data/dme209/sampleBarcodesRobinson2014.txt} within the
data zip archive (\autoref{app:codeanddata}). \\
\end{table}

%%%%%%%%%%%%%%%%%%%%%%%%%%%%%%%%%%%%%%%%%%%%%%%%%%%%%%%%%%%%
%%% APPENDICES
%%%%%%%%%%%%%%%%%%%%%%%%%%%%%%%%%%%%%%%%%%%%%%%%%%%%%%%%%%%%

\appendix

\begin{appendixbox}
\label{app:shiny}
\section{Visualizing data with a Shiny application}

A Shiny application is available to explore the two main 
datasets in this paper, at
\url{http://shiny.bio.nyu.edu/users/dhm267/}. It
is also included as a separate zipped archive for local
installation and long-term archiving. 
To use the Shiny applications from the zipped archive: 

\begin{enumerate}
\item Download
\href{https://osf.io/ecyj9/}{\texttt{independent\_shiny\_archive.zip}}.
\item Unzip this archive.
\item Open \href{http://cran.r-project.org/}{R}.
\item Install the `shiny` and `tidyverse` packages by entering the command
  \begin{verbatim} install.packages(c("shiny","tidyverse")) \end{verbatim}
\item Enter the command: 
  \begin{verbatim} shiny::runApp("shiny",port=5000) \end{verbatim}
  where "shiny" is the path to the unziped folder and "5000" is 
  an arbitrarily selected port number.
\item Point your web browser at the URL \texttt{\url{127.0.0.1:5000}}
    and follow the instructions. The application has two tabs,
    one for the label-chase RNAseq and one for the BFF experiment.
\end{enumerate}

\end{appendixbox}

%%%
\appendix
%%%

\begin{appendixbox}
\label{app:codeanddata}
\section{Organization and availability of code and data}

\begin{flushleft}
The computer analysis code is available as a git
repository on Github: \break
\texttt{\url{https://github.com/darachm/millerBrandtGresham2018}} \break
The data are available as a set of `zip` format archives on OSF: \break
\texttt{\url{https://osf.io/7ybsh/}}
\end{flushleft}

To reproduce the entire analysis, or to access a particular 
analysis, clone the git repo. For example, on a 
Linux/Unix/MacOSX system install `git` and run:

\texttt{git clone https://github.com/darachm/millerBrandtGresham2018.git}

\noindent and change into that directory.
Then, download the `zip` data archives from the above
OSF link, and put them inside this git repo folder 
(here, `millerBrandtGresham2018`).
At minimum, you should have the `data.zip` archive in that directory,
although records of all R analyses are in `html\_reports.zip`
and intermediate files are in `tmp.zip`.

Consult the `README.md` file in the repository for more instructions
and options, including to unzip intermediate files and HTML
reports generated for every R script which detail the results.

