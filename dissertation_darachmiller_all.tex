\documentclass[12pt,letterpaper]{memoir}
\usepackage{graphicx}
\usepackage[margin=1in]{geometry}

\usepackage[utf8]{inputenc}
\usepackage{DejaVuSansCondensed}
%\usepackage{helvet}
\renewcommand\familydefault{\sfdefault} 
\usepackage[T1]{fontenc}

\usepackage{wrapfig}
\usepackage{framed}
%\usepackage{subfigure}

\usepackage{pdfpages}

\usepackage{float}
\let\origfigure\figure
\let\endorigfigure\endfigure
\renewenvironment{figure}[1][2] {
  \expandafter\origfigure\expandafter[H]
  } { \endorigfigure }

\usepackage{amsmath}

\usepackage{tikz}
\usepackage{pgfmath}
%\usepackage{pgffor}
\usetikzlibrary{arrows,positioning,shapes,patterns,decorations} 
\tikzstyle{every picture}+=[remember picture]

\usepackage[backend=bibtex8%biber%bibtex8
  ,backref=true,hyperref=true
  ,refsection=none %  ,refsection=chapter
  ,maxcitenames=2
  ,style=authoryear-comp]{biblatex}
\addbibresource{references_dissertation_darachmiller.bib}

\usepackage{hyperref}
\hypersetup{colorlinks, citecolor=blue
  , filecolor=black, linkcolor=blue, urlcolor=black }

\makeatletter 
\def\makemytitle{
  \newpage
\begin{center}
\vfill
  \@title\\
\vfill
  by\\
\vfill
  \@author\\
\vfill
  A dissertation submitted in partial fulfillment \\
  of the requirements for the degree of \\
  Doctor of Philosophy \\
  Department of Biology \\
  New York University \\
\vfill
  \monthz, \yearz\\
  \end{center}
\vfill
\begin{flushright}
  $\rule{5cm}{0.01mm}$\\
  David Gresham
\end{flushright}
  \pagebreak
\begin{center}
  \textcopyright \hspace{0.2em} \@author \\
  All rights reserved, \yearz \\
\end{center}
  \pagebreak
}
\makeatother

\title{
The extent of mRNA destabilization and 
genetic factors of rapid \textit{GAP1} repression 
upon a nitrogen upshift.
}
\author{Darach Miller}
\def\monthz{May}
\def\yearz{2018}

%
%
%
%
%
%

% Memoir layout options
\setulmargins{1.0in}{*}{*}
\setheadfoot{15pt}{0.75in}
\settypeblocksize{8.25in}{*}{*}
\setlrmarginsandblock{1.0in}{1.0in}{*}
%
\openright
\chapterstyle{dash} %dash, dowding, ntglike, thatcher
\setlength\beforechapskip{0pt}%-\baselineskip}
\aliaspagestyle{cleared}{plain}
\aliaspagestyle{chapter}{plain}
%
\setlength\floatsep{0em}
\setlength\textfloatsep{0em}
\setlength\intextsep{0em}
\renewcommand{\topfraction}{0.90}
\renewcommand{\bottomfraction}{0.9}
\renewcommand{\textfraction}{0.10}
%
\setbeforesecskip{-1.0em plus -0em minus -0em}
\setbeforesubsecskip{-1.0em plus -0em minus -0em}
\setbeforesubsubsecskip{-1.0em plus -0em minus -0em}
\setaftersecskip{1em}
\setaftersubsecskip{1em}
\setaftersubsubsecskip{1em}
\setsecheadstyle{\large\bfseries\raggedright}
\setsubsecheadstyle{\normalsize\bfseries\raggedright}
\setsubsubsecheadstyle{\normalsize\bfseries\raggedright}
%
\abnormalparskip{0em}
%
\checkandfixthelayout

%\newcommand{\openrule}{
%  \par\noindent\rule{.5\textwidth}{1pt}\par\vskip-0.2em
%  \noindent\rule[\baselineskip]{1pt}{\baselineskip}}
%
%\newcommand{\closerule}{
%  \par\hfill\rule[-\baselineskip]{1pt}{\baselineskip}
%  \par\vskip-0.2em\hfill\rule{.5\textwidth}{1pt}\par}

\newcommand{\afig}[3]{
  \begin{figure}[t]
%    \vspace{-10pt}
    \begin{center}
    #1
    \end{center}
    \vspace{-10pt}
    \caption[#3]{\textbf{#3} #2}
%    \vspace{-10pt}
  \end{figure}
  }

%
\includeonly{
  dissertation_darachmiller_1,
  dissertation_darachmiller_2,
  dissertation_darachmiller_3,
  dissertation_darachmiller_4,
  dissertation_darachmiller_5
}


% hax to get the compiler to shut up, ripped from SO
\hfuzz=40pt
\vfuzz=40pt
\hbadness=5000
\vbadness=\maxdimen

\setsecnumdepth{subsubsection}
\settocdepth{subsection}

\begin{document}
\DoubleSpacing
\frontmatter
\pagestyle{empty}
%\renewcommand{\thefigure}{\arabic{figure}}

\makemytitle

\iftrue
\begin{quote}
\SingleSpace
"Science is a match that a person has just got alight. 
They thought they were in a room --- 
in moments of devotion, a temple --- 
and that this light would be reflected from and display walls 
inscribed with wonderful secrets and pillars carved with 
philosophical systems wrought into harmony. 
\vfill
It is a curious sensation, now that the preliminary splutter is 
over and the flame burns up clear, to see lit just their hands and 
just a glimpse of themselves and the patch they stand on visible, and 
around them, in place of all that comfort and beauty they
anticipated, \vspace{0.5em} darkness still."

\hfill - H.G. Wells, 1891, \\\vspace{1em}\hfill adapted for 2018
\end{quote}
\fi

\newpage

\section*{Acknowledgements}
\addcontentsline{toc}{section}{\numberline{}Acknowledgements}

\label{section:acknow}

\iffalse

A sailboat is means of transit. 
The aim of the method is to use lines and fabric and structure to
juxtapose a variety of forces against each other, 
and thus give them the only option of causing progress. 
Without the right constraints and re-direction, 
the system is a mess, destined for the rocks.
A sailboat needs the wind as much as it needs the water,
for without a good keel there's little for the winds to push off of.
Without wind there's little action, and a tiller's important for
deciding where the action should proceed.
Many human endeavors can be described with 
this metaphor --- a variety of forces, pressures, opportunities, 
all swirling around to make something flow in a particular
direction. When all the components align with and against each
other harmoniously, then the effect is made. Just a rope does not make
much progress upwind, and it depends on the cleats and pulleys and
sails and so forth to give it purpose and allow it to play any role in
effecting progress.
 
I would like to acknowledge the system of people that has made 
progress possible.

The culture built over these many years at NYU Biology is a unique 
moment of growth, and these informal linkages play a surprisingly
important role in facilitating the collaboration and support
necessary.As Benjy Neymotin 
pointed out several years ago: "it takes a village".

Specifically,
I thank the kind folks on the 4th floor of CGSB, namely
the research group led by Christine Vogel. I am glad we moved to share
space and scientific discussions.
I also thank Ken Birnbaum and his group for helpful conversations and
access to the lucky FISH incubator.
Across the street, I thank Viji Subramanian, Andreas Hochwagen, 
and his research group for microscope access and tolerating my
questions.
I thank Andres Mansisidor and Matt Paul for helpful reading and 
comments on the paper, and otherwise.

Away from NYU, I thank
Evelina Tutucci for demonstrating the mRNA FISH method to me,
helping me understand just how finicky it can be.
I thank Megan McClean for openly sharing a FISH protocol on
OpenWetWare (although we've never talked).
On the west coast (and now elsewhere) I thank Carole Hom, Ian Korf, 
and Patrice Kohel at UC Davis, Nitin Baliga Aaron Brooks at ISB
Seattle, but most importantly Marc Facciotti at UC Davis for 
giving me a chance at getting into science late.

I want to thank Sarah Nguyen for keeping me together, like 

I thank Michi, Henri(ci), and Tai for making me get up and come 
home on time.


I thank Sue Miller for devising a clever genetic screen for yeast
strains that fail to re-initiate growth programs, as well as for
being so supportive through the years.
I also thank my Kendra Miller for inspiring and egging me on, 
a knowingly efficacious combination of support for folks like me.
I thank Harry Miller for support and inspiring me to wander this way,
and to never stop learning new things.

I thank staff at eBioscience/Affymetrix/ThermoFisher for support with
the Quantigene/FlowRNA probe set and patience through the years.

I thank the Cold Spring Harbor Yeast Course, especially the instructors
Maitreya Dunham, Marc Gartenberg, and Grant Brown for introducing
me to the community and (re)-lighting a fire to help me finish.
That course is the reason I'm staying in science as long as I can.

I thank past students including Daniel, Alex, and Stephen.

My committee for the most fun three hours I had each year.

Past and present members of the Gresham and Vogel labs for 
discussions and support.
Especially Niki, Steff, Nathan, and Benjy; Naomi, Farah, Siyu,
Pieter.

To Rodoniki Athansidou for steeling me against the alluring
and corrupting temptation of Kits, and Terrance Cooper for sharing
with me the wisdom of when Kits are necessary for a field to
progress. Threading the gradient between these two extremes seems to
contain some degree of powerful wisdom, something about life and
practice and cargo cults and the very nature of scientific
investigations and Darwinian versus Lamarkian biology, but I have 
not yet had the time to condense it into a sharable form (or a Kit).


the NYU Genomics Core facility for sequencing, flow cytometry,

\fi

\newpage
\pagestyle{plain}

\section*{Abstract}
\addcontentsline{toc}{section}{\numberline{}Abstract}

Cellular responses to changing environments frequently involve rapid
reprogramming of the transcriptome. Regulated
changes in mRNA degradation rates can accelerate reprogramming by
clearing or stabilizing extant transcripts. 
During a nitrogen upshift, the budding yeast \textit{Saccharomyces
cerevisiae} reprograms its physiology in order to enter into  
a state of rapid growth. 
I sought to understand the role of transcriptome reprogramming
in effecting this transition.
During the first minutes of a nitrogen upshift,
preceding the changes in population growth rate,
the transcriptome quickly into a state transiently distinct from
either rapid or slow growth.
During this, the five fastest decreasing mRNA are all NCR-regulated 
transporters, suggesting that this reprogramming is specifically
targeted to set the transcriptional program for rapid growth.
Many transcripts decrease significantly faster than expected from
their stability measured during steady-state growth during
rich media conditions, so I set out to determine the stability of
the yeast transcriptome preceding and during a nitrogen upshift.
I used 4-thiouracil labeling to track mRNA stability, and found 
that 78 mRNAs are subject to destabilization. 
These transcripts include
Nitrogen Catabolite Repression (NCR) and carbon metabolism mRNAs,
suggesting that mRNA destabilization is a mechanism for targeted
reprogramming. 
To explore the molecular basis of destabilization and possible
mechanisms of specifying or effecting the destabilization, I
developed a novel method combining mRNA FISH, fluorescence-activated
cell sorting, and DNA barcode sequencing to screen the pooled deletion
collection library for \textit{trans} factors that mediate rapid GAP1 mRNA
repression. This required that I completely redesign the methods for 
low-input multiplexed barcode sequencing, as well as adapt 
branched-DNA single-molecule mRNA FISH protocols to work in budding
yeast. Modeling of the data identifies known factors of mRNA
degradation, namely all three non-essential components of the
Lsm1-7p/Pat1p complex, as being important for proper \textit{GAP1}
mRNA dynamics. However, the modulators \textit{EDC3} and \textit{SCD6}
have more complex phenotypes. Re-analyzing previously collected data,
I identified that a \textit{scd6}$\Delta$, \textit{tif4632}$\Delta$
(eIF4G2 knock-out), and a \textit{GAP1} 5' UTR delete strain all
share a similar phenotype of lower \textit{GAP1} expression preceding
the upshift, and reduced decay upon the upshift. This suggests that 
translation initiation dynamics may play a role in mRNA levels and 
their stability, and suggests that the destabilization phenomenon 
may in fact be the release of a stabilization effect.

\newpage

{\SingleSpacing
\tableofcontents}

\listoffigures

\mainmatter

\chapter{Introduction}

Organisms adapt to
their environment by expressing different phenotypes as environments
change. We expect this to be an advantageous strategy, depending on
the various parameters of fitness advantages of each phenotype in each
environment balanced against the costs of innovating and maintaining
the machinery for adaptive differential gene expression
\citep{kussell2005phenotypic}. 
If different mechanisms have different properties of
efficacy and energetic costs, and if the phenotype to fitness
relationship varies, then different conditions may select
for the use of different mechanisms of achieving appropriately
regulated gene expression.
Thus, understanding the adaptive basis of the diversity of 
regulatory mechanisms may inform our understanding of selection 
on complex systems.

\section{Regulated gene expression} 

Expression of a protein-coding gene
product involves many complex steps, each with their own opportunities
for a variety of regulatory mechanisms. At the outset, DNA sequences
that encode genetic elements are transcribed into corresponding chains
of messenger RNA (mRNA). The rate of this transcription is helped by
factors that facilitate recruitment of RNA polymerase II (RNA Pol II)
and is hindered by factors that block this process by physical
occlusion or changes in the accessibility to the chromatin 
\citep{hahn2011transcriptional}. 
The translation of mRNA into a protein product by
ribosomes occurs at different rate for different genes and different
environments, regulated by a complex interplay between ribosomes and
associated translation factors, RNA binding proteins (RBPs), and
intrinsic factors of the mRNA like length or codon-usage 
\citep{dever2016mechanism}. 
For both the mRNA and its protein product, stability is also
important 
\textbf{(cite something about protein stability)} 
\citep{perez2013eukaryotic}. 
Additionally, localization or allosteric regulation can change
the activity of a gene product. Myriad factors contribute to the
expression of a gene product, and determining the functional adaptive
basis for particular regulatory mechanisms, if they are indeed
adaptive, would help us better understand the diversity of gene
regulatory mechanisms.

Functional transcriptome reprogramming during a
nitrogen upshift The budding yeast \textit{Saccharomyces cerevisiae}
is a
classically studied model system for many fields, including gene
expression regulation. In response to different nutrient
availabilities, yeast will change gene expression programs on the
whole transcriptome and adopt different rates of growth. The
particular nutrient environment is described in terms of the quality
and quantity of the nutrient provided. Quantity refers to just that,
the molar availability of the nutrient that the yeast can take up,
while quality is more of a empirical reference to how rapidly budding
yeast can biochemically incorporate the nutrient into their metabolism
--- ie rate of growth. One prediction from this understanding is that
altering the quantity of the nutrient availability to vary growth
rates in a range below which the quality limits growth rates will
elicit a common response between various nutrient limitations.
Indeed, studies systematically varying nutrient environments have
shown that about a quarter of the transcriptome is differentially
expressed at different steady-state nutrient-limited growth states,
regardless of nutrient used to limit growth 
\citep{brauer2008coordination,regenberg2006growth}. 
Statistical modeling of this process
determined that the molecular signature of this growth-rate signalling
could be capture in a small number of “calibrator” genes whose
expression was very well correlated with changes in growth rate or
perturbation of signalling pathways associated with this process, and
importantly also changed during dynamic transitions or upon
perturbation of growth signalling pathway PKA 
\citep{airoldi2009predicting}.
Dynamic transitions to better nutrient environments (nitrogen, carbon,
and phosphorus upshifts) shared a similar pattern 
\citep{conway2012glucose}, 
and the pattern of gene expression associated with increased
nutrient availability and growth rates is the opposite of the
Environmental Stress Response (ESR) --- a shared in co-regulation of
~600 mRNA across dynamic responses to various stressors 
\citep{gasch2000genomic}. 
Together, this shows how yeast has a common response that
largely corresponds to the suitability of the sensed sensed
environment. A better environment transduces to faster growth with
more growth associated mRNA and less stress response mRNA, and this
holds true in steady-state differences and dynamic transitions.  

One
classically studied transition between growth rates is the nitrogen
upshift. Yeast grows quickly when provided with nitrogen sources like
glutamine or ammonium sulfate, but can make use of various
less-preferred nitrogen sources like proline or urea by expressing
overlapping sets of specific and general nitrogen-source permeases
that concentrate these sources inside the cell for use.  

Various
nitrogen sources are then catabolized to eventually make glutamate and
glutamine, with an estimated ~85\% of macromolecular nitrogen coming
from the amino nitrogen in glutamate and the rest from the amide group
of glutamine 
\citep{magasanik2002nitrogen}. The addition of glutamine to
a nitrogen-limited culture, for example grown with only the
non-preferred proline as a nitrogen source, is called an upshift
because it is the change from a slow growing condition to one of rapid
growth. Through the use of a temperature-sensitive glutamine synthase
allele or treatment with methionine sulfoximine, it has been shown
that NCR is a response to intracellular glutamine availability
\textbf{(cites)}
. Upon an upshift, a regulatory phenomenon called nitrogen
catabolite repression (NCR) ensures that the set of NCR transporters,
metabolic enzymes, and regulatory factors are repressed.  

One layer of
the repression occurs at the level of transcript synthesis. Four of
the five GATA factors in yeast coordinate to control transcription of
NCR genes, with two factors (Gln3p and Gat1p) activating transcription
while two (Gzf3p and Dal80p) repress transcription 
\citep{hahn2011transcriptional,stanbrough1995transcriptional}. 
These factors are also subject to NCR
control to different extents, with the activators increasing the
expression of the repressive factors. This is thought to be an
adaptation to enable quick repression upon a nitrogen upshift, as may
be encountered when yeast is introduced to a new abundant nutrient
environment of grape or wort 
\textbf{(cite)}
. It has been long known that the
eukaryotic growth signalling pathway TORC1 largely regulates these
factors by controlling the activity of phosphatases and thus
localization of these transcription factors, via Ure2p for Gln3p 
\citep{beck1999tor} 
and unknown mechanisms for Gat1p
\textbf{(citez, also classic ure2 papz)}
. However, careful genetics over the last two decades in
Terrance Cooper’s laboratory has identified that the genetic
requirements for phenotypes differ in different environments, with
comparisons of nitrogen “starvation” (8+ hours) versus “limitation”
(<3 hours, or proline) showing that a large part of NCR regulation was
still unexplained by TORC1 signalling 
\citep{tate2013five}. By way
of a temperature-sensitive tRNA allele, they have since identified
that Gcn2p impinges in a parallel pathway through the 14-3-3 proteins
Bmh1/2 to promote the export of Gln3p and Gat1p 
\citep{tate2015gata,tate2015nitrogen,tate2017general}. 
Additionally, others have suggested that the
amino-acid permease Gap1p may directly signal to PKA, but more has to
be done 
(theivelen eh)
. Thus, multiple signalling pathways converge to
affect the import and export of NCR GATA factors to effect multiply
redundant layers of NCR transcript synthesis control.  

Gene product
regulation can also occur post-translationally. NCR has primarily
referred to the control of transcript synthesis rates, but it has been
long observed that upon addition of a preferred nitrogen source the
enzymatic and permease activities are repressed faster than can be
caused by a shut-off of synthesis 
\citep{cooper1983function}. A
classical NCR-regulated gene is the general amino-acid permease GAP1.
GAP1 mRNA is repressed much faster than the repression of the
protein-product 
\citep{stanbrough1995transcriptional}
, and we know that this
Gap1p shut-off is adaptive \citep{risinger2006activity}, 
perhaps due to an
excess of amino-acid transport causing ammonia toxicity 
\citep{hess2006ammonium}
or excess proton symport driving a depolarization against futile
Pma1p proton-export activity. This growth phenotype allowed the early
identification of mutants in this process, and this indicates that it
is mediated by a uniquitinyation mark that inactivates the permease
and leads to relocalization and degradation 
\textbf{(more risinger? (Risinger and Kaiser 2008) earlier? smething
from magasanik?)}
. Thus multiple layers redundantly repress
the NCR-regulated Gap1p.  

In Chapters 2 and 3, I show how mRNA
degradation also plays a role in this repression.  

\section{mRNA degradation and its regulation}

Even simply considering the regulation of mRNA
abundance, there are at least two processes that contribute --- that
of synthesis and degradation. We know much about transcript synthesis,
perhaps owing to the fact that virtually all events of mRNA synthesis
pass through a well-characterized reaction of synthesis by RNA Pol II,
capping and polyadenylation, and export into the cytoplasm. The
details may vary, but the common pathway is the same. Conversely, mRNA
degradation does have a main pathway that performs the bulk of mRNA
degradation but mRNA are also subject to divergent redundant pathways
that have been challenging to measure. Moreover, the rates of this
various process are subject to control in ways less well-understood.
While some similarity is thought to exist in how RBPs may recognize
cis-element sequences in RNA similar to how TFs recognize upstream
activating or repressing sequences in DNA, the single-stranded
nature[d] of mRNA complicates this process by blocking linear
cis-elements (Li et al. 2010). Additionally, these secondary
structures of RNA may be recognized as the cis-element, complicating
our approaches to recognize these patterns (Goodarzi et al. 2012).
Primary pathways of mRNA degradation The canonical protein-coding mRNA
is synthesized in the nucleus from a DNA template by RNA Pol II, and
is capped co-transcriptionally at the 5’ end with a m7G cap. As Pol II
transcribes sequence 3’ of the stop codon the cleavage factor complex
recognizes cis element binding motifs in the RNA to direct cleavage
and polyadenylation to specific sites in the mRNA. Upon successful
completion of this process, the nascent mRNA is exported to the
cytoplasm where it enters into the pool of translatable mRNA.
Typically, translation begins when initiation factors load ribosomal
subunits to scan the 5’ leader or untranslated region (UTR) for the
start codon where the process of coding sequence translation begins.
These initiation factors (eIF4F) bind the m7G cap to load ribosome
subunits (Dever et al. 2016), and thus most translation depends on the
cap ( with an exception demonstrated by internal ribosome entry sites
(Gilbert et al. 2007). The m7G cap is also critical for mRNA
stability. Xrn1p is a highly-processive combination of helicase and
exonucleolytic domains that as a single protein rapidly degrades
transcripts from a 5’ to 3’ end, recognizing unprotected 5’
phosphorylated ribonucleotides as substrates (Parker 2012). Thus, the
inverted linkage of the m7G escapes degradation.  

During rounds of
translation the poly-adenosine tail is shortened from about 65-90
adenosines to about 10 adenosines by a combination of the Pan2/3 and
Ccr4/Pop2 deadenylase complexes, with activity antagonized by the
poly-A binding protein Pab1p (Parker 2012; Decker and Parker 1993).
When the tail is thus shortened to ~10 adenosines, the Lsm1-7p/Pat1p
complex binds the remainder of the poly-A tail (Tharun et al. 2000).
This complex is a heptameric ring of the Lsm1-7 proteins with the
Lsm1p’s C-terminal domain elegantly spanning the center (Sharif and
Conti 2013) and the last eight residues projecting into this center
and critical for binding the shortened poly-A tail (Chowdhury et al.
2016). The critical function of this complex is to recruit and promote
activity of the decapping complex to the 5’ end of the mRNA, and in
cooperation with Pat1p (Chowdhury et al. 2014) the binding of this
complex to mRNA and to decapping factors is indeed correlated with
decapping of the mRNA (Chowdhury and Tharun 2009). Thus, the complex
maps the deadenylated status to the next step in mRNA degradation.  

A
cytoplasmic mRNA without a 5’ m7G cap is not long lived, by virtue of
Xrn1p, thus the recruitment and activation of the decapping complex is
thought to be the key regulatory step in rates of mRNA degradation
(Coller and Parker 2004). Dcp2p carries out the catalytic activity of
the holoenzyme but is promoted by the effects of Dcp1p, and in
comparing in vitro to functional in vivo assays of mutants it appears
that the catalytic rate of the enzyme is not the limiting step (Tharun
and Parker 1999). Rather it is re-modeling of the mRNP complex that
leads to association of the decapping enzyme complex with the 5’ cap,
and the rate of this process determines the activity of this
degradation pathway (Tharun and Parker 2001). This
decapping-enzyme-localization process is promoted by the Lsm1-7p/Pat1p
complex and inhibited by poly-A binding protein Pab1p (perhaps by
competition of Pab1p with Lsm1-7p/Pat1p for the poly-A tail) and eIF4E
(Coller and Parker 2004; Caponigro and Parker 1996). eIF4E, eIF4G, and
eIF4A comprise the cap-dependent translation-initiation factor eIF4F
(citez), which gives rise to an elegant model for the observed effect
of translation initiation inhibiting decapping by competition for the
5’ m7G cap (Huch and Nissan 2014), while the requirement of sufficient
poly-A tail for Pab1p to bind is consistent with the effect of
deadenylation in promoting deapping (Parker 2012). The Lsm1-7p/Pat1p
complex and Dcp2p/Dcp1p are physically associated (by
co-immunoprecipitation) with several factors that genetically modulate
the activity of remodeling step ---Dhh1p, Edc3p, and Scd6p (Nissan et
al. 2010).  

Dhh1p is a helicase that associates with polysomes (mRNA
with multiple ribosomes bound) and has been recently demonstrated to
be genetically required for the relationship between codon optimality
and mRNA stability (Radhakrishnan et al. 2016; Presnyak et al. 2015;
Sweet et al. 2012). It has been demonstrated to play a critical
genetic role in mapping codon-optimality and translation speed back to
changes in mRNA stability. Curiously, tethering Dhh1p to a 3’ UTR
using the MS2 system resulted in lower translation rate of an mRNA
despite causing more ribosomes to be associated with the mRNA (Sweet
et al. 2012), suggesting that Dhh1p resolves slowly translating
ribosomes by promoting decapping and mRNA degradation.  

Edc3p
physically associates with the decapping complex and stimulates its
activity (Nissan et al. 2010), but it has also been shown to be
important (Decker et al. 2007) but not critical (Huch et al. 2016; Rao
and Parker 2017) for the formation of processing-bodies. These are
microscopically visible foci of mRNA and degradation factors that form
in response to stress conditions but are assumed to be condensed from
mRNA-protein complexes that exist before stress (Sheth and Parker
2003; Lui et al. 2014; Rao and Parker 2017). Interestingly, the
canonical role of processing-bodies had been as a site of mRNA
degradation, but recent work with refined genetic tools demonstrated
that mRNA degradation takes place outside of these sites (Tutucci et
al. 2017). A mutant deleted of EDC3 and the C-terminal domain of the
essential LSM4 is deficient in processing-body formation with , and is
surprisingly this processing-body deficiency also correlates with a
deficiency in the stabilization of several mRNA upon osmotic stress
(Huch and Nissan 2017). Thus, the role of Edc3p may be to promote
interactions between these mRNPs (mRNA-protein complexes) in a manner
that affects how well mixed the degradation factors and targeted mRNAs
are.  

Scd6p inhibits the formation of the 48S pre-initation complex
(when the 40S subunit associates with eIF4E cap-dependent initiation
factors and begins to scan the 5’ UTR) via forming its own cap-binding
complex with eIF4G subunit eIF4G1 in an arginine-methylation-dependent
manner (Rajyaguru et al. 2012; Poornima et al. 2016). This is thought
to physically occlude the normal initiation complex from binding.
Scd6p also binds to several other factors in the Lsm1-7p/Pat1p
deadenylation-promoting complex (Nissan et al. 2010), and thus may
play an indirect role in preventing the pre-initiation complex from
binding and stabilizing the mRNA through translation.  

Other pathways
of mRNA degradation exist. If the poly-A tail is completely removed,
the cytoplasmic exosome complex can degrade the mRNA from 3’ to 5’.
This redundant mechanism allows a xrn1$\Delta$ mutant to grow, although
slowly, as the 5’ to 3’ pathway is throught to effect the bulk of mRNA
degradation (Parker 2012). The balance between the two may be because
of the enzymatic rate of digestion, but more likely because the
binding of the Lsm1-7 complex to the shortened poly-A tail protects
the mRNA from further deadenylation and thus represses this 3’ pathway
(cite Tharun on this).  

\section{Alternative pathways of mRNA degradation ---
quality control and other }

Three other pathways are known to act as a
co-translational layer of quality control, where errors detected by
abnormal translation processes result in destruction of the presumably
defective mRNA. Nonsense-Mediated Decay (NMD) is the canonical example
of this. A mutation, transcriptional error, alternative splicing
event, or abnormal translational event (like leaky scanning or a uORF,
explained later) can cause an mRNA to have a stop codon well before
the usual position, which is recognized for destruction by a
deadenylation-independent decapping and 5’->3’ decay (Muhlrad and
Parker 1994). How the aberrant nature of the misplaced stop codon is
detected is still a mystery, but NMD sensitivity is known to be
maximally active after Xaa have been translation and this effect
reduces to minimal activity towards the 3’ end of the transcript
(Losson and Lacroute 1979). Non-Stop Decay refers to the inverse of
NMD, no stop codon. Ribosomes that over-run into the poly-A tail
recruit degradation by the 3’ exonuclease (exo review). No-go Decay is
named after the phenomenon that triggers it, when ribosomes no go.
Difficult to translate sequences (lysine repeats) trigger the
endonucleolytic cleavage of the offending mRNA, which is then degraded
from the cut towards both ends (Doma and Parker 2006). Very recently,
it has been shown that it is likely the ribosome collisions that
promote the ribosome ubiquitination associated with the triggering of
No-Go decay (Simms et al. 2017), and other work has suggested that the
protein Asc1p may play a key role in this process (Ikeuchi and Inada
2016), perhaps via K63 ubiquitination (Saito et al. 2015). Together
these pathways surveil translating mRNAs for defects, but it is likely
that false positives in the recognition process also contribute to
their regulatory effects.  

Disruption of the NMD pathway is associated
with different expression of many transcripts. Recent genome-wide
analysis identifying ~900 mRNA upregulated upon deletion of any of
UPF1-3, and subsequent ribosome profiling found this targeting to be
associated with out-of-frame translation effects and non-optimal
codons (Celik et al. 2017). NMD has been implicated in the regulation
of ribosomal protein subunit pre-mRNA (Garre et al. 2013) in different
environmental conditions, has been shown to interact genetically with
Hrp1p and cis-elements spanning the start codon of PPR1 mRNA to target
this mRNA for degradation (Pierrat et al. 1993; Kebaara et al. 2003),
and may be triggered by upstream open reading frames (uORFs, discussed
later). These could be specific regulation, or aberrant probabilistic
activation due to the sensitivity of co-translational quality control
(Celik et al. 2017).  

% Chen 1998 hrp1 was figured using adh2 !!!!

The interaction of translation and mRNA
degradation Codon-optimality refers to the concept that certain codons
are translated by the ribosome more quickly than other codons. This is
thought to result in part from changes in tRNA abundance and in part
due to intrinsic differences in the decoding rates (Curran and Yarus
1989; Thomas et al. 1988), and often quantified using the tAI index
(dos Reis et al. 2004). The expectation that tRNA availability is
associated with increased rates of translation has been tested with
more recent ribosome footprint profiling experiments, and consistent
with this ribosomes tend to occupy optimal codons less often (Weinberg
et al. 2016).  

The functional relationship between codon-optimality
and mRNA degradation rate had been considered and rejected by a review
of single-transcript studies (Caponigro and Parker 1996). However,
with the advent of accurate genome-wide measurements of mRNA
degradation rates, we are able to explore the generality of this
principle in a relatively unbiased way. Several groups (Presnyak et
al. 2015; Neymotin et al. 2016; Harigaya and Parker 2016; Cheng et al.
2017) have found that poor codon-optimality and lower ribosome density
is associated with a higher degradation rate when considered on a
per-transcript basis. This can be explained through multiple models.
One model is that translation elongation rates are sensed, with slower
elongation accelerating the degradation of mRNA. Jeff Coller’s group
has worked extensively on Dhh1p, and found that it is genetically
required for the clear relationship between codon-optimality and mRNA
stability (Radhakrishnan et al. 2016; Presnyak et al. 2015). Although
the mechanism is at this point unclear, Dhh1p’s genetic association is
a fruitful hub to work out from.  

Alternatively, competition between
decapping enzymes and translation initiation factors for access to the
5’ m7G cap has long been proposed as a mechanism by which the two
processes interact (Schwartz and Parker 1999; Schwartz and Parker
2000). Karsten Weis’ group (Chan et al. 2017) reproduced the result
that slowing elongation with cycloheximide, sordarin, or 3AT treatment
slows mRNA degradation, but conversely inhibition of initiation with
hippuristanol or a dominant negative eIF4E increased degradation
rates. These measurements were made on the whole-transcriptome using
4-thiouracil and RNA sequencing, similar to RATEseq (Neymotin et al.
2014). The connection between the effect of elongation rates and
initiation rates could be explained by the known effect of slow
elongation rates inhibiting initiation events, as predicted (Shah et
al. 2013) and measured (Chu et al. 2014).  

Thus, much evidence points
to competition between translation initiation and 5’ to 3’ degradation
initiation at the cap as a major determinant of mRNA stability,
although the molecular work with Dhh1p suggests that events after
initiation still play a role. Other mRNA degradation pathways like NMD
or NGD during elongation (as discussed earlier) could also possibly
contribute to the effect.  

Regulation of mRNA degradation mRNA
degradation can be affected by various trans factors. While micro RNAs
are prolific in regulating mRNA in animals and plants, budding yeast
do not make use of this mechanism. Instead, in yeast mRNA degradation
appears to be determined by a combination of intrinsic properties like
length or codon-optimality, and trans factor RNA binding proteins
(RBPs) that can bind cis element sequences in the mRNA sequence to
effect changes in stability (cite? morris, cheng2017). The best
example of this is Puf3p, which binds motifs in mRNA with products
destined for mitochondrial function and degrades these in the
appropriate environment (Olivas 2000; Miller et al. 2013), perhaps by
mapping phosphorylation of Puf3p to association of these mRNA to
cytoplasmic granules (Lee and Tu 2015). Secondary structures may
complicate the recognition of linear cis elements, or be used as cis
elements in their own right (Li et al. 2010), for example Vts1p (Smaug
homolog) recognizes a small sequence motif in the context of the loop
of a stem-loop hairpin (She et al. 2017; Aviv et al. 2003).
Degradation rates can be affected by many mechanisms. Elements in
promoters (cis when in DNA but not part of the affected mRNA) can
“mark” transcripts for differential stability (Haimovich et al. 2013).
One of the most well known examples of this is Dbf2p loading onto SWI5
and CLB2 mRNA to effect destabilization upon mitosis (Trcek et al.
2011). In a direct example, the transcription activation domain of
Adr1p fused to a different DNA binding domain has been shown to be
sufficient to mark ADH2 mRNA for destabilization upon a glucose
upshift (Braun et al. 2015). Thus, RBPs may recognize sequence
elements in the mRNA or be loaded onto messenger ribonucleo-protein
complexes (RNPs) at synthesis (Gupta et al. 2016) to effect control of
mRNA stability in response to events in the cytoplasm.  

Non-RBP
mechanisms can also be used. Upstream open reading frames (uORFs) were
originally characterized using the phenotype of post-transcriptional
regulation of the Gcn2-regulated Gcn4p (Dever et al. 1992) in part
through quality control pathways (Ruiz-Echevarria and Peltz 1996).
Canonical and non-canonical start-codons can recruit initiation of
scanning ribosome subunits with a variety of effects on the
translation of the main coding sequence and the mRNA stability
(Spealman et al. 2017). Ribosomes may skip re-initiation at the
primary start codon to generate N-terminal diversity by initiating at
alternative start codons, or upon termination of a uORF very distant
from the 3’ end of the transcript trigger the NMD pathway to destroy
the mRNA (Dever et al. 2016). Additionally, modification of
nucleotides such as m6A can play a regulatory effect (citez).  

mRNA
localization within the cytoplasm may affect degradation by virtue of
regulating the accessibility of degradation factors. Processing-bodies
were originally described as cytoplasmic co-localized foci of 5’->3’
degradation factors that formed under stress induction conditions, and
on the basis of steady-state genetics and experiments with an MS2
aptamer-based live-imaging system, it was concluded that
processing-bodies are foci of active mRNA degradation (Sheth and
Parker 2003). These foci are usually studied by microscopy during
stresses, entry into stationary phase, and in the use of mutants in
degradation pathways, but recent advances in microscopy and nanoscopy
particle tracking have identified that these complexes are likely
condensations of previously-existing RNPs and depend on a network of
redundant interactions between mRNA 5’ to 3’ degradation protein
factors (Lui et al. 2014; Rao and Parker 2017). Additionally, recent
adjustments to the aforementioned MS2 aptamer system and explorations
during dynamic conditions point towards processing-bodies being sites
of degradation factor sequestration (Huch and Nissan 2017; Tutucci et
al. 2017). Interestingly, the formation of these processing-bodies are
halted upon cycloheximide treatment (Sheth and Parker 2003),
suggesting that translational status of the transcriptome and
processing-body composition may be related. In recent work, inhibition
of translation initiation was demonstrated to increase p-body
formation in correlation with increased degradation rates (Chan et al.
2017). Together, these observations indicate that processing-bodies
result from a complex balance of mRNA degradation initiation,
resolution, and mRNA degradation factor interactions with impacts on
the accessibility of degradation factors to mRNA targets of
degradation.  

The role of stability control in transcriptome
reprogramming The change in concentration of an mRNA ($R_t$) depends
on the rates of mature transcript synthesis ($k_s$) and mRNA
degradation ($k_d$). We assume that the cell volume is fixed, that
synthesis is a constant rate dependent on the unchanging concentration
of the DNA encoding the gene, and that degradation is a random first
order process of the mRNA interacting with a fixed and unsaturated
factor degradation. Thus, the change in mRNA over time is $$
\frac{dR_t}{dt} = k_s - R_t k_d$$ From this, the two rates determine
the steady-state equilibrium of $\frac{k_s}{k_d}$. Given a change in
these rates, a faster mRNA degradation rate will approach or relax to
the new equilibrium value quicker (Hargrove and Schmidt 1989), as the
doubling time (or half-life) of the mRNA is dependent on only the
degradation rate $\frac{log(2)}{k_d}$.  

While both synthesis and
degradation contribute to changes in abundance, changes in degradation
rates can cause the changes to occur more rapidly. If we expect that
the existence of a mechanism implies a selective pressure specifically
for it (Gould and Lewontin 1979), then we would expect that studying
an example of a transcript subject to both synthesis and degradation
regulation might reveal a balance of selection during steady-state and
dynamic conditions. 

\section{Stress conditions trigger rapid regulation of mRNA stability}

mRNA degradation rate changes have been characterized
to play a role in responses to heat-shock, osmotic stress, pH
increases, and oxidative stress, sharing a similar program of
destabilization of mRNA coding for ribosomal-biogenesis gene products
and stabilization of stress-responsive mRNA (Canadell et al. 2015;
Molina-Navarro et al. 2008; Shalem et al. 2011; Romero-Santacreu et
al. 2009; Molin et al. 2009; Castells-Roca et al. 2011; Miller et al.
2011; Garre et al. 2013). Simultaneous increases in both synthesis and
degradation rates of some of these mRNA are thought to serve to return
the transcriptome quickly to a new steady-state after effecting a
transient pulse of regulation (Shalem et al. 2008; Rabani et al.
2011), demonstrating a key functional role in stability control in
achieving a particular pattern of mRNA dynamics. Interestingly, these
stability changes appear to be a singular regulatory event
(Pérez-Ortín et al. 2013). Glucose deprivation stress also impacts
mRNA stability (Munchel et al. 2011). 

\subsection{Nutrient shifts also trigger mRNA stability changes }

In response to a carbon-source downshift
(glucose-grown cells resuspended in media with only galactose
available), functionally important regulatory changes in mRNA
stability occur (Munchel et al. 2011). Ribosome biogenesis associated
mRNAs are are destabilized, an effect that can be phenocopied by the
addition of rapamycin (inhibitor of the central growth signalling
TORC1 pathway) (Munchel et al. 2011). Conversely, a carbon source
upshift (galactose to glucose) triggers a destabilization of inducible
GAL genes, an effect that appeared to be restricted to the dynamic
condition as mRNA transgenically overexpressed in glucose media were
stable (Munchel et al. 2011).  

Global changes in transcription and
mRNA destabilization has been observed before (Jona et al. 2000), and
recently systematically measured to be correlated with changes in
growth rate (García-Martínez et al. 2016). The involvement of the
TORC1 pathway in this process has identified that its effect is
specified to differentially regulate the stability of certain
transcript sets (Albig and Decker 2001; Talarek et al. 2010).
Recently, a phosphoproteomics approach to studying signallng of the
AMPK homolog Snf1p during a carbon upshift identified a role in Xrn1p
phosphorylation in the specification by this factor (Braun et al.
2014). Thus, specific signalling pathways appear to effect large
changes in mRNA stability in response to different nutrient conditions
for growth. 

Relieving nutrient limitation with a glucose upshift has
been shown to mediate both stabilization of mRNA in the ribosomal
protein subunit regulon (Yin et al. 2003) and destabilization of
gluconeogenic transcripts (de la Cruz et al. 2002; Mercado et al.
1994; Scheffler et al. 1998; Lombardo et al. 1992). Mapping the
determinants of this effect has been met with mixed success. The
destabilization of SDH2 and GAL1 mRNA have been mapped to elements in
the 5’ UTR (Bennett et al. 2008) with destabilization of GAL1 being
associated with a growth advantage in switching carbon-source
environments (Baumgartner et al. 2011). JEN1 has been demonstrated to
be destabilized upon glucose addition, and this has been mapped to cis
elements of different transcription start sites which can act in trans
to regulate other co-expressed engineered alleles of JEN1 through
unknown mechanisms (Andrade et al. 2005) although subsequent work has
identified DHH1 as being a genetic factor of the destabilization (Mota
et al. 2014). Some transcripts respond at different levels of glucose
addition and differently in different genetically-perturbed metabolic
backgrounds (Yin et al. 2000), and disrupting signalling through the
PKA pathway affects destabilization of some mRNA but not others (Yin
et al. 2003). Thus, a systematic measurement of mRNA stability and a
broad determination of genetic factors of the transcript dynamics
would be useful for making progress at untangling the regulation of
mRNA stability in response to the increase in growth rate upon a
nutrient upshift.  

\section{What is the function of rapid txtome changes upon upshift?}

Up-regulation of mRNA abundance during an increase in growth
rate serves a clear functional purposes. The ribosomal protein (RP)
and ribosome biogenesis (RiBi) regulons are swiftly upregulated upon
repletion of nutrients to nutrient-limited cultures of yeast
(Jorgensen et al. 2004), and are well-correlated with growth rate in
both dynamic and steady-state conditions (does brauer or airoldi show
this?). Ribosomes are lower in slow growing conditions and need to be
upregulated upon resumption of rapid growth. (von der Haar estimates,
Tu, Hwa) (Estimated this is probably from 50k to about 200k.) The
relative allocation of gene expression resources in the cell is a
fundamentally important decision cells must make. Modeling of this
phenomenon across various conditions, mainly in E. coli as a model
system, as identified that a simple “pie-chart” model explains the
up-regulation of ribosome biogenesis relative to the rest of the
cellular investments well during steady-state.  

A recent study has
explored the theoretically best strategy during an increase in growth
rates, and found that a “bang-bang singular” strategy of complete
focus on generating gene expression machinery at the neglect of
metabolism machinery would be the optimal strategy for resuming growth
most rapidly (Giordano et al. 2016). With that, transcripts that are
stress responsive or important for metabolism in the old environment
are repressed upon a nutrient upshift.  

What does rapid transcriptional reprogramming achieve with respect to gene regulation?
In an integrative study of proteome and transcriptome dynamics from
the Gasch laboratory (Lee et al. 2011), the authors found that while
upregulation of mRNA did correlate with an increase in protein
abundance, the repression of mRNA did not correlate with a
downregulation of protein products on the timescales they measured.
This asymmetry makes sense, as with protein approximately 20 times
more stable than the mRNA they derive from (one of those reviews).
This suggests that accelerated mRNA degradation may serve a different
role. Others have suggested that degradation can help recycle
nucleotides (Kresnowati et al. 2006) or that reprogramming the
transcriptome would help to reallocate the extant translational
capacity of the cell to enact a growth-optimal program (Kief and
Warner 1981; Giordano et al. 2016; Shachrai et al. 2010). Identifying
the genetic factors responsible for the degradation would allow us to
test if the destabilization is adaptive, and if so to make progress in
understanding the mechanistic basis of this phenomenon.[e] 

\section{Measuring mRNA dynamics}

It is easier to measure the abundance of something than
it is to measure the change in abundance of something. While mRNA
abundance measurements for entire transcriptomes are now routine,
determining the rates that underlie this molecular phenotype has
lagged. Synthesis rate control has largely been assayed by techniques
like Genome Run On (GRO) sequencing (discussed below) to measure
transcription rates or measuring intron-exon ratios (pick one, NET
seq?) as a proxy for synthesis rates (Pérez-Ortín et al. 2013).
Degradation rate measurements have used a variety of methods, but are
now applied to the whole transcriptome with enough accuracy to enable
systematic modeling of the determinants of mRNA degradation rates
(Pérez-Ortín et al. 2013; Neymotin et al. 2016; Cheng et al. 2017).
Pioneering studies used pulse-chase experiments with radioactive
nucleotides to study turnover of the whole transcriptome, but were
unable to assay the process at the single-gene level (that one that
David sent me, and the one that Leon Chan sent me).  

To study
particular genes people have used promoters with inducible repression
characteristics to halt transcription. Transgenes like the
doxycycline-inducible TetOff (g something 1998?) use a heterologous
system. Researchers have also made use of the GAL1 promoter. Upon
addition of glucose, transcription of the GAL1 mRNA is immediately
halted. This property has been exploited to study mRNA stability in a
technically simple manner ( parker?, coller green ) and the promoter
has been adapted to higher throughput studies ( that bullshit short
half-life recent paper ). However, the last 100bp before the start
codon of GAL1 has been shown to be necessary for accelerated
degradation of this mRNA upon glucose addition (baumgarner, bennet,
hasty), challenging the interpretation of mRNA stability measured in
glucose-containing media with the GAL1 promoter fused directly
upstream of the start codon. While some researchers take care to limit
the glucose in the system (0.0?5\% in that bullshit paper) while using
heterologous binding sites, researchers have shown that glucose
concentrations of 0.0?1\% (56mM, check Ziv’s paper for making sure I’m
not screwing up a decimal place) have triggered instability of
gluconeogenic mRNA ( yin brown 2000 or 2003 ). Thus, while the GAL1
system is a convenient system for studying degradation intermediates (
parker and coller ), its use for studying the native stability of
different mRNA and in different environments is uncertain.  

A system
that does not rely on engineered cis-elements would avoid these issues
and scale to genome-wide assays, and thus two methods of
transcriptional inhibition were applied to study mRNA degradation
rates in landmark studies. A temperature sensitive rpb1-1 allele was
demonstrated to halt most Pol II transcription at non-permissive
temperatures, while the drugs thiolutin and 1,10-phenanthroline
inhibited polymerases including Pol II to mostly halt transcription.
These have been used widely, and are still used to this day. However,
it has been shown that used of thiolutin or 1,10-phenalanthroline
induces some heat-shock genes (Adams and Gross 1991), thiolutin
inhibits mRNA degradation in a dose-dependent way (Pelechano and
Pérez-Ortín 2008) (perhaps via inducing processing-body formation
(Huch et al. 2016)), and eliminating the essential RNA Pol II complex
from the nucleus has complex effects on the transcriptome dynamics (Yu
et al. 2016). While it may seem logical that studies of mRNA
associated with processes distinct from heat-shocks may be unaffected
by these, the complexity of the cell demonstrated itself in vital
controls run by (Mercado et al. 1994) which demonstrated that
gluconeogenic mRNA were subject to destabilization upon a
heat-shock[f]. Shutting off transcription has complex and difficult to
predict effects on transcript abundance as the cells die over the
course of the experiment.  

Orthogonal to these approaches is GRO-seq
(Genomic Run On)[g]. This method uses a sarkosyl treatment (and
salt?)[h] to fix RNA Pol II complexes onto genomic DNA by freezing
their elongation. Extraction and a defined in-vitro polymerase
extension by reversing the fixation before profiling the resulting
mRNA with microarrays or RNAseq allows for an estimate of the
instantaneous transcription rate status for each gene in a population
of cells. Interpretation of these numbers must be considered in the
context of the in-vitro environment of the elongation step, but this
method serves as an valuable orthogonal measure of transcript dynamics
--- and an instantaneous one.  

The development of 4-thiouracil
metabolic labeling of RNA (Dölken et al. 2008) has enabled a return to
the pulse-chase methodology in the development of genome-wide assays
of mRNA dynamics. Fundamentally, these assays work by changing the
labelling frequency of mRNA and tracking the dynamics as the labeled
mRNA abundance relaxes towards that new equilibrium. Below I review
the basis of these assays, then focus on their applications and where
the dissertation work is placed.  

If we consider mRNA abundance at a
certain time as being denoted as $M_t$, then I expect this number to
change as a zeroth order rate of synthesis per time ($k_s$) and a
first order rate of degradation per mRNA ($k_d$). While mRNA
degradation is a multi-step process (above) and more complex models
may identify nuances in the rates of progression through these
intermediates (Deneke et al. 2013), at steady-state the rate of
degradation of mRNA in a population should be well modeled by a single
first order rate ( ?can I find the cite for that? ).
$$\frac{dM_t}{dt} = k_s - M_t*k_d$$ Introducing a term $L$ that
denotes the fraction of newly synthesized mRNA that are labeled and
measured after purifying mRNA for the labeled mRNA (thus $M_t$ is just
labeled and captured mRNA), we can now model the changes as simply
$$\frac{dM_t}{dt} = L k_s - M_t*k_d$$ I introduce the superscript
notation of $L^o$ for the old labeling frequency and $L^n$ for the new
labeling frequency, and solving for the change of $M_t$ from some
steady-state equilibrium $L^o \frac{k_s^o}{k_d^o}$ to a new
equilibrium $L^n \frac{k_s^n}{k_d^n}$, and rearranging terms we get
$$M_t =  L^o \frac{k_s^o}{k_d^o} e^{-k_d^n t} + L^n \frac{k_s^n}{k_d^n} ( 1- e^{-k_d^n t} )$$ 
This matches well with our
intuition. On the right, the nascent transcripts are labeled at the
new rate and approach this new equilibrium controlled by the term $(
1- e^{-k_d^n t} )$, while on the left the extant transcripts approach
zero in an exponential decrease controlled by the term $e^{-k_d^n t}$.
These are both controlled by $k_d^n$, or the rate of mRNA degradation
after chasing the label. Thus, by measuring the transition between the
equilibrium we get the mRNA degradation rate, assuming that synthesis
rates do not change.  

Measuring specific rates with high confidence
requires a steady-state approximation. RATEseq is one method to do so,
using many timepoints to accurately model the approach of labeled mRNA
abundance to a new equilibrium (Neymotin et al. 2014). This
experimental design is theoretically the most accurate, although it
requires the assumption that the total mRNA (labeled + unlabeled) is
indeed at a steady-state abundance. Dynamic Transcriptome Analysis
(Miller et al. 2011) violates this assumption to explore changes in
degradation rates during 6 minute windows. While they sacrifice
high-confidence of an exact rate, the temporal resolution of stability
changes during osmotic stress has revealed an unprecedented dynamic
view of the regulation of mRNA dynamics during complex processes. This
approach requires that 4-thiouracil transport and incorporation into
nucleotide metabolism occurs during the course of the perturbation
experiment, but with the right measurements, normalization, and
integration with other datasets an accurate and dynamic picture of
transcriptome dynamics can be built. To assess mRNA stability changes
during dynamic processes, one can also label the transcriptome to
equilibrium and then chase out the label by adding an excess of
unlabeled nucleotides. This approach was used by researchers in the
Weis group to demonstrate changes in the stability of groups of mRNA
in response to environmental changes, namely shifts in carbon sources
and with rapamycin treatment inhibiting TORC1 (Munchel et al. 2011).
They found that the RP (ribosomal protein) regulon was destabilized
upon induction of nutrient starvation, demonstrating that mRNA
degradation is under tight regulation from nutrient-sensing pathways.
In Chapter 3 I demonstrate the use of a similar label-chase
experimental design, with with refined analysis to explore
single-transcript destabilization upon a nitrogen upshift.  

\section{Methods for determining the genetic basis of a transcript
dynamics phenotype}

mRNA is an intermediate in the expression of a protein product, and
has the key virtue of being much easier to measure than to measure
abundance of the protein product. This became especially true with the
advent of massively-parallelized DNA sequencers and the methods to
accurately convert transcriptomes to DNA libraries, ie RNAseq
(Shendure et al. 2017). For this reason, it is often used as a proxy
of gene expression at the protein level. Although the relationship is
strong when correcting for experimental noise (Csárdi et al. 2015),
the quantitative functional nature of this relationship within a
particular gene in different environments depends on the particular
gene in question (Franks et al. 2017). It is also clear that
transcriptomic and proteomic responses greatly vary in the timescales
of effect, with the transcriptome subject to rapid impulses of
changing abundance that may or may not result in longer term
regulation of the protein product (Cheng et al. 2016; Lee et al.
2011). Even then, protein abundance in a cell does not correspond
perfectly to its activity, be that regulated allosterically or by
localization.  

Given this disparity, what can we learn about adaptive
gene expression from mRNA abundance regulation? First, the expression
of a gene product requires mRNA, thus the binary expression of mRNA is
a predictor of the possibility of protein expression. Additionally,
cellular processes often impinge upon changes in mRNA abundance, be
they direct via regulation of abundance, activity, and localization of
activity of specific effectors or by indirect effects on common gene
expression machinery or cellular metabolism. In this way, a specific
perturbation of a signalling pathway is expected to broadcast to
changes in mRNA abundance. Quantification of the thousands of mRNA
that are expressed in a cell is a sensitively quantitative measurement
on thousands of dimensions, and can thus be used as a
relatively-unbiased indicator of cellular status with which to explore
the genetic requirements of particular signalling perturbations (that
mol sys bio paper). Thus, efficient methods to explore the genetic
basis of transcript dynamics upon a perturbation should help to
accelerate the study of cellular signalling pathways.  

Genetic screens
in yeast have been a powerful tool to narrow down the immense search
space of possibilities to a narrow set of hypotheses about a
biological process. Classically, these function by mapping some
phenotype of interest to a change in growth rate. For example, mutants
in transporters of a particular amino-acid can be isolated by feeding
the cells a toxic stereoisomer (like D-histidine). A more complex
method in Lee Hartwell’s classic screen for cell-cycle mutants used
the assay of growth at a low temperature and cessation of growth at a
high temperature to identify mutants in critically important pathways
(hartwell 1970), work that contributed to a 2001 Nobel Prize for
advancing our understanding of the cell cycle. However, this concept
becomes problematic when studying a molecular phenotype which is not
known to be adaptive. 

For example, gene regulation might not have a
clear phenotypic outcome or be subject to redundant layers of
regulation that mask the effect of a mutation. One solution is to
engineer a specific reporter into the expressed gene, such that
defects in gene expression can be assayed. It becomes more difficult
if the phenotype is a transient one, such that a reporter through
growth rate (perhaps a toxic peptide) does not have time to accumulate
the signal of growth. A fluorescent tag is one approach that bypasses
this requirement, as cells can be instantaneously assayed for the
level of GFP fluorescence through methods like flow cytometry. The GFP
can be fused to the protein of interest or simply placed downstream of
an appropriate reporter, such as the strategy employed by (Neklesa and
Davis 2009). They were able to use a DAL80 promoter upstream of a GFP
reporter to explore the genetic requirements for the NCR-regulated
expression of this promoter, discovering the SEACIT complex components
Npr2/3p upstream of TORC1. However, this method requires that the gene
expression phenotype be regulated at both the transcript synthesis
level and be relevant at the level of protein expression. Additionally
it requires that the GFP tag be a relevant and faithful reporter of
the protein abundance, a condition which is not always satisfied given
the stability of the GFP tag in the vacuole (find that cite).
Previous work to do genetics of transcript dynamics neklesa - sortseq,
reporters Mention problems of GFP stability worley - robotics exotic
methods, like COE, so antibodies scRNAseq I think tavazoie’s stuff
Fluorescent readout of mRNA degradation mechanisms at the protein
level. (Aviv et al. 2003) GAP1 mRNA degradation, which we identify as
being subject to accelerated mRNA degradation in Chapter 3 and 4,
occurs much faster than the repression of the protein-product. We also
know that Gap1p, the protein product of GAP1 mRNA, is subject to
de-activation and re-localization in response to a nitrogen upshift.
Thus, a functional assay of Gap1p is irrelevant to the dynamics of
GAP1 mRNA repression, and requires a novel method to screen for
genetic factors of this molecular phenotype.  

Ambitious work from the
Capaldi group developed a workflow using extensive automation to
perform qPCR assays for NSR1 mRNA abundance 19 minutes after induction
of an osmotic stress response (Worley et al. 2015). While accurate and
reproducible, the extensive automation and reagent usage to perform
qPCR on ~4700 mutant strains poses a financial and logistical
challenge to performing the assay, and to perform the assay in
different genetic backgrounds, in larger libraries, in replicates, or
in different timepoints.  

I composed several existing technologies to
develop a direct assay of transcript abundance in a high-throughput
pooled format. This is discussed in depth in Chapter 3. 


\chapter{Modeling transcript dynamics upon a nitrogen upshift}

This chapter was published as part of the article 
\textit{"Steady-state and dynamic gene expression programs in 
Saccharomyces cerevisiae in response to variation in 
environmental nitrogen”} in \textit{Molecular Biology of the Cell}
(vol. 27 no. 8 1383-1396. April 15, 2016.
\url{doi.org/10.1091/mbc.E14-05-1013}).

Authorship of this article was: Edoardo M. Airoldi, 
\textbf{Darach Miller},
Rodoniki Athanasiadou, Nathan Brandt, Farah Abdul-Rahman, Benjamin
Neymotin, Tatsu Hashimoto, Tayebeh Bahmani, and David Gresham. 

Below is reprinted the abstract, then excerpts of the introduction,
results, and conclusion to which I contributed. 
The text has been edited for clarity. Supplemental tables,
figures, and files are available on the MBoC article website (
\url{doi.org/10.1091/mbc.E14-05-1013} ).  

\section{Abstract} 

Cell growth rate is
regulated in response to the abundance and molecular form of essential
nutrients. In \textit{Saccharomyces cerevisiae} (budding yeast), the molecular
form of environmental nitrogen is a major determinant of cell growth
rate, supporting growth rates that vary at least threefold.
Transcriptional control of nitrogen use is mediated in large part by
nitrogen catabolite repression (NCR), which results in the repression
of specific transcripts in the presence of a preferred nitrogen source
that supports a fast growth rate, such as glutamine, that are
otherwise expressed in the presence of a nonpreferred nitrogen source,
such as proline, which supports a slower growth rate. Differential
expression of the NCR regulon and additional nitrogen-responsive genes
results in >500 transcripts that are differentially expressed in cells
growing in the presence of different nitrogen sources in batch
cultures. Here we find that in growth rate–controlled cultures using
nitrogen-limited chemostats, gene expression programs are strikingly
similar regardless of nitrogen source. NCR expression is derepressed
in all nitrogen-limiting chemostat conditions regardless of nitrogen
source, and in these conditions, only 34 transcripts exhibit nitrogen
source–specific differential gene expression. Addition of either the
preferred nitrogen source, glutamine, or the nonpreferred nitrogen
source, proline, to cells growing in nitrogen-limited chemostats
results in rapid, dose-dependent repression of the NCR regulon. Using
a novel means of computational normalization to compare global gene
expression programs in steady-state and dynamic conditions, we find
evidence that the addition of nitrogen to nitrogen-limited cells
results in the transient overproduction of transcripts required for
protein translation. Simultaneously, we find that that accelerated
mRNA degradation underlies the rapid clearing of a subset of
transcripts, which is most pronounced for the highly expressed
NCR-regulated permease genes \textit{GAP1}, \textit{MEP2}, 
\textit{DAL5}, \textit{PUT4}, and \textit{DIP5}. Our
results reveal novel aspects of nitrogen-regulated gene expression and
highlight the need for a quantitative approach to study how the cell
coordinates protein translation and nitrogen assimilation to optimize
cell growth in different environments.  

\section{Introduction}

The rate at which budding yeast cells grow is sensitive to the
molecular form of nitrogen in the environment. Yeast cells are able to
use and discriminate between different nitrogen sources 
\parencite{cooper1982nitrogen,magasanik2002nitrogen}. 
When a variety of nitrogen sources are
available, a yeast cell will preferentially transport and metabolize
particular nitrogen-containing compounds by decreasing levels of
transcripts and proteins required for use of nonpreferred nitrogen
sources 
\parencite{cooper1982nitrogen,magasanik2002nitrogen}. 
A study of yeast
cells growing in the presence of different individual nitrogen sources
provided a genome-wide view of nitrogen-regulated gene expression and
suggested that >500 genes are differentially expressed as a function
of environmental nitrogen source 
\parencite{godard2007effect}. On the basis
of differential gene expression, promoter sequence elements, and
published literature, \cite{godard2007effect} assigned membership of many
of these transcripts to five regulons that are responsive to
environmental nitrogen: the nitrogen catabolite repression A (NCR-A)
regulon, which includes bona fide NCR targets; the potential NCR
target (NCR-P) regulon; the general amino acid control (GAAC) regulon;
the unfolded protein response (UPR) regulon; and the \textit{SSY1}-PTR3-SSY5
(SPS) regulon.

Transcriptional control of the NCR regulon (i.e., both NCR-A and NCR-P
regulons) is mediated by the transcription factors 
\textit{GLN3}, \textit{GAT1}, \textit{DAL8}0,
and \textit{GZF3}, which bind to the 5′-GATAA-3′ consensus sequence in target
gene promoter regions 
\parencite{cooper1982nitrogen,magasanik2002nitrogen}. 
Whereas \textit{DAL8}0 and \textit{GZF3} act as repressors of 
NCR transcription, \textit{GLN3}
and \textit{GAT1} activate the transcription of NCR genes in a nitrogen
source–dependent manner. The evolutionarily conserved TOR complex 1
(TORC1) is believed to be an upstream regulator of NCR expression, as
it promotes the nuclear exclusion of \textit{GLN3} by physical association with
\textit{URE2} in a phosphorylation-dependent manner
\parencite{beck1999tor}.

%To study the effect on mRNA expression of environmental
%nitrogen source variation in nitrogen-limited, growth rate–controlled
%conditions, we studied cells growing in chemostats using six different
%nitrogen sources at four different dilution rates. We show that
%differential expression of the NCR (Nitrogen Catabolite Repression)
%and SPS (SSY1-PTR3-SSY5) regulons is primarily a function of growth in
%a nitrogen-limited environment, with the molecular form of nitrogen
%having minimal effect on differential gene expression when cells are
%limited for nitrogen. By contrast, the GAAC (General Amino Acid
%Control) and UPR (Unfolded Protein Response) regulons do not respond
%specifically to nitrogen limitation compared with other
%nutrient-limited conditions. 
To study the dynamics of
nitrogen-responsive gene expression, we performed transient
perturbation experiments in which different quantities and sources of
nitrogen were added to cells growing in nitrogen-limited chemostats.
The addition of either the preferred nitrogen source, glutamine, or
the nonpreferred nitrogen source, proline, to cells growing in
nitrogen-limited conditions results in rapid repression of the NCR
regulon in a dose-dependent manner. Surprisingly, a sudden increase in
environmental nitrogen does not correspond to a detectable increase in
biomass production or cell number, consistent with a time delay
between activation of the transcriptional growth program and its
manifestation in an increased rate of cell growth. To compare global
gene expression in dynamic conditions with mRNA expression in
steady-state conditions, we used computational estimation of
instantaneous growth rate from gene expression profiles 
\parencite{brauer2008coordination,airoldi2009predicting} 
and defined gene expression responses to
growth rate in both steady-state and dynamic conditions using linear
regression. We find that the response of transcripts required for
protein translation (RP and RiBi) in cells provided with an increase
in nitrogen exceeds the response to growth rate in cells growing in
steady-state conditions consistent with a transient overproduction of
RP and RiBi transcripts. Finally, we show that accelerated degradation
of some NCR transcripts underlies gene expression remodeling in
response to sudden relief from nitrogen limitation, indicating the
activity of a posttranscriptional mechanism controlling
nitrogen-responsive gene expression.  

\section{Results} 

To obtain a high-resolution view of mRNA abundance
changes during the first 10 min after addition of nitrogen, when
changes in gene expression are maximal, we repeated the
pulse experiments (addition of nitrogen source to yeast grown in
steady-state nitrogen-limited chemostat cultures) and assayed global
gene expression at 1-2 min intervals after the addition of 40 $\mu$M
glutamine or 80 $\mu$M proline. We observed a rapid increase in expression
of the RiBi and RP regulons in response to a pulse of glutamine, with
a concomitant rapid decrease in expression of the NCR-A and NCR-P
regulons. Consistent with our initial observation, we observed a
similar response to a pulse of proline. 

\subsection{Accelerated degradation of
mRNAs contributes to remodeling of the transcriptome }

The majority of
NCR transcripts are strongly repressed in response to a nitrogen
pulse. If gene expression is repressed at the promoters of these genes
and mRNA synthesis ceases, the decrease in mRNA abundance is expected
to be a function of the degradation rate of the corresponding mRNA.
Using our high-density time-series data, we estimated the rate of
change in abundance for all transcripts, assuming a first-order
exponential degradation model (Materials and Methods; Supplemental
Table S7), which is the standard method for estimating mRNA
degradation rates. We found that in response to a glutamine pulse, 269
genes fit a first-order exponential decay model (FDR < 0.05;
Supplemental Table S4), whereas 458 transcripts fit a first-order
exponential decay model in response to the proline pulse (Supplemental
Table S4).  

\newpage

\afig{
  \includegraphics[width=\textwidth,bb=0 0 310 300]{img/airoldi2016_F6.large.jpg}
  }{
  \textcolor{white}{\label{fig:airoldi2016f6}}
  }{
  Accelerated mRNA degradation contributes to gene expression 
  remodeling
  }

\begin{framed}
\noindent
\autoref{fig:airoldi2016f6} ---
Accelerated mRNA degradation contributes
to gene expression remodeling. Upon addition of glutamine to
NCR-derepressed cells, a subset of transcripts degrade more rapidly
than their steady-state degradation rate both (A) in cells grown in
ammonia-limited chemostats and (B) in cells growing in proline media
in batch cultures. All points are genes that fit a model of
exponential decrease in abundance (FDR < 0.05). Orange points are NCR
genes that show significant accelerated degradation, blue points are
NCR genes that are not significant, green points are non-NCR genes
that show significantly accelerated degradation, and gray points are
genes that are neither accelerated nor NCR. The dashed line denotes
equal degradation rates in both conditions (i.e., slope equal to 1).
Names of nitrogen transporter genes are displayed. We measured the
transient changes in the degradation rates of (C) \textit{GAP1} and (D) \textit{DIP5}
mRNA using a pulse-chase experiment. Cells were grown for 24 h in the
presence of 4-thiouracil, which was chased at t = 0 min by the
addition of excess uracil. At t = 13 min, we added either glutamine in
water (orange) or equal volume of water (blue). We extracted and
quantified the abundance of 4-thiouracil–labeled mRNA relative to a
thiolated external spike-in using qPCR. We found significant
acceleration of degradation for both \textit{GAP1} and \textit{DIP5} mRNAs (p < 0.001).
Points are the mean of triplicate qPCR measurements, error bars are
the propagated SD of transcript and spike-in measurements, and dotted
lines are the log-linear model fit.  
\end{framed}

We compared the half-lives of rapidly degraded transcripts
after the glutamine pulse with half-life estimates in steady-state
conditions determined using RATE-seq 
\parencite{neymotin2014determination}. We found
that some transcripts decay significantly faster than expected,
suggesting that their degradation rate is accelerated in response to
the glutamine pulse (\autoref{fig:airoldi2016f6}a). 
Batch culture growth in proline also
results in derepression of the NCR regulon 
\parencite{godard2007effect}. To
test whether accelerated mRNA decay is specifically a response to the
nitrogen-limited conditions of a chemostat, we added a pulse of
glutamine to cells growing in batch cultures containing proline as a
sole nitrogen source and measured genome-wide gene expression
(Supplemental Table S7). The half-lives of transcripts that exhibit an
exponential decrease is similar in chemostat and batch cultures
(Supplemental Figure S7B), and many of the same transcripts show
evidence of accelerated degradation rates in batch cultures 
(\autoref{fig:airoldi2016f6}B)
and Supplemental Table S4). Strikingly, the five nitrogen permease
genes \textit{GAP1}, \textit{DIP5}, \textit{MEP2}, 
\textit{PUT4}, and \textit{DAL5} are the most rapidly cleared
mRNAs in both the chemostat and batch culture experiments.  

To verify
that the addition of glutamine stimulates accelerated degradation of
specific NCR transcripts, we performed pulse-chase experiments using
the metabolic label 4-thiouracil (4-tU). After several generations of
batch culture growth in proline medium in the presence of 4-tU to
allow complete labeling of mRNAs, we added unlabeled uracil to the
culture. We allowed the chase to occur for 13 min and then added
either glutamine or water (mock) to the cells. We purified labeled
transcripts and analyzed \textit{GAP1} and \textit{DIP5} 
mRNAs using quantitative PCR
(qPCR) and normalization to external spike-ins. Consistent with our
genome-wide assay, the addition of glutamine results in a clear
accelerated degradation of both \textit{GAP1} mRNA 
(\autoref{fig:airoldi2016f6}C)
and \textit{DIP5} mRNA (\autoref{fig:airoldi2016f6}D), 
confirming that the transition from NCR-derepressed to
NCR-repressed conditions results in the accelerated degradation of
some transcripts.  

\section{Discussion} 

Some mRNAs are rapidly degraded when cells
transition from NCR-activating to NCR-repressing conditions in both
chemostats and batch culture. Comparison with mRNA degradation rates
suggests that the degradation of some of these transcripts is
accelerated. Using in vivo metabolic labeling with 4-tU, we provide
additional evidence that the addition of glutamine to nitrogen-limited
cells accelerates the degradation of specific transcripts. A previous
study of the transcriptional response to glucose addition in
carbon-limited chemostats suggested a role for accelerated degradation
of mRNAs 
\parencite{kresnowati2006transcriptome}
, and there is increasing evidence
that mRNA stability plays an important role in regulating gene
expression programs 
\parencite{puig2005coordinated,bennett2008metabolic,baumgartner2011antagonistic}. 
Consistent with a posttranscriptional
mechanism underlying the rapid clearing of some NCR transcripts,
previous work showed that \textit{GAP1} mRNA transiently decreases in abundance
during a nitrogen up-shift in the absence of \textit{URE2} 
\parencite{ter1998repression}, 
which is required for NCR repression by sequestering \textit{GLN3} in
the cytoplasm. Several studies have shown that T\textit{ORC1} can affect
transcript stability 
\parencite{albig2001target,munchel2011dynamic}.
Our results suggest that posttranscriptional regulation of mRNA
stability may play an important role in remodeling gene expression in
response to changes in environmental nitrogen. Transient stabilization
of the RP and RiBi regulons also could contribute to their rapid
increase in expression 
\parencite{yin2003glucose}. Defining the role of
regulated changes in mRNA stability in dynamic conditions is an
important area for further study. 

What is the underlying rationale
for rapid induction of RP/RiBi transcripts occurring in parallel with
accelerated degradation of NCR transcripts? We propose that
accelerated degradation of NCR transcripts may allow for reallocation
of ribosomes to transcripts required for growth and proliferation
\parencite{kief1981coordinate,lee2011dynamic}. Our observations are
consistent with a model in which TORC1 orchestrates the balance
between transcripts required for protein production and transcripts
required for the acquisition and assimilation of nitrogen. When
nitrogen is abundant, TORC1 activates the expression of the RP and
RiBi regulons while actively repressing the NCR-A and NCR-P regulons.
Conversely, when nitrogen levels are in growth-limiting
concentrations, T\textit{ORC1} activity decreases, leading to reduced
activation of the RP and RiBi regulons and derepression of the NCR-A
and NCR-P regulons. In NCR-derepressing conditions, NCR transcripts,
including \textit{GAP1}, \textit{MEP2}, and \textit{PUT4}, 
are the most abundant transcripts
(Supplemental Table S5). When a cell encounters a sudden increase in
environmental nitrogen, some highly expressed transcripts may be
targeted for accelerated degradation to increase the pool of free
ribosomes facilitating rapid translation of newly transcribed RiBi and
RP transcripts, thereby accelerating physiological remodeling of the
cell for rapid growth.  

\section{Materials and methods}

\subsubsection{Strains and culturing conditions}

We used the prototrophic haploid strain FY4 (MATa), which
is isogenic to the S288c reference strain, for all experiments. We
used minimal defined media for all experiments, using a common base
medium for nitrogen limitation, as described previously 
\parencite{brauer2008coordination,boer2010growth}. The appropriate concentrations of
allantoin, glutamine, glutamate, urea, ammonium sulfate, proline, and
arginine were added from 100 mM stock. Batch culture experiments were
performed in 30$^{\circ}$C shaking incubators using 100-ml cultures. Continuous
culturing in chemostats using Sixfors bioreactors (Infors, Laurel, MD)
was performed as described 
\parencite{brauer2008coordination,boer2010growth}
using a 300-ml working volume. Culture parameters were determined
using either a Klett colorimeter or a Coulter counter after
sonication. For perturbation studies, a single bolus of proline,
glutamine, or a mix of both was added to the chemostat to a final
concentration of 80 or 800 $\mu$M nitrogen.  

\subsubsection{RNA analysis} 

Cell samples for
mRNA analysis were preserved by rapid filtration and quick freezing
using liquid nitrogen. We isolated total RNA using hot acid–phenol
extraction and subsequently purified RNA samples using RNeasy columns.
We performed gene expression profiling using Agilent (Santa Clara, CA)
60-mer DNA microarrays and Cy3 and Cy5 incorporation as previously
described 
\parencite{brauer2008coordination}. We used a common reference obtained
from a sample growing in an ammonium sulfate–limited chemostat at a
dilution rate of 0.12 hours$^1$ for all hybridization experiments and
hybridized labeled cRNA to Agilent Yeast DNA microarrays for 20 h at
65$^{\circ}$C. We washed arrays and scanned microarrays using an Agilent
two-color scanner and extracted hybridization signals using Agilent
Feature Extractor Software. Supplemental Table S6 gives the entire
data set of processed log$_2$ ratios.  

\subsubsection{Pulse chase} 

Cells were grown in
600 ml of minimal medium containing 800 $\mu$M proline, 500 $\mu$M uracil, and
500 $\mu$M 4-thiouracil at 30$^{\circ}$C for 24 h. The culture was divided into two
300-ml cultures, and uracil was added to a final concentration of 2mM. 
We acquired 20-ml samples after the chase using rapid filtration
and flash freezing in liquid nitrogen. At 13 min after starting the
chase, we added either glutamine to a final concentration of 400 $\mu$M or
an equal volume of water and acquired additional samples.  

After RNA
extraction, samples were mixed with an in vitro–transcribed thiolated
spike-in (\textit{BAC1}200) at a ratio of 1 ng of spike-in to 25 $\mu$g of total
RNA and reacted with EZ-Link HPDP-Biotin (ThermoFisher Scientific,
Waltham, MA) at 2 mg/ml for 200 min. Reactions were cleaned up by
centrifugation and ethanol precipitation and then conjugated with 180
$\mu$l of streptavidin magnetic beads (M0253L; NEB, Ipswich, MA). Labeled
RNA was eluted using 5\% $\beta$-mercaptoethanol.  

Samples were reverse
transcribed with Moloney murine leukemia virus reverse transcriptase
(NEB) and random hexamer priming. We performed qPCR in technical
triplicate on a LightCycler 480 (Roche, Branchburg, NJ) using the
following primers: \\[1em]
\begin{tabular}{l | p{15em} p{15em}}
%  \toprule
%  mRNA & Forward & Reverse \\                                                
%  \midrule
  \textit{GAP1} & 5'-\texttt{ACGGTATCAAGGGTTTGCCAAG}-3' &
    5'-\texttt{GCATAAATGGCAGAGTTAC}-3' \\
  \textit{DIP5} & 5'-\texttt{TGGCGTACATGAATGTGTCTTCA}-3' &
    5'-\texttt{GGTGATCCAACTCAAGATTC}-3' \\
  BAC1200 & 5'-\texttt{CTGGACGACTTCGACTACGG}-3' & 
    5'-\texttt{ATCAGCCTTTCCTTTCGTCA}-3' \\
\end{tabular} \\[1em]
$C_p$ values
were calculated for each sample and the spike-in and log-linear
regression performed using the ratio of either \textit{GAP1} mRNA or \textit{DIP5} mRNA
to the spike-in in R.  

\subsubsection{mRNA decay estimation}

We estimated rates of
mRNA decay for all transcripts using high–temporal resolution data. We
used ratios ($y_t$) of hybridization intensities for each transcript
obtained from two-color DNA microarrays co-hybridized with a common
reference. Data were normalized to the initial data point ($y_0$) and
then log-transformed. We modeled the degradation rate $k_deg$ of each
gene:
$$ ln\left(\frac{y_t}{y_0}\right)=k_{\text{deg}}\times t$$
where $t$ is the sampling time in minutes. Transcript
half-lives were computed as $\frac{ln(2)}{k_{deg}}$. 
Accelerated degradation was
assessed by fitting the model 
$$ ln\left(\frac{y_t}{y_0}\right)=(k_{\text{transient
deg}}+k_{\text{steady-state deg}})\times t$$
where $k_{steady-state deg}$ 
is the specific degradation rate for transcript $i$ as reported in 
\cite{neymotin2014determination}.
For all linear modeling, we assessed statistical
significance of coefficients using a t-statistic and determined
empirical p-values by permuting data for each gene 1000 times. The
false discovery rate was determined using the \texttt{qvalue} package in R.
Data availability DNA microarray data are available through gene
expression omnibus (GEO) GSE57293.


\chapter{Measuring the extent of mRNA destabilization and screening 
for genetic factors of \textit{GAP1} repression}

This chapter is similar to an article currently submitted to a 
journal for review and publication. 
It is also posted on \textit{biorxiv}, titled:
\textit{"Global analysis of gene expression dynamics identifies factors
required for accelerated mRNA degradation"}.
Authorship of this article is: \textbf{Darach Miller}, Nathan Brandt, 
and David Gresham.
Darach Miller did most of the benchwork, analysis, and writing.
Nathan Brandt did all benchwork to generate mutants and collect qPCR 
for \autoref{fig:figure5qpcr}.
David Gresham helped with discussions and co-writing the article.
The \textit{biorxiv} version is at \url{doi.org/10.1101/254920}.

The below is adapted for the dissertation, incorporating important
text from the supplementary methods into the chapter text.
Supplemental tables are available from the OSF repository linked 
with this work (\url{https://osf.io/7ybsh/files/}), 
and are reproducible
using the \texttt{Makefile} and associated scripts
in the git repository distributed with the paper
(\url{http://github.com/darachm/millerBrandtGresham2018}).

%The methods here allow
%for the use of fixed-cell flow-cytometry assays in pooled Sort-seq
%assays on yeast, and would be useful to inform the development of
%similar assays in other systems. Development of this approach to
%estimating mRNA abundance on pooled mutants would enable the
%combination of transcriptomics as a high-dimensional marker of
%cellular signalling pathways with the use of transcript markers to
%explore the genetics of these pathways.  Supplementary issues
%Pulse-chase modeling ( I basically want to reprint the stuff in the
%supplement here ) BFF rationale, methodology, and future directions (
%I basically want to reprint the stuff in the supplement here, then
%waste paper speculating )

\section{Abstract}

Cellular responses to changing environments frequently
involve rapid reprogramming of the transcriptome.
Regulated changes in mRNA degradation rates can
accelerate reprogramming by clearing or stabilizing extant transcripts. 
%Budding yeast respond to an improvement in
%nitrogen-availability by triggering a transcriptional reprogramming
%that functions to upregulate ribosome biogenesis and repress
%alternative nitrogen-source catabolism. 
Here, we measured mRNA stability using 4-thiouracil labeling
in the budding yeast \textit{Saccharomyces cerevisiae}
during a nitrogen upshift and found that 78 mRNAs are subject
to destabilization. These transcripts include Nitrogen
Catabolite Repression (NCR) and carbon metabolism mRNAs,
suggesting that mRNA destabilization is a mechanism 
for targeted reprogramming.
To explore the molecular basis of
destabilization we implemented a SortSeq approach to
screen using the pooled deletion collection library
for \textit{trans} factors that mediate rapid \textit{GAP1}
mRNA repression.
We combined low-input multiplexed \underline{B}arcode sequencing 
with branched-DNA single-molecule mRNA \underline{F}ISH and 
\underline{F}luorescence-activated cell sorting (\underline{BFF})
to identify that the Lsm1-7p/Pat1p complex and general mRNA
decay machinery are important for \textit{GAP1} mRNA clearance.
We also find that the decapping modulator \textit{SCD6}, translation
factor eIF4G2, and the 5' UTR of \textit{GAP1}
are important for this repression, 
suggesting that translational control may impact the 
post-transcriptional fate of mRNAs in response to 
environmental changes.

\section{Introduction}

Regulated changes in mRNA abundance are a primary cellular response
to external stimuli.
Both the rate of synthesis and the rate of degradation determine the
steady-state abundance of a particular mRNA and the kinetics
with which abundance changes occur
\parencite{hargrove1989role,perez2013eukaryotic}. 
Changes in mRNA degradation rates fulfill an important 
mechanistic role in diverse systems, including 
development \parencite{alonso2012complex,west2018developmental} and disease
\parencite{aghib19903}.
In budding yeast, the rate of
mRNA degradation is affected by environmental stresses
\parencite{canadell2015impact}, cellular growth rate
\parencite{garcia2016growth}, as well as by improvements in 
nutrient conditions \parencite{scheffler1998control}.

Environmental shifts trigger rapid reprogramming of the budding yeast
transcriptome in response to stresses and nutritional
changes \parencite{gasch2000genomic,conway2012glucose}. mRNA degradation rate changes
have been shown to play a role in responses to heat-shock, osmotic
stress, pH increases, and oxidative stress
\parencite{castells2011heat,romero2009specific,canadell2015impact,molina2008comprehensive}. 
In response to these
diverse stresses destabilization of mRNAs encoding 
ribosomal-biogenesis gene products, and  
stress-induced mRNA occurs \parencite{canadell2015impact}. 
Simultaneous increases in both synthesis and
degradation rates of some  mRNAs may serve to speed the return to a
steady-state following a transient pulse of regulation
\parencite{shalem2008transient}. Addition of glucose to carbon-limited cells 
results in both stabilization of 
ribosomal protein mRNAs \parencite{yin2003glucose} and destabilization
of gluconeogenic transcripts \parencite{de2002role,mercado1994levels}.
Destabilization of transcripts can
have a delayed effect on reducing protein levels compared to
up-regulated genes \parencite{lee2011dynamic}. This suggests that accelerated
mRNA degradation may serve additional purposes. For example, clearance
of specific mRNAs could increase nucleotide pools
\parencite{kresnowati2006transcriptome} or facilitate reallocation of
translational capacity 
\parencite{kief1981coordinate,giordano2016dynamical,shachrai2010cost}.
%Identifying the genetic
%factors responsible for the accelerated mRNA degradation would allow
%us to test if regulated destabilization of specific transcripts is
%adaptive.

Yeast cells metabolize a wide variety of nitrogen sources, but
preferentially assimilate and metabolize specific nitrogen compounds.
Transcriptional regulation, known as
“nitrogen catabolite repression” (NCR)
\parencite{magasanik2002nitrogen},
controls the expression of mRNAs
encoding transporters, metabolic enzymes, and regulatory
factors required for utilization of alternative nitrogen sources. 
NCR-regulated transcripts are expressed in the
absence of a readily metabolized (preferred) nitrogen sources or in
the presence of growth-limiting concentrations (in the low $\mu$M range)
of any nitrogen source \parencite{godard2007effect,airoldi2016steady}. Regulation
of NCR targets is mediated by two activating GATA
transcription factors, Gln3p and Gat1p, and two repressing
GATA factors, Dal80p and Gzf3p. \textit{GAT1}, \textit{GZF3}, and
\textit{DAL80} promoters
contain GATAA motifs, and thus transcriptional regulation of NCR
targets entails self-regulatory and cross-regulatory loops. When
supplied with a preferred nitrogen source such as glutamine, the
NCR-activating transcription factors Gat1p and Gln3p are excluded from
the nucleus by TORC1-dependent and -independent mechanisms
\parencite{beck1999tor,tate2013five,tate2017general} and NCR transcripts are strongly
repressed. The activity of some NCR gene
products is also controlled by post-translational mechanisms
\parencite{cooper1983function} such as the General Amino-acid Permease
(Gap1p) which is rapidly inactivated upon a nitrogen 
upshift via ubiquitination
\parencite{stanbrough1995transcriptional,risinger2006activity,merhi2012internal}. Recently, we have
identified an additional level of regulation of NCR transcripts: cells
growing in NCR de-repressing conditions accelerate the degradation
of \textit{GAP1} %and \textit{DIP5} mRNAs
mRNA upon addition of glutamine
\parencite{airoldi2016steady}. Thus, mRNA degradation rate regulation may be an
additional mechanism for clearing NCR-regulated transcripts upon 
improvements in environmental nitrogen availability.

Multiple pathways mediate the degradation of mRNAs. The main pathway
of mRNA degradation occurs by deadenylation and decapping
prior to 5' to 3' exonucleolytic degradation by Xrn1p; however,
transcripts are also degraded 3' to 5' via the exosome, or via
activation of co-translational quality control mechanisms
\parencite{parker2012rna}. Deadenylation of mRNAs by the Ccr4-Not complex
allows the mRNA to be bound at the 3' end by the 
Lsm1-7p/Pat1p complex, a heptameric
ring comprising the SM-like proteins Lsm2-7p and the
cytoplasmic-specific Lsm1p
\parencite{tharun2000yeast,sharif2013architecture}, which then
recruits factors for decapping by Dcp2p. 
Recruitment of the decapping enzyme \parencite{coller2004eukaryotic} is the 
rate-limiting step for canonical 5'-3' degradation.
Therefore Lsm1-7p, Pat1p,
and associated factors play a key role \parencite{nissan2010decapping}. 

Regulation of mRNA degradation pathways can alter the stability of
specific mRNAs. For example, the RNA-binding protein (RBP) Puf3p
recognizes a \textit{cis}-element in 3' UTRs \parencite{olivas2000puf3}
and affects mRNA degradation rates depending on
Puf3p phosphorylation status \parencite{lee2015glucose}. 
%Transcript properties
%also associated with translation dynamics affect mRNA degradation, at
%the level of elongation \parencite{Sweet2012,Presnyak2015,Neymotin2016} or
%competition between the decapping enzymes and translation initiation
%\parencite{Schwartz2000}. 
In addition to \textit{cis}-elements within the transcirpt, 
promoters have
been shown to mark certain RNA-protein (RNP) complexes to specify
their post-transcriptional regulation
\parencite{mercado1994levels,haimovich2013gene,trcek2011single,braun2016snf1}. These
mechanisms may be controlled by a variety of different signalling
pathways including Snf1
\parencite{young2012amp,braun2014phosphoproteomic}, PKA
\parencite{ramachandran2011camp}, Phk1/2 \parencite{luo2011nutrients}, and TORC1
\parencite{talarek2010initiation}. Thus, regulated changes in  mRNA degradation
rates entails numerous mechanisms that collectively tune stability of
mRNAs in response to the activity of signalling pathways. 

Here, we studied the global regulation of mRNA degradation rates upon
improvment in environmental nitrogen using 4-thiouracil (4tU) 
label-chase and RNAseq.
We found that a set of 78 mRNAs are subject to accelerated mRNA
degradation, including many NCR transcripts as well as mRNAs
encoding components of
carbon metabolism. To identify the mechanism underlying accelerated
mRNA degradation we designed a high-throughput genetic screen using 
\underline{B}arcode-sequencing of a pooled library which was
fractionated using \underline{F}luorescence-activated cell 
sorting of single molecule mRNA \underline{F}ISH signal (BFF). 
We screened the barcoded
yeast deletion collection to test the effect of each gene deletion
on the abundance of \textit{GAP1} mRNA in NCR de-repressing 
conditions and its clearance following the 
addition of glutamine. We
find that the Lsm1-7p/Pat1p complex and decapping modifiers affect
both \textit{GAP1} mRNA steady-state expression and its 
accelerated degradation.
This work expands our
understanding of mRNA stability regulation in remodeling the
transcriptome during a relief from growth-limitation and demonstrates
a generalizable approach to the study of genetic determinants of mRNA
dynamics.


%%%%%
%%%%%
%%%%%
%%%%%
%%%%%

\section{Characterizing transcriptome dynamics upon a nitrogen upshift}

\subsection{Transcriptional reprogramming precedes physiological remodeling}

Cellular responses to environmental signals entail coordinated changes
in both gene expression and cellular physiology.  Previously, we
studied the steady-state and dynamic responses of 
\textit{Saccharomyces cerevisiae} 
(budding yeast) to environmental nitrogen
\parencite{airoldi2016steady}, and found that the transcriptome is rapidly
reprogrammed following a single pulsed addition of glutamine to
nitrogen-limited cells in either a chemostat or
batch culture. To study physiological changes in response to a
nitrogen upshift, we measured growth rates of a population of 
cells. A prototrophic haploid lab strain 
(FY4, isogenic to S288c) grows with a
4.5 hour doubling time in batch culture in minimal media 
containing proline as a sole
nitrogen source (\autoref{fig:figure1a}). Upon addition of 400$\mu$M glutamine
the cells undergo a 2-hour lag period during which no change in
population growth rate is detected, but the average cell size
continuously increases ($\sim$21\% increase in mean volume 
\autoref{fig:figure1b}). Following the lag, the population adopts a 2.1 
hour doubling time.
%This lag in population growth rate upon an upshift has been 
%described before \parencite{Carter1978}.
By contrast, global gene expression changes are detected
within three minutes of the upshift \parencite{airoldi2016steady}. 
Thus, transcriptome remodeling precedes
physiological remodeling in response to a nitrogen upshift.

\afig{
  \includegraphics[width=.8\textwidth]{img/Figure1a.png}
  \includegraphics[width=.8\textwidth]{img/Figure1b.png}
  }{
    \textbf{(Top)} 
    400$\mu$M glutamine was added to a
    culture of yeast cells growing in minimal media containing 800$\mu$M
    proline as a sole nitrogen source. Measurements
    of culture density across the upshift are plotted. 
    Dotted lines denote linear regression of the
    natural log of cell density against time before the upshift and 
    after the 2 hour lag. \textbf{(Bottom)} Average cell size.
    Dotted lines denote the mean cell diameter before the upshift
    and after the 2 hour lag. 
    \label{fig:figure1a}
    \label{fig:figure1b}
  }{The nitrogen upshift of population and cellular growth rate.}

\afig{
  \includegraphics[width=\textwidth]{img/Figure1c.png}
  }{
    PCA analysis of global
    mRNA expression in steady-state chemostats and following an upshift
    \parencite{airoldi2016steady}. Steady-state nitrogen-limited chemostat
    cultures maintained at different growth rates (colored circles)
    primarily vary along principal component 2. Expression following a
    nitrogen-upshift in either a chemostat (squares) or batch culture
    (triangles) show similar trajectories and primarily vary along
    principal component 1. Grey lines depict the major trajectory
    of variation for the steady-state and upshift experiments.
    \label{fig:figure1c}
  }{Dynamics of transcriptome remodeling during a nitrogen upshift, 
    on a coarse scale.}

\afig{
  \includegraphics[width=\textwidth]{img/Figure1_S_longTermPCA.png}
  }{
  Principal components analysis (SVD) of microarray data from 
  \cite{airoldi2016steady}. 
  Colored points are from steady-state chemostats grown in
  limitation for various nitrogen sources, at different growth rates.
  Time-series experiments are show in grey points, connected by lines,
  and line-type is the type of upshift (in batch or in chemostat).
  \label{fig:longTermPCA}
  }{The coarse long-term transcriptome dynamics of a glutamine upshift.}

To evaluate concordance in transcriptome remodeling between chemostat
and batch nitrogen upshifts, and the extent to which they reflect
changes in gene expression observed during systematic steady-state 
changes in growth rates using chemostats, we
performed principal component analysis of global gene expression
(\autoref{fig:figure1c}). The first two principal components, which
account for almost half of the total variation, clearly separate
steady-state and nitrogen upshift cultures.  Systematic changes in
growth rate primarily results in
separation of gene expression states along the second principal
component, whereas upshift experiments vary along the first 
principal component.  This suggests that
although a nitrogen upshift results in a gene expression state 
reflecting increased cell growth rates \parencite{airoldi2016steady}, the
transcriptome is remodeled through a distinct state. 
In upshift experiments in
chemostats, the gene expression trajectory ultimately returns to 
the initial steady-state condition as excess nitrogen is 
depleted by consumption and dilution 
(\autoref{fig:longTermPCA}). 


To investigate the functional basis of gene expression programs
in the upshift and steady-state conditions, we computed the
correlation of each transcript with the loadings on these first two
principal components and performed gene-set enrichment analysis
(\nameref{itm:microarrayPCAgsea}). 
Component 1 is positively correlated with functions like
mRNA processing, transcription from RNA polymerases (I,II,and III),
and chromatin organization, and negatively correlated with
cytoskeleton organization,
vesicle organization, membrane fusion, and cellular respiration.
Both steady-state and upshift gene expression
trajectories increase with principal component 2, but they diverge
along principal component 1. Components 1 and 2 are 
strongly enriched for terms including ribosome biogenesis, 
nucleolus, and
tRNA processing, and negatively correlated with
vacuole, cell cortex, and carbohydrate metabolism terms. 
Together, this analysis suggests that both upshift and
increased steady-state growth rates share upregulation of
ribosome-associated components, but the reprogramming
preceding the upshift in growth reflects an immediate focus on 
gene expression machinery instead of structural cellular components.
Importantly,
dynamic reprogramming is similar in both the chemostat and batch
upshift (\autoref{fig:figure1c}). As batch cultures are a technically
simpler experimental system, we performed all subsequent experiments
using batch culture nitrogen upshifts. 

\label{subsection:pcaGoCorr}

\subsection{Global analysis of mRNA stability changes during the
nitrogen upshift}

Previously, we found that \textit{GAP1} and \textit{DIP5} mRNAs 
are destabilized in
response to a nitrogen upshift \parencite{airoldi2016steady}. We sought to
determine if mRNA destabilization is specific to NCR transporter
mRNAs by measuring global mRNA stability across the upshift
using 4-thiouracil (4tU) labeling and RNA-seq 
\parencite{neymotin2014determination,munchel2011dynamic}.
As 4tU labeling requires nucleotide transport, which may be altered
upon a nitrogen-upshift \parencite{hein1995npi1}, we designed experiments such
that following complete 4tU labeling and metabolism to nucleotides 
the chase was initiated prior to addition of glutamine or water (mock).
We normalized data using \textit{in vitro} synthesized thiolated 
spike-ins by fitting a log-linear model to spike-in counts
across time (\nameref{subsection:bff}), which reduced noise and increased
our power to detect stability changes (\nameref{itm:dme211filterModel}).
Data and models for each transcript can be visualized in browser
using a Shiny appplication (see
\url{http://shiny.bio.nyu.edu/users/dhm267/} or \nameref{subsubsection:codeanddata} ). 

We modeled the
log-transformed normalized signal for each mRNA using linear
regression (\nameref{itm:dme211resultsModel}).
Of 4,859 mRNAs measured we identified 94 that increased in 
degradation rate and 38 that decreased (FDR < 0.01, using
\cite{storey2003statistical}). 
We generated a model of nucleotide
labeling kinetics to assess the effect of an incomplete label 
chase on our experimental design ( \nameref{subsec:4tuNormalization} ),
 and found that complete transcriptional inhibition alone could 
theoretically result in a 17\% increase in the apparent 
degradation rate. In order to eliminate the possibility that
rapid synthesis changes could affect our estimates,
we only considered destabilization of at least a
doubling (100\% increase) of apparent degradation rates between 
pre-upshift and post-upshift.
This conservative cutoff 
left 78 mRNA that are significantly destabilized 
upon a nitrogen upshift. 

The vast majority of transcripts (4,781 of 4,859) do not show
individual evidence for stability changes upon addition of glutamine
(e.g. \textit{HTA1}, \autoref{fig:figure2a}). 
The median pre-upshift half-life is 6.92 minutes and the median
half-life following the upshift is 6.32 minutes (\autoref{tab:table1})
suggesting that there is not a global change in mRNA stability.
Global stability estimates are
considerably lower than previous estimates in rich medium
\parencite{munchel2011dynamic,neymotin2014determination,miller2011dynamic}, 
which may reflect the
different nutrient conditions used in our study. 
The 78 transcripts significantly destabilized upon the 
glutamine-upshift include
mRNAs encoding NCR transporters \textit{GAP1}, \textit{DAL5}, and
\textit{MEP2} (blue label, \autoref{fig:figure2a}), the pyruvate metabolism enzymes
\textit{PYK2} and \textit{PYC1} (orange label), and trehalose synthase
subunits \textit{TPS1} and
\textit{TPS2} (yellow label).
Destabilized mRNA tend to be more stable before the upshift
(\autoref{fig:figure2bc}),
(median half-life of 9.46 minutes) and exhibit 
a median 3.06-fold increase in degradation rates (median half-life of
3.02 minutes following the upshift). 

\label{subsection:stabilityChanges}

\begin{table}[h]
\small
\caption{\label{tab:table1} Summary of mRNA stability, median values}
\begin{tabular}{p{8em} | l | l | l | l | l | l}
%\toprule
& \multicolumn{2}{c}{Pre-shift} & \multicolumn{2}{c}{Post-shift} &
Change in & Fold-change\\
 & specific & half-life & specific  & half-life & specific & specific \\
 & rate & & rate & & rate & rate \\
 & (min$^{-1}$) & (min) & (min$^{-1}$) & (min) & (min$^{-1}$) & \\
\midrule
\raggedright All transcripts & 0.100 & 6.92 & 0.110 & 6.32 & 0.00865 & 1.08\\
\midrule
\raggedright Destabilized (n=78) & 0.0732 & 9.46 & 0.229 & 3.02 & 0.158 & 3.06\\
\midrule
\raggedright Unchanged (n=4781) & 0.101 & 6.89 & 0.108 & 6.40 & 0.00728 & 1.07\\
\bottomrule
\end{tabular}
\end{table}

\afig{
  \includegraphics[width=\textwidth]{img/Figure2a.png}
  }{
    \textbf{a)} 4tU-labeled mRNA from each gene was measured over time, before and
    after the addition (vertical dotted line) of glutamine 
    (nitrogen-upshift) or water (mock). Linear regression models were 
    fit to the data with a rate before the upshift (solid line) 
    and a rate after glutamine addition (dashed line). 
    \textit{HTA1} is not significantly destabilized, 
    whereas mRNAs encoding NCR-regulated transporters or 
    pyruvate and trehalose metabolism enzymes are significantly destabilized. 
    \label{fig:figure2a}
  }{4tU label-chase RNA sequencing measures mRNA stability changes 
    following a nitrogen upshift.}

We tested for
functional enrichment among the set of 78 destabilized
mRNAs and found that they are strongly enriched for NCR
transcripts (16 of 78, p < $10^{-11}$). Almost half of the
destabilized transcripts are annotated as “ESR-up” genes
(\autoref{fig:comparisonESR}), on the basis of  their rapid induction
during the environmental stress response \parencite{gasch2000genomic}. These 78
destabilized mRNA are enriched (FDR < 0.05) for GO terms and KEGG 
pathways (\nameref{itm:dme211goAndKegg}) including
glycolysis/gluconeogenesis (6 genes), 
carbohydrate metabolic process (24),
trehalose-phosphatase activity (3), 
pyruvate metabolic process (6), 
and secondary active transmembrane transport
(8, a subset of the enriched 11 ion transmembrane transport genes).
%We also see destabilization of \textit{PYK2} and \textit{HXK1},
%both of which are isozymes expressed highly in poor nutrient conditions.
Thus destabilized mRNA upon a nitrogen upshift regulates, 
but is not restricted to, NCR-regulated mRNA and reflects broader
metabolic changes in the cell. 

To investigate the extent to which mRNA stability changes contribute
to transcriptome reprogramming, we compared degradation rates
to abundance changes following the upshift 
(\cite{airoldi2016steady}, \autoref{fig:figure2bc}). 
Changes in mRNA degradation rates
and expression change rates are anti-correlated (Pearson's $r$ = -0.598,
p-value < $10^{-15}$, \autoref{fig:kkdComparison}),
consistent with stability changes impacting gene expression dynamics.
However, they are not entirely co-incident, as some destabilized
transcripts do not exhibit decreases in abundance (red points in
\autoref{fig:figure2bc}, \autoref{fig:comparisonDestabilized},
and \autoref{fig:compareSix}).
This analysis shows that increases in degradation rates play a 
significant role
in the rapid reprogramming of the transcriptome upon a glutamine
upshift, but that in some cases cases they are counteracted by
increases in mRNA synthesis rates
\parencite{shalem2008transient,canadell2015impact}.

\afig{
  \includegraphics[width=\textwidth]{img/Figure2bc.png}
  }{
    Comparison between the pre-upshift mRNA
    degradation rate (y-axis) and the post-upshift mRNA degradation rate
    (x-axis). Negative values result from noise on the slope estimate.
    Comparison between changes in mRNA expression following
    upshift \parencite{airoldi2016steady} (y-axis) and the post-upshift
    degradation rate (x-axis). Both plots share the same x-axis.
    Transcripts that are significantly destabilized are colored red, and
    shown with red rug-marks in the marginal histograms.
    \label{fig:figure2bc}
  }{Global mRNA stability changes following a nitrogen upshift.}

\afig{
  \includegraphics[width=\textwidth]{img/Figure2_S_globalComparisons.png}
  }{
  Pre-upshift decay rates \textbf{(top)} don't explain the
  abundance change. 
  The degradation rate changes (\textbf{middle}, difference between pre
  and post upshift) and the post-upshift rates \textbf{(bottom)} 
  are anti-correlated with the abundance changes.
  \label{fig:kkdComparison}
  }{Comparison between rates of mRNA degradation and abundance
    changes.}

\afig{
  \includegraphics[width=\textwidth]{img/Figure2_S_comparisonToESR.png}
  }{
  Comparisons of degradation rates from this study with mRNA abundance change rates
  from \cite{airoldi2016steady}. Destabilized transcripts are colored based on
  their membership in the ESR gene set \parencite{gasch2000genomic}, 
  as described in the supplement 
  of \cite{brauer2008coordination}. 
  Many of the destabilized set are ESR "up" genes, as they
  are increased in expression in response to stresses.
  \label{fig:comparisonESR}
  }{Many of the destabilized mRNA are members of the ESR-up regulon.}

\afig{
  \includegraphics[width=\textwidth]{img/Figure2_S_justDestabilizedDecayvsDynamics.png}
  }{
    For each, the x-axis is
    the fit rate of degradation rate post-upshift. On the y-axis is the mRNA abundance
    (expression) change rate \parencite{airoldi2016steady} after the upshift.
    These values were modeled to normalized
    sequencing signal (x-axis) and normalized microarray ratio (y-axis). The dashed
    line is a 1:1 line of equality.
    \label{fig:comparisonDestabilized}
  }{Significantly destabilized transcripts are not always strongly
    repressed.}

\afig{
  \includegraphics[width=\textwidth]
    {img/Figure2_S_sixExamplesOfDestabilizationWithoutRepression.png} 
  }{
    For several examples of the slowest decreasing (in the microarray fits)
    transcripts, we plot the microarray (abundance) and sequencing (decaying labeled
    abundance) data normalized to be on the same relative y-axis scale (subtracted
    t\_0 y-intercepts of fits).
    Destabilization does not necessarily result in a rapid clearance
    of the mRNA.
    \label{fig:compareSix}
  }{Six examples of individual mRNA whose regulation is more
    complex than a homo-directional destabilization and synthesis
    repression.}

Functional coordination of mRNA stability changes suggests  a possible
role for \textit{cis}-element regulation. We analyzed UTRs and coding
sequence for enrichment of new motifs or known RNA binding protein
(RBP) motifs.
3' UTRs of destabilized transcripts are
depleted of Puf3p binding sites, and we found no enriched sequence
motif in the 3' UTRs.
5' UTRs are enriched for a GGGG motif, which
may be explained by convergence between mRNA stability changes and
transcriptional control by Msn2/4 on the ESR “up” genes
(\autoref{fig:comparisonESR},
\cite{gasch2000genomic,canadell2015impact}). 
5' UTRs are also enriched for binding motifs reported for Hrp1p 
(\autoref{fig:hrp1}),
a canonical member of the nuclear cleavage factor I complex
\parencite{chen1998specific}.
However, this protein has been shown to shuttle to the cytoplasm
and where it may play a regulatory role
\parencite{kessler1997hrp1,kebaara2003upf,guisbert2005functional}.
On average,
destabilized mRNAs are longer and contain more optimal codons
(\autoref{fig:lengthAndCodons}, \cite{khong2017stress}). 
Together, this analysis suggests that the
mechanism of destabilization may act through cis elements in the 5'
UTR and or biased codon usage.


\afig{
  \includegraphics[width=.8\textwidth]{img/Figure2_S_averageMotifsPerSection.png}
  }{
    Sequences of destabilized and unaltered mRNAs were analyzed for
    RBP binding motif enrichment
    using the AME program in MEME, then significant hits were confirmed by using a
    logistic model predicting destabilization based on motif content per sequence
    length. Hrp1p is significantly ( p<0.0001 ) enriched in
    the 5' UTRs of destabilized transcripts. For this plot, motif matches were
    counted using the GRanges package \parencite{lawrence2013software}
     for the 5' UTRs, 3' UTRs,
    and coding sequence of transcripts using the largest isoforms detected in
    \cite{pelechano2014genome}.
    \label{fig:hrp1}
  }{Enrichment of Hrp1p motif in 5' UTRs of destabilized transcripts.}

\afig{
  \includegraphics[width=.8\textwidth]{img/Figure2_S_lengthAndCodons.png}
  }{
    Comparisons of destabilized mRNAs with the rest of the transcriptome.
    \textbf{a)} 
    Destabilized transcripts tend to have longer CDS lengths ( p-value < 2e-5
    by Wilcoxon rank sum test ). 
    \textbf{b)} 
    On average, the destabilized transcripts have more optimal codons
    than the rest of the transcriptome ( p-value < 2e-8 Wilcoxon rank
    sum test).
    The fraction of optimal codons per feature
    was obtained from the supplement of \cite{khong2017stress} using definitions
    from \cite{presnyak2015codon}. 
    \label{fig:lengthAndCodons}
  }{The destabilized set is longer and has a higher frequency
    of optimal codons than the rest of the transcriptome.}

\subsection{Methods and materials}

\subsubsection{Measurement of growth during upshift}

A single colony of FY4 was inoculated in 5mL NLimPro 
media and grown to exponential phase, then back diluted in NLimPro media
in a baffled flask. 
Samples were collected into an eppendorf, sonicated,
diluted in isoton solution, and analyzed with a Coulter Counter Z2
(Beckman Coulter).

\subsubsection{Re-analysis of microarray data} 

Supplemental files from \cite{airoldi2016steady} were 
downloaded, read
into \texttt{localc} (an open-source spreadsheet software), 
a small Excel-generated auto-correction error was 
fixed ("Oct-1" -> "OCT1"), and the file saved as a CSV. Microarray 
intensity ratios were processed with 
\texttt{pcaMethods} to perform a SVD PCA on scaled data. 

\subsubsection{Synthetic RNA spike-in generation}

\label{subsubsection:spikeins}

Poly-adenylated RNA molecules were synthesized \emph{in vitro} using a
Promega Riboprobe SP6 kit (P1420) with 4-thiouridine, as previously
described \parencite{neymotin2014determination}, to serve as
spike-in calibrators for RNAseq normalization across samples.
Ampure XP beads were used to clean up the reaction.
Products were quantified using the Qubit HS RNA assay, and 
equivalent mass amounts of spike-ins were pooled to create a 
8ng/\(\mu\)L stock containing all four 4-thiouridine-labeled 
spike-ins.

We also prepared total 4-thiouracil labeled \emph{E. coli} RNA to 
use as another spike-in. We grew strain MG1655 (a gift of Edo Kussell)
overnight in 5mL of LB with 20\(\mu\)M of 4-thiouracil, then outgrew
this for 2.5 hours in the presence of 4-thiouracil, then extracted
by boiling the pellet with 1\% SDS, 100mM NaCl, and 8mM EDTA for 5
minutes with intermittent vortexing, before cleaning up with a phenol
chloroform extraction. The product was quantified using qubit and 
diluted to a 5ng/\(\mu\)L solution of thiolated total \emph{E. coli} 
RNA.

\subsubsection{Culturing and sampling}

FY4 was grown in nitrogen-limitation conditions overnight (26 hours) 
with a mixture of 50\(\mu\)M:50\(\mu\)M of 4-thiouracil:uracil. 
The culture was split into two 450ml cultures 5 hours before
the label chase began. During exponential phase growth ( \(\sim\) 5
\(\times 10^6\) cells per mL), uracil from a 400mM DMSO stock was added
to a final concentration of 4mM (41-fold excess) to chase the label.
30mL samples from the culture were collected by filtration and
flash-freezing within a minute of removal from the flask, and 
sampling time is recorded as the time of flash-freezing. 
After letting the chase proceed for 12.5 minutes, I added glutamine
from 100mM stock (dissolved in water) to a final concentration of
400\(\mu\)M to one flask, or an equal volume of water to the control
flask.

Timepoints were chosen to sample five times before the intervention,
but timepoints actually used are the times that the sample was dropped
into liquid nitrogen for fixation.
For mock treatment (water at 13 minutes), this was 
3.85, 6.02, 7.92, 9.90, 11.8, 15.1, 17.0, 18.8, 20.8, 22.9, 26.1, 
and 50.5 minutes. 
For nitrogen upshift (glutamine at 12.5 minutes), this was
3.30, 5.32, 7.65, 9.47, 11.3, 14.4, 16.4, 18.2, 20.0, 23.8, 28.8, 
and 49.1 minutes.

\subsubsection{RNA Extraction}

Since equal volume (30mL) of culture was taken for each sample, an equal
volume of synthetic spike-ins was added to each RNA extraction reaction
(hot acid-phenol method).
The extraction yielded at least 3.3 \(\mu\)g of RNA per \(10^7\) cells.

\subsubsection{Biotinlyation and fractionation}

The total RNA (yeast and spike-ins, mixed) was reacted by adding
a total of 20$\mu$g MTSEA-biotin diluted in DMF to
200$\mu$L of HEPES buffered total RNA solution, 
and incubating for 2 hours.
Biotinylated total RNA was fractionated with streptavidin bead
pulldown, using 200\(\mu\)L of NEB (S1420S) streptavidin beads. 
These were bound for 20 minute incubation, then pulled down and
solution discarded, then washed four times with bead buffer, three
times at room-temperature and once at 65C. Labeled mRNA was eluted
using beta-mercaptoethanol, twice, then precipitated.

\subsubsection{rRNA depletion}

Fractionated RNA was depleted of rRNA using the RiboZero kit (Illumina
RZY1324) according to manufacturer instructions, except that for
every reaction of 2\(\mu\)g input RNA, we used half-reactions 
(half of every supplied reagent). 
Final RNA was ethanol precipitated, as above. Agilent
Tapestation measurements of the RNA size histograms confirmed that
virtually all of the rRNA was removed, and later computation analysis
found very few rRNA reads in the sequencing results.

\subsubsection{Preparing sequencing libraries}

RNA samples were converted into Illumina sequencing libraries using a
strand-specific (UNG) protocol.  Briefly, 1st strand cDNA was 
synthesized using a SuperScript III kit
(Invitrogen), primed with random hexamers. RNA was fragmented by
98C hybridization for 1 minute in 4.17mM MgCl$_2$, then actinomycin
and the SuperScriptIII enzyme were added after annealing.  
This was incubated to extend first strand in a series of increasing
temperature steps over the course of an hour.
This reaction was ethanol precipitated
and resuspended for a second-strand reaction that incorporated dUTP
in the place of dTTP, using a cocktail of DNA PolI, Ecoli ligase, and
RNAseH. This was reacted at 16C for 2 hours, then cleaned up on
MinElute nucleic acid purification columns (Qiagen 28004).
This product was eluted, end-repaired using T4 DNA polymerase and PNK,
then cleaned up on MinElute columns.
This product was eluted, A-tailed with Klenow (exo-), 
then cleaned up on MinElute columns.

This product, fresh from A-tailing, was ligated with "TrUMISeq"
adapters made by a former graduate student in the lab
\parencite{hong2017method}. These are TruSeq adapters, but the
sample index has been incorporated as the first six bases sequenced
from the sequencing priming site and a final T exists to help with
ligation. In place of the sample index (interior to the adapter),
a degenerate sequence is incorporated during synthesis.
These adapters, theoretically, mark each unique ligation event
with a DNA barcode sampled from a degenerate pool of $4^6 = 4096$
different barcodes. This greatly reduces the chance that two
molecules that appear to be PCR duplicates (false double-counting
by virtue of the amplification scheme) are actually considered
to be duplicates in the analysis. For a more in-depth discussion of
UMIs, please see
\nameref{subsubsection:rarefractionDiscussion}.

These ligations were all of the A-tailed product with 20nM annealed
adapter, each reaction with a different sample index. These were
reacted using Quick Ligase (NEB), reacted at 22C for 15 minutes
before immediate clean-up with Ampure XP beads (BeckmanCoulter).
These were selected twice on beads to remove small adapters.
To amplify libraries and select the strand-specificity, I prepared a
master-mix of NEB Phusion buffer with primers DGO366 and DGO367,
reacted this with UNG (Thermo EN0361) to digest the dUTP containing
strand, then added NEB Phusion polymerase and PCR amplified for
18 cycles.
These reactions were cleaned up using a MinElute column, then diluted
and concentration estimated using qPCR on a Roche 480 (using KAPA
Library Quant Kit Illumina REF 07960281001), and submitted as a 1nM pool
to the NYU GenCore system for sequencing on a NextSeq using the 75bp
format in High-Output mode.

\subsection{4tU label-chase sequencing analysis and modeling}

\subsubsection{Quantifying sequencing reads}

Following base-calling and demultiplexing by NYU GenCore, the sequencing
reads were quantified using the following pipeline:

\begin{enumerate}
  \setlength\itemsep{0em}
  \item Raw reads were trimmed using \texttt{cutadapt} \parencite{martin2011cutadapt}
  \item Trimmed reads were aligned using \texttt{tophat2} \parencite{kim2013tophat2}
    to a reference genome that included the yeast reference 
     genome (assembly R64), the Ecoli genome (assembly 
     GCF\_000005845.2), and the four synthetic in-vitro transcribed 
     spike-ins (termed BES and available in the \texttt{data.zip} 
     archive of the OSF archive associated with this paper). 
     This was done with parameters optimized 
     against \emph{in silico} data generated by Flux Simulator
     \url{http://sammeth.net/confluence/display/SIM/Home} from this 
     reference genome, in replicates.
  \item Reads with mapping quality above 20 and with at least 50 
    matched bases with 
    \texttt{samtools} \parencite{li2009sequence}
    were processed with \texttt{umi\_tools} \parencite{smith2017umi}
    in ``dir'' mode to
    de-duplicate possible PCR duplicates.
  \item The demultiplexed \texttt{.bam} file was processed with the
    \texttt{htseq-count} \parencite{anders2015htseq}
    script to generate counts files per gene feature (according to 
    the GFF file in the \texttt{data/BES} directory).
\end{enumerate}

\subsubsection{Normalization of counts into signal for modeling}

\label{subsec:4tuNormalization}

In order to accurately estimate degradation rates, I must accurately
estimate labeled mRNA abundance within each timepoint. This is
achieved by normalizing the signal of the mRNA (counts) to the signal
of a spiked-in reference transcript pool (the four spike-ins added
during extraction).

The simplest normalization is to divide each feature counts
by the sum of the counts of all the spike-ins (personal communication,
Daniel Tranchina). However, the low RNA abundances of several 
samples in this experiment had poor quantification of the spike-in 
which required me to remove outlier measurements to prevent 
systematically noisy data from
disrupting the quantification. 

I decided to smooth the normalization across the samples, as we expect
the proportion of counts that are the spike-ins to increase over time.
This assumes that the whole transcriptome decays with exponential
kinetics \parencite{petersen1976half}.

In the sequencing data, we clearly see that the proportion of counts
that are the spike-ins increases with time.
\autoref{fig:propIncrease}.
We modeled this increase using the \texttt{lm} function to linearly
regress this in natural-log-space. We see that
the residuals are randomly distributed around the fit across time 
for both treatments \autoref{fig:propIncrease}.
Using an ANCOVA (\texttt{aov}/\texttt{lm}), I found the effect of 
treatment was associated with a p-value < 0.301 and the p-value
associated with time estimated as ``1'', so it does not appear that the
residuals particularly depend on time or treatment.

\afig{
  \includegraphics[width=.49\textwidth]{img/fig2s_increase_props.png}
  \includegraphics[width=.49\textwidth]{img/fig2s_increase_residuals.png}
  }{
  The observed proportion of the reads is on the left, and the
  residuals against this fit model is on the right. 
  \label{fig:propIncrease}
  }{The proportion of spike-in counts increase over the course of the
    experiment.}

How do the normalizations compare on a per-gene basis? 
\ref{fig:normgenes} compares this.


\afig{
  \includegraphics[height=7in]{img/fig2s_norm_example.png}
  }{
  The results of the two strategies are shown on the left (direct
  normalization) and on the right (model-based). For several example
  genes, we see that the model-based smoothing reduces the noise.
  \label{fig:normgenes}
  }{Several examples of alternative normalization strategies.}

The direct normalization of each gene in each timepoint combines the 
noise of both the measured gene and the measured spike-ins. 
By smoothing across timepoints, 
given our expectation of whole-transcriptome exponential
decay dynamics, we can deliver a more reliable estimate for each gene
feature. Thus more observations can be reliably analyzed
without single sample spike-in errors systematically skewing the fits
across all features.

We also tried to spike-in labeled \textit{E. coli} total RNA; 
however, we found
that those counts were low, noisy, and did not behave as expected
from previous work \parencite{neymotin2014determination}. We
hypothesize that this was due to lower addition of \textit{E. coli} 
total RNA than synthetic spike-ins, 
combined with noise associated with amplifying a
random sub-sample of a more complex spike-in pool of total \textit{E.
coli} RNA. Each sub-sample may have very different GC-content, and
thus may be amplified to a different degree.
Thus, we normalized all yeast mRNA to the synthetic spike-ins previously
demonstrated.
Sampling from a complex spike-in library may be an important
consideration for their use in normalization, and may warrant further
technical experimentation.

\subsubsection{Model of transcript dynamics as a function of 
dynamics and labeling parameters}

To analyze this data, I fit a model of 
labeled transcript dynamics.
I used this to analyze the dataset for expected label-chase
dynamics, and also to exclude effects that may result from a
confounding of new synthesis with changes in degradation rates.

\(m_t\) is the labeled mRNA at time \(t\). It changes according to the
equation: \[ \frac{d m_t}{dt} = L k_s - k_d m_t\] where \(L\) is the
fraction of new mRNA that is labeled and pulled down, \(k_s\) is the
rate of synthesis, and \(k_d\) is the rate of degradation. Our
experimental design is to change \(L\) from an initial fraction of
transcripts that are pulled down by a 4tU-incorporation-dependent
mechanism of \(L^o\) (old) to a new fraction \(L^n\) (new). Note that
I use the notation as a superscript, so that I can also share that
notation with the synthesis rates as \(k_s^o\) and degradation rates as
\(k_d^o\).

I assume that the culture begins at a steady state of
\(L^{o}\frac{k_s^{o}}{k_d^{o}}\), from solving the above equation.
I assume this because I grow the cells for 24 hours in conditions
of labeling, and they are well below saturating conditions.
Solving the above differential equations with the assumption that
everything changes once, which is a simplifying assumption but supported
by previous studies of transcript stability changes during shifts
\parencite{perez2013eukaryotic}, I expect that \(m_t\) should behave
as, \[ m_t = L^o \frac{k_s^o}{k_d^o} e^{-k_d^n t} + 
  L^n\frac{k_s^n}{k_d^n}(1-e^{-k_d^n t}) \] Nicely, the solution is
similar to what I would expect intuitively - extant transcripts decay
(left), and nascent transcripts approach the new equilibrium (right).
The equilibriums are set by all parameters, but the change between them
is dictated by the new degradation rate operative during the transition.

In the case were either \(L^o\) or \(L^n\) is 0, then the transcript
behaves just as one side of the equation. With the label-chase, I are
trying to get \(L^n\) as low as is possible without perturbing the
system being measured by killing the cell.
To analyze this dataset for potential changes in transcript stability,
I approximated this by fitting a linear regression model to the
normalized signal. I explore the sufficiency of this model later in
this document using simulations. This model was fit using the
\texttt{lm} function in R, with the formula

\begin{verbatim}
log( NormalizedSignal ) ~ Minutes + Minutes:Treated + 1
\end{verbatim}

where ``\texttt{NormalizedSignal}'' is the signal of the gene feature
normalized as described in the previous section, ``\texttt{Minutes}'' is
minutes relative to the glutamine (or water) addition,
\texttt{Minutes:Treated}'' is an additional slope of the observations
after glutamine addition, and ``\texttt{+\ 1}'' denotes to fit a single
intercept for the model (data are centered around the moment of 
treatment, $t_0$ is the addition of glutamine or water). 
From this fit, I took the p-values associated
with the t-statistic of the additional slope fit to the glutamine
treated samples, then adjusted the p-values using the \texttt{qvalue}
package from BioConductor using default settings. I chose to use a FDR
cut-off of less than 0.01 for this analysis.

\subsubsection{Estimating possible effects of synthesis changes on 
labeled abundance}

In our experimental design I initially grow the cells in a
50\(\mu M\):50\(\mu M\) mix of uracil and 4-thiouracil, so I will set
as a labeling ratio \(L^o\) of 1 for simplicity. I add 4,000 \(\mu M\)
uracil to begin the chase, so this is a shift to a \(L^n\) of
\(\frac{50 \mu M}{4100 \mu M} / \frac{50 \mu M}{100 \mu M}\) , or
\(\frac{1}{41}\). Since I am not reducing this number to zero, there
is still residual labeling incorporated into nascent transcription.
\(\frac{1}{41}\) is a small number, but is still not zero and should
not be neglected.
Thus, there is a potential that residual label could confound our
estimate of degradation rates. This is an inherent tradeoff in a
label-chase design \parencite{perez2013eukaryotic}, 
especially since the low RNA content of the cells
and low cell density in these nitrogen limited conditions make necessary
the use of a more efficient pull-down reagent (MTSEA-biotin). This could
be circumvented by comparing abundance and synthesis measurements, but
the uracil transporter responding to glutamine in the media makes this
technically difficult with 4tU incorporation. Comparing abundance and
mRNA synthesis by other means is feasible, but introduces a compounding
of errors from both methods. Thus performing one direct assay is
preferable for precision, and the drawbacks described here are
unavoidable with current technology (although progress is being made
\parencite{chan2017non}).

Therefore, I used simulations to investigate how varying the labelling
parameter changes the expected dynamics if we also vary the synthesis
parameter. \autoref{fig:modelingNoChange} (left) shows a plot of the modeled
labeled transcript abundance, with no change in synthesis parameter.
How does this estimate of change in degradation look if we decrease the
\(k_s\)? For example, the NCR regulon is expected to be shut-off at the
synthesis level quickly upon glutamine addition, so how would that swift
repression affect the apparent change in labeled mRNA dynamics?
\autoref{fig:modelingChange} (right) shows a plot of the modeled
labeled transcript abundance, with a $k_d$ of 0.1 (similar to the
median rate of degradation \autoref{fig:figure2bc}) 
and an instantaneous shutoff of transcript synthesis.

\afig{
  \includegraphics[width=.49\textwidth]{img/fig2s_no_change_synth.png}
  \includegraphics[width=.49\textwidth]{img/fig2s_shutoff_synth.png}
  }{
  \textbf{(Left)} is with no change in synthesis.
  \textbf{(Right)} is with a complete halt in synthesis. 
  \label{fig:modelingNoChange}
  \label{fig:modelingChange}
  }{Modeling of labeled transcript dynamics assuming an incomplete 
    chase and synthesis rate changes.}

The effect of reduced synthesis on apparent slope change of the 
labeled RNA here is a 13.3\% increase in the rate.
I conclude that there's a slight effect from synthesis rate changes.
For transcripts that are putatively stabilized, the effect is 
much larger ($\sim$60\%), likely because of the delay in pertubation
until 13 minutes after the label chase begins.

What does this mean for our estimates of destabilization? What effect
sizes are estimated, and how do they compare to this effect size?
\autoref{fig:changeRateDist} shows the distribution of the fold changes in
stability:

\afig{
  \includegraphics[width=.7\textwidth]{img/fig2s_distribution_of_rates.png}
  }{
  Histogram of the fold-changes in rate of labeled transcript
  clearance upon glutamine treatment, determined as significant by 
  adjusting the parametric (t-statistic of slope) with a
  multiple-hypothesis correction (\texttt{qvalue} package
  \cite{storey2015qvalue}). The blue line denotes a 13.3\%
  increase in the rate of clearance, the red line denotes a doubling
  of the rate. Some transcripts appear stabilized (left of 0), most
  are destabilized, many are destabilized more than the red cutoff.
  \label{fig:changeRateDist}
  }{Distribution of all fit changed rates versus effect size 
    thresholds from modeling.}

I see that all of the significant changes are in great excess to that
blue line. To be careful, I choose to use a cut off of a 100\%
increase, a doubling, of apparent degradation rate to call a feature
destabilized (right of the red line). Since I cannot place an upper
bound on the synthesis rates after a glutamine upshift, I cannot
definitively say that the potentially stabilized transcripts (left of 0)
are stabilized without additional experiments.
Could these fits just be on the right side of the blue line by chance?
Given that the t-statistics (ratio of effect size over standard error)
for the fits of ones over this line are a median of -5.66,
I'm not going to have fits within several standard errors of
crossing that threshold by a reasonably expected error.

I conclude that the RNA from 78 gene
features appear to be degraded much more quickly than can be reasonably
explained by labelling carry-over, and are thus accelerated in
degradation upon the nitrogen upshift.
I encourage the interested reader to go to
\url{http://shiny.biology.nyu.edu/users/dhm267} to explore examples of
the data under both normalizations and for a range of features.

\subsubsection{\textit{Cis} element analysis}

To detect if \textit{de novo} or known \textit{cis} elements were 
associated with destabilization upon a nitrogen upshift,
I used a variety of bioinformatic methods.  
For each transcript, I used a GFF file to extract the coding sequence
of each annotated mRNA and four different definitions of it's
untranslated regions. This is an ambiguous definition, and a more
rigorous definition using 5' and 3' end sequencing methods in this
particular condition would be necessary for best exploring this with
certainty. 

Here, I used four definitions --- 
200bp upstream of the start codon or downstream
of the stop codon, the largest detected isoform in TIF-seq from
\parencite{pelechano2014genome}, 
or the most distal detected gPAR-CliP sites in
exponential-phase or nitrogen-limited growth in 
\parencite{freeberg2013pervasive}.
To find putative cis-elements, 
I used DECOD \parencite{huggins2011decod}, 
FIRE \parencite{elemento2007universal},
TEISER \parencite{goodarzi2012systematic},
and the \#ATS pipeline \parencite{li2010predicting},
I also scanned for RBP binding sites from
CISBP-RNA \parencite{ray2013compendium}
using AME from the MEME suite \parencite{mcleay2010motif}.

The enrichment for the Hrp1p motif is described in \autoref{fig:hrp1}.
It is worth noting that this is the core-element of Hrp1p (also known
as Nab4p) described in \cite{guisbert2005functional,chen1998specific}, 
but is mistakenly recorded in the CISBP-RNA database as being the 
motif for Npl3p. 
\cite{guisbert2005functional} did not find a motif for Npl3p.

%%%%%
%%%%%
%%%%%
%%%%%
%%%%%

\section{Estimating \textit{GAP1} mRNA for every mutant in a pool}

\subsection{A genome-wide screen for \textit{trans}-factors 
  regulating \textit{GAP1} mRNA repression}

We sought to identify \textit{trans}-factors mediating accelerated mRNA
degradation in response to a nitrogen upshift. We selected \textit{GAP1} 
as representative of transcript destabilization, as it is abundant in
nitrogen-limiting conditions and is rapidly cleared upon addition of
glutamine  (3.24-fold increase in degradation rate, \autoref{fig:figure3a},
\nameref{itm:dme211resultsModel}). Previous approaches to high-throughput
genetics of transcriptional activity have used protein expression
reporters \parencite{neklesa2009genome,sliva2016barcode} or automation of qPCR 
\parencite{worley2016genome}. However, for our
purposes, we required direct measurement of \textit{GAP1} mRNA 
changes on a rapid timescale.
Therefore, we applied single molecule fluorescent \textit{in situ}
hybridization (smFISH) to quantify 
native \textit{GAP1} transcripts in yeast cells in the pooled
prototrophic yeast deletion collection \parencite{vandersluis2014broad}.
Using fluorescence activated cell sorting (FACS) and Barseq
\parencite{smith2009quantitative,robinson2014design,giaever2014yeast},
we aimed to quantify and model the distribution of \textit{GAP1} mRNA
in each mutant \parencite{kinney2010using,peterman2016sort}.

\afig{
  \includegraphics[width=.6\textwidth]{img/Figure3a.png}
  }{
  \textit{GAP1} mRNA following upshift measured using RT-qPCR, 
  relative to an external spike-in mRNA standard. The dashed line 
  is fit to points after 2 minutes. This shows that the rapid
  clearance of \textit{GAP1} mRNA is not an artifact of normalization
  to an increasing internal reference.
  \label{fig:figure3a}
  }{\textit{GAP1} mRNA dynamics by spike-in normalized qPCR.}

%Development of our screen required that we could detect and
%sort cells using \textit{GAP1} mRNA signal. 
We found that
individually labeled probes tiled across \textit{GAP1} mRNA
\parencite{raj2008imaging} were insufficiently bright for
\textit{GAP1} mRNA quantification using flow cytometry (data not shown),
likely due to the small cell size of nitrogen-limited cells and the
low transcript numbers in yeast cells compared to mammalian cells
\parencite{klemm2014transcriptional}. Therefore, we used branched DNA probes
(Quantigene), which serve to amplify the FISH signal
\parencite{hanley2013detection}. We developed a fixation and permeabilization
protocol (\nameref{subsection:bff}) that enabled detection of the
distribution of  \textit{GAP1} mRNA in steady-state nitrogen-limited conditions
and its repression following the  upshift (\autoref{fig:figure3b}). In control
experiments, we found that the signal is eliminated in a \textit{GAP1} deletion
or by omitting the targeting probe% that confers specificity
(\autoref{fig:figure3}b and \autoref{fig:gap1Delete}). To validate
sorting, we sorted a sample of cells into quartiles and used
microscopy to count fluorescent foci per cell
(\autoref{fig:figure3}c) .
We found that increased flow cytometry signal is associated with an
increase in the number of foci in the cells (\autoref{fig:figure3}d, $R^2$ = 0.607,
p < $10^{-11}$ ). 

\afig{
  \includegraphics[width=.7\textwidth]{img/Figure3_S_gap1deleteControl.png}
  }{
  Wild-type or \textit{gap1}$\Delta$ cells were grown in 
  proline-media, which induces expression of \textit{GAP1}. 
  As seen in the positive control, there is heterogeneity in the
  induction. This is likely due to technical issues, namely over
  fixation. Importantly, a \textit{gap1}$\Delta$ does not have any
  positive signal.
  \label{fig:gap1Delete}
  }{\textit{GAP1} delete or omission of the targeting probe removes
    signal of GAP1 FISH.}

\afig{
  \includegraphics[width=.7\textwidth]{img/Figure3b.png}
  }{
  Flow cytometry of wild-type yeast in nitrogen-limited conditions 
  and following an upshift. The vertical grey lines indicate FACS 
  gate boundaries used for cell sorting, for \autoref{fig:figure3c}
  microscopy.
  \label{fig:figure3b}
  }{\textit{GAP1} mRNA dynamics measured by flow cytometry.}

\afig{
  \includegraphics[width=.7\textwidth]{img/Figure3c.png}
  }{
  Representative cells from each bin sorted from the experiment in 
  \autoref{fig:figure3b}. Alexa647 signal is imaged with a Cy5
  filter set.
  \label{fig:figure3c}
  }{Example microscopy of cells sorted by \textit{GAP1} mRNA.}

\afig{
  \includegraphics[width=.7\textwidth]{img/Figure3d.png}
  }{
  Cells sorted from \autoref{fig:figure3b} were manually scored in
  z-stacks for Alexa647 foci.
  Each black dot represents the total for a single cell. 
  The mean number of foci per cell in each bin is displayed as a red point.
  \label{fig:figure3d}
  }{Quantification of microscopy of cells sorted by \textit{GAP1} mRNA.}

Previous SortSeq studies of
the yeast deletion collection have used outgrowth 
to generate sufficient material for 
Barseq \parencite{sliva2016barcode}. However, formaldehyde fixation precludes
outgrowth. We found that below approximately $10^6$ templates, the
Barseq reaction produces primer dimers
that outcompete the intended PCR product (\nameref{subsection:bff}). 
Therefore, we re-designed the
PCR reaction \parencite{robinson2014design,smith2009quantitative} to be robust for
very low sample inputs (\nameref{subsection:bff}). Our protocol
incorporates a 6-bp unique molecular identifier (UMI) into the first
round of extension to identify PCR duplicates, 
and uses 3'-phosphorylated oligonucleotides and a
strand-displacing polymerase (Vent exo-) to block primer dimer formation and 
off-target amplification. 
%We developed a bioinformatics pipeline 
%using pairwise alignment
%for per-read quality-filtering and compatibility with variable barcode
%length, and using the degenerate UMI barcodes to help account for PCR
%duplicates. 
%UMIs to identify duplicates.
Because strain barcodes are of variable lengths, 
we developed a bioinformatic pipeline to extract barcodes and UMIs 
using pairwise alignment to invariant flanking sequences.
Based on \textit{in silico} benchmarks, this
approach was robust to systematic and simulated random errors 
that can confound analysis of the yeast deletion barcodes 
(\nameref{subsubsection:codeanddata}, \nameref{subsection:bff}). 

We refer to this experimental approach as BFF (Barseq after FACS after FISH). 
We used BFF to estimate \textit{GAP1} mRNA abundance for every mutant in the
haploid prototrophic deletion collection
\parencite{vandersluis2014broad} in
nitrogen-limiting conditions and 10 minutes following the upshift. 
This approach facilitates identification of mutants with
defects in mRNA regulation at both the transcriptional and
post-transcriptional level without altering \textit{GAP1} mRNA 
\textit{cis}-elements that may affect its regulation. 
Moreover, this design enables identification of factors that 
regulate both the steady-state abundance of \textit{GAP1} mRNA and 
its transcriptional repression following an upshift.
We analyzed the deletion pool in biological triplicate
(\autoref{fig:figure4a}). We found that UMIs 
approached saturation at a slower rate than expected for random sampling,
consistent with PCR amplification bias 
(\autoref{fig:umi}), and therefore we adopted the 
correction of \cite{fu2011counting} (dicussed in depth later). 
After filtering, we calculated a
pseudo-events metric that approximates the number of each mutant sorted
into each bin. 
Principal components analysis shows that the samples are 
separated primarily by FACS bin within each
condition and biological replicates are clustered indicating that our
approach reproducibly captures the variation of  \textit{GAP1} mRNA flow
cytometry signal across the library (\autoref{fig:pca}). 



\afig{
  \includegraphics[width=\textwidth]{img/Figure4_S_PCAonFilteredQCdData.png}
  }{
    Each
    color is a type of sample, from low to high gates (with black denoting the
    input samples before sort). Technical replicates are connected by dashed lines,
    biological replicates are each letter A B or C. At top, the first two prinicpal components
    show the separation of gates by signal intensity, and reflects that the lower
    gates on the upshifted samples were very close (blue and red samples on far right
    panel), within the distribution of the negative population. This is consistent
    with their tight sampling of the "GAP1-off" population, as seen in
    \autoref{fig:figure4a}.
    \label{fig:pca}
  }{Principal components analysis of the abundance estimates for
    samples.}

\subsection{Estimating \textit{GAP1} mRNA abundance for individual mutants}

We estimated the distribution of \textit{GAP1} mRNA for each mutant by
modeling pseudo-events in each quartile as a
log-normal distribution using likelihood maximization  
(\autoref{fig:figure4b}). 
%Specifically, replicate A had a consistently lower estimate of
%\textit{GAP1} FISH fluoresence in both flow cytometry and modeling.
%Replicate C had fewer mutants sorted (\autoref{subsection:bff}), 
%reflected in the wider distribution of estimated means.
From model fits we estimated the mean expression value for each
mutant and found that the distribution of means estimated for
3,230 strains (\nameref{itm:dme209pooledFits}, \autoref{fig:figure4c}) 
recapitulates the overall
distribution of flow cytometry signal (\autoref{fig:figure4a}). 
%To estimate \textit{GAP1}
%mRNA per strain, we used all replicate measurement to perform model
%fitting and filtered models for sufficient measurements  (at least two
%of three replicates in at least three of the four bins). We generated
%expression distribution estimates for 3,230 strains, and used the mean
%of each distribution as the estimate of \textit{GAP1} mRNA abundance for each
%strain (\nameref{itm:dme209pooledFits}). %added after commented out
To validate our approach we first examined
strains for which we expected to have a specific phenotype and
compared their mean expression level to the distribution of expression
for the entire population (\autoref{fig:figure4d}). 

\afig{
  \includegraphics[width=.8\textwidth]{img/Figure4b.png}
  }{
    Measurements for individual genes before and
    after the upshift. Black dashed lines indicate maximum-likelihood 
    fits of a log-normal to pseudo-events for each mutant. 
    Each color is a biological replicate.
    \label{fig:figure4b}
  }{Data and models for several individual sample genes from the BFF
    modeling.}
\afig{
  \includegraphics[width=\linewidth]{img/Figure4a.png}
  }{
    Flow cytometry analysis of \textit{GAP1} mRNA 
    abundance in the prototrophic
    deletion collection before and after the upshift. 
    The vertical gray
    lines denote FACS gates. Biological replicates are
    indicated by color. 
    \label{fig:figure4a}
  }{Flow cytometry of the \textit{GAP1} mRNA distribution across all
    mutants.}
\afig{
  \includegraphics[width=\textwidth]{img/Figure4c.png}
  }{
    Distribution of mean modeled GAP1 mRNA levels for each mutant.
    \label{fig:figure4c}
  }{Global distribution of BFF estimates of \textit{GAP1} mRNA abundance.}
\afig{
  \includegraphics[width=.8\linewidth]{img/Figure4d.png}
  }{
    The mean \textit{GAP1} mRNA expression levels for 
    individual mutants before and after
    the upshift are shown as points connected by a line, colored
    according to the type of gene. 
    For reference, the background violin plot shows the distribution 
    of all 3,230 mutants fit.
    \label{fig:figure4d}
  }{Individual BFF estimates of \textit{GAP1} mRNA abundance for
    mutants known to play a role in \textit{GAP1} regulation.}

We found that the wildtype
genotype (\textit{his3}$\Delta$, complemented by the spHis5 in
library construction) has an expression level that is centrally
located in the distribution both before and following the upshift. The
\textit{gap1}$\Delta$ genotype is a negative control and 
we estimate that it is at the extreme
low end of the distribution before and following the upshift. 
\textit{dal80}$\Delta$ is a direct transcriptional repressor
of NCR transcripts %like \textit{GAP1}, 
and we found that this is defective in
repression of \textit{GAP1} before and after the upshift. 
Counter-intuitively, deletion of \textit{GAT1}, a transcriptional activator
of \textit{GAP1}, appears to have higher steady-state expression of
\textit{GAP1} mRNA.
However, increased expression of \textit{GAP1} mRNA in a
\textit{gat1}$\Delta$ background has
previously been reported \parencite{scherens2006identification} and is thought to
result from the complex interplay of NCR transcription factors on
their own expression levels. 
Data and models for each mutant strain can be visualized in browser
using a Shiny appplication (see
\url{http://shiny.bio.nyu.edu/users/dhm267/} or \nameref{subsubsection:codeanddata}). 

To identify new cellular processes that regulate \textit{GAP1} mRNA abundance, we
used gene-set enrichment analysis (\nameref{itm:dme209gsea}).
Following the upshift we found mutants that have high \textit{GAP1}
expression before the upshift are enriched for mutants in sulfate
assimilation, and mutants that 
maintain high \textit{GAP1} mRNA expression are enriched for negative
regulation of gluconeogenesis (\autoref{fig:gluco}).
\afig{
  \includegraphics[width=.49\textwidth]{img/Figure5_S_negGluconeogenesis.png}
  \includegraphics[width=.49\textwidth]{img/Figure5_S_sulfateAssimilation.png}
  }{
  Knock-out mutants of involved in negative regulation of
  gluconeogenesis (left) or sulfate assimilation (right)
  are associated with higher \textit{GAP1} expresion 
  after or before (respectively) the upshift, 
  by GSEA analysis of GO-terms (p-value < 0.05).
  \label{fig:gluco}
  \label{fig:sulfate}
  }{Mutants of negative regulators of gluconeogenesis or sulfate 
    assimilation are associated with defects in \textit{GAP1} 
    expression.}
However, the strongest enrichment for high \textit{GAP1} expression 
was components Lsm1-7p/Pat1p complex (\autoref{fig:figure5a}). 
Mutants in the TORC1 signalling pathway were not enriched,
but, I did find that a \textit{tco89}$\Delta$ mutant has
greatly increased \textit{GAP1} mRNA expression before and after the 
upshift (\autoref{fig:tco89}), consistent with the repressive role 
of TORC1 on the NCR regulon.
\afig{
  \includegraphics[width=.7\textwidth]{img/Figure4_S_poorlyQuantifiedStrains.png}
  }{
    Data and fits for several mutants. \textit{xrn1}$\Delta$ 
    mutant (left) is lowly abundant in
    the library and is only observed in the highest bin of \textit{GAP1} signal, consistent
    with the role of Xrn1p as a global exonuclease. 
    \textit{tco89}$\Delta$ is the only detected member that would abrogate TORC1 activity.
    This mutant (right) has elevated \textit{GAP1} mRNA before and after the upshift,
    consistent with the role of TORC1 in repressing the NCR regulon. 
    \label{fig:tco89}
  }{\textit{tco89}$\Delta$ and \textit{xrn1}$\Delta$ show
    defects in \textit{GAP1} mRNA regulation in the BFF assay.}
To compare expression before and after the upshift for each mutant,
we regressed the post-upshift mean expression against the pre-upshift 
mean expression for each genotype (\autoref{fig:prePredictPost}). 
\afig{
  \includegraphics[width=.8\textwidth]{img/Figure4_S_PreShiftPredictingPostShiftLM.png}
  }{
    The relationship between the estimated mean before the upshift and after the
    upshift. Scatter plot of the estimated means, with marginal histograms along
    top and right. Red vertical line on top histogram is a cut-off of
    \textit{GAP1} mRNA induction for this analysis,
    and is the mean of the fit to wild-type minus the standard deviation of that
    distribution. The red linear regression line is fit to all points above this
    threshold, in which expression was detected before the upshift.
    \label{fig:prePredictPost}
  }{The relationship between the estimated mean before the 
    shift and after the upshift.}
We used the residuals for each
strain to identify mutants that clear \textit{GAP1} mRNA with kinetics slower
than expected by this model.
We found that the Lsm1-7p/Pat1p complex is again strongly 
enriched for slower than
expected \textit{GAP1} mRNA clearance (\nameref{itm:dme209pooledFits}). 
Specifically
the \textit{lsm1}$\Delta$, \textit{lsm6}$\Delta$, and 
\textit{pat1}$\Delta$ strains are highly elevated in \textit{GAP1}
expression before the upshift and strongly impaired in the 
repression of \textit{GAP1} mRNA after the upshift
(\autoref{fig:figure5a}). 

\afig{
  \includegraphics[width=\linewidth]{img/Figure5a.png}
  }{
    In the background is the distribution of 
    fit \textit{GAP1} mRNA mean expression levels for all mutants
    in the pool. Indicated by colored points and lines are the means for
    individual knockout strains, as labeled.
  \label{fig:figure5a}
  }{Disrupting the Lsm1-7p/Pat1p complex impairs
    clearance of \textit{GAP1} mRNA.}

\subsection{Testing the roles of decapping modulators and associated
components}

As these factors are associated with processing-body dynamics, 
we tested if microscopically-observable processing-bodies form or
disassociate during the upshift, using microscopy of Dcp2-GFP. 
We did not observe qualitative changes
in Dcp2-GFP distribution (\autoref{fig:pbodyScope}),
and thus the upshift does not
result in a microscopically visible changes in processing-body foci
as seen in other stresses. This is consistent with previous
investigations of amino-acid limitation stress
\parencite{hoyle2007stress} and
suggests that the defects in \textit{GAP1} mRNA clearance likely 
result from their roles in decapping or associated processes.

\afig{
  \includegraphics[width=\textwidth]{img/Figure5_S_pbodyMicroscopy.png}
  }{
  A strain harboring a copy of Dcp2p-GFP expressed from a plasmid
  was grown in conditions of exponential phase in YPD or 10 minutes of
  starvation in water (first row). Starvation in water is a common 
  condition known to result in the strong formation of processing-body 
  foci of Dcp2-GFP, and is thus a positive control.
  The bottom row shows microscopy during the upshift.
  We do not see either formation or dissolution of Dcp2-GFP foci 
  resembling p-bodies during the nitrogen upshift.
  \label{fig:pbodyScope}
  }{Processing-body dynamics are not associated with the
    nitrogen upshift, by Dcp2p-GFP microscopy.}

To confirm the role of the Lsm1-7p/Pat1p  complex in clearing \textit{GAP1}
mRNA during the nitrogen upshift we measured \textit{GAP1} mRNA
repression using qPCR normalized to
\textit{HTA1}, which is not subject to destabilization upon the upshift
(\autoref{fig:figure2a}). We also tested mutants that were not detected using BFF,
or were only detected in the highest \textit{GAP1} bin and therefore
not suitable for modeling
(e.g. \textit{xrn1}$\Delta$ \autoref{fig:tco89}). 
Using this assay we found that the main 5'-3' 
exonuclease \textit{xrn1}$\Delta$ 
and mRNA deadenylase complex (\textit{ccr4}$\Delta$ and
\textit{pop2}$\Delta$) are impaired in \textit{GAP1} repression 
(\autoref{fig:figure5qpcr}).
We found that qPCR confirms results from BFF.
We confirmed that the accelerated degradation of \textit{GAP1} mRNA is impaired
in \textit{lsm1}$\Delta$ and \textit{lsm6}$\Delta$ 
(\autoref{fig:figure5qpcr}). 
We also tested
\textit{scd6}$\Delta$ and \textit{edc3}$\Delta$, two modifiers of the
decapping or processing-body
assembly functions associated with this complex, and found two
distinct phenotypes (\autoref{fig:figure5qpcr}). \textit{edc3}$\Delta$ has similar expression 
as wild-type before the upshift, but is cleared much more slowly.
\textit{scd6}$\Delta$ has a greatly reduced \textit{GAP1} expression
before the upshift but is impaired in \textit{GAP1} clearance. 
\textit{tif4632}$\Delta$, a deletion of the eIF4G2
known to interact with Scd6p \parencite{rajyaguru2012scd6}, 
has a similar phenotype. 

\afig{
  \includegraphics[width=.49\textwidth]{img/Figure5b.png}
  \hfill
  \includegraphics[width=.49\textwidth]{img/Figure5c.png}
  \includegraphics[width=.49\textwidth]{img/Figure5d.png}
  \hfill
  \includegraphics[width=.49\textwidth]{img/Figure5_S_bothutr.png}
  }{
    Points are the ratio of \textit{GAP1} mRNA to \textit{HTA1} mRNA 
    before and 10 minutes after a glutamine upshift, in biological 
    triplicates. Lines are a log-linear regression fit. 
    Points are dodged horizontally for clarity, but this is not used
    for modeling. Wild-type is FY4.
    \textbf{(Top Left)} \textit{xrn1}$\Delta$, \textit{ccr4}$\Delta$,
    \textit{pop2}$\Delta$ are all slowed in clearance 
    (p-values < 0.004).
    \textbf{(Top Right)} \textit{lsm1}$\Delta$ and \textit{lsm6}$\Delta$ 
    are slowed in clearance (p-values < 0.0132 and 0.0299, 
    respectively).
    \textbf{(To Left)} \textit{edc3}$\Delta$ is slowed in clearance 
    (p-value < $10^{-4}$). \textit{scd6}$\Delta$ and 
    \textit{tif4632}$\Delta$ are slowed in clearance 
    (p-values < $10^{-5}$) and have lower levels of expression
    before the upshift (p-values < 0.003).
    \textbf{(Top Right)} A deletion of 150bp 3' of \textit{GAP1} stop 
    codon has no significant effect, but a deletion of 100bp 5' of 
    the start codon has slower clearance (p-value < $10^{-4}$) and 
    lower level of expression before the upshift (p-value < 0.0015).
    During strain construction, a deletion of 152bp 5' of the start
    codon was also generated, and see a similar phenotype with this
    strain ( p-value < $10^{-5}$ clearance defect, 
    p-value < $0.0061$ ).
    \label{fig:figure5qpcr}
  }{Disrupting core pathways of mRNA degradation, decapping
    modulators, or the 5' UTR impairs the clearance of \textit{GAP1} 
    mRNA, by qPCR}

Identification of an initiation factor subunit with defects in
\textit{GAP1} mRNA clearance suggests that translation control may
impact stability changes. Therefore we deleted sequence of 
the 5' UTR and 3' UTR of \textit{GAP1}, specifically 100bp and 152bp 
upstream of the start codon (approximate 5' UTR) or the
100bp downstream of the stop codon (approximate 3' UTR), 
Whereas the 3' UTR deletion does not have an effect the 5' UTR deletion
exhibit the phenotype of reduced \textit{GAP1} mRNA before the upshift
and a reduced rate of transcript clearance following the upshift
(\autoref{fig:figure5qpcr}). 
We observed a similar phenotype with a different deletion of 152bp upstream
of the \textit{GAP1} start codon (\autoref{fig:figure5qpcr}). 
%These measurements suggest altered mRNP composition of the
%Lsm1-7p/Pat1p complex and associated decapping factors are associated
%with defects in \textit{GAP1} mRNA expression dynamics upon a 
%nitrogen upshift. Importantly the phenotype of the \textit{scd6}$\Delta$, 
%\textit{tif4632}$\Delta$, or 5' sequence deletions preceed the 
%addition of glutamine, suggesting that the observed destabilization
%of \textit{GAP1} may be the halt of a stabilization effect, perhaps
%due to changes in translational status of \textit{GAP1}.
This indicates that \textit{cis}-elements responsible for the
rapid clearance of \textit{GAP1} are unlikely to be located in the
3' UTR, and instead may be exerting an effect at the 5' end of the
mRNA.

\subsection{Methods and materials}

\subsubsection{Strains}

Strains with deletions 5' of the start codon and 3' of the stop
codon were generated by the "delitto-perfeto" 
method \parencite{storici2006delitto}, 
by inserting the pCORE-Kp53 casette
at either the 5' or 3' end of the coding sequence, then transforming
with a short oligo product spanning the deletion junction and
counter-selecting against the casette with Gal induction of p53 
from within the cassette.
These strains were generated and confirmed by Sanger sequencing,
and traces are available in directory \texttt{data/qPCRfollowup/} 
within the data zip archive (\nameref{subsubsection:codeanddata}).

\subsubsection{qPCR}

Each strain was grown from single colonies.
Samples were collected before, during the first ten minutes of
the nitrogen upshift (\autoref{fig:figure3a}),
or at ten minutes after the upshift (\autoref{fig:figure3}).
For the experiments described in
\autoref{fig:figure5qpcr}, all work
was done in biological replicates.
Each 10mL sample was collected by vacuum filtration, and RNA extracted
using the hot-acid phenol technique .
For \autoref{fig:figure3a} only, at the beginning of this extraction 
incubation I added 10$\mu$L of a 0.1ng/$\mu$L in-vitro synthesized 
spike-in mRNA BAC1200 (as generated
for the label-chase RNAseq (\nameref{subsubsection:spikeins},
but without 4-thiouridine). 
RNA was treated DNAse RQ1 (Promega) according to manufacturer
instructions, cleaned and precipitated.
All samples were reverse transcribed. 
For \autoref{fig:figure3a} 2$\mu$g RNA was primed with 2.08ng/$\mu$L
random hexamers (Invitrogen) and 2.5mM total dNTPs (Promega),
while for \autoref{fig:figure5qpcr} 1$\mu$g RNA was primed with 
5.6mM Oligo(dT)18 primers (Fermentas) and
0.56mM total dNTPs (Promega).
These mixtures were reverse transcribed with M-MulvRT (NEB) according
to manufacturer's instructions, then diluted 1/8 with hyclone water 
and used as direct template in 10$\mu$L reactions with SybrGreen I 
Roche qPCR master-mix (Roche). These were measured
on a Roche Lightcycler 480. 
For \autoref{fig:figure3a}, I used primers 
DGO230,DGO232 to quantify \textit{GAP1} and 
DGO605,DGO606 to quantify the synthetic spike-in BAC1200.
For \autoref{fig:figure5qpcr}, Nathan Brandt used primers
DGO229, DGO231 to quantify \textit{GAP1} and
DGO233, DGO236 to quantify \textit{HTA1}.  
See \autoref{tab:primerTable} for sequence.
These were run on a Roche480 Lightcycler, 
with a max-second derivative estimate
of the cycles-threshold (the $C_p$ value output by analysis) used 
for analysis by scripts included in the git repo 
(\nameref{subsubsection:codeanddata}).
Linear regression of the log-transformed values was used to quantify
the dynamics and assess significance of changes in expression
levels or rates of change.

\subsubsection{Microscopy of Dcp2-GFP}

To look for processing-body dynamics in response to
a nitrogen upshift, I used strain DGY525, which is FY3
containing plasmid pRP1315 (gift from Roy Parker).
Samples were collected before and following a nitrogen upshift
(4, 10, 12, 19, or 25 minutes later),
from exponential growth in YPD, 
or 10 minutes after resuspending YPD-grown cells in DI water.
All samples were collected by brief centrifugation (1 minute)
then resuspension in PBS buffered 4\% PFA for 
aspirating most supernatant, then centrifugation for 20 seconds
and aspirating all media. Each pellet was 
immediately resuspended in 4\% PFA 
(diluted from EMS 16\% PFA ampule RT15710) 
with 1x PBS for 15 minutes on bench, 
then spun at 10,000g for 1 minute, aspirated, 
then washed once and resuspended with 1x PBS. 
Samples were kept on ice, then put onto a coverslip
for imaging on a DeltaVision scope. Raw images available in the
microscopy zip archive (\nameref{subsubsection:codeanddata}).

\subsection{Methods and materials of Barseq after FACS after FISH experiment}

\subsubsection{Culturing and sampling}

An aliquot of the prototrophic deletion collection
\parencite{vandersluis2014broad} was thawed and diluted, with 
approximately 78 million cells added to 500mL of NLimPro media 
in a 1L baffled flask. This was shaken at 30$^{\circ}$C overnight, 
then split into three flasks (A, B, and C). 
After three hours (at mid-exponential)
we collected samples of 30mL culture by filtration and flash-freezing.
The collection times for each
sample were A: 10 minutes and 38 seconds, B: 10 minutes and 12 seconds
C: 10 minutes and 17 seconds.

\subsubsection{Fixation and permeabilization}

Frozen samples of the pool were fixed with formaldehyde
(4\% PFA diluted in PBS, 2 hours room-temperature) and digested
with lyticase (in BufferB with VRC 37C 1 hour).
Microscopy monitoring of the reaction showed the classic
greying of the cells under phase contrast microscopy to a dark grey, 
but did not digest to ghosts and fragments. 
Critically, 1/5th volume 2.5M glycine was added to quench the 
fixation before pelleting and washing the cells with centrifugation
at 1200g room temperature. An experiment where quenching did not
take place resulted in smaller fixed cells that were less accessible
to the FISH hybridization.
After washing three times with Buffer B (from Pringle's
immunofluoresence method) these were permeabilized with ethanol at
4C overnight.

\subsubsection{Hybridization}

The samples were processed with a Quantigene Flow RNA kit purchased in
March of 2015 (product code 15710), and designed for GAP1 mRNA in
\emph{Saccharomyces cerevisiae}. The probe sequences are proprietary.
This procedure is largely as described by the manufacturer, with some
critical modifications.
The incubator used was calibrated to 40\(^{\circ}\)C using a Traceable
4004 Type-K thermometer, with the probe inserted into an eppendorf tube
through a hole and sealed with parafilm, and inserted into the aluminum
heatblock in the air incubator as used for incubating samples. 
The ethanol-permeabilized samples were pelleted by centrifuging 
1200g for 5 minutes room temperature, then washed with "Solution D"
from the kit before proceeding with the kit instructions, with the
exception that all reactions were conducted with 1/4 volume. 
For each wash, the complete supernatant was discarded, and cells
resuspended in wash buffer to 25$\mu$L (to replicate the "about
100$\mu$L residual" in the instructions).
Upon final wash, these were incubated for 5 minutes with DAPI
to counter-stain.

\subsubsection{Flow cytometry and FACS}

Samples were sonicated, then run through a BD FACSAria II
by a NYU GenCore technician.
Cells were gated for singlets and DAPI content 
(estimated 1N or more), then sorted based on emission area from a
660/20nm filter with a 633nm laser activation
into four gates within each timepoint, across replicates.
Importantly, the sorting gates were set with a GUI interface until they
approximated splitting the libraries into quartiles for the six samples.
These were sorted using PBS sheath fluid at room-temperature, into
poly-propylene FACS tubes, then stored at -20$^{\circ}$C.

Note that we later in analysis add a fixed number to all observations
and gates in linear scale in order to get into positive values.

\subsubsection{Cell collection and DNA extraction}

For each gate, cells were collected in eppendorfs via laborious 
repeated gentle centrifugation steps and genomic
DNA extracted by NaCl reverse-crosslinking at 65$^{\circ}$ for 16
hours, inspired by \cite{klemm2014transcriptional}, with
subsequent proteinase K and RNase A digestions.
DNA was extracted from each sample with a very careful
phenol:chloroform extraction, with back-extraction, estimated to yield
1/3rd the theoretical input (by qubit and estimation), twice as good
as by either Zymo columns or Ampure beads. This may be a result of
trying to obtain genomic DNA.

\subsubsection{Construction of amplicon sequencing libraries for barcode counting}

gDNA was amplified in technical triplicates in a heavily modified 
BarSeq protocol

A master mix of buffer, BSA, dNTPs, MgSO$_4$, and Vent (exo-)
polymerase (NEB) is made containing three oligos DGO1562, DGO1588, 
and DGO1589.
This is combined with about 1/3rd of the genomic DNA extracted
from the sample, and the reaction is loaded into individual PCR
tubes (the optimization was in individual PCR tubes and 96 well 
plates did not work, likely due to particularities of the sealing
mechanism).
30 reactions were run in each batch of preparation. All 30 reactions
were put into a BioRad T100 thermocycler set for a 30\(\mu\)L
reaction.

This was cycled through 4 minutes at 95C to denature, then a single
annealing and extension step. This was then cooled to 37C and
immediately ExoI (Thermo) was added and incubated (with periodic
mixing) for twenty minutes. This exonuclease proceeds from
un-annealed 3' ends. 


The exonuclease was then inactivated with a 80C incubation, 5 minutes,
then DGO1576, DGO1567, and DGO1519 were added in a master mix
containing glycerol to bring the whole reaction to about 5\% glycerol
content. This became essential for keeping the forward primer
effective at lower primer concentrations, presumably due to the
computationally predicted hairpin in the 3' end.

This reaction was cycled 40 times to anneal and extend the product.
I do not expect all the cycles to be productive. 
Rather, the lower concentrations used
(10nM forward DGO1567) limit the product generation.

To this generated intermediate product, I then added more primers,
more buffer, and indexed forward primers. These incorporate a 5bp
sample index. This is of insufficient complexity for error-correction
bioinformatically, but should be sufficient to demultiplex most
samples. This was cycled 12 times to extend the product.

All reactions were then frozen, then thawed on ice and pooled into
four pools. These were purified, then one last reaction used the
same polymerase to do three rounds of extension to add on the final
Illumina P5 adapter onto the molecule. This was resolved on a 3\%
agarose gel, then the only visible bands were purified, quantified
using qPCR, and submitted for sequencing on an Illumina NextSeq.

Based on previous runs, the Genomics Core spiked-in 5\% PhiX onto
the run, to help maintain diversity. However, during the run there was
a massive failure in Illumina software to discern clusters given
low base-diversity.  This run only yielded approximately 127 million
reads (out of 400 million listed yield). Thus, 5\% PhiX is too low,
and I would recommend trying $\sim$25\% in future runs. 

\subsection{Design and analysis of Barseq after FACS after FISH experiment}

\subsubsection{Design of Barseq after FACS after FISH experiment}

\label{subsection:bff}

First, we motivate this development, as it departs from previous procedures
in a few ways.  
The main impetus for this was the generation of primer dimers that form
when the forward universal primer primes off the reverse universal
primer. For example, \autoref{fig:dimer} shows a failed experiment
that shows dimer formation in the sample lanes (on right). 

\vspace{3em}

\afig{
  \includegraphics[width=.5\textwidth]{img/exampleDimers.png}
  }{
\begin{tikzpicture}[overlay
    ,font=\small,%font=\ttfamily
    ,inner sep=0pt,outer sep=0pt
    ,shift={(0.7,-3)}]
    ]
  \node[align=left] at (-5.8,5) (ladder200) {200bp};
  \node[align=left,below=0.6cm of ladder200] (ladder100) {100bp};
%
  \node[below right=0.025cm and 1cm of ladder200] (200L) {};
  \draw[->] (ladder200) -- (200L);
  \node[above right=0.000cm and 1cm of ladder100] (100L) {};
  \draw[->] (ladder100) -- (100L);
%
  \node[align=left,anchor=south] at (-3.5,5.6) {Ladder\\Lane 1};
  \node[align=left,anchor=south] at (-2.1,5.6) {Lane 2};
  \node[align=left,anchor=south] at (-0.7,5.6) {Lane 3};
  \node[align=left,anchor=south] at (0.7,5.6) {Lane 4};
  \node[align=left,anchor=south] at (2.1,5.6) {Lane 5};
  \node[align=left,anchor=south] at (3.4,5.6) {Lane 6};
\end{tikzpicture}
    This is a 3\% TAE agarose gel, stained with Sybr Safe dye.
    The left-most lane is a NEB 100bp ladder, with the bottom 
    two bands as 100bp and 200bp. The red is due to overexposure.
    The right five bands are from samples prepared with an earlier
    version of this protocol. The band approximately 190bp is
    throught to be the library product, and the band approximately
    160bp is thought to be the inhibitory and unwanted 
    barcode-less dimer. Lane 3 clearly shows both bands, while lane
    6 is all dimer.
  \label{fig:dimer}
  }{An example of a dimer.}

Below a critical threshold, this dimer greatly out-competes the desired
product and can result in a loss of amplicon before the amplicon is
amplified enough to gel extract (above figure, lane 6). The dimer is
also sequenced via Illumina chemistry (not desired). By Sanger
sequencing we found that it appears to result from a three base
trucation of the forward primer priming perfectly for about 6 bases off
the reverse primer. This was not solved by switching to a polymerase
without 3' exonuclease activity, or by using HPLC purified primers.
Using different reverse primers lead to off-target products.

We saw these dimers before incorporating a UMI step into the protocol.
We used a UMI because we wanted the protocol to be as quantitative as
possible, despite the multiple amplification steps that would introduce
randomly sampled noise at each cycle. The design of this was 6 bp
degenerate sequence spaced with fixed bases, in the design of
\texttt{NCNCNCNTNCN} because we estimated this would best block
annealing to any 3' ends of the primers used. In future work, we would
strongly recommend using more degenerate bases for such a low-complexity
library \parencite{fu2011counting}. 

In order to digest the
excess un-incorporated UMI primers it requires the addition of a
exonuclease. ExoI is characterized to be maximally effective at
37\(^{\circ}\)C, and although it can have activity at 42\(^{\circ}\)C
for a some time \parencite{fei2015structural} it
will be inactivated. This low temperature requirement likely exacerbates
dimer and off-target product formation.
The common, but incorrect, one-UMI-one-molecule assumption is based on
the belief that each UMI and each genomic template is used once in the
first round. The exonuclease step must therefore digest all
un-extended UMI-containing primers for this to be true. One control
for this is to monitor the primer concentrations using a technology
like HPLC to measure abundance, but a small number is, again, not
necessarily negligable. The proper control is to not use the
polymerase to extend the first round, and instead immediately digest
the UMI-containing primers. This is a crucial control, and on the
basis of experiments I did using this setup using dilutions of
template, I estimate that the effect of the exonuclease, in the 
absence of the first round of extension, reduced product formation 
by about 20 fold. However, it was not absent. Thus, I suspect that 
the exonuclease digestion only ensures that $\approx$20/21 UMIs are
from the first round of incorporation. While small, this is not
negigable, especially in conditions where template number changes
dramatically (as here it does not, given similar inputs to extractions).
Optimization of this step did not improve upon this efficacy, and in
the lack of commercially available thermostable (and active) 
exonucleases \parencite{fei2015structural}, this background noise 
must be understood and accepted.

To address this, we optimized the reactions on a dilution series of gDNA
from a different experiment with the same knockout library. By balancing
MgSO\(_4\) and glycerol concentrations we got better amplification, and
tried to use a ``booster'' \parencite{ruano1989biphasic} PCR approach. 
This gave some improvements in how low we
could detect before saturating the reaction with dimers, but we could
not go lower in primer concentration and attributed this to the
predicted secondary structure in the 3' end of the primer amplifying
from the outside of the UPTAG barcodes. Adding DMSO helped with this,
but we still had to leave the reaction with plenty of primer as
intra-molecular interference from this process would out-compete
inter-molecular productive annealing. We still could not get reliable
amplification from \textless{} \(10^5\) templates (estimated by qubit
assuming 12.5 picograms gDNA per genome).

The major solution to this problem was the addition of 3' phosphorylated
blocker oligos. These are not extended by DNA polymerases but are
displaced by a strand-displacing polymerase like Vent exo-. By using
this polymerase and blocker combo, we prevent new 3' ends from annealing
but allow properly annealed primers to extend through this region. This,
in combination with the exonuclease digestion of most of the reverse
primer, prevented dimer formation. This revealed that these universal
priming sequences will amplify from two loci near \emph{CIA1} and
\emph{RDN37}. This was identified by Sanger sequencing gel-extracted
bands, so we designed more 20-mers that again block off-target annealing
and found they worked wonderfully. In test experiments, we believe we
got amplicons of the correct size from as low as \(\sim\) 300 targets
but have not sequence verified this.

To simplify the addition of the last 5' Illumina P5 adpater, we kept
this as a separate reaction. To minimize chimera formation between
different samples in this reaction, this is a 2-step polymerase
extension reaction which partially forms the sequencing product (1/3 of
results, theoretically). This is sufficient for qPCR quantification of
the library and Illumina sequencing. Given our gel-extraction clean up
and small product size, we do not expect formation of chimeras on the
flow cell \footnote{http://dnatech.genomecenter.ucdavis.edu/2017/04/11/update-on-barcode-mis-assignment-issue/}.

\autoref{fig:cartoon} shows a cartoon of
the amplicon library-making procedure, up to the generation of the
product before the Illumina P5 addition \autoref{fig:amplicon}. 
Given the simplicity of this
last reaction (simply adding sequence on the 5' end), it is omitted.

\afig{
  \vspace{-4em}
  \includegraphics[height=10.0in]{img/sobaseqPCRschematic-1.png}
  \vspace{-7em}
  }{
  \label{fig:cartoon}
  }{A schematic of the barcode sequencing strategy.}

%  \raggedright
%  \begin{minipage}{1cm}
%    \caption{}
%  \end{minipage}
%  \includepdf[pages={2},scale=0.8]{img/sobaseqPCRschematic.pdf}


\afig{
\begin{tikzpicture}[overlay
    ,font=\small,font=\ttfamily
    ,inner sep=0pt,outer sep=0pt
    ,scale=0.8, every node/.style={scale=0.7}
    ,shift={(-9.25,-3)}
    ]
  \node[fill=orange!50,align=left,anchor=west] at (-1,0) (1) 
    {\scriptsize ACGCTCTTCCGATCT};
  \node[fill=green!30,align=left,anchor=west] (2) at (1.east) 
    {\scriptsize NNNNN};
  \node[fill=yellow!100,align=left,anchor=west] (3) at (2.east) 
    {\scriptsize GTCCACGAGGTCTCT};
  \node[fill=white,align=left,anchor=west] (4) at (3.east) 
    {\scriptsize NNNNNNNNNNNNNNNNNNNN};
  \node[fill=yellow!100,align=left,anchor=west] (5) at (4.east)
    {\scriptsize CGTACGCTGCAGGTCGAC};
  \node[fill=purple!30,align=left,anchor=west] (6) at (5.east)
    {\scriptsize NGNANGNGNGN};
  \node[fill=orange!50,align=left,anchor=west] (7) at (6.east)
    {\scriptsize GATGTGACTGGAGTTCAGAC};
  \node[fill=cyan!50,align=left,anchor=west] (8) at (7.east)
    {\scriptsize ATCTCGTATGCCGTCTTCTGCTTG};
%
  \node[align=left,below=0.5cm of 1] (1l) {adaptor\\sequence};
  \node[align=left,above=0.5cm of 2] (2l) {sample\\index\\sequence};
  \node[align=left,below=0.5cm of 3] (3l) {fixed\\sequence};
  \node[align=left,above=0.5cm of 4] (4l) {strain barcode};
  \node[align=left,below=0.5cm of 5] (5l) {fixed\\sequence};
  \node[align=left,above=0.5cm of 6] (6l) {UMI};
  \node[align=left,below=0.5cm of 7] (7l) {adaptor\\sequence};
  \node[align=left,below=0.5cm of 8] (8l) {illumina\\adaptor sequence};
%
  \draw[->] (1l) -- (1);
  \draw[->] (2l) -- (2);
  \draw[->] (3l) -- (3);
  \draw[->] (4l) -- (4);
  \draw[->] (5l) -- (5);
  \draw[->] (6l) -- (6);
  \draw[->] (7l) -- (7);
  \draw[->] (8l) -- (8);
\end{tikzpicture}
\\\vspace{9em}
  }{
  \label{fig:amplicon}
  }{The expected amplicon, before adding P5 sequences at the 5' end.}

These were checked with sanger sequencing, for pools 1 and 3, using
primers DGO 276 or DGO 1519, with Genewiz sequencing. Representative
Sanger sequencing image of library pool 1, sequenced forward (with DGO
216) is shown in figure \ref{fig:sanger}.

Trace colors: red is T, green is A, blue is C, black is G.

\afig{
\includegraphics[trim={0cm 1.6cm 0cm 3cm},clip,width=\textwidth]
  {img/represtativeSangerImage.png}
\begin{tikzpicture}[overlay,shift={(-8.25,0)}]
  \node[rotate=90,anchor=west] at (3,5) (sample) {5bp sample barcode};
  \draw[->] (sample) to (2.7,3.5);
  \draw[->] (sample) to (3.3,3.5);
  \node[rotate=0,anchor=west,align=left] at (6,5) (strain) {15-22bp \\strain barcode};
  \draw[->,bend right=20] (strain) to (6,2.5);
  \draw[->,bend left=20] (strain) to (9,2.5);
  \node[rotate=0,align=left,anchor=west] at (10,7) (umi) 
    {UMI barcode, 5bp fixed \\and 6bp degenerate};
  \draw[->,bend left=00] (umi) to (12.20,2.5);
  \draw[->,bend left=04] (umi) to (12.50,2.5);
  \draw[->,bend left=08] (umi) to (12.80,2.5);
  \draw[->,bend left=12] (umi) to (13.10,2.5);
  \draw[->,bend left=16] (umi) to (13.40,2.5);
  \draw[->,bend left=20] (umi) to (13.75,2.5);
\end{tikzpicture}
  }{\label{fig:sanger}}
  {Sanger sequencing of the produced library shows the expected
  degenerate positions.}

\subsubsection{Designing an analysis pipeline to filter dimers and 
extract UMIs from indeterminate locations}

We previously used the one-program solution of BarNone 
(\url{http://varianceexplained.org/BarNone/})
to rapidly and easily quantify barcode counts from yeast barcode
sequencing experiments. However, our new amplicon design makes use of
UMIs to help account for the amplicon noise inherent in the BarSeq
method, and BarNone does not account for these. We also wanted to devise
a pipeline that would be modular, consisting of multiple well-designed
tools that could be modified independently and would maintain read
information along the pipeline to assist in debugging and benchmarking.

\begin{enumerate}
  \setlength\itemsep{1em}
  \item
  FASTQ files of the reads are fed into a custom python script called
  \texttt{SLAPCHOP.py}. This is named because it Simply Looks At
  Pair-wise Comparisons to Help Optimize Parsing 
  (\url{https://github.com/darachm/slapchop}). This
  parallelized script takes each read, aligns the expected fixed
  sequences that bracket the informative barcodes (using
  BioPython \cite{cock2009biopython}), decides if the read matches the
  expected structure based on a specified criteria, then extracts out 
  the sample index, strain barcode, and UMI degenerate sequence into 
  appropriate positions in a FASTQ format. This keeps the filtering 
  and strain barcode identification separate from the fixed sequences.
  \item
  A simple perl script (\texttt{pickyDemuxer.pl}) demultiplexes the
  processed FASTQ file on perfect matches of the 5bp index sequence and
  generates a demultiplexing report.
  \item
  The strain barcode regions, padded with flanking sequence to a uniform
  length of 26bp, from the demultiplexed FASTQ files are aligned using
  \texttt{bwa\ mem} \parencite{li2013aligning}
  to the
  expected barcodes as re-annotated by \cite{smith2009quantitative}.
  \item
  From the resulting \texttt{bam} alignment files, we extract the strain
  identification and UMI. Using the UMI-collision / label-saturation
  concept and equation of Fu et. al. 2011 (PNAS), we adjust the
  saturated pool to estimate the input of strain genomes into the
  library preparation.
\end{enumerate}

This allowed us to recognize and extract barcodes from indeterminate
positions in the amplicon and filter the reads for real, intact
amplicons in one step. This extraction improved our accuracy, for
example eliminating spurious alignments of the forward priming sequence
against the barcode of \emph{ymr258c}\(\Delta\), a barcode re-annotated
with striking similarity to the fixed priming sequence
\cite{smith2009quantitative}.

There are several different ways to use the UMI information to estimate
unique input molecules from a sequencing assay. The naive approach is to
assume that every combination of strain barcode with a certain UMI
sequence in a sample is a unique event, and any repetitions of this are
only PCR duplicates. 
The molecular biology caveats for this assumption are addressed 
earlier.
However, we only have \(4^6=4096\) possible UMIs for \(\sim4500\) 
possible strain
barcodes, with approximately a million reads per sample. Thus, due to a
short length of UMI we cannot make this assumption, and instead are
confronted with a space of UMIs with a fairly high chance of two
UMI-strain combinations being generated by random chance alone. 

We refer
to this as a UMI-collision (similar concept to a hash collision) or the
phenomenon of label saturation. 
There are multiple ways to deal with this.
The simplest is to simple take each unique UMI, but errors in library
amplification could diversify spuriously duplicated UMIs.
Error-correcting algorithms 
exist that use graph information to improve accuracy of this method, but
these require that the space of all usable UMIs is sparse. 
Another solution is the label saturation correction of 
\cite{fu2011counting}. This depends
on treating the chance that any UMI-collisions are a random and rare
event, thus modeled as a Poisson distribution, similar to the classic
Luria-Delbruck method of mutation rate estimation. 
If we have 4096 possible
UMIs, and for one strain in one sample we observe \(x\) different UMIs
associated, then we estimate that there were \(z\) different original
molecules in the sample, where
\(z=4096 (1-e^{-\frac{x}{4096}})\) .

\label{subsubsection:rarefractionDiscussion}

Figure \ref{fig:umi} plots this function, and compares it to
actual sequencing data from the experiment.
I also attempted to use the error-correcting approach of UMI-tools
\parencite{smith2017umi}, but found that the results of this
tool did not conform to this expectation \autoref{fig:umi}. 

\afig{
  \includegraphics[width=.49\textwidth]{img/Figure4_S_umiSaturationCurve.png}
  \includegraphics[width=.49\textwidth]{img/Figure4_S_totalInputReadsVsUMItoolsCounts.png}
  }{
  For each sample, each combination of UMIs and strain barcodes was
  collected. For each point, the y-axis denotes the unique UMIs
  observed for that combination of strain and sample.
  On the x-axis is the raw counts of observing that strain barcode
  in that sample, without error-correction
  \textbf{(Left)} and with \texttt{umi-tools} error-correction
  algorithim on the right \textbf{(Right)}.
  The line denotes the curve expected just from label saturation
  with increased re-sampling of a limited pool as it approaches 4096. 
  \label{fig:umi}
  }{A comparison of the unique UMIs versus input UMIs for un-corrected
    and error-corrected UMIs.}

We see that the actual observations (the points) follow a similar
pattern as the expectation, saturation towards the higher end of counts.
However, they are greatly depressed below this line. As shown in
\autoref{fig:umihistz}, we see that for some UMI-strain barcode
combinations, we see much more counts per UMI label than expected by a
Poisson distribution. 

\afig{
  \includegraphics[width=.8\textwidth]{img/umi_histograms.png}
  }{
  In black is the observed data, in red is the expected distribution
  if Poisson.
  We see that there is a long-tail of UMIs with more reads than 
  observed for that combination of strain and sample.
  \label{fig:umihistz}
  }{Histograms of UMI observations associated with the YOR202W barcode
    in three samples.}

Both label saturation and PCR duplication
are at play to distort the mapping between either raw counts or unique
labels and the actual underlying estimate of input genome targets per
strain in each sample
It appears that UMI-tools scavenges the poisson distribution of
"counts per UMI" into a small number of real clusters. This is not
surprising given that we are using this tool in a labeling regime that
is wholy inappropriate for what it is trying to do. Thus, I do not use
this tool an instead used the more conservative 
correction of \cite{fu2011counting}.
An appropriately complex UMI design would allow a future user to make
use of error-correcting algorithims in this method.

\subsubsection{Benchmarking this pipeline}

We benchmarked this pipeline against an \emph{in silico} dataset to
determine performance across a range of mutation rates and read depths
similar to what we would expect for this experiment. We also compared
the two UMI correction approaches described in the previous section, the
``UMI-collision correction'' and the ``number unique'' method. We used a
python script \texttt{makingFakeReads.py} to generate several datasets
with the following parameters:

\begin{itemize}
  \setlength\itemsep{0em}
  \item
  16 million reads per FASTQ, split amongst 32 samples
  \item
  each strain barcode is sampled from an emipirically observed
  distribution averaged from the first timepoints of an unpublished
  dataset, quantified by BarNone
  \item
  each amplicon has a poisson number of random single nucleotide
  mutations to a different base, based on a given parameter of
  \texttt{0}, \texttt{1}, \texttt{2}, or \texttt{3} lambda of mutations
  per amplicon.
  \item
  each generated amplicon is added to the file \(x\) number of times,
  where \(x\) is an exponential distribution with mean 5
  \item
  3 ``biological'' replicate datasets are generated per set of
  parameters
\end{itemize}

After quantification, we calculate pearson correlation, spearman
correlation, and the number of mistaken strain-identifications.
Tolerated mismatches is a parameter set in BarNone or by the score
requirements for alignment in \texttt{bwa}.

As seen in \autoref{fig:wholeSampleDme234}, 
it would appear that by pearson correlation, the filtration step of the
\texttt{bwa} alignment allows us to make more robust assignment of
strain barcodes. The spearman correlation tells us that as mutation rate
increases, high mismatch tolerance on the \texttt{bwa} tool is very
dangerous for misaligning and can cause large rank changes.

\afig{
  \includegraphics[width=\textwidth]{img/fig4s_whole_sample_pearson.png}
  \includegraphics[width=\textwidth]{img/fig4s_whole_sample_spearman.png}
  }{
  Tool quantification was compared to the \textit{in silico} ground
  truth using correlations of Pearson \textbf{(top)} or Spearman
  \textbf{(bottom)}.
  \label{fig:wholeSampleDme234}
  }{Optimization of mutant quantification methods, looking at
    simulated data on a whole-lane comparison.}

How does the UMI-collision correction perform?
We see that on the whole, un-demultiplexed datasets (16 million reads
across \textasciitilde{}4000 strains and 4096 possible UMIs) that the
performance is best with the UMI-collision correction 
\autoref{fig:dedupSampleDme234whole}. We see that just
using unique UMI counts in this regime leads to a good reconstruction of
the rank order (Spearman), but inaccurate of the magnitude (Pearson)
against ground truth.
Does this change with lower read density? 
Each point is one of 32 demultiplexed samples of three whole-library
replicates, and we see that in this lower read regime, we get similar
performance from both methods \autoref{fig:dedupSampleDme234per}.
To avoid biases arising from differences in abundance between strains,
I keep the UMI-collision correction.

\afig{
  \includegraphics[width=\textwidth]{img/fig4s_whole_sample_dedup_pearson.png}
  }{
  On a whole-sample basis, the error of the "number unique" quantifier
  of UMIs is evident, in this saturated regime.
  \label{fig:dedupSampleDme234whole}
  }{Optimization of mutant quantification methods, looking at
    deduplicating methods on whole comparisons.}

\afig{
  \includegraphics[width=\textwidth]{img/fig4s_demux_dedup_pearson.png}
  }{
  On a per-multiplexed-sample basis with less total counts, 
  the error of the "number unique" quantifier
  of UMIs is less evident.
  \label{fig:dedupSampleDme234per}
  }{Optimization of mutant quantification methods, looking at
    deduplicating methods on per-sample comparisons.}

BarNone appears to be more robust to mutations, in that it maintains a
flatter profile in the higher regimes of mutations. However,
\texttt{bwa} starts out higher. Accounting for duplication, as generated
by the exponential distribution described above, greatly improves
performance. At an average coverage here of a half-million reads, the
difference between the UMI-collision and unique counts is less that
with large coverage.  
In conclusion, our pipline for analysis is able to use an early
filtration step (SLAPCHOP) to improve strain barcode
identification and to extract UMIs that are useful for de-duplicating
PCR duplicates back to better estimates of the ground truth.

\subsubsection{Modeling \textit{GAP1} FISH signal per strain in the pool}

In order to use the counts of each mutant in each sample to estimate
\emph{GAP1} mRNA abundance per strain, we used a maximum-likelihood
modeling approach.

We are interested in the number of cells of a certain strain that went
into each bin. We estimate this as a metric we define as
``pseudocounts'', or \(u_{ik}\) where \(i\) is the strain index, and
\(k\) is the FACS bin. We call the sequencing counts \(c_{ijk}\), where
\(j\) is the particular PCR replicate out of \(J\) PCR replicates that
were successfully sequenced. We assume the sequencing counts are
linearly amplified from the ``events'' of actual cells being sorted into
each collection tube, and we assume that all of these ``events'' have
equal chance to be amplified and detected by this sequencing assay. Then
we scale this estimate by the total number of ``events'' we observed
during the FACS procedure going into each bin \(e_k\). We assume that
all ``events'' had equal chances in all bins to get sequenced. Then we
have

\[ u_{ik} = 
  \frac{ \sum_j \frac{c_{ijk}}{\sum_i c_{ijk}} }{J}
  \;\;\; \frac{ e_{k}}{\sum_{k} e_{k}} \]

This is intuitively more simple than the notation used here to describe
it precisely. Since we split the library into quartiles for the
sequencing, \(\frac{ e_{k}}{\sum_{k} e_{k}}\) is about one quarter for
each bin. \(\frac{ c_{ijk}}{\sum_{i} c_{ijk}}\) is just the proportion
of counts in that sample that are that mutant \(i\).
\(\frac{ \sum_j \frac{c_{ijk}}{\sum_i c_{ijk}} }{J}\) is simply the
average of the proportion of counts, across the PCR replicates.

Thus, \(u_{ik}\) is essentially the proportion of the
original library that is the mutant in that bin, and 
\(\sum_k u_{ik}\) is the total proportion of the library that is
that mutant.
So then if we divide \(\frac{u_{ik}}{\sum_k u_{ik}}\),
we have an estimate of the proportion of that mutant that
went into each bin, out of all the mutant that was in the experiment.

Once we have this normalized pseudocounts metric within each biological
replicate, then we fit a log-normal model. We explored a logistic model
and several mixture models (similar to DNA content flow cytometry with
two log-normals and a middle quasi-uniform distribution), and found that
the log-normal robustly fit well. The log-normal and logistic largely
agreed on ranking of estimated means, but the likelihood was slightly
higher for the log-normal fits on the whole library, so we used that
model.

From this model (fit using \texttt{mle()} in R), we used the fit 
mean as the estimate for the GAP1 abundance for that strain in that
sample.

\section{Discussion}

Regulated changes in mRNA stability allows cells to rapidly reprogram
gene expression, clearing extant transcripts that are no longer
required and potentially reallocating translational capacity.
%Despite progress in understanding the pathways that mediate
%mRNA degradation, the functional role of mRNA degradation and the
%factors that control regulated changes in mRNA stability remain poorly
%understood. 
Pioneering work in budding yeast has shown that mRNA
stability changes facilitate gene expression remodeling in response to
changes in nutrient availability including changes in carbon sources
\parencite{scheffler1998control} and iron starvation
\parencite{puig2005coordinated}. 
Here, we characterized genome-wide changes
in mRNA stability in response to changes in nitrogen availability and
identified factors that mediate the rapid repression of the
destabilized mRNA, \textit{GAP1}. Our study extends our previous work
characterizing the dynamics of transcriptome changes using chemostat
cultures \parencite{airoldi2016steady} and shows that accelerated mRNA
degradation targets a specific subset of the transcriptome in response
to changes in nitrogen availability. We developed a novel approach to
identify regulators of mRNA abundance using pooled mutant screens and
find that modulators of decapping activity, and core degradation
factors, are required for accelerated degradation of 
\textit{GAP1} mRNA. 
 
Measuring the stability of the transcriptome requires the ability to
separate pre-existing and newly synthesized transcripts. We modified
existing methods to measure 
post-transcriptional regulation of the yeast transcriptome in a
nitrogen upshift using 4-thiouracil labeling
\parencite{miller2011dynamic,neymotin2014determination,munchel2011dynamic}. These
modifications entailed improved normalization and quantification of
extant transcripts and explicit modeling of labelling dynamics to
account for some of the inherent limitations of metabolic labeling
approaches \parencite{perez2013eukaryotic}. Continued development of
fractionation biochemistry \parencite{duffy2015tracking} and incorporation of
explicit per-transcript efficiency terms will improve these
methods further \parencite{chan2017non}.

Our experiments show that the process of physiological and gene
expression remodeling occur on very different timescales in response
to a nitrogen upshift. Cellular physiology is remodeled over the
course of two hours to achieve a new growth rate.
By contrast, transcriptome remodeling occurs rapidly and through
states that are distinct
from increases in steady-state growth rates. 
%Interestingly,
%we found that the yeast transcriptome is on average less stable in
%using metabolic labeling
%\parencite{Munchel2011,Neymotin2014,Miller2011}. This relative
%reduction in mRNA stability could be an adaptation to potentially
%limiting ribonucleotides, but further work exploring differences in
%mRNA degradation rates during growth limited by different nutrients is
%required to test this concept \parencite{Garcia-Martinez2016}.
%Stability changes upon the nitrogen upshift generally exhibited the
%expected relationship with rates of abundance change (
%anti-correlation, $R^2=$-0.376 ); however, we found multiple cases in
%which increased mRNA degradation rates did  not result in rapid
%decreases in mRNA abundance. This has been observed for transcripts
%up-regulated in stress conditions, and has been proposed as a
%mechanism to effect a rebalancing of the transcriptome after a
%transient phase of reprogramming \parencite{Shalem2008}. Importantly,
%the changes in mRNA stability that we detect are nearly coincident
%with the environmental perturbation suggesting that a signal is sensed
%and the effect propagated to impact post-transcriptional regulation
%with rapid kinetics.
Previous studies have shown that transcriptional activation of the NCR
regulon is rapidly repressed upon a nitrogen upshift
\parencite{airoldi2016steady}. Our
results indicate that accelerated degradation of 
%at least 16 of the 77 probable 
many NCR transcripts \parencite{godard2007effect} contributes to this
repression. 
A three-fold increase in
the degradation rate of \textit{GAP1} mRNA provides an additional layer of
repressive control. Importantly, our results show that accelerated
degradation is not limited to NCR transcripts but also targets
transcripts enriched in carbon metabolism pathways, particularly
pyruvate metabolism. Conversely, we also detect an apparent reduction in the 
degradation rate for some transcripts 
%enriched in ribosome biogenesis mRNAs (for example \textit{NSR1}) as well as 
including \textit{MAE1}. \textit{MAE1} encodes
an enzyme responsible for the conversion of malate to pyruvate, and
combined with the accelerated degradation of \textit{PYK2} mRNA 
may reflect an adaptive shunt of carbon skeletons from glutamine 
to alanine via the TCA cycle \parencite{boles1998identification}. 
%Due to the limitations of
%labeling approaches \parencite{Perez-Ortin2013} we cannot conclude 
%here that these transcripts are indeed stabilized, 
%however we can conclude that they are strongly upregulated. 
Recently, \cite{tesniere2017relief}
described destabilization  of carbon metabolism mRNAs after repletion
of nitrogen following 16 hours of starvation. We do
not detect destabilization of \textit{PGK1} mRNA and note that
the basal half-life of 6.2 minutes estimated in our study is similar
to the accelerated rate reported by \cite{tesniere2017relief}.

%To identify the factors that underlie accelerated mRNA degradation, we
%developed a global \textit{trans}-factor screen using mRNA FISH, FACS, and
%sequencing. 
%BFF identified mutants in the Lsm1-7p/Pat1p
%complex as having elevated \textit{GAP1} mRNA levels both before and after the
%upshift.
%Given that the \textit{GAP1} mRNA is destabilized during this
%transition we suspect that these core mRNA degradation factors are
%directly involved. 
%Because factors associated with the Lsm1-7p/Pat1p
%complex are also involved in processing-body formation we looked for
%processing-body dynamics during the nitrogen upshift, but did not see
%qualitative changes in Dcp2-GFP distribution (raw data available in
%supplement). However, it has been proposed that pre-existing mRNPs
%seed the formation of processing-bodies \parencite{Lui2014}, thus the
%phenotype may require assays at a finer spatial scale to eliminate
%this possibility \parencite{Rao2017}. Interestingly, during
%cross-comparisons with a recent dataset exploring mRNA localization to
%RNP condensates \parencite{Khong2017} we found that the set of
%destabilized transcripts in the label-chase experiment are on average
%longer in CDS and have an increased codon-optimality, two factors that
%were shown to be associated with differences in stress-granule
%localization of mRNA \parencite{Khong2017}. 

%Regulated changes in mRNA stability can be mediated by RBP binding the
%3' UTR of specific transcripts.  However, cis element analysis are
%inconsistent with a role for known RBPs in the observed
%destabilization. In particular, Puf3p motifs are de-enriched from the
%destabilized set. We failed to detect new 3' UTR sequence motifs that
%are enriched in destabilized transcripts, but these 
Destabilized
transcripts are enriched for a binding motif of Hrp1p in
the 5' UTR. This essential component of mRNA cleavage for
poly-adenylation in the nucleus has been shown to shuttle to the
cytoplasm and bind to amino-acid metabolism mRNAs
\parencite{guisbert2005functional} and been shown to interact genetically to
mediate nonsense-mediated decay (NMD) of a \textit{PGK1} mRNA harboring a
premature stop-codon \parencite{gonzalez2000yeast} or a \textit{cis}-element spanning
the 5' UTR and first 92 coding bp of \textit{PPR1} mRNA
\parencite{kebaara2003upf}.
A potential role for these Hrp1p sites warrants further investigation. 

BFF identified mutants in the Lsm1-7p/Pat1p
complex as having elevated \textit{GAP1} mRNA levels both before and after the
upshift. This is expected given their central role in mRNA 
metabolism, and experiments using \textit{GAP1} normalized to
\textit{HTA1} demonstrate that the effect before the upshift is
likely a global effect (\autoref{fig:figure5qpcr}). 
However, these mutants still have a
significant defect in clearance of \textit{GAP1},
and the assay demonstrates that associated decapping factors 
\textit{EDC} and \textit{SCD6} have specific effects
(\autoref{fig:figure5qpcr}).
Given that the \textit{GAP1} mRNA is destabilized during this
transition we suspect that these mRNA degradation factors are
directly involved. 
While we found that the \textit{edc3}$\Delta$ mutant has defects in
clearance of \textit{GAP1}, we also 
found that \textit{scd6}$\Delta$,
%mutant shares a phenotype of reduced \textit{GAP1} mRNA
%expression during nitrogen limitation and reduced rate of \textit{GAP1} mRNA
%clearance with a 
\textit{tif4632}$\Delta$, and deletion of the 5' UTR
of \textit{GAP1} impairs clearance (\autoref{fig:figure5qpcr}). 
This deletion does not include the TATA box (ending at -179) or
GATAA sites (nearest at -237) responsible for NCR GATA-factor
regulation of \textit{GAP1} \parencite{stanbrough1996two}.
This suggests that interactions of
these factors with \textit{cis}-elements in the 5' UTR might be responsible for
stabilizing \textit{GAP1} mRNA during limitation, although the 
truncation of the 5' sequence may be enough to inhibit translation 
initiation by virtue of the shorter length
\parencite{arribere2013roles}.
Elements in the 5' UTR have
also been demonstrated to modulate \textit{GAL1} mRNA stability
\parencite{baumgartner2011antagonistic} and destabilize \textit{SDH2} mRNA upon glucose
addition, perhaps due to the competition between translation
initiation and decapping mechanisms \parencite{de2002role}.
Interestingly, both \textit{GAP1} and \textit{SDH2} 
share the feature of a second start
codon downstream of the canonical start
\parencite{neymotin2016multiple} and
we have previously found that mutation of
the start codon of \textit{GAP1} results in lower
steady-state mRNA abundances \parencite{neymotin2016multiple}.
This
%in light of recent analyses further highlighting the contribution of
%translation dynamics to mRNA stability 
%\parencite{Presnyak2015,Neymotin2016,Cheng2017}, 
suggests a mechanism of degradation through dynamic changes in 
translation initiation that triggers decapping of \textit{GAP1} 
and other mRNA. 
%However, the deletion of \textit{SCD6} would be
%expected to promote translation of mRNA on the basis of it's measured
%repressive activity in cell extracts, suggesting that if Scd6p does play a role
%that it may be specified by some condition-specific modulation of its
%activity \parencite{Rajyaguru2012,Poornima2016}. 
Future work interrogating
this possible interaction of translational status and mRNA
stability during dynamic conditions could also expand our understanding of
the relationship between these two processes.

To our knowledge, this is the first time mRNA abundance has
been directly estimated using a SortSeq approach, although 
%sorting on indirect markers or 
using mRNA FISH and FACS to enrich subpopulations of cells has been
previously reported
\parencite{klemm2014transcriptional,hanley2013detection,sliva2016barcode}. This
approach could be used with other barcoding mutagenesis technologies,
like transposon-sequencing or Cas9 mediated perturbations, to
systematically test the genetic basis of transcript dynamics.
%phenotypes. A strategy combining this technology with transcriptomics
%as a high-dimensional marker could accelerate unbiased investigation
%of cellular signalling pathways \parencite{Gapp2016}. Additionally, 
The use of branched-DNA mRNA FISH, or other methods
\parencite{rouhanifard2017single}, allows for mRNA abundance estimation without
requiring genetic manipulation which makes it suitable for a variety
of applications such as extreme QTL mapping. 
%While the cell wall
%of yeast makes optimization crucial to this assay, future development
%of hybridization protocols may improve accuracy and make the assay
%more robust \parencite{Richter2017,Wadsworth2017}. 
Furthermore, our methods for library construction should permit accurate
quantification of pooled barcode libraries with small inputs, 
expanding the possibilities for flow cytometry markers to fixed-cell assays.

Why is \textit{GAP1} subject to multiple layers of gene product repression upon
a nitrogen upshift, at the level of transcript synthesis, degradation,
protein maturation, and post-translational inactivation? Given the
strong fitness cost of inappropriate activity
\parencite{risinger2006activity},
this overlap could ensure mechanistic redundancy for robust repression in
the face of phenotypic or genotypic variation. Alternatively, it could
reflect a systematic need to free ribonucleotides or
translational capacity, or result from some as yet uncharacterized
process.
%, or could simply be an effect of some unrelated function. 
%While this question remains open,
%we have made progress towards this goal by identifying 
%factors required for its accelerated degradation.
%of decapping associated with the Lsm1-7p/Pat1p complex play a role. 
Future work aimed at determining the adaptive basis of accelerated
mRNA degradation will serve to illuminate the functional role of
post-transcriptional gene expression regulation.
%dissecting thisb
%mechanism and contrasting the dynamic process of mRNA destabilization
%during other growth transitions would greatly inform our understanding
%of mRNA stability specification at steady-state, possibly in light of
%the relationship between translation and stability of mRNAs. 

\section{General methods and materials}

\subsubsection{Availability of data and analysis scripts}

\label{subsubsection:codeanddata}

Computer scripts used for all analyses are available as a git repository
on GitHub\\
(\url{https://github.com/darachm/millerBrandtGresham2018})
and data is available as zip archives on the Open Science
Framework (\url{https://osf.io/7ybsh/}).

To reproduce the entire analysis, or to access a particular 
analysis, clone the git repo.  
Download the \texttt{zip} data archives from the above
OSF link, and put them inside this git repo folder 
(here, \texttt{millerBrandtGresham2018}).
At minimum, you should have the \texttt{data.zip} archive in that directory,
although records of all R analyses are in \texttt{html\_reports.zip}
and intermediate files are in \texttt{tmp.zip}.
Consult the \texttt{README.md} file in the repository for more instructions
and options, including to unzip intermediate files and HTML
reports generated for every R script which detail the results.

A Shiny application is also available to explore the two main 
datasets in this paper more easily, at
\url{http://shiny.bio.nyu.edu/users/dhm267/}. It
is also included in the OSF as a separate zipped archive for local
installation and long-term archiving. 
To use the Shiny applications from the zipped archive, download,
unzip the archive, and direct R to run the \texttt{runApp} command
to use the directory as a Shiny app. It can thus be used as an
interactive tool for visualization.

\subsubsection{Supplementary tables}

These are available on the Open Science Framework archive
\url{https://osf.io/9ct3m/} .

\begin{itemize}
  \setlength\itemsep{0em}
  \item Gene set enrichment analysis of loadings on principal 
    components one and two.
    \\\texttt{Figure1\_Table\_GSEofGOtermsAgainstPCcorrelation.csv}
    \label{itm:microarrayPCAgsea}
  \item Raw counts of labeled mRNA quantified by RNAseq in 
    label-chase experiment.
    \\\texttt{Figure2\_Table\_RawCountsTableForPulseChase.csv}
    \label{itm:dme211raw}
  \item Filtered label-chase RNAseq data for modeling, normalized 
    directly within sample.
    \\\texttt{Figure2\_Table\_PulseChaseDataNormalizedDirectAndFiltered.csv}
    \label{itm:dme211filterDirect}
  \item Filtered label-chase RNAseq data for modeling, normalized by
    modeling across samples.
    \\\texttt{Figure2\_Table\_PulseChaseDataNormalizedByModel.csv}
    \label{itm:dme211filterModel}
  \item Degradation rate modeling results, from data normalized 
    within samples.
    \\\texttt{Figure2\_Table\_PulseChaseModelingResultTable\_DirectNormalization.csv}
    \label{itm:dme211resultsDirect}
  \item Degradation rate modeling results, from data normalized 
    across samples.
    \\\texttt{Figure2\_Table\_PulseChaseModelingResultTable\_ModelNormalization.csv}
    \label{itm:dme211resultsModel}
  \item Enriched GO and KEGG terms within the set of mRNA 
    destabilized upon a nitrogen upshift, across sample normalization.
    \\\texttt{Figure2\_Table\_AcceleratedDegradationTranscripts\_EnrichedGOandKEGGterms.csv}
    \label{itm:dme211goAndKegg}
  \item Raw counts of strain barcode quantification within each bin
    in the BFF experiment, and gate settings for the observations.
    \\\texttt{Figure4\_Table\_BFFcountsAndGateSettingsFACS.csv}
    \label{itm:dme209rawCountsGates}
  \item BFF data filtered for modeling.
    \\\texttt{Figure4\_Table\_BFFmodelingData.csv}
    \label{itm:dme209modelData}
  \item The parameters of all models fit to the BFF data.
    \\\texttt{Figure4\_Table\_BFFallFitModels.csv}
    \label{itm:dme209allFits}
  \item All 3230 models used for identifying strains with defective 
    \textit{GAP1} dynamics.
    \\\texttt{Figure4\_Table\_BFFfilteredPooledModels.csv}
    \label{itm:dme209pooledFits}
  \item Gene-set enrichment analysis results using \textit{GAP1} 
    estimates.
    \\\texttt{Figure4\_Table\_GSEanalysisOfBFFresults.csv}
    \label{itm:dme209gsea}
\end{itemize}

\subsubsection{Media and upshifts of media}

Nitrogen-limited media (abbreviated as "Nlim") is a minimal media
supplemented with various salts, metals, minerals, vitamins, and
2\% glucose, as previously described
\parencite{airoldi2016steady,brauer2008coordination}. 
For proline limitation, 
Nlim base media was made with 800$\mu$M L-proline as the sole
nitrogen source (NLim-Pro).
YPD media was made using standard recipes \parencite{amberg2005methods}.
All growth was at 30$^{\circ}$C, in an air-incubated 200rpm shaker  
using baffled flasks with foil caps, or roller drums for 
overnight cultures in test tubes.
For glutamine upshift experiments, 
400$\mu$M L-glutamine was added from a 100mM stock solution dissolved 
in MilliQ double-deionized water and filter sterilized.
All upshift experiments were
performed at a cell density between 1 and 5 million cells per mL,
in media where saturation is approximately 30 million cells per mL. 
For all experiments, 
a colony was picked from a YPD plate and grown in a 5mL NLimPro 
pre-culture overnight at 30$^{\circ}$C, then innoculated into
the experimental culture from mid-exponential phase.

\subsubsection{Strains}

See \autoref{tab:strainsTable} for details.  
The wild-type strain used is FY4, a S288C derivative. 
The pooled deletion collection is as published in 
\cite{vandersluis2014broad}.
For all experiments with single strains, colonies were struck 
from a -80$^{\circ}$C frozen stock onto YPD (or YPD+G418 for
deletion strains) to isolate single colonies.
For pooled experiments we inoculated directly into NLim media
from aliquots of frozen glycerol stocks.

\begin{table}%[bt]
\caption{Yeast strains used in this study}
% Use "S" column identifier to align on decimal point 
\label{tab:strainsTable}
\begin{tabular}{l l p{.5\textwidth}}
\toprule
Strain ID & Short description & Details \\
\midrule
DGY1 & FY4 & Isogenic to S288C, prototrophic, MATa \\
- & Deletion collection pool & Haploid (MATa) prototrophic deletion
collection as described in the publication of
\cite{vandersluis2014broad}\\
DGY410 &xrn1$\Delta$::KanMX &   ygl173c$\Delta$::KanMX from the prototrophic deletion collection \\
DGY564 &ccr4$\Delta$::KanMX &   yal021c$\Delta$::KanMX from the prototrophic deletion collection \\
DGY565 &pop2$\Delta$::KanMX &   ynr052c$\Delta$::KanMX from the prototrophic deletion collection \\
DGY547 &lsm1$\Delta$::KanMX &   yjl124c$\Delta$::KanMX from the prototrophic deletion collection \\
DGY571 &lsm6$\Delta$::KanMX &   ydr378c$\Delta$::KanMX from the prototrophic deletion collection \\
DGY545 &pat1$\Delta$::KanMX &   ycr077c$\Delta$::KanMX from the prototrophic deletion collection \\
DGY554 &edc3$\Delta$::KanMX &   yel015w$\Delta$::KanMX from the prototrophic deletion collection \\
DGY552 &scd6$\Delta$::KanMX &   ypr129w$\Delta$::KanMX from the prototrophic deletion collection \\
DGY611 &tif4632$\Delta$::KanMX &   ygl049c$\Delta$::KanMX from the prototrophic deletion collection \\
DGY539 & \textit{GAP1} 5' UTR delete & confirmed by Sanger sequencing to have 152bp deleted 5' of the start codon \\
DGY576 & \textit{GAP1} 5' UTR delete & confirmed by Sanger sequencing to have 100bp deleted 5' of the start codon \\
DGY577 & \textit{GAP1} 3' UTR delete & confirmed by Sanger sequencing to have 150bp deleted 3' of the stop codon \\
DGY525 & FY3 + pRP1315 & FY3, a ura- auxotroph (ura3-52), transformed with pRP1315 (URA3 marker, expressing a Dcp2-GFP fusion) \\
\bottomrule
\end{tabular}
\end{table}

\begin{table}%[bt]
\caption{Primers used in this study}
\label{tab:primerTable}
% Use "S" column identifier to align on decimal point 
\begin{tabular}{p{.1\textwidth} l p{.25\textwidth}}
\toprule
ID & Sequence & Description \\
\midrule
DGO230 & \scriptsize\ttfamily ACGGTATCAAGGGTTTGCCAAG & Figure 3 qPCR \textit{GAP1} reverse \\
DGO232 & \scriptsize\ttfamily GCATAAATGGCAGAGTTAC & Figure 3 qPCR \textit{GAP1} forward \\
DGO229 & \scriptsize\ttfamily CTCTACGGATTCACTGGCAGCA & Figure 5 qPCR \textit{GAP1} reverse \\
DGO231 & \scriptsize\ttfamily TTTGTTCTGTCTTCGTCAC & Figure 5 qPCR \textit{GAP1} forward \\
DGO236 & \scriptsize\ttfamily TTACCCAATAGCTTGTTCAATT & qPCR HTA1 forward  \\
DGO233 & \scriptsize\ttfamily GCTGGTAATGCTGCTAGGGATA & qPCR HTA1 reverse  \\
DGO605 & \scriptsize\ttfamily CTGGACGACTTCGACTACGG & qPCR 1200 spike-in forward \\
DGO606 & \scriptsize\ttfamily ATCAGCCTTTCCTTTCGTCA & qPCR 1200 spike-in reverse \\
DGO1562 & \scriptsize\ttfamily
GTCTGAACTCCAGTCACATCNCNCNCNTNCNGTCGACCTGCAGCGTA & Degenerate first round primer \\
DGO1588 & \scriptsize\ttfamily CCATTGGTGAGCAGCGAAGGATTTGGTGGA/3Phos/ & First round blocker oligo \\
DGO1589 & \scriptsize\ttfamily AGAAAAAGCAGCGTAGATGTAGAAGCAAGA/3Phos/ & First round blocker oligo \\
DGO1567 & \scriptsize\ttfamily GATGTCCACGAGGTCTCT & Second round outside primer \\
DGO1576 & \scriptsize\ttfamily CGTACGCTGCAGGTCGAC/3Phos/ & Second round blocker oligo \\
DGO1519 & \scriptsize\ttfamily CAAGCAGAAGACGGCATACGAGATGTCTGAACTCCAGTCAC & Second and third round inside primer and P7 adapter \\
Forward index primer & \scriptsize\ttfamily
ACGCTCTTCCGATCTXXXXXGTCCACGAGGTCTCT & Multiplexing primer, where 
XXXXX is one of 120 different barcodes (see below).
\autoref{tab:indexbarcodes}.\\
DGO276 & \scriptsize\ttfamily AATGATACGGCGACCACCGAGATCTACACTCTTTCCCTACACGACGCTCTTCCGATCT & Illumina P5 adapter incorporation primer \\
DGO366 & \scriptsize\ttfamily AATGATACGGCGACCACCGAGATCTACAC & RNAseq Illumina library amplification, forward \\
DGO367 & \scriptsize\ttfamily CAAGCAGAAGACGGCATACGAGAT & RNAseq Illumina library amplifcation, reverse \\
\bottomrule
\end{tabular}
\end{table}

All primers were synthesized by Integrated DNA Technologies (IDT).
\texttt{N} refers to a standard degenerate position.
\label{tab:indexbarcodes}
Barseq multiplexing barcode sequences and index numbers available in 
the file 
\texttt{data/dme209/sampleBarcodesRobinson2014.txt} within the
data zip archive (\nameref{subsubsection:codeanddata}). \\


\chapter{Investigations of physiological remodeling upon a nitrogen 
upshift}
\label{chapter:four}

This chapter describes lines of investigation that were not
pursued deeply, but may inform future investigations of the
physiological remodeling that occurs as yeast resumes rapid
growth.

The study of microbial physiology is a long standing area of
investigation, and with modern systems biology approaches this
question of how the physiological composition of microbes change in
order to accomplish the essential project of growth is still the
subject of advances both quantitative and conceptual 
\parencite{slator1918some,henrici1928morphologic,schaechter1958dependency,kjeldgaard1958transition,wehr1969macromolecular,waldron1977synthesis,carter1978protein,waldron1975effect,kief1981coordinate,scott2010interdependence,erickson2017global,kafri2016cost,metzl2017principles}.
In recent years, the study of changes in abundance of specific mRNA
factors in the budding yeast has characterized a phenomenon in which
approximately one quarter of the yeast transcriptome scales with
growth rate \parencite{brauer2008coordination,airoldi2009predicting}.
This phenomenon is characterized at the level of molecular species,
and thus can be compared to changes that occur in response to
stressors, summarized as the Environmental Stress Response 
\parencite{gasch2000genomic}. 
In addition, a shared signature of knockout mutants, commonly used to
probe gene function, is that associated with changes in cell-cycle
progression distribution due to growth rate changes
\parencite{o2014cell}.
Thus the appreciation of the systematic physiological changes that
occur in response to genetic perturbations holds light to many
biological problems, even if only to identify the domineering and 
confounding factor of growth-associated physiological changes.

%
%These, and other observations, described microbial growth generally as 
%possessing a kind of "interia" [@henrici1928morphologic], where the
%growth rate 
%of the old culture is maintained for a short period after a
%nutrient-shift.
%@sherman1924function studied changes in resistance in bacterial
%salt-stress 
%during this lag-phase to conclude that old cells would remodel their
%physiological state prior to initiating growth at the maximal rate
%possible of the new environment.
%@kjeldgaard1958transition, @wehr1969macromolecular
%waldron1977synthesis  found that following a nitrogen-source upshift,
%a yeast culture will continue its rate of accumulation of optical
%density
%for about 2 hours before increasing the rate to that appropriate to
%the new
%media.
%In contrast, the RNA accumulation appeared to lag only 10 minutes
%before
%accumulating at a rate temporarily faster than the new steady-state
%rate.
%This suggests that rRNA, thus ribosome content, 
%is a leading feature of this physioloical remodeling.
%A carbon-source upshift also results in massive regulation 
%that reprograms the yeast physiology and transcriptome (and other
%-omes) for 
%rapid growth \cite{kief1981coordinate}.
%It was originally observed that yeast paradoxically halts growth
%upon a glucose up-shift for about 60 minutes before resuming
%growth and protein-synthesis [@kief1981coordinate]. However,
%ribosomal RNA and protein were shown to still rapidly increase
%across the span of an hour, even while the rest of the cellular
%growth and division are halted. It is important to note that
%this study measured nascent relative to extant - that is, an
%increase in the ratio of nascent to extant can result from either
%accelerated synthesis or degradation of extant transcripts 
%[@kief1981coordinate].

%
%
%
\section{Changes in poly-adenylated transcript content per cell 
upon changes in growth rates}
%
%
%

This section describes work that contributed to a submitted article 
%that 
%has been submitted, rejected, and is currently undergoing revision 
%of the text to address a particular biological question with more
%focus.
%The article was 
titled:
\textit{"Growth Rate-Dependent Global Amplification of Gene Expression."}
Authorship of this article is: 
Niki Athanasiadou, Benjamin Neymotin, Nathan Brandt, 
\textbf{Darach Miller}, Daniel Tranchina, and David Gresham.
The \textit{biorxiv} draft is at \url{doi.org/10.1101/044735}

The writing and figures of this chapter are original to this document.

%
%
%
\subsection{Introduction}
%
%
%


We know that the total RNA content of a cell changes upon changes in
growth rates \parencite{waldron1975effect}.
We know that specific mRNA, each a small component of the cell's 
total RNA also change in relative abundance. A less-characterized 
question is if the whole mRNA transcriptome changes
and if this has a significant effect on the regulatory role
of absolute or relative changes in mRNA abundance.
Transcriptomic measurements are usually normalized to relative
measures, and is thus based (sometimes explicitly
\parencite{love2014moderated})
on the assumption that the total transcriptome does not change in 
abundance.  However, we now know of cases of where this 
assumption is violated \parencite{nie2012c}.

Spike-in normalized RNA sequencing can estimate absolute mRNA
abundance per cell, but has been criticized before for
being "too noisy" and instead computational methods of "removing
unwanted variation" were used \parencite{risso2014normalization}. 
Led by Rodoniki Athansidou, our group pursued a more thorough approach
to this design by normalizing RNA sequencing data using the ERCC
spike-in set \parencite{jiang2011synthetic}, 
using preliminary sequencing runs
to first determine the appropriate amount of spike-ins necessary for
accurate sequencing. Then, yeast were grown in systematically varied
nutrient limitations of growth, then RNA sequencing using a known
quantity of the exogenous spike-ins was used to normalize the
measurements to absolute mRNA per cell.

I sought to complement this work by orthogonally estimating the size 
of the whole yeast transcriptome. To do this, I adapted the screening 
strategy of \cite{amberg1992isolation} to flow cytometry. 
Essentially, this utilized a poly-deoxythymidine oligo singly labeled
with a fluorophore. This was hybridized in with the fixed and
permeabilized yeast cell, and the resulting fluorescence after washing
is taken to be a proxy for the number of hybridized poly-dT probes,
presumably hybridized to a poly-adenosine sequence, and thus mRNA.

Another motivation of this was to serve as a fixation-digestion
control for methods involving single-gene mRNA FISH.
We had patterns of mRNA FISH hybridization signal that appeared
bimodal (\autoref{fig:gap1Delete}).
This could be a technical issue of incomplete permeabilization due to
over-fixing, or a biological phenomenon.
To distinguish the two would take two-color FISH, with a positive
control \parencite{andersen2014genetic}.
Since nitrogen-limitation causes a severe restriction of the total
transcriptome content, we don't have an obvious pick for a uniformly
expressed positive control. 
However, most of the mRNA should be poly-adenylated, so FISH against
that sequence should be present in all cells, and in high-copy.
While I did not integrate this into the single gene mRNA FISH as an
internal control, I did use it to optimize fixation/permeabilization 
conditions.

%
%
%
\subsection{The assay design}
%
%
%

The assay uses a similar fixation permeabilization method as the
single-gene mRNA FISH assay, then an overnight hybridization 
against a poly-dT probe, then flow cytometry.

The yeast cells are sampled via vacuum filtration onto nylon
filters, then the filters are quickly flash-frozen in liquid nitrogen.
These are resuspended in 0.75x PBS buffered 4\% PFA (from ampules from
EMS), the cells vortexed off the filter, then the filter discarded.
The cell suspension in the fixative is incubated for hours at RT to
complete fixation, with the assumption that rapid fixation halts
RNA metabolism in the cell and long-term fixation stabilizes the fixed
components into a configuration that can survive digestion and 
permeabilization. The fix is critically quenched using 2.5M glycine,
then collected by centrifugation and washed with PBS. The cells are
digested for one hour at 37C using lyticase and beta-mercaptoethanol
in 1.2M sorbitol buffered
by potassium phosphate at about 7.4 pH, with 20mM vanadyl
ribonucleoside complex to inhibit RNAses. 
This is washed and further permeabilized with 70\% ethanol overnight,
then is resuspened using hybridization buffer
(10\% dextran sulfate w/v, 2x SSC final, 100ug/ml ecoli tRNA, 
250mg/ml salmon sperm DNA) plus 100nM of a (dT)50+V oligo 5'-labeled
with with Alexa 488, as ordered from IDT. 
This is incubated for 14+ hours on a 37C roller drum, then washed with
2x SSC several times before resuspending in PBS and flowing through an
Accurri flow cytometer.
Poly(A) content signal was determined by the signal area on the
514/20nm detector.

To test this procedure, I used RNAseA treated cells as a negative
control.
\autoref{fig:ypdnlim} shows the RNAseA-treated controls for two
samples, where the treatment abrogates the signal for the vast
majority of the cells in the sample.

To optimize this design, I varied formamide from 0\% to 50\%, and
probe concentrations from 10nM to 1$\mu$M. I found that 100nM and 0\%
formamide saturated the signal of YPD-grown cells without largely
increasing the signal on the RNAseA-treated cells.
This assay takes approximately 4 hours of work spread over 3 days.
More detailed protocol is maintained by the Gresham laboratory.

%%%%% PUT A CONTROL FIG HERE.

%
%
%
\subsection{Nutrient limitation and transcriptome size}
%
%
%

Yeast growing in
YPD complete a division approximately every 1.5 hours (0.45 specific
growth rate), while 
proline-limited media (NLimPro) only supports division approximately
every 4.5 hours (.15 specific growth rate). 
Using this poly-dT FISH method, we see differences in the total 
poly-adenylated mRNA signal between the different media conditions
(\autoref{fig:ypdnlim}).
The distributions are significantly
different (KS test and Wilcoxon, p-value < $2.2 \times 10^{-16}$).
We know that fast growing cells (YPD) have more RNA per cell, so it
appears that part of this difference is contributed by a global
scaling of the mRNA content as well.
The fold-changes in the mean and median of the YPD-grown cells versus
the proline-limited cells were 3.34 and 3.68, respectively.

\afig{
  \includegraphics[width=.7\textwidth]{img/polya_ypdnlim_controls.png}
  }{
  Wild-type yeast grown in proline-limited media (left) or YPD rich
  media (right) were assayed in exponential growth for poly-A content.
  Included are RNAsed controls treated with RNAseA, to show negative
  samples.
  The plot is cropped from 0 to $10^5$ arbitrary units of polyA
  signal to show the center of the distributions.
  The distributions from different media have different means by 
  KS or Wilcoxon tests, with unreasonably small p-values.
  \label{fig:ypdnlim}
  }{Changes in whole cell polyA content in YPD or nitrogen-limitation.}

To investigate the dynamics of changes in poly-A abundance between 
different growth conditions, I grew cells in proline-limited media
overnight to reach a steady-state of growth, then collected samples
during a nitrogen-upshift.
I assayed the poly-A content of the cells using the above assay
(\autoref{fig:upshift}).
I found that the total poly-A content took about two hours to increase
to the new steady-state of a larger transcriptome, a similar timescale
as the changes in cell size and lag in population growth rate
(\autoref{fig:figure1a}).
The final steady-state differences were of a fold-change of 2.16 and
1.93 for the mean and median poly-A content, consistent with the
change between specific growth rates of 0.15 and 0.35 being lower
than the difference with YPD (0.45 specific growth rate). 

\afig{
  \includegraphics[width=\textwidth]{img/polya_upshift.png}
  }{
  Wild-type yeast were grown in proline-limited media, then glutamine
  was added at time 0 minutes. Samples were assayed for polyA content
  using the poly-dT assay.
  \label{fig:upshift}
  }{Changes in polyA content upon a nitrogen upshift.}

Previously, others in the lab (as described at the beginning of this
section) had used ERCC-normalized RNA sequencing
to assay the absolute abundance of mRNA in yeast grown at
systematically varied growth rates (0.12, 0.2, 0.3 specific growth 
rate) in chemostats. In a repeat experiment of this, I took samples 
from chemostats limited by nitrogen or carbon at these growth rates, 
and processed them to assay the distribution of poly-A content of the 
cells. \autoref{fig:nikis} shows the distributions and the
relationship between the distribution means and the estimates from
SPARQ (the spike-in normalized RNAseq method). We see that the poly-dT
method also captures the scaling of the whole yeast transcriptome
across different growth rates, and correlates well with the spike-in
normalized method.

\afig{
  \includegraphics[width=.48\textwidth]{img/polya_niki_box.png}
  \includegraphics[width=.48\textwidth]{img/polya_niki_summary.png}
  }{
  (Left) PolyA content was estimated for cultures grown in two nutrient
  limitations at three different dilution (growth) rates.
  (Right) Comparing these measurements to SPARQ (the spike-in 
  normalized RNAseq method) shows that two methods are well 
  correlated (Pearson's r=0.95, \texttt{cor.test} p-value = 0.003684,
  dashed-line shows linear regression through all points),
  although the poly-dT method remains uncalibrated.
  \label{fig:nikis}
  }{Measuring polyA content across systematically varied growth rates
    in chemostats, and comparison to a spike-in normalized RNA 
    sequencing method.}

\subsection{Conclusion and future directions}

This assay appears to detect changes in the scaling of the yeast
mRNA content between different growth rates. 
It is consistent with spike-in normalized RNAseq (random 
hexamer-primed) estimates of the total mRNA content.
As a flow cytometry 
assay this has the potential to be used as a marker for
high-throughput investigations of the genetics of transcriptome size
changes (or regulation), using methods as described in
\nameref{subsection:bff}. 
This method offers a conveniently high-throughput assay for total
transcriptome size, and as such is one more tool that microbial
physiologists can use to probe the functional changes that occur as
organisms systemically adapt to their environments and growth
programs.

However, more work remains to use the assay to reliably inform on
these changes without incorporating an orthogonal measure.
Changes in polyA tail length could hypothetically affect
hybridization, and a distribution shifting such that more of the
functional mRNA have a tail length less than minimum tail length
requisite for hybridization would produce a similar graded effect.
Hybridization of this probe to synthetic mRNA cross-linked to a nylon
substrate would allow quantitative testing of this in similar
conditions as the hybridization occurs, provided a method for
manufacturing accurately generated poly-A tail lengths exists.

Future investigations of mRNA content per cell will illuminate
the role or significance of total mRNA abundance 
versus relative mRNA abundance in gene regulation and physiological
adjustments to changing environments. Adjustment is apparent during a 
nitrogen upshift, what causes it, and is it adaptive?

\section{Screening for genes important for remodeling physiology for
growth}

\subsection{Introduction}



With changing physiology in response to growth rate changes, many
molecular and functional phenotypes change. One of these is the
resistance to stress. 
It has been long known that slow growing cells are more resistant to
stressors
\parencite{sherman1923physiological,elliott1993stress,lu2009slow}.
Yeast appears to have adapted to its ecological niche by adopting a
boom/bust, feast or famine approach to quickly
growing during favorable conditions at the expense of stress
resistance. 
Resistance to stress seems to offer "cross-protection", and the
anti-correlation of growth rate and stress resistance suggests that
the two processes might be opposed in mechanisms to
achieve these objectives.
The dimension of coordinated cellular growth may be a simple axis that
explains much of the variation in gene expression and phenotypic
differences in budding yeast
\parencite{brauer2008coordination,lu2009slow}.

One approach to identify the characteristics required for yeast to 
achieve a faster growth rate is to monitor the regulated changes that
occur upon the upshift. We could infer
that since the most logical response to a stress is to express this
adaptation, then the gene expression increasing upon a stress must be
adaptive \parencite{gould1979spandrels}. 
This has been demonstrated to be a false assumption, at least for the
case of heatshocks, as the genes whose expression increases do not 
overlap well with the
genes important for resistance \parencite{gibney2013yeast}.
The later functional genetic measurement is possible to do in 
high-throughput, as
the yeast community has access to a yeast deletion collection and
high-throughput means of assaying genetic effects on the quantitative
phenotype of continued existence. 
Thus, an assay of the functional consequences is a more direct
approach to understand these processes.

The nitrogen upshift enriches for %offers an obvious process to enrich for
differences in growth rate, by growth. 
Subtle effects can be magnified over time, for example a
1\% growth rate defect over 7 hours would be magnified to an abundance
change of at least 20\%. However, the phenotype I am interested in is
in the completion of remodeling for rapid growth, so I am most
interested in the duration of the lag between nitrogen addition and
increased growth. Thus, the compounding of growth rate effect does not 
apply. One approach would be to repeat the upshift many times on the
same batch of cells, but this greatly confounds the fitness between 
various growth stages and does not offer the reproducibility of cells
being in a particular physiological status --- nitrogen-limitation can
take hours to reach a steady-state of signalling
\parencite{tate2013five}, and the life history of an individual cell
could have physiological consequences.

After practicing the Feynmen method with a scientific advisor in
California (\nameref{section:acknow}), and given
the opportunity to work with a talented young scientist named Stephen
Nyarko, we decided to pursue this question by using the correlated
phenotypes of growth and susceptibility to stress.
The logic is that if we are interested in isolating mutants that are
defective in increasing their growth rate upon a nitrogen upshift, and
an increase in growth is associated with a susceptibility to stress,
then a somewhat-lethal stress should enrich for mutants defective in
susceptibility to the stress --- ie defective in rapidly increasing 
growth rate.
Upon further reading, we found that the group of Johan Theiveilen had
used a similar approach to isolate mutants defective for increasing
growth upon repletion of glucose, and had identified new critical
components of the PKA pathway, \textit{CYR1} and \textit{GPR1}
\parencite{van2000characterization}.
Thus encouraged, we intended to use the anti-correlation of growth
and stress resistance to isolate mutants defective in resuming growth
rapidly.

\subsection{Results}

We first determined if the heatshock resistance of the wild-type
FY4 yeast changed during a nitrogen upshift.
I grew cells in proline-limited media (approximately 4.5 hour doubling
time), then added glutamine to induce the nitrogen upshift.
For each sample, cells were subject to a 52C heatshock for 30 minutes
by the addition of pre-warmed media, or for negative control were
simply kept at room temperature. The processed samples were arrayed 
in a 96 well plate, then pinned onto YPD, grown at 30C for
approximately 40 hours, and imaged (\autoref{fig:dme180}).

\newpage
{\color{white}{z}}\\[1em]
\afig{
  \hfill
  \includegraphics[width=.2\textwidth]{img/160328exp180sample1.png}
  \includegraphics[width=.2\textwidth]{img/160328exp180sample2.png} 
  \includegraphics[width=.2\textwidth]{img/160328exp180sample3.png}
  \includegraphics[width=.2\textwidth]{img/160328exp180sample4.png}
  \\
  \begin{tikzpicture}[overlay
      ,font=\small,%font=\ttfamily
      ,inner sep=0pt,outer sep=0pt
      ,shift={(-8,0)}]
      ]
    \node[align=left] at (0,0) (o) {};
%
    \node[align=left] at ($(o)+(5,0)$) (dil) {Culture dilutions (5-fold)};
    \draw[->] (dil) to ($(dil)+(10,0)$);
%
    \node[align=left,anchor=east] at ($(o)+(2,3.00)$) {\tiny Heatshocked};
    \node[align=left,anchor=east] at ($(o)+(2,2.65)$) {\tiny Negative};
    \node[align=left,anchor=east] at ($(o)+(2,2.30)$) {\tiny Heatshocked};
    \node[align=left,anchor=east] at ($(o)+(2,1.95)$) {\tiny Negative};
    \node[align=left,anchor=east] at ($(o)+(2,1.60)$) {\tiny Heatshocked};
    \node[align=left,anchor=east] at ($(o)+(2,1.25)$) {\tiny Negative};
%
    \node[align=left] at ($(o)+(4,4.1)$) {Before upshift};
    \node[align=left] at ($(o)+(7.8,4.4)$) {61 minutes\\after upshift};
    \node[align=left] at ($(o)+(11.2,4.4)$) {132 minutes\\after upshift};
    \node[align=left] at ($(o)+(14.6,4.4)$) {204 minutes\\after upshift};
  \end{tikzpicture}
  }{
    Wild-type (FY4) cells were subject to nitrogen-limitation, then
    a nitrogen upshift with 400$\mu$M glutamine. Cultures were
    heatshocked at 52C for 30 minutes, or at roomtemperature 
    (negative). 5-fold serial dilutions were plated on YPD plates.
    \label{fig:dme180}
  }{Glutamine upshift causes a lost in heatshock-resistance.}

Thus, the glutamine upshift triggers a loss in resistance to the
heatshock. We then devised a screen, wherein a barcoded 
and pooled yeast deletion collection
is grown in conditions of nitrogen-limitation then upshifted.
Samples were taking before or 
after 120 minutes after the glutamine upshift,
heatshocked or not (negatives),
then outgrown to enrich for living mutants.
These libraries were sequenced using an amplicon-sequencing procedure
to quantify the mutants in the resulting library.

Direct measurement of mutant abundance is preferred, but we used
outgrowth of the heatshocked population, counting on the severe
selection of a heatshock to appropriately select.
We did this in six biological replicates in order to generate robust
signal.
These were extracted with standard Hoffman-Winston DNA preparations,
then amplified using the same primers and protocol as described in
\parencite{robinson2014design}. These were sequenced along with other
samples on an Illumina MiSeq run.

Barcode sequencing, like other molecule-counting applications of
sequencing like RNAseq, is presented to the researcher as a relative 
measurement in integer quantities. 
One of the first steps in reading this data in is to look at the
distribution of counts per strain barcode identified
\autoref{fig:sneDist}.
We see that our heatshock and outgrowth has a much more profound
distribution of effects.
For which genes is this significant?


Numerous statistical approaches
exist to normalize the data for accurate detection of differential
abundance. One flexible and robust method is using the \texttt{voom}
statistical pre-processing step with \texttt{limma}. 
This calculates
the expected noise contributed by low integer count observation, but
has the advantage of converting the measurement to a "counts per
million" relative metric for normalization. 
It also has useful visualizations for characterizing the 
distribution of signal across complex experimental designs.

One other observation in \autoref{fig:sneDist} is that the histograms
show a log-normal distribution of high counts, then a long tail
downwards. 
Then, there appears to be a low distribution of single digit counts
which enter the distribution from around zero \autoref{fig:sneDistSum}.
These are believed to occur from spurious amplicon products or
software misalignment counting barcodes that do not exist.
To characterize this further, I used \texttt{limma/voom} to generate
plots of variance against abundance for different thresholds of
cutoffs based on total counts across the entire library
\autoref{fig:vooming}.
I found that a threshold of 30 counts in total across the library
was sufficient to remove these effects.

\afig{
  \includegraphics[width=.6\textwidth]{img/sne_histogram.png}
  }{
  Histograms of counts, for each mutant in each sample. Three
  histograms show the occurrences of these observations for the
  library before the upshift (top), 2 hours after adding glutamine
  (middle), and after the heatshock and outgrowth (bottom). The wider
  spreading is a good indication of complex selection of large
  effects occurring in the library.  \label{fig:sneDist}
  }{Histogram of mutant counts, within each sample.}
\afig{
  \includegraphics[width=.6\textwidth]{img/sne_histogram_sum.png}
  }{
  Histograms of counts, for each mutant, summed across all samples
  in the three treatments: before the upshift (top), 2 hours after 
  adding glutamine (middle), and after the heatshock and outgrowth 
  (bottom). We see that most features are log normally distributed,
  but some appear to be noisy counts near zero, due to unknown
  factors. 
  \label{fig:sneDistSum}
  }{Histogram of mutant counts, summed across samples.}
\afig{
  \includegraphics[width=.7\textwidth]{img/sne_voomer.png}
  }{
  Each plot shows each gene average abundance (x-axis) against its
  residual variation (y-axis), with a line smoothing the relationship
  as expected by \texttt{limma} modeling. The threshold of minimum
  total counts per feature is shown for each plot in the grey bar.
  We see that thresholding above 30 counts \textbf{(right)} gives us the 
  expected relationship, while not thresholding \textbf{(left)}
  demonstrates how lowly abundance counts behave aberrantly with
  artificially reduced variance that may confound statistical
  analyses of barcode sequencing data (\texttt{limma} uses the model 
  fit as the line). 
  \label{fig:vooming}
  }{Diagnostic \texttt{limma/voom} plots show the effects of 
    low-count barcodes in confounding the noise model.}

I used this tool's
flexible general linear modeling interface to ask how our treatment
enriched for particular mutants.
We saw no significant effects from a glutamine upshift, 
confirming the intuition that this
is such a subtle effect of momentary fitness that it becomes hard to 
detect without amplification. 
Testing for the effect of changes in abundance based on a
glutamine treatment before the heatshock, we find that four deletion 
strains significantly
(multiple-hypothesis adjusted p-values < 0.05) increase in abundance 
specifically after glutamine treatment, and 41 are decreased in
abundance.

Of the four genes increased in abundance (suggesting a failure to
resume rapid growth), \textit{SLA1} and
\textit{CAP2} are involved in actin binding and dynamics.
\textit{SLA1} is involved in assembly of the cortical actin
cytoskeleton \parencite{holtzman1993synthetic}, 
while \textit{CAP2} is an actin barbed-end
capping protein that localizes to cortical actin patches
\parencite{amatruda1990disruption}. 
This suggests that these mutants specifically are involved in
remodeling the cortical exoskeleton in a way that makes cells more
susceptible to heatshock, or that these mutants are defective in
increases in stress-resistance associated with slow growth rates.
\textit{SXM1} over-expression rescues mutants defective in mRNA export 
from the nucleus \parencite{seedorf1997importin}, suggesting that it
may play a role in mRNA export itself and that mRNA export may
regulate some important downstream factor associated with increasing
growth.
\textit{MAE1} encodes a malate dehydrogenase. This reaction takes
malate, a citric-acid cycle metabolite, and converts it to pyruvate
\parencite{boles1998identification}.
Pyruvate is an essential substrate for the
biogenesis of the carbon structures of alanine, valine, and other 
amino-acids.
Carbon-skeletons of glutamine can enter the citric-acid cycle from a
point between the entry of pyruvate, and shunt from malate to pyruvate
via \textit{MAE1}.
Considering the enrichment of pyruvate metabolism mRNA identified as 
destabilized in the 4tU label-chase work in 
\autoref{subsection:stabilityChanges}, and that the same experiment 
showed either a stabilization or dramatic synthesis up-regulation of 
\textit{MAE1} mRNA upon the nitrogen upshift, one prediction might 
be that Mae1p provides a shunt by which yeast re-directs the excess of
carbon skeletons from glutamine deamination through the citric-acid
cycle to provide substrates for alanine biosynthesis. This may be
adaptive.

To explore this, I regenerated a \textit{mae1}$\Delta$ mutant using
a KanMX knockout cassette amplified from the yeast deletion
collection, confirming incorporation by PCR. I subjected this mutant
to a glutamine upshift, and saw an increase in the lag-phase duration
(\autoref{fig:mae1})
compared to wild-type (\autoref{fig:figure1a}). I sought to test if this
was specifically due to disruption of the malate to pyruvate shunt for
the effect of alanine metabolism, and so repeated the experiment but
added 200$\mu$M alanine, 200$\mu$M pyruvate, or water (mock) to the 
cell culture at the same time as glutamine.
I did not see a significant effect on the growth rate increase
(\autoref{fig:mae1}).
Thus, disruption of this gene may not result in slower upshift in
growth by virtue of blocking this metabolic pathway, but instead the
metabolic state of the cell before the upshift may not be well 
prepared to resume rapid growth.

\afig{
  \includegraphics[width=.7\textwidth]{img/dme231.png}
  \includegraphics[width=.7\textwidth]{img/dme242.png}
  \includegraphics[width=.7\textwidth]{img/dme244.png}
  }{
  A \textit{mae1}$\Delta$ strain was subject to a glutamine upshift.
  \textbf{(Top)} The mutant alone appears to show a slight defect in the lag 
  phase (approximately 3 hours compared to approximately 2 hours
  \autoref{fig:figure1a}).
  \textbf{(Middle and bottom)} The mutant had glutamine or glutamine and either
  alanine (middle) or pyruvate (bottom) added, with two cultures per
  treatment. Neither showed a significant effect in reducing lag phase
  or increasing growth rate.
  \label{fig:mae1}
  }{A \textit{mae1}$\Delta$ mutant is slower in a glutamine upshift,
    but this is not rescued by supplementation with alanine or
    pyruvate.}

\subsection{Conclusion}

We found mutants knocked out for several non-essential genes changed
their relative susceptibility upon heatshock treatment, suggesting
that their resistance does not decrease as much as wild-type upon the
re-addition of a nitrogen source with adding glutamine.
These mutants could be involved in either the increase in stress
resistance upon slow growth conditions, or the decrease in stress
resistance upon increase in growth. 
For the factors of the actin cytoskeleton, this points towards a
hypothesis that the increase in stress resistance results from
specifically the cortical actin network, and would be testable by
determining when these mutants are more or less resistant to stress
than the wild-type. For \textit{MAE1}, I found that there appears to
be a longer lag phase, but this is not rescued by addition of pyruvate
or alanine. This suggests that the deletion of this gene puts the cell
in a metabolic configuration less capable of rapidly increasing growth
rate upon glutamine addition.

\iffalse
%
%
%
\section{Apparent cell-cycle halt upon nitrogen-upshift}
%
%
%

Apparent cell-cycle halt upon
glutamine addition Previous work on cell cycle halt, basically just
alberghina, PKA and CLN1 This phenomenon has been previously seen.
Upshift ecoli, they get bigger.  This is thought to be because the
critical cell size threshold has been reset by growth signalling
pathways to a larger size. However, this result might argue that
instead it could be regulated by CLN1 transcript abundance.
Alberghina’s demonstrated that depends on Swi4?p, so there you go

Later, it has been shown that this halt in growth seems to occur
through a PKA-mediated repression of
CLN1$\cite{jiang1998??,$tokiwa1994??}



Experiment, results Conclusion ( anything that didn’t get into chapter
3 )
\fi

\chapter{Conclusion}

\section{Possible mechanisms of \textit{GAP1} clearance}

The potential role of Scd6p and eIF4G2 in
regulating mRNA in different environmental conditions Scd6p
interaction with eIF4G1 has been mainly studied, but here we see a
similar phenotype with eIF4G2 (eIF4G1 was not tested). These are
homologs with similar function, although eIF4G1 is expressed higher
during “normal” laboratory growth conditions, ie rich media. eIF4G2
has been shown to be more highly expressed and localized to
processing-bodies during conditions of nutrient limitations 
(Brengues and Parker 2007)
, thus this initiation factor may help to specify
proper storage of certain mRNA during conditions of low amino-acid,
and thus low charged tRNA, availability to help ensure translation
with low resource availability. 

Similarities between different
phenomena of destabilization Hrp1p binding sites are enriched in the
set of transcripts destabilized during a nitrogen upshift 
(González et al. 2000) (Kebaara et al. 2003) (Kessler et al. 1997) (Guisbert et al.
2005) 
The translation and degradation of mRNA are intimately coupled
processes which compete for the same substrate, the mRNA. Perhaps this
is playing a role. But this could also be very indirect.  Degradation
is mediated by connections between 3’ and 5’, and affected by the
process of translation What if mRNA degradation simply depends on the
Lsm1-7 complex interacting with the cap? This is facilitated by
deadenylation and antagonized by translation. One idea could be that
translation initiation and elongation just wiggles the 3’ away from
the 5’, a sort of ribosome-dependent 3’ extrusion ( referencing the
cohesin-dependent loop extrusion model ).  This would sorta explain
Zenklusen’s latest, that cycloheximide preserves distance from 5’ to
3’, and puromycin doesn’t.  

An efficient method for estimating mRNA
abundance in barcoded mutant pools 


Much of what we know about mRNA
degradation has been worked out using genetics. While appealing in its
ease of use and functional consequences, the use of genetic
perturbations also comes with the caveat. Perturbing an exquisitely
homeostatic system, like a living cell, can result in indirect effects
as systems of regulatory feedback percolate the signal through the
network. Thus in this work, the use of knockout mutants is the use of
an indeterminately perturbed system. Additionally, the laborious
creation and maintenance of mutants introduces potential artifacts of
suppressor mutations and additional mutations in the mutant background
(Markowitz et al. 2017). Technologies have been recently developed to
scale heterologous repression of expression to genome-wide scales
(CRISPRi (Smith et al. 2016) and are currently being developed to
deliver programmed single residue editing at thousands of thousands of
sites (personal communication). With the ability to induce programmed
mutagenesis, with  replicates, provides for each experiment a pooled
de novo mutant library. The pooled assay concept demonstrated here is
easily composable with these approaches, and should be the next step.
Critical improvements to this method include a better design of the
gates to improve estimates of mutants with largely divergent
phenotypes. A suboptimal design was used here because the samples were
collected and sorted before the library preparation procedure was
improved to the point where low-input was a feasible option.  The
scalable efficiency of the pooled approach could be used to limit the
search space of more precise but resource-intensive automated assays
(Worley et al. 2015), and thus this hybrid approach could be used to
efficiently increase throughput of measured transcript dynamics. In
particular, the explicit measurement of degradation instead of
dynamics by use of transcriptional inhibition could be used on a
case-by-case basis (Worley et al. 2015), so long as the inhibition was
optimized to offer a relevant measurement (Pelechano and Pérez-Ortín
2008).

\section{Hypotheses}

\subsection{Possible mechanisms of \textit{GAP1} clearance}

More or less ribsomes could affect stability.

Scd6

eIF4G2, sloppy initiation, Nterminal extensions, out of frame stuff
with the Celiak2017 paper.

\subsection{The role of pH in sensing nutrient upshifts}

Glucose upshifts result in a rapid but transient reduction in pH
that is not mediated by xyz
\parencite{kresnowati2008quantitative}
Hesso etc have shown that a similar thing could be happening.
Shifting pH can have strong biophysical effects, including
the aggregation of the poly-A binding protein pab1
Riback 2017
It could also be Gcn2 mediated (huesso)



\SingleSpacing

\printbibliography[heading=bibintoc,title={References}]

\end{document}


