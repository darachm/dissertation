\documentclass[12pt,letterpaper]{memoir}
\usepackage{graphicx}
\usepackage[margin=1in]{geometry}

\usepackage[utf8]{inputenc}
\usepackage{DejaVuSansCondensed}
%\usepackage{helvet}
\renewcommand\familydefault{\sfdefault} 
\usepackage[T1]{fontenc}

\usepackage{wrapfig}
\usepackage{framed}
%\usepackage{subfigure}

\usepackage{pdfpages}

\usepackage{float}
\let\origfigure\figure
\let\endorigfigure\endfigure
\renewenvironment{figure}[1][2] {
  \expandafter\origfigure\expandafter[H]
  } { \endorigfigure }

\usepackage{amsmath}

\usepackage{tikz}
\usepackage{pgfmath}
%\usepackage{pgffor}
\usetikzlibrary{arrows,positioning,shapes,patterns,decorations} 
\tikzstyle{every picture}+=[remember picture]

\usepackage[backend=bibtex8%biber%bibtex8
  ,backref=true,hyperref=true
  ,refsection=none %  ,refsection=chapter
  ,maxcitenames=2
  ,style=authoryear-comp]{biblatex}
\addbibresource{references_dissertation_darachmiller.bib}

\usepackage{hyperref}
\hypersetup{colorlinks, citecolor=blue
  , filecolor=black, linkcolor=blue, urlcolor=black }

\makeatletter 
\def\makemytitle{
  \newpage
\begin{center}
\vfill
  \@title\\
\vfill
  by\\
\vfill
  \@author\\
\vfill
  A dissertation submitted in partial fulfillment \\
  of the requirements for the degree of \\
  Doctor of Philosophy \\
  Department of Biology \\
  New York University \\
\vfill
  \monthz, \yearz\\
  \end{center}
\vfill
\begin{flushright}
  $\rule{5cm}{0.01mm}$\\
  David Gresham
\end{flushright}
  \pagebreak
\begin{center}
  \textcopyright \hspace{0.2em} \@author \\
  All rights reserved, \yearz \\
\end{center}
  \pagebreak
}
\makeatother

\title{
The extent of mRNA destabilization and 
genetic factors of rapid \textit{GAP1} repression 
upon a nitrogen upshift.
}
\author{Darach Miller}
\def\monthz{May}
\def\yearz{2018}

%
%
%
%
%
%

% Memoir layout options
\setulmargins{1.0in}{*}{*}
\setheadfoot{15pt}{0.75in}
\settypeblocksize{8.25in}{*}{*}
\setlrmarginsandblock{1.0in}{1.0in}{*}
%
\openright
\chapterstyle{dash} %dash, dowding, ntglike, thatcher
\setlength\beforechapskip{0pt}%-\baselineskip}
\aliaspagestyle{cleared}{plain}
\aliaspagestyle{chapter}{plain}
%
\setlength\floatsep{0em}
\setlength\textfloatsep{0em}
\setlength\intextsep{0em}
\renewcommand{\topfraction}{0.90}
\renewcommand{\bottomfraction}{0.9}
\renewcommand{\textfraction}{0.10}
%
\setbeforesecskip{-1.0em plus -0em minus -0em}
\setbeforesubsecskip{-1.0em plus -0em minus -0em}
\setbeforesubsubsecskip{-1.0em plus -0em minus -0em}
\setaftersecskip{1em}
\setaftersubsecskip{1em}
\setaftersubsubsecskip{1em}
\setsecheadstyle{\large\bfseries\raggedright}
\setsubsecheadstyle{\normalsize\bfseries\raggedright}
\setsubsubsecheadstyle{\normalsize\bfseries\raggedright}
%
\abnormalparskip{0em}
%
\checkandfixthelayout

%\newcommand{\openrule}{
%  \par\noindent\rule{.5\textwidth}{1pt}\par\vskip-0.2em
%  \noindent\rule[\baselineskip]{1pt}{\baselineskip}}
%
%\newcommand{\closerule}{
%  \par\hfill\rule[-\baselineskip]{1pt}{\baselineskip}
%  \par\vskip-0.2em\hfill\rule{.5\textwidth}{1pt}\par}

\newcommand{\afig}[3]{
  \begin{figure}[t]
%    \vspace{-10pt}
    \begin{center}
    #1
    \end{center}
    \vspace{-10pt}
    \caption[#3]{\textbf{#3} #2}
%    \vspace{-10pt}
  \end{figure}
  }

%
\includeonly{
  dissertation_darachmiller_1,
  dissertation_darachmiller_2,
  dissertation_darachmiller_3,
  dissertation_darachmiller_4,
  dissertation_darachmiller_5
}


% hax to get the compiler to shut up, ripped from SO
\hfuzz=40pt
\vfuzz=40pt
\hbadness=5000
\vbadness=\maxdimen

\setsecnumdepth{subsubsection}
\settocdepth{subsection}

\begin{document}
\DoubleSpacing
\frontmatter
\pagestyle{empty}
%\renewcommand{\thefigure}{\arabic{figure}}

\makemytitle

\iftrue
\begin{quote}
\SingleSpace
"Science is a match that a person has just got alight. 
They thought they were in a room --- 
in moments of devotion, a temple --- 
and that this light would be reflected from and display walls 
inscribed with wonderful secrets and pillars carved with 
philosophical systems wrought into harmony. 
\vfill
It is a curious sensation, now that the preliminary splutter is 
over and the flame burns up clear, to see lit just their hands and 
just a glimpse of themselves and the patch they stand on visible, and 
around them, in place of all that comfort and beauty they
anticipated, \vspace{0.5em} darkness still."

\hfill - H.G. Wells, 1891, \\\vspace{1em}\hfill adapted for 2018
\end{quote}
\fi

\newpage

\section*{Acknowledgements}
\addcontentsline{toc}{section}{\numberline{}Acknowledgements}

\label{section:acknow}

\iffalse


The culture built over these many years at NYU Biology is a unique 
moment of growth, and these informal linkages play a surprisingly
important role in facilitating the collaboration and support
necessary.As Benjy Neymotin 
pointed out several years ago: "it takes a village".

Specifically,
I thank the kind folks on the 4th floor of CGSB, namely
the research group led by Christine Vogel. I am glad we moved to share
space and scientific discussions.
I also thank Ken Birnbaum and his group for helpful conversations and
access to the lucky FISH incubator.
Across the street, I thank Viji Subramanian, Andreas Hochwagen, 
and his research group for microscope access and tolerating my
questions.
I thank Andres Mansisidor and Matt Paul for helpful reading and 
comments on the paper, and otherwise.

Away from NYU, I thank
Evelina Tutucci for demonstrating the mRNA FISH method to me,
helping me understand just how finicky it can be.
I thank Megan McClean for openly sharing a FISH protocol on
OpenWetWare (although we've never talked).
On the west coast (and now elsewhere) I thank Carole Hom, Ian Korf, 
and Patrice Kohel at UC Davis, Nitin Baliga Aaron Brooks at ISB
Seattle, but most importantly Marc Facciotti at UC Davis for 
giving me a chance at getting into science late.

I want to thank Sarah Nguyen for keeping me together, like 

I thank Michi, Henri(ci), and Tai for making me get up and come 
home on time.


I thank Sue Miller for devising a clever genetic screen for yeast
strains that fail to re-initiate growth programs, as well as for
being so supportive through the years.
I also thank my Kendra Miller for inspiring and egging me on, 
a knowingly efficacious combination of support for folks like me.
I thank Harry Miller for support and inspiring me to wander this way,
and to never stop learning new things.

I thank staff at eBioscience/Affymetrix/ThermoFisher for support with
the Quantigene/FlowRNA probe set and patience through the years.

I thank the Cold Spring Harbor Yeast Course, especially the instructors
Maitreya Dunham, Marc Gartenberg, and Grant Brown for introducing
me to the community and (re)-lighting a fire to help me finish.
That course is the reason I'm staying in science as long as I can.

I thank past students including Daniel, Alex, and Stephen.

My committee for the most fun three hours I had each year.

Past and present members of the Gresham and Vogel labs for 
discussions and support.
Especially Niki, Steff, Nathan, and Benjy; Naomi, Farah, Siyu,
Pieter.

To Rodoniki Athansidou for steeling me against the alluring
and corrupting temptation of Kits, and Terrance Cooper for sharing
with me the wisdom of when Kits are necessary for a field to
progress. Threading the gradient between these two extremes seems to
contain some degree of powerful wisdom, something about life and
practice and cargo cults and the very nature of scientific
investigations and Darwinian versus Lamarkian biology, but I have 
not yet had the time to condense it into a sharable form (or a Kit).


the NYU Genomics Core facility for sequencing, flow cytometry,


The aim of sailing is to use lines and fabric and structure to
juxtapose a variety of forces against each other, 
and thus give them the only option of causing progress. 
Without the right constraints and re-direction, 
the system is a mess, destined for the rocks.
A sailboat needs the wind as much as it needs the water,
for without a good keel there's little for the winds to push off of.
Without wind there's little action, and a tiller's important for
deciding where the action should proceed.
Many human endeavors can be described with 
this metaphor --- a variety of forces, pressures, opportunities, 
all swirling around to make something flow in a particular
direction. When all the components align with and against each
other harmoniously, then the effect is made. Just a rope does not make
much progress upwind, and it depends on the cleats and pulleys and
sails and so forth to give it purpose and allow it to play any role in
effecting progress.
 
I would like to acknowledge the system of people that has made 
progress possible.

\fi

\newpage
\pagestyle{plain}

\section*{Abstract}
\addcontentsline{toc}{section}{\numberline{}Abstract}

Cellular responses to changing environments frequently involve rapid
reprogramming of the transcriptome. Regulated
changes in mRNA degradation rates can accelerate reprogramming by
clearing or stabilizing extant transcripts. 
During a nitrogen upshift, the budding yeast \textit{Saccharomyces
cerevisiae} reprograms its physiology in order to enter into  
a state of rapid growth. 
I sought to understand the role of transcriptome reprogramming
in effecting this transition.
During the first minutes of a nitrogen upshift,
preceding the changes in population growth rate,
the transcriptome quickly into a state transiently distinct from
either rapid or slow growth.
During this, the five fastest decreasing mRNA are all NCR-regulated 
transporters, suggesting that this reprogramming is specifically
targeted to set the transcriptional program for rapid growth.
Many transcripts decrease significantly faster than expected from
their stability measured during steady-state growth during
rich media conditions, so I set out to determine the stability of
the yeast transcriptome preceding and during a nitrogen upshift.
I used 4-thiouracil labeling to track mRNA stability, and found 
that 78 mRNAs are subject to destabilization. 
These transcripts include
Nitrogen Catabolite Repression (NCR) and carbon metabolism mRNAs,
suggesting that mRNA destabilization is a mechanism for targeted
reprogramming. 
To explore the molecular basis of destabilization and possible
mechanisms of specifying or effecting the destabilization, I
developed a novel method combining mRNA FISH, fluorescence-activated
cell sorting, and DNA barcode sequencing to screen the pooled deletion
collection library for \textit{trans} factors that mediate rapid GAP1 mRNA
repression. This required that I completely redesign the methods for 
low-input multiplexed barcode sequencing, as well as adapt 
branched-DNA single-molecule mRNA FISH protocols to work in budding
yeast. Modeling of the data identifies known factors of mRNA
degradation, namely all three non-essential components of the
Lsm1-7p/Pat1p complex, as being important for proper \textit{GAP1}
mRNA dynamics. However, the modulators \textit{EDC3} and \textit{SCD6}
have more complex phenotypes. Re-analyzing previously collected data,
I identified that a \textit{scd6}$\Delta$, \textit{tif4632}$\Delta$
(eIF4G2 knock-out), and a \textit{GAP1} 5' UTR delete strain all
share a similar phenotype of lower \textit{GAP1} expression preceding
the upshift, and reduced decay upon the upshift. This suggests that 
translation initiation dynamics may play a role in mRNA levels and 
their stability, and suggests that the destabilization phenomenon 
may in fact be the release of a stabilization effect.

\newpage

{\SingleSpacing
\tableofcontents}

\listoffigures

\mainmatter

\chapter{Introduction}

Organisms adapt to
their environment by expressing different phenotypes as environments
change. We expect this to be an advantageous strategy, depending on
the various parameters of fitness advantages of each phenotype in each
environment balanced against the costs of innovating and maintaining
the machinery for adaptive differential gene expression
\parencite{kussell2005phenotypic}. 
If different mechanisms have different properties of
efficacy and energetic costs, and if the phenotype to fitness
relationship varies, then different conditions may select
for the use of different mechanisms of achieving appropriately
regulated gene expression.
Thus, understanding the adaptive basis of the diversity of 
regulatory mechanisms may inform our understanding of selection 
on complex systems.

\section{Regulated gene expression} 

Expression of a protein-coding gene
product involves many complex steps, each with their own opportunities
for a variety of regulatory mechanisms. At the outset, DNA sequences
that encode genetic elements are transcribed into corresponding chains
of messenger RNA (mRNA). The rate of this transcription is helped by
factors that facilitate recruitment of RNA polymerase II (RNA Pol II)
and is hindered by factors that block this process by physical
occlusion or changes in the accessibility to the chromatin 
\parencite{hahn2011transcriptional}. 
The translation of mRNA into a protein product by
ribosomes occurs at different rate for different genes and different
environments, regulated by a complex interplay between ribosomes and
associated translation factors, RNA binding proteins (RBPs), and
intrinsic factors of the mRNA like length or codon-usage 
\parencite{dever2016mechanism}. 
For both the mRNA and its protein product, stability is also
important 
\textbf{(cite something about protein stability)} 
\parencite{perez2013eukaryotic}. 
Additionally, localization or allosteric regulation can change
the activity of a gene product. Myriad factors contribute to the
expression of a gene product, and determining the functional adaptive
basis for particular regulatory mechanisms, if they are indeed
adaptive, would help us better understand the diversity of gene
regulatory mechanisms.

Functional transcriptome reprogramming during a
nitrogen upshift The budding yeast \textit{Saccharomyces cerevisiae}
is a
classically studied model system for many fields, including gene
expression regulation. In response to different nutrient
availabilities, yeast will change gene expression programs on the
whole transcriptome and adopt different rates of growth. The
particular nutrient environment is described in terms of the quality
and quantity of the nutrient provided. Quantity refers to just that,
the molar availability of the nutrient that the yeast can take up,
while quality is more of a empirical reference to how rapidly budding
yeast can biochemically incorporate the nutrient into their metabolism
--- ie rate of growth. One prediction from this understanding is that
altering the quantity of the nutrient availability to vary growth
rates in a range below which the quality limits growth rates will
elicit a common response between various nutrient limitations.
Indeed, studies systematically varying nutrient environments have
shown that about a quarter of the transcriptome is differentially
expressed at different steady-state nutrient-limited growth states,
regardless of nutrient used to limit growth 
\parencite{brauer2008coordination,regenberg2006growth}. 
Statistical modeling of this process
determined that the molecular signature of this growth-rate signalling
could be capture in a small number of “calibrator” genes whose
expression was very well correlated with changes in growth rate or
perturbation of signalling pathways associated with this process, and
importantly also changed during dynamic transitions or upon
perturbation of growth signalling pathway PKA 
\parencite{airoldi2009predicting}.
Dynamic transitions to better nutrient environments (nitrogen, carbon,
and phosphorus upshifts) shared a similar pattern 
\parencite{conway2012glucose}, 
and the pattern of gene expression associated with increased
nutrient availability and growth rates is the opposite of the
Environmental Stress Response (ESR) --- a shared in co-regulation of
~600 mRNA across dynamic responses to various stressors 
\parencite{gasch2000genomic}. 
Together, this shows how yeast has a common response that
largely corresponds to the suitability of the sensed sensed
environment. A better environment transduces to faster growth with
more growth associated mRNA and less stress response mRNA, and this
holds true in steady-state differences and dynamic transitions.  

One
classically studied transition between growth rates is the nitrogen
upshift. Yeast grows quickly when provided with nitrogen sources like
glutamine or ammonium sulfate, but can make use of various
less-preferred nitrogen sources like proline or urea by expressing
overlapping sets of specific and general nitrogen-source permeases
that concentrate these sources inside the cell for use.  

Various
nitrogen sources are then catabolized to eventually make glutamate and
glutamine, with an estimated ~85\% of macromolecular nitrogen coming
from the amino nitrogen in glutamate and the rest from the amide group
of glutamine 
\parencite{magasanik2002nitrogen}. The addition of glutamine to
a nitrogen-limited culture, for example grown with only the
non-preferred proline as a nitrogen source, is called an upshift
because it is the change from a slow growing condition to one of rapid
growth. Through the use of a temperature-sensitive glutamine synthase
allele or treatment with methionine sulfoximine, it has been shown
that NCR is a response to intracellular glutamine availability
\textbf{(cites)}
. Upon an upshift, a regulatory phenomenon called nitrogen
catabolite repression (NCR) ensures that the set of NCR transporters,
metabolic enzymes, and regulatory factors are repressed.  

One layer of
the repression occurs at the level of transcript synthesis. Four of
the five GATA factors in yeast coordinate to control transcription of
NCR genes, with two factors (Gln3p and Gat1p) activating transcription
while two (Gzf3p and Dal80p) repress transcription 
\parencite{hahn2011transcriptional,stanbrough1995transcriptional}. 
These factors are also subject to NCR
control to different extents, with the activators increasing the
expression of the repressive factors. This is thought to be an
adaptation to enable quick repression upon a nitrogen upshift, as may
be encountered when yeast is introduced to a new abundant nutrient
environment of grape or wort 
\textbf{(cite)}
. It has been long known that the
eukaryotic growth signalling pathway TORC1 largely regulates these
factors by controlling the activity of phosphatases and thus
localization of these transcription factors, via Ure2p for Gln3p 
\parencite{beck1999tor} 
and unknown mechanisms for Gat1p
\textbf{(citez, also classic ure2 papz)}
. However, careful genetics over the last two decades in
Terrance Cooper’s laboratory has identified that the genetic
requirements for phenotypes differ in different environments, with
comparisons of nitrogen “starvation” (8+ hours) versus “limitation”
(<3 hours, or proline) showing that a large part of NCR regulation was
still unexplained by TORC1 signalling 
\parencite{tate2013five}. By way
of a temperature-sensitive tRNA allele, they have since identified
that Gcn2p impinges in a parallel pathway through the 14-3-3 proteins
Bmh1/2 to promote the export of Gln3p and Gat1p 
\parencite{tate2015gata,tate2017general}. 
Additionally, others have suggested that the
amino-acid permease Gap1p may directly signal to PKA, but more has to
be done 
(theivelen eh)
. Thus, multiple signalling pathways converge to
affect the import and export of NCR GATA factors to effect multiply
redundant layers of NCR transcript synthesis control.  

Gene product
regulation can also occur post-translationally. NCR has primarily
referred to the control of transcript synthesis rates, but it has been
long observed that upon addition of a preferred nitrogen source the
enzymatic and permease activities are repressed faster than can be
caused by a shut-off of synthesis 
\parencite{cooper1983function}. A
classical NCR-regulated gene is the general amino-acid permease GAP1.
GAP1 mRNA is repressed much faster than the repression of the
protein-product 
\parencite{stanbrough1995transcriptional}
, and we know that this
Gap1p shut-off is adaptive \parencite{risinger2006activity}, 
perhaps due to an
excess of amino-acid transport causing ammonia toxicity 
\parencite{hess2006ammonium}
or excess proton symport driving a depolarization against futile
Pma1p proton-export activity. This growth phenotype allowed the early
identification of mutants in this process, and this indicates that it
is mediated by a uniquitinyation mark that inactivates the permease
and leads to relocalization and degradation 
\textbf{(more risinger? (Risinger and Kaiser 2008) earlier? smething
from magasanik?)}
. Thus multiple layers redundantly repress
the NCR-regulated Gap1p.  

In Chapters 2 and 3, I show how mRNA
degradation also plays a role in this repression.  

\section{mRNA degradation and its regulation}

Even simply considering the regulation of mRNA
abundance, there are at least two processes that contribute --- that
of synthesis and degradation. We know much about transcript synthesis,
perhaps owing to the fact that virtually all events of mRNA synthesis
pass through a well-characterized reaction of synthesis by RNA Pol II,
capping and polyadenylation, and export into the cytoplasm. The
details may vary, but the common pathway is the same. Conversely, mRNA
degradation does have a main pathway that performs the bulk of mRNA
degradation but mRNA are also subject to divergent redundant pathways
that have been challenging to measure. Moreover, the rates of this
various process are subject to control in ways less well-understood.
While some similarity is thought to exist in how RBPs may recognize
cis-element sequences in RNA similar to how TFs recognize upstream
activating or repressing sequences in DNA, the single-stranded
nature of mRNA complicates this process by blocking linear
cis-elements 
\parencite{li2010predicting}
Additionally, these secondary
structures of RNA may be recognized as the cis-element, complicating
our approaches to recognize these patterns
\parencite{goodarzi2012systematic}.
Primary pathways of mRNA degradation The canonical protein-coding mRNA
is synthesized in the nucleus from a DNA template by RNA Pol II, and
is capped co-transcriptionally at the 5’ end with a m7G cap. As Pol II
transcribes sequence 3’ of the stop codon the cleavage factor complex
recognizes cis element binding motifs in the RNA to direct cleavage
and polyadenylation to specific sites in the mRNA. Upon successful
completion of this process, the nascent mRNA is exported to the
cytoplasm where it enters into the pool of translatable mRNA.
Typically, translation begins when initiation factors load ribosomal
subunits to scan the 5’ leader or untranslated region (UTR) for the
start codon where the process of coding sequence translation begins.
These initiation factors (eIF4F) bind the m7G cap to load ribosome
subunits (Dever et al. 2016), and thus most translation depends on the
cap ( with an exception demonstrated by internal ribosome entry sites
\parencite{gilbert2007cap}.
The m7G cap is also critical for mRNA
stability. Xrn1p is a highly-processive combination of helicase and
exonucleolytic domains that as a single protein rapidly degrades
transcripts from a 5’ to 3’ end, recognizing unprotected 5’
phosphorylated ribonucleotides as substrates \parencite{parker2012rna}. Thus, the
inverted linkage of the m7G escapes degradation.  

During rounds of
translation the poly-adenosine tail is shortened from about 65-90
adenosines to about 10 adenosines by a combination of the Pan2/3 and
Ccr4/Pop2 deadenylase complexes, with activity antagonized by the
poly-A binding protein Pab1p
\parencite{parker2012rna,decker1993turnover}.
When the tail is thus shortened to ~10 adenosines, the Lsm1-7p/Pat1p
complex binds the remainder of the poly-A tail (Tharun et al. 2000).
This complex is a heptameric ring of the Lsm1-7 proteins with the
Lsm1p’s C-terminal domain elegantly spanning the center 
\parencite{sharif2013architecture}
and the last eight residues projecting into this center
and critical for binding the shortened poly-A tail 
\parencite{chowdhury2016mutagenic}.
The critical function of this complex is to recruit and promote
activity of the decapping complex to the 5’ end of the mRNA, and in
cooperation with Pat1p \parencite{chowdhury2014pat1}.
the binding of this
complex to mRNA and to decapping factors is indeed correlated with
decapping of the mRNA 
\parencite{chowdhury2009activation}.
Thus, the complex
maps the deadenylated status to the next step in mRNA degradation.  

A
cytoplasmic mRNA without a 5’ m7G cap is not long lived, by virtue of
Xrn1p, thus the recruitment and activation of the decapping complex is
thought to be the key regulatory step in rates of mRNA degradation
\parencite{coller2004eukaryotic}.
Dcp2p carries out the catalytic activity of
the holoenzyme but is promoted by the effects of Dcp1p, and in
comparing in vitro to functional in vivo assays of mutants it appears
that the catalytic rate of the enzyme is not the limiting step 
\parencite{tharun1999analysis}. 
Rather it is re-modeling of the mRNP complex that
leads to association of the decapping enzyme complex with the 5’ cap,
and the rate of this process determines the activity of this
degradation pathway 
\parencite{tharun2001targeting}.
This
decapping-enzyme-localization process is promoted by the Lsm1-7p/Pat1p
complex and inhibited by poly-A binding protein Pab1p (perhaps by
competition of Pab1p with Lsm1-7p/Pat1p for the poly-A tail) and eIF4E
\parencite{coller2004eukaryotic,caponigro1996mechanisms}
. eIF4E, eIF4G, and
eIF4A comprise the cap-dependent translation-initiation factor eIF4F
\textbf{(citez)}
, which gives rise to an elegant model for the observed effect
of translation initiation inhibiting decapping by competition for the
5’ m7G cap 
\parencite{huch2014interrelations},
while the requirement of sufficient
poly-A tail for Pab1p to bind is consistent with the effect of
deadenylation in promoting deapping (Parker 2012). The Lsm1-7p/Pat1p
complex and Dcp2p/Dcp1p are physically associated (by
co-immunoprecipitation) with several factors that genetically modulate
the activity of remodeling step ---Dhh1p, Edc3p, and Scd6p 
\parencite{nissan2010decapping}.

Dhh1p is a helicase that associates with polysomes (mRNA
with multiple ribosomes bound) and has been recently demonstrated to
be genetically required for the relationship between codon optimality
and mRNA stability 
\parencite{radhakrishnan2016dead,presnyak2015codon,sweet2012dead}. 
It has been demonstrated to play a critical
genetic role in mapping codon-optimality and translation speed back to
changes in mRNA stability. Curiously, tethering Dhh1p to a 3’ UTR
using the MS2 system resulted in lower translation rate of an mRNA
despite causing more ribosomes to be associated with the mRNA
(Sweet et al. 2012)
, suggesting that Dhh1p resolves slowly translating
ribosomes by promoting decapping and mRNA degradation.  
However, physically tethering Dhh1p may affect its role if
topology of interactions is important or movement along the mRNA
is important for its function.


Edc3p
physically associates with the decapping complex and stimulates its
activity \parencite{nissan2010decapping}, 
but it has also been shown to be
important \parencite{decker2007edc3p,huch2016decapping} but not critical 
\parencite{rao2017numerous}
for the formation of processing-bodies. 
These are
microscopically visible foci of mRNA and degradation factors that form
in response to stress conditions but are assumed to be condensed from
mRNA-protein complexes that exist before stress 
\parencite{sheth2003decapping,lui2014granules,rao2017numerous}. 
Interestingly, the
canonical role of processing-bodies had been as a site of mRNA
degradation, but recent work with refined genetic tools demonstrated
that mRNA degradation takes place outside of these sites 
\parencite{tutucci2017improved}. 
A mutant deleted of EDC3 and the C-terminal domain of the
essential LSM4 is deficient in processing-body formation with , and is
surprisingly this processing-body deficiency also correlates with a
deficiency in the stabilization of several mRNA upon osmotic stress
\parencite{huch2017mrna}. 
Thus, the role of Edc3p may be to promote
interactions between these mRNPs (mRNA-protein complexes) in a manner
that affects how well mixed the degradation factors and targeted mRNAs
are.  

Scd6p inhibits the formation of the 48S pre-initation complex
(when the 40S subunit associates with eIF4E cap-dependent initiation
factors and begins to scan the 5’ UTR) via forming its own cap-binding
complex with eIF4G subunit eIF4G1 in an arginine-methylation-dependent
manner 
\parencite{rajyaguru2012scd6,poornima2016arginine}. 
This is thought
to physically occlude the normal initiation complex from binding.
Scd6p also binds to several other factors in the Lsm1-7p/Pat1p
deadenylation-promoting complex 
\parencite{nissan2010decapping}
, and thus may
play an indirect role in preventing the pre-initiation complex from
binding and stabilizing the mRNA through translation.  

Other pathways
of mRNA degradation exist. If the poly-A tail is completely removed,
the cytoplasmic exosome complex can degrade the mRNA from 3’ to 5’.
This redundant mechanism allows a xrn1$\Delta$ mutant to grow, although
slowly, as the 5’ to 3’ pathway is throught to effect the bulk of mRNA
degradation 
\parencite{parker2012rna}. 
The balance between the two may be because
of the enzymatic rate of digestion, but more likely because the
binding of the Lsm1-7 complex to the shortened poly-A tail protects
the mRNA from further deadenylation and thus represses this 3’ pathway
\textbf{(cite Tharun on this)}
.

\section{Alternative pathways of mRNA degradation ---
quality control and other }

Three other pathways are known to act as a
co-translational layer of quality control, where errors detected by
abnormal translation processes result in destruction of the presumably
defective mRNA. Nonsense-Mediated Decay (NMD) is the canonical example
of this. A mutation, transcriptional error, alternative splicing
event, or abnormal translational event (like leaky scanning or a uORF,
explained later) can cause an mRNA to have a stop codon well before
the usual position, which is recognized for destruction by a
deadenylation-independent decapping and 5’->3’ decay 
\parencite{muhlrad1994premature}. 
How the aberrant nature of the misplaced stop codon is
detected is still a mystery, but NMD sensitivity is known to be
maximally active after Xaa have been translation and this effect
reduces to minimal activity towards the 3’ end of the transcript
\parencite{losson1979interference}. 
Non-Stop Decay refers to the inverse of
NMD, no stop codon. Ribosomes that over-run into the poly-A tail
recruit degradation by the 3’ exonuclease 
(exo review)
. No-go Decay is
named after the phenomenon that triggers it, when ribosomes no go.
Difficult to translate sequences (lysine repeats) trigger the
endonucleolytic cleavage of the offending mRNA, which is then degraded
from the cut towards both ends 
\parencite{doma2006endonucleolytic}. Very recently,
it has been shown that it is likely the ribosome collisions that
promote the ribosome ubiquitination associated with the triggering of
No-Go decay 
\parencite{simms2017ribosome}, and other work has suggested that the
protein Asc1p may play a key role in this process 
\parencite{ikeuchi2016ribosome}, perhaps via K63 ubiquitination 
\parencite{saito2015inhibiting}. Together
these pathways surveil translating mRNAs for defects, but it is likely
that false positives in the recognition process also contribute to
their regulatory effects.  

Disruption of the NMD pathway is associated
with different expression of many transcripts. Recent genome-wide
analysis identifying ~900 mRNA upregulated upon deletion of any of
UPF1-3, and subsequent ribosome profiling found this targeting to be
associated with out-of-frame translation effects and non-optimal
codons \parencite{celik2017high}.
NMD has been implicated in the regulation
of ribosomal protein subunit pre-mRNA 
\parencite{garre2013nonsense}
in different
environmental conditions, has been shown to interact genetically with
Hrp1p and cis-elements spanning the start codon of PPR1 mRNA to target
this mRNA for degradation 
\parencite{pierrat19935,kebaara2003upf},
and may be triggered by upstream open reading frames (uORFs, discussed
later). These could be specific regulation, or aberrant probabilistic
activation due to the sensitivity of co-translational quality control
\parencite{celik2017high}.

% Chen 1998 hrp1 was figured using adh2 !!!!

The interaction of translation and mRNA
degradation Codon-optimality refers to the concept that certain codons
are translated by the ribosome more quickly than other codons. This is
thought to result in part from changes in tRNA abundance and in part
due to intrinsic differences in the decoding rates 
\parencite{curran1989rates,thomas1988codon}, and often quantified using the tAI index
\parencite{reis2004solving}. The expectation that tRNA availability is
associated with increased rates of translation has been tested with
more recent ribosome footprint profiling experiments, and consistent
with this ribosomes tend to occupy optimal codons less often 
\parencite{weinberg2016improved}.  

The functional relationship between codon-optimality
and mRNA degradation rate had been considered and rejected by a review
of single-transcript studies 
\parencite{caponigro1996mechanisms}. However,
with the advent of accurate genome-wide measurements of mRNA
degradation rates, we are able to explore the generality of this
principle in a relatively unbiased way. Several groups 
\parencite{presnyak2015codon,neymotin2016multiple,harigaya2016codon,cheng2017cis}
have found that poor codon-optimality and lower ribosome density
is associated with a higher degradation rate when considered on a
per-transcript basis. This can be explained through multiple models.
One model is that translation elongation rates are sensed, with slower
elongation accelerating the degradation of mRNA. Jeff Coller’s group
has worked extensively on Dhh1p, and found that it is genetically
required for the clear relationship between codon-optimality and mRNA
stability 
\parencite{presnyak2015codon,radhakrishnan2016dead}. Although
the mechanism is at this point unclear, Dhh1p’s genetic association is
a fruitful hub to work out from.  

Alternatively, competition between
decapping enzymes and translation initiation factors for access to the
5' m7G cap has long been proposed as a mechanism by which the two
processes interact 
\parencite{schwartz1999mutations,schwartz2000mrna}. Karsten Weis' group 
(Chan et al. 2017) 
\parencite{chan2017non}
reproduced the result
that slowing elongation with cycloheximide, sordarin, or 3AT treatment
slows mRNA degradation, but conversely inhibition of initiation with
hippuristanol or a dominant negative eIF4E increased degradation
rates. These measurements were made on the whole-transcriptome using
4-thiouracil and RNA sequencing, similar to RATEseq 
\parencite{neymotin2014determination}. 
The connection between the effect of elongation rates and
initiation rates could be explained by the effect of slow
elongation rates inhibiting initiation events, as predicted 
\parencite{shah2013rate} and measured \parencite{chu2014translation}.

Thus, much evidence points
to competition between translation initiation and 5’ to 3’ degradation
initiation at the cap as a major determinant of mRNA stability,
although the molecular work with Dhh1p suggests that events after
initiation still play a role. Other mRNA degradation pathways like NMD
or NGD during elongation (as discussed earlier) could also possibly
contribute to the effect.  

\subsection{ Regulation of mRNA degradation }

mRNA
degradation can be affected by various trans factors. While micro RNAs
are prolific in regulating mRNA in animals and plants, budding yeast
do not make use of this mechanism. Instead, in yeast mRNA degradation
appears to be determined by a combination of intrinsic properties like
length or codon-optimality, and trans factor RNA binding proteins
(RBPs) that can bind cis element sequences in the mRNA sequence to
effect changes in stability 
\parencite{li2010predicting,cheng2017cis}. The best
example of this is Puf3p, which binds motifs in mRNA with products
destined for mitochondrial function and degrades these in the
appropriate environment 
\parencite{olivas2000puf3,miller2013carbon}, perhaps by
mapping phosphorylation of Puf3p to association of these mRNA to
cytoplasmic granules 
\parencite{lee2015glucose}. Secondary structures may
complicate the recognition of linear cis elements, or be used as cis
elements in their own right \parencite{li2010predicting},
for example Vts1p (Smaug homolog) recognizes a small sequence 
motif in the context of the loop of a stem-loop hairpin 
\parencite{she2017comprehensive,aviv2003rna}.
Degradation rates can be affected by many mechanisms. Elements in
promoters (cis when in DNA but not part of the affected mRNA) can
“mark” transcripts for differential stability 
\parencite{haimovich2013gene}.
One of the most well known examples of this is Dbf2p loading onto SWI5
and CLB2 mRNA to effect destabilization upon mitosis 
\parencite{trcek2011single}. 
In a direct example, the transcription activation domain of
Adr1p fused to a different DNA binding domain has been shown to be
sufficient to mark ADH2 mRNA for destabilization upon a glucose
upshift \parencite{braun2016snf1}.
Thus, RBPs may recognize sequence
elements in the mRNA or be loaded onto messenger ribonucleo-protein
complexes (RNPs) at synthesis 
\parencite{gupta2016translational}
to effect control of
mRNA stability in response to events in the cytoplasm.  

Non-RBP
mechanisms can also be used. Upstream open reading frames (uORFs) were
originally characterized using the phenotype of post-transcriptional
regulation of the Gcn2-regulated Gcn4p 
\parencite{dever1992phosphorylation}
in part
through quality control pathways 
(Ruiz-Echevarria and Peltz 1996)
\parencite{ruiz1996utilizing}.
Canonical and non-canonical start-codons can recruit initiation of
scanning ribosome subunits with a variety of effects on the
translation of the main coding sequence and the mRNA stability
\parencite{spealman2017conserved}. Ribosomes may skip re-initiation at the
primary start codon to generate N-terminal diversity by initiating at
alternative start codons, or upon termination of a uORF very distant
from the 3’ end of the transcript trigger the NMD pathway to destroy
the mRNA \parencite{dever2016mechanism}. 

Additionally, modification of
nucleotides such as m6A can play a regulatory effect \textbf{(citez)}.
While the primary-sequence of the mRNA is often thought to be the
primary source of \textit{cis}-elements,
mRNA can be modified in a variety of ways
(psuedo uridine, methylation)
\parencite{walters2017identification}.
Psu and methyl have been shown to affect stability.
The critical nature of the 5' m7G cap suggests that the
discovery of NAD+ capped mRNA may have impacts on
stability, and the introduction of this modification by
initiating transcription with NAD+ capped mRNA suggests an
elegant mechanism to explain how factors regulating synthesis 
processes can mark mRNA for differential 
stability \parencite{walters2017identification}.
Studies using methods to study the m7G caps may need to be revisited
to understand the contribution of this modification.

%check redundancy with intro around edc3 intro
mRNA
localization within the cytoplasm may affect degradation by virtue of
regulating the accessibility of degradation factors. Processing-bodies
were originally described as cytoplasmic co-localized foci of 5’->3’
degradation factors that formed under stress induction conditions, and
on the basis of steady-state genetics and experiments with an MS2
aptamer-based live-imaging system, it was concluded that
processing-bodies are foci of active mRNA degradation 
\parencite{sheth2003decapping}.
These foci are usually studied by microscopy during
stresses, entry into stationary phase, and in the use of mutants in
degradation pathways, but recent advances in microscopy and nanoscopy
particle tracking have identified that these complexes are likely
condensations of previously-existing RNPs and depend on a network of
redundant interactions between mRNA 5’ to 3’ degradation protein
factors 
\parencite{lui2014granules,rao2017numerous}.
Additionally, recent
adjustments to the aforementioned MS2 aptamer system and explorations
during dynamic conditions point towards processing-bodies being sites
of degradation factor sequestration 
\parencite{huch2017mrna,tutucci2017improved}.
Interestingly, the formation of these processing-bodies are
halted upon cycloheximide treatment 
\parencite{sheth2003decapping},
suggesting that translational status of the transcriptome and
processing-body composition may be related. In recent work, inhibition
of translation initiation was demonstrated to increase p-body
formation in correlation with increased degradation rates 
\parencite{chan2017non}. 
Together, these observations indicate that processing-bodies
result from a complex balance of mRNA degradation initiation,
resolution, and mRNA degradation factor interactions with impacts on
the accessibility of degradation factors to mRNA targets of
degradation.  

\subsection{The role of stability control in transcriptome
reprogramming}

The change in concentration of an mRNA ($R_t$) depends
on the rates of mature transcript synthesis ($k_s$) and mRNA
degradation ($k_d$). We assume that the cell volume is fixed, that
synthesis is a constant rate dependent on the unchanging concentration
of the DNA encoding the gene, and that degradation is a random first
order process of the mRNA interacting with a fixed and unsaturated
factor degradation. Thus, the change in mRNA over time is $$
\frac{dR_t}{dt} = k_s - R_t k_d$$ From this, the two rates determine
the steady-state equilibrium of $\frac{k_s}{k_d}$. Given a change in
these rates, a faster mRNA degradation rate will approach or relax to
the new equilibrium value quicker (Hargrove and Schmidt 1989), as the
doubling time (or half-life) of the mRNA is dependent on only the
degradation rate $\frac{log(2)}{k_d}$.  

While both synthesis and
degradation contribute to changes in abundance, changes in degradation
rates can cause the changes to occur more rapidly. If we expect that
the existence of a mechanism implies a selective pressure specifically
for it 
\parencite{gould1979spandrels},
then we would expect that studying
an example of a transcript subject to both synthesis and degradation
regulation might reveal a balance of selection during steady-state and
dynamic conditions. 

\section{Stress conditions trigger rapid regulation of mRNA stability}

mRNA degradation rate changes have been characterized
to play a role in responses to heat-shock, osmotic stress, pH
increases, and oxidative stress, sharing a similar program of
destabilization of mRNA coding for ribosomal-biogenesis gene products
and stabilization of stress-responsive mRNA 
\parencite{canadell2015impact,molina2008comprehensive,shalem2011transcriptome,romero2009specific,molin2009mrna,castells2011heat,miller2011dynamic,garre2013nonsense}.
Simultaneous increases in both synthesis and
degradation rates of some of these mRNA are thought to serve to return
the transcriptome quickly to a new steady-state after effecting a
transient pulse of regulation 
\parencite{shalem2008transient,rabani2011metabolic},
demonstrating a key functional role in stability control in
achieving a particular pattern of mRNA dynamics. Interestingly, these
stability changes appear to be a singular regulatory event
\parencite{perez2013eukaryotic}. 
Glucose deprivation stress also impacts mRNA stability 
\parencite{munchel2011dynamic}.

To methods.
\parencite{wada2017impact}

\subsection{Nutrient shifts also trigger mRNA stability changes }

In response to a carbon-source downshift
(glucose-grown cells resuspended in media with only galactose
available), functionally important regulatory changes in mRNA
stability occur 
\parencite{munchel2011dynamic}. Ribosome biogenesis associated
mRNAs are are destabilized, an effect that can be phenocopied by the
addition of rapamycin (inhibitor of the central growth signalling
TORC1 pathway). Conversely, a carbon source
upshift (galactose to glucose) triggers a destabilization of inducible
GAL genes, an effect that appeared to be restricted to the dynamic
condition as mRNA transgenically overexpressed in glucose media were
stable \parencite{munchel2011dynamic}.

Global changes in transcription and
mRNA destabilization has been observed before 
\parencite{jona2000glucose}, and
recently systematically measured to be correlated with changes in
growth rate 
\parencite{garcia2016growth}. The involvement of the
TORC1 pathway in this process has identified that its effect is
specified to differentially regulate the stability of certain
transcript sets 
\parencite{albig2001target,talarek2010initation}.
Recently, a phosphoproteomics approach to studying signallng of the
AMPK homolog Snf1p during a carbon upshift identified a role in Xrn1p
phosphorylation in the specification by this factor 
\parencite{braun2014phosphoproteomic}. 
Thus, specific signalling pathways appear to effect large
changes in mRNA stability in response to different nutrient conditions
for growth. 

Relieving nutrient limitation with a glucose upshift has
been shown to mediate both stabilization of mRNA in the ribosomal
protein subunit regulon 
\parencite{yin2003glucose}
and destabilization of
gluconeogenic transcripts 
\parencite{de2002role,mercado1994levels,scheffler1998control
  ,lombardo1992control}. 
Mapping the
determinants of this effect has been met with mixed success. The
destabilization of SDH2 and GAL1 mRNA have been mapped to elements in
the 5' UTR 
\parencite{bennett2008metabolic}
with destabilization of GAL1 being
associated with a growth advantage in switching carbon-source
environments 
\parencite{baumgartner2011antagonistic}. 
JEN1 has been demonstrated to
be destabilized upon glucose addition, and this has been mapped to cis
elements of different transcription start sites which can act in trans
to regulate other co-expressed engineered alleles of JEN1 through
unknown mechanisms 
\parencite{andrade2005multiple}
although subsequent work has
identified DHH1 as being a genetic factor of the destabilization 
\parencite{mota2014role}. 
Some transcripts respond at different levels of glucose
addition and differently in different genetically-perturbed metabolic
backgrounds 
\parencite{yin2000differential}, and disrupting signalling through the
PKA pathway affects destabilization of some mRNA but not others 
\parencite{yin2003glucose}. 
Thus, a systematic measurement of mRNA stability and a
broad determination of genetic factors of the transcript dynamics
would be useful for making progress at untangling the regulation of
mRNA stability in response to the increase in growth rate upon a
nutrient upshift.  

\section{What is the function of rapid txtome changes upon upshift?}

Up-regulation of mRNA abundance during an increase in growth
rate serves a clear functional purposes. The ribosomal protein (RP)
and ribosome biogenesis (RiBi) regulons are swiftly upregulated upon
repletion of nutrients to nutrient-limited cultures of yeast
\parencite{jorgensen2004dynamic}, 
and are well-correlated with growth rate in
both dynamic and steady-state conditions 
(does brauer or airoldi show this?)
. Ribosomes are lower in slow growing conditions and need to be
upregulated upon resumption of rapid growth. 
(von der Haar estimates, Tu, Hwa) (Estimated this is probably from 50k to about 200k.) 
The
relative allocation of gene expression resources in the cell is a
fundamentally important decision cells must make. Modeling of this
phenomenon across various conditions, mainly in E. coli as a model
system, as identified that a simple “pie-chart” model explains the
up-regulation of ribosome biogenesis relative to the rest of the
cellular investments well during steady-state.  

A recent study has
explored the theoretically best strategy during an increase in growth
rates, and found that a “bang-bang singular” strategy of complete
focus on generating gene expression machinery at the neglect of
metabolism machinery would be the optimal strategy for resuming growth
most rapidly 
\parencite{giordano2016dynamical}. With that, transcripts that are
stress responsive or important for metabolism in the old environment
are repressed upon a nutrient upshift.  

% Shachari 20?? upshift uri alon?
% also mention my experiment where I didn't see increased growth upon
% inhibition?

What does rapid transcriptional reprogramming achieve with respect to gene regulation?
In an integrative study of proteome and transcriptome dynamics from
the Gasch laboratory 
\parencite{lee2011dynamic}, the authors found that while
upregulation of mRNA did correlate with an increase in protein
abundance, the repression of mRNA did not correlate with a
downregulation of protein products on the timescales they measured.
This asymmetry makes sense, as with protein approximately 20 times
more stable than the mRNA they derive from 
\textbf{(one of those reviews).}
This suggests that accelerated mRNA degradation may serve a different
role. Others have suggested that degradation can help recycle
nucleotides 
\parencite{kresnowati2006transcriptome}
or that reprogramming the
transcriptome would help to reallocate the extant translational
capacity of the cell to enact a growth-optimal program 
\parencite{kief1981coordinate,giordano2016dynamical,shachrai2010cost}. 
Identifying
the genetic factors responsible for the degradation would allow us to
test if the destabilization is adaptive, and if so to make progress in
understanding the mechanistic basis of this phenomenon.

\section{Measuring mRNA dynamics}

It is easier to measure the abundance of something than
it is to measure the change in abundance of something. While mRNA
abundance measurements for entire transcriptomes are now routine,
determining the rates that underlie this molecular phenotype has
lagged. Synthesis rate control has largely been assayed by techniques
like Genome Run On (GRO) sequencing (discussed below) to measure
transcription rates or measuring intron-exon ratios 
\textbf{(pick one, NET seq?)}
as a proxy for synthesis rates 
\parencite{perez2013eukaryotic}.
Degradation rate measurements have used a variety of methods, but are
now applied to the whole transcriptome with enough accuracy to enable
systematic modeling of the determinants of mRNA degradation rates
\parencite{perez2013eukaryotic,neymotin2016multiple,cheng2017cis}.
Pioneering studies used pulse-chase experiments with radioactive
nucleotides to study turnover of the whole transcriptome, but were
unable to assay the process at the single-gene level 
\textbf{(that one that
David sent me, and the one that Leon Chan sent me).}

To study
particular genes people have used promoters with inducible repression
characteristics to halt transcription. Transgenes like the
doxycycline-inducible Tet-Off 
\parencite{gari1997set} use a heterologous
system. Researchers have also made use of the GAL1 promoter. Upon
addition of glucose, transcription of the GAL1 mRNA is immediately
halted. This property has been exploited to study mRNA stability in a
technically simple manner 
( parker?, coller green )
and the promoter
has been adapted in combination with the Tet system 
\parencite{baudrimont2017multiplexed}. 
However, this method is difficult to interpret given that 
the last 100bp before the start
codon of GAL1 has been shown to be required for accelerated
degradation of GAL1 mRNA upon glucose addition 
\parencite{baumgartner2011antagonistic},
thus challenging the interpretation of mRNA stability measured in
glucose-containing media with the GAL1 promoter fused directly
upstream of the start codon. While the researchers took care to limit
the glucose in the system (0.0?5\% in that bullshit paper) while using
heterologous binding sites, researchers have shown that glucose
concentrations of 0.0?1\% 
\textbf{(56mM, check Ziv’s paper for making sure I’m
not screwing up a decimal place)} have triggered instability of
gluconeogenic mRNA \textbf{\parencite{yin2000,yin2003}}. 
Thus, while the GAL1
system is a convenient system for studying degradation intermediates 
( parker and coller )
, its use for studying the native stability of
different mRNA in different environments is uncertain.  

A system
that does not rely on engineered cis-elements would avoid these issues
and scale to genome-wide assays, and thus two methods of
transcriptional inhibition were applied to study mRNA degradation
rates in landmark studies. A temperature sensitive rpb1-1 allele was
demonstrated to halt most Pol II transcription at non-permissive
temperatures, while the drugs thiolutin and 1,10-phenanthroline
inhibited polymerases including Pol II to mostly halt transcription.
These have been used widely, and are still used to this day. However,
it has been shown that used of thiolutin or 1,10-phenalanthroline
induces some heat-shock genes \parencite{adams1991yeast}, thiolutin
inhibits mRNA degradation in a dose-dependent way 
\parencite{pelechano2008transcriptional}
(perhaps via inducing processing-body formation,
\cite{huch2016decapping}), 
and eliminating the essential RNA Pol II complex
from the nucleus has complex effects on the transcriptome dynamics 
\parencite{yu2016rna}. While it may seem logical that studies of mRNA
associated with processes distinct from heat-shocks may be unaffected
by these, the complexity of the cell demonstrated itself in vital
controls run in \cite{mercado1994levels} which demonstrated that
gluconeogenic mRNA were subject to destabilization upon a
heat-shock. Shutting off transcription has complex and difficult to
predict effects on transcript abundance as the cells die over the
course of the experiment.  

Orthogonal to these approaches is GRO-seq
(Genomic Run On)
\parencite{garcia2004genomic}. This method uses a sarkosyl treatment (and
salt?) to fix RNA Pol II complexes onto genomic DNA by freezing
their elongation. Extraction and a defined in-vitro polymerase
extension by reversing the fixation before profiling the resulting
mRNA with microarrays or RNAseq allows for an estimate of the
instantaneous transcription rate status for each gene in a population
of cells. Interpretation of these numbers must be considered in the
context of the in-vitro environment of the elongation step, but this
method serves as an valuable orthogonal measure of transcript dynamics
--- and an instantaneous one.  

The development of 4-thiouracil
metabolic labeling of RNA 
\parencite{dolken2008high}
has enabled a return to
the pulse-chase methodology in the development of genome-wide assays
of mRNA dynamics. Fundamentally, these assays work by changing the
labelling frequency of mRNA and tracking the dynamics as the labeled
mRNA abundance relaxes towards that new equilibrium. Below I review
the basis of these assays, then focus on their applications and where
the dissertation work is placed.  

If we consider mRNA abundance at a
certain time as being denoted as $M_t$, then I expect this number to
change as a zeroth order rate of synthesis per time ($k_s$) and a
first order rate of degradation per mRNA ($k_d$). While mRNA
degradation is a multi-step process (above) and more complex models
may identify nuances in the rates of progression through these
intermediates \parencite{deneke2013complex}, at steady-state the rate of
degradation of mRNA in a population should be well modeled by a single
first order rate 
\textbf{( ?can I find the cite for that? ).}
$$\frac{dM_t}{dt} = k_s - M_t*k_d$$ Introducing a term $L$ that
denotes the fraction of newly synthesized mRNA that are labeled and
measured after purifying mRNA for the labeled mRNA (thus $M_t$ is just
labeled and captured mRNA), we can now model the changes as simply
$$\frac{dM_t}{dt} = L k_s - M_t*k_d$$ I introduce the superscript
notation of $L^o$ for the old labeling frequency and $L^n$ for the new
labeling frequency, and solving for the change of $M_t$ from some
steady-state equilibrium $L^o \frac{k_s^o}{k_d^o}$ to a new
equilibrium $L^n \frac{k_s^n}{k_d^n}$, and rearranging terms we get
$$M_t =  L^o \frac{k_s^o}{k_d^o} e^{-k_d^n t} + L^n \frac{k_s^n}{k_d^n} ( 1- e^{-k_d^n t} )$$ 
This matches well with our
intuition. On the right, the nascent transcripts are labeled at the
new rate and approach this new equilibrium controlled by the term $(
1- e^{-k_d^n t} )$, while on the left the extant transcripts approach
zero in an exponential decrease controlled by the term $e^{-k_d^n t}$.
These are both controlled by $k_d^n$, or the rate of mRNA degradation
after chasing the label. Thus, by measuring the transition between the
equilibrium we get the mRNA degradation rate, assuming that synthesis
rates do not change.  

Measuring specific rates with high confidence
requires a steady-state approximation. RATEseq is one method to do so,
using many timepoints to accurately model the approach of labeled mRNA
abundance to a new equilibrium 
\parencite{neymotin2014determination}. This
experimental design is theoretically the most accurate, although it
requires the assumption that the total mRNA (labeled + unlabeled) is
indeed at a steady-state abundance. Dynamic Transcriptome Analysis
\parencite{miller2011dynamic}
violates this assumption to explore changes in
degradation rates during 6 minute windows. While they sacrifice
high-confidence of an exact rate, the temporal resolution of stability
changes during osmotic stress has revealed an unprecedented dynamic
view of the regulation of mRNA dynamics during complex processes. This
approach requires that 4-thiouracil transport and incorporation into
nucleotide metabolism occurs during the course of the perturbation
experiment, but with the right measurements, normalization, and
integration with other datasets an accurate and dynamic picture of
transcriptome dynamics can be built. To assess mRNA stability changes
during dynamic processes, one can also label the transcriptome to
equilibrium and then chase out the label by adding an excess of
unlabeled nucleotides. This approach was used by researchers in the
Weis group to demonstrate changes in the stability of groups of mRNA
in response to environmental changes, namely shifts in carbon sources
and with rapamycin treatment inhibiting TORC1
\parencite{munchel2011dynamic}
They found that the RP (ribosomal protein) regulon was destabilized
upon induction of nutrient starvation, demonstrating that mRNA
degradation is under tight regulation from nutrient-sensing pathways.
In Chapter 3 I demonstrate the use of a similar label-chase
experimental design, with with refined analysis to explore
single-transcript destabilization upon a nitrogen upshift.  

\section{Methods for determining the genetic basis of a transcript
dynamics phenotype}

mRNA is an intermediate in the expression of a protein product, and
has the key virtue of being much easier to measure than to measure
abundance of the protein product. This became especially true with the
advent of massively-parallelized DNA sequencers and the methods to
accurately convert transcriptomes to DNA libraries, ie RNAseq
\parencite{shendure2017dna}. For this reason, it is often used as a proxy
of gene expression at the protein level. Although the relationship is
strong when correcting for experimental noise 
\parencite{csardi2015accounting},
the quantitative functional nature of this relationship within a
particular gene in different environments depends on the particular
gene in question 
\parencite{franks2017post}. It is also clear that
transcriptomic and proteomic responses greatly vary in the timescales
of effect, with the transcriptome subject to rapid impulses of
changing abundance that may or may not result in longer term
regulation of the protein product 
\parencite{cheng2016differential,lee2011dynamic}. 
Even then, protein abundance in a cell does not correspond
perfectly to its activity, be that regulated allosterically or by
localization.  

Given this disparity, what can we learn about adaptive
gene expression from mRNA abundance regulation? First, the expression
of a gene product requires mRNA, thus the binary expression of mRNA is
a predictor of the possibility of protein expression. Additionally,
cellular processes often impinge upon changes in mRNA abundance, be
they direct via regulation of abundance, activity, and localization of
activity of specific effectors or by indirect effects on common gene
expression machinery or cellular metabolism. In this way, a specific
perturbation of a signalling pathway is expected to broadcast to
changes in mRNA abundance. Quantification of the thousands of mRNA
that are expressed in a cell is a sensitively quantitative measurement
on thousands of dimensions, and can thus be used as a
relatively-unbiased indicator of cellular status with which to explore
the genetic requirements of particular signalling perturbations 
\parencite{gapp2016parallel}. Thus, efficient methods to explore the genetic
basis of transcript dynamics upon a perturbation should help to
accelerate the study of cellular signalling pathways.  

Genetic screens
in yeast have been a powerful tool to narrow down the immense search
space of possibilities to a narrow set of hypotheses about a
biological process. Classically, these function by mapping some
phenotype of interest to a change in growth rate. For example, mutants
in transporters of a particular amino-acid can be isolated by feeding
the cells a toxic stereoisomer (like D-histidine). A more complex
method in Lee Hartwell’s classic screen for cell-cycle mutants used
the assay of growth at a low temperature and cessation of growth at a
high temperature to identify mutants in critically important pathways
\parencite{hartwell1970genetic}, work that contributed to a 2001 Nobel Prize for
advancing our understanding of the cell cycle. However, this concept
becomes problematic when studying a molecular phenotype which is not
known to be adaptive. 

For example, gene regulation might not have a
clear phenotypic outcome or be subject to redundant layers of
regulation that mask the effect of a mutation. One solution is to
engineer a specific reporter into the expressed gene, such that
defects in gene expression can be assayed. It becomes more difficult
if the phenotype is a transient one, such that a reporter through
growth rate (perhaps a toxic peptide) does not have time to accumulate
the signal of growth. A fluorescent tag is one approach that bypasses
this requirement, as cells can be instantaneously assayed for the
level of GFP fluorescence through methods like flow cytometry. The GFP
can be fused to the protein of interest or simply placed downstream of
an appropriate reporter, such as the strategy employed by 
\parencite{neklesa2009genome}. 
They were able to use a DAL80 promoter upstream of a GFP
reporter to explore the genetic requirements for the NCR-regulated
expression of this promoter, discovering the SEACIT complex components
Npr2/3p upstream of TORC1. However, this method requires that the gene
expression phenotype be regulated at both the transcript synthesis
level and be relevant at the level of protein expression. Additionally
it requires that the GFP tag be a relevant and faithful reporter of
the protein abundance, a condition which is not always satisfied given
the stability of the GFP tag in the vacuole (find that cite).
Previous work to do genetics of transcript dynamics neklesa - sortseq,
reporters Mention problems of GFP stability worley - robotics exotic
methods, like COE, so antibodies scRNAseq I think tavazoie’s stuff
Fluorescent readout of mRNA degradation mechanisms at the protein
level. GAP1 mRNA degradation, which we identify as
being subject to accelerated mRNA degradation in Chapter 3 and 4,
occurs much faster than the repression of the protein-product. We also
know that Gap1p, the protein product of GAP1 mRNA, is subject to
de-activation and re-localization in response to a nitrogen upshift.
Thus, a functional assay of Gap1p is irrelevant to the dynamics of
GAP1 mRNA repression, and requires a novel method to screen for
genetic factors of this molecular phenotype.  

Ambitious work from the
Capaldi group developed a workflow using extensive automation to
perform qPCR assays for NSR1 mRNA abundance 19 minutes after induction
of an osmotic stress response 
\parencite{worley2016genome}. While accurate and
reproducible, the extensive automation and reagent usage to perform
qPCR on ~4700 mutant strains poses a financial and logistical
challenge to performing the assay, and to perform the assay in
different genetic backgrounds, in larger libraries, in replicates, or
in different timepoints.  

I composed several existing technologies to
develop a direct assay of transcript abundance in a high-throughput
pooled format. This is discussed in depth in Chapter 3. 

% How about ammonia and nitrogen metaoblism, what's the singal?  \parencite{ter2000role}
%This lag in population growth rate upon an upshift has been 
%described before \parencite{Carter1978}.

\chapter{Modeling transcript dynamics upon a nitrogen upshift}

This
chapter was published as part of the article “Steady-state and dynamic
gene expression programs in Saccharomyces cerevisiae in response to
variation in environmental nitrogen” in Molecular Biology of the Cell
vol. 27 no. 8 1383-1396. April 15, 2016. doi: 10.1091/mbc.E14-05-1013 


Authorship of this article was: Edoardo M. Airoldi, Darach Miller,
Rodoniki Athanasiadou, Nathan Brandt, Farah Abdul-Rahman, Benjamin
Neymotin, Tatsu Hashimoto, Tayebeh Bahmani, and David Gresham. 


Below is reprinted the abstract, the sections of the results, and the
sections of the conclusion relevant to this dissertation. The text has
been edited for clarity, primarily of acronyms. Supplemental tables,
figures, and files are available on the MBoC article website (
doi.org/10.1091/mbc.E14-05-1013 ).  

\section{Abstract} 

Cell growth rate is
regulated in response to the abundance and molecular form of essential
nutrients. In Saccharomyces cerevisiae (budding yeast), the molecular
form of environmental nitrogen is a major determinant of cell growth
rate, supporting growth rates that vary at least threefold.
Transcriptional control of nitrogen use is mediated in large part by
nitrogen catabolite repression (NCR), which results in the repression
of specific transcripts in the presence of a preferred nitrogen source
that supports a fast growth rate, such as glutamine, that are
otherwise expressed in the presence of a nonpreferred nitrogen source,
such as proline, which supports a slower growth rate. Differential
expression of the NCR regulon and additional nitrogen-responsive genes
results in >500 transcripts that are differentially expressed in cells
growing in the presence of different nitrogen sources in batch
cultures. Here we find that in growth rate–controlled cultures using
nitrogen-limited chemostats, gene expression programs are strikingly
similar regardless of nitrogen source. NCR expression is derepressed
in all nitrogen-limiting chemostat conditions regardless of nitrogen
source, and in these conditions, only 34 transcripts exhibit nitrogen
source–specific differential gene expression. Addition of either the
preferred nitrogen source, glutamine, or the nonpreferred nitrogen
source, proline, to cells growing in nitrogen-limited chemostats
results in rapid, dose-dependent repression of the NCR regulon. Using
a novel means of computational normalization to compare global gene
expression programs in steady-state and dynamic conditions, we find
evidence that the addition of nitrogen to nitrogen-limited cells
results in the transient overproduction of transcripts required for
protein translation. Simultaneously, we find that that accelerated
mRNA degradation underlies the rapid clearing of a subset of
transcripts, which is most pronounced for the highly expressed
NCR-regulated permease genes GAP1, MEP2, DAL5, PUT4, and DIP5. Our
results reveal novel aspects of nitrogen-regulated gene expression and
highlight the need for a quantitative approach to study how the cell
coordinates protein translation and nitrogen assimilation to optimize
cell growth in different environments.  

\section{Introduction}

To study the effect on mRNA expression of environmental
nitrogen source variation in nitrogen-limited, growth rate–controlled
conditions, we studied cells growing in chemostats using six different
nitrogen sources at four different dilution rates. We show that
differential expression of the NCR (Nitrogen Catabolite Repression)
and SPS (SSY1-PTR3-SSY5) regulons is primarily a function of growth in
a nitrogen-limited environment, with the molecular form of nitrogen
having minimal effect on differential gene expression when cells are
limited for nitrogen. By contrast, the GAAC (General Amino Acid
Control) and UPR (Unfolded Protein Response) regulons do not respond
specifically to nitrogen limitation compared with other
nutrient-limited conditions. To study the dynamics of
nitrogen-responsive gene expression, we performed transient
perturbation experiments in which different quantities and sources of
nitrogen were added to cells growing in nitrogen-limited chemostats.
The addition of either the preferred nitrogen source, glutamine, or
the nonpreferred nitrogen source, proline, to cells growing in
nitrogen-limited conditions results in rapid repression of the NCR
regulon in a dose-dependent manner. Surprisingly, a sudden increase in
environmental nitrogen does not correspond to a detectable increase in
biomass production or cell number, consistent with a time delay
between activation of the transcriptional growth program and its
manifestation in an increased rate of cell growth. To compare global
gene expression in dynamic conditions with mRNA expression in
steady-state conditions, we used computational estimation of
instantaneous growth rate from gene expression profiles (Brauer et al.
2008; Airoldi et al. 2009) and defined gene expression responses to
growth rate in both steady-state and dynamic conditions using linear
regression. We find that the response of transcripts required for
protein translation (RP and RiBi) in cells provided with an increase
in nitrogen exceeds the response to growth rate in cells growing in
steady-state conditions consistent with a transient overproduction of
RP and RiBi transcripts. Finally, we show that accelerated degradation
of some NCR transcripts underlies gene expression remodeling in
response to sudden relief from nitrogen limitation, indicating the
activity of a posttranscriptional mechanism controlling
nitrogen-responsive gene expression.  

\section{Results} 

To obtain a high-resolution view of mRNA abundance
changes during the first 10 min after addition of nitrogen, when
changes in gene expression are maximal (Figure 4), we repeated the
pulse experiments (addition of nitrogen source to yeast grown in
steady-state nitrogen-limited chemostat cultures) and assayed global
gene expression at 1–2 min intervals after the addition of 40 $\mu$M
glutamine or 80 $\mu$M proline. We observed a rapid increase in expression
of the RiBi and RP regulons in response to a pulse of glutamine, with
a concomitant rapid decrease in expression of the NCR-A and NCR-P
regulons. Consistent with our initial observation, we observed a
similar response to a pulse of proline. 

Accelerated degradation of
mRNAs contributes to remodeling of the transcriptome The majority of
NCR transcripts are strongly repressed in response to a nitrogen
pulse. If gene expression is repressed at the promoters of these genes
and mRNA synthesis ceases, the decrease in mRNA abundance is expected
to be a function of the degradation rate of the corresponding mRNA.
Using our high-density time-series data, we estimated the rate of
change in abundance for all transcripts, assuming a first-order
exponential degradation model (Materials and Methods; Supplemental
Table S7), which is the standard method for estimating mRNA
degradation rates. We found that in response to a glutamine pulse, 269
genes fit a first-order exponential decay model (FDR < 0.05;
Supplemental Table S4), whereas 458 transcripts fit a first-order
exponential decay model in response to the proline pulse (Supplemental
Table S4).  

We compared the half-lives of rapidly degraded transcripts
after the glutamine pulse with half-life estimates in steady-state
conditions determined using RATE-seq (Neymotin et al. 2014). We found
that some transcripts decay significantly faster than expected,
suggesting that their degradation rate is accelerated in response to
the glutamine pulse (Figure 6A). Batch culture growth in proline also
results in derepression of the NCR regulon (Godard et al. 2007). To
test whether accelerated mRNA decay is specifically a response to the
nitrogen-limited conditions of a chemostat, we added a pulse of
glutamine to cells growing in batch cultures containing proline as a
sole nitrogen source and measured genome-wide gene expression
(Supplemental Table S7). The half-lives of transcripts that exhibit an
exponential decrease is similar in chemostat and batch cultures
(Supplemental Figure S7B), and many of the same transcripts show
evidence of accelerated degradation rates in batch cultures (Figure 6B
and Supplemental Table S4). Strikingly, the five nitrogen permease
genes GAP1, DIP5, MEP2, PUT4, and DAL5 are the most rapidly cleared
mRNAs in both the chemostat and batch culture experiments.  

To verify
that the addition of glutamine stimulates accelerated degradation of
specific NCR transcripts, we performed pulse-chase experiments using
the metabolic label 4-thiouracil (4-tU). After several generations of
batch culture growth in proline medium in the presence of 4-tU to
allow complete labeling of mRNAs, we added unlabeled uracil to the
culture. We allowed the chase to occur for 13 min and then added
either glutamine or water (mock) to the cells. We purified labeled
transcripts and analyzed GAP1 and DIP5 mRNAs using quantitative PCR
(qPCR) and normalization to external spike-ins. Consistent with our
genome-wide assay, the addition of glutamine results in a clear
accelerated degradation of both GAP1 mRNA (Figure 6C) and DIP5 mRNA
(Figure 6D), confirming that the transition from NCR-derepressed to
NCR-repressed conditions results in the accelerated degradation of
some transcripts.  

FIGURE 6: Accelerated mRNA degradation contributes
to gene expression remodeling. Upon addition of glutamine to
NCR-derepressed cells, a subset of transcripts degrade more rapidly
than their steady-state degradation rate both (A) in cells grown in
ammonia-limited chemostats and (B) in cells growing in proline media
in batch cultures. All points are genes that fit a model of
exponential decrease in abundance (FDR < 0.05). Orange points are NCR
genes that show significant accelerated degradation, blue points are
NCR genes that are not significant, green points are non-NCR genes
that show significantly accelerated degradation, and gray points are
genes that are neither accelerated nor NCR. The dashed line denotes
equal degradation rates in both conditions (i.e., slope equal to 1).
Names of nitrogen transporter genes are displayed. We measured the
transient changes in the degradation rates of (C) GAP1 and (D) DIP5
mRNA using a pulse-chase experiment. Cells were grown for 24 h in the
presence of 4-thiouracil, which was chased at t = 0 min by the
addition of excess uracil. At t = 13 min, we added either glutamine in
water (orange) or equal volume of water (blue). We extracted and
quantified the abundance of 4-thiouracil–labeled mRNA relative to a
thiolated external spike-in using qPCR. We found significant
acceleration of degradation for both GAP1 and DIP5 mRNAs (p < 0.001).
Points are the mean of triplicate qPCR measurements, error bars are
the propagated SD of transcript and spike-in measurements, and dotted
lines are the log-linear model fit.  

\section{Discussion} 

Some mRNAs are rapidly degraded when cells
transition from NCR-activating to NCR-repressing conditions in both
chemostats and batch culture. Comparison with mRNA degradation rates
suggests that the degradation of some of these transcripts is
accelerated. Using in vivo metabolic labeling with 4-tU, we provide
additional evidence that the addition of glutamine to nitrogen-limited
cells accelerates the degradation of specific transcripts. A previous
study of the transcriptional response to glucose addition in
carbon-limited chemostats suggested a role for accelerated degradation
of mRNAs (Kresnowati et al., 2006), and there is increasing evidence
that mRNA stability plays an important role in regulating gene
expression programs (Puig et al., 2005; Bennett et al., 2008;
Baumgartner et al., 2011). Consistent with a posttranscriptional
mechanism underlying the rapid clearing of some NCR transcripts,
previous work showed that GAP1 mRNA transiently decreases in abundance
during a nitrogen up-shift in the absence of URE2 (ter Schure et al.,
1998), which is required for NCR repression by sequestering GLN3 in
the cytoplasm. Several studies have shown that TORC1 can affect
transcript stability (Albig and Decker, 2001; Munchel et al., 2011).
Our results suggest that posttranscriptional regulation of mRNA
stability may play an important role in remodeling gene expression in
response to changes in environmental nitrogen. Transient stabilization
of the RP and RiBi regulons also could contribute to their rapid
increase in expression (Yin et al., 2003). Defining the role of
regulated changes in mRNA stability in dynamic conditions is an
important area for further study. 

What is the underlying rationale
for rapid induction of RP/RiBi transcripts occurring in parallel with
accelerated degradation of NCR transcripts? We propose that
accelerated degradation of NCR transcripts may allow for reallocation
of ribosomes to transcripts required for growth and proliferation
(Kief and Warner, 1981; Lee et al., 2011). Our observations are
consistent with a model in which TORC1 orchestrates the balance
between transcripts required for protein production and transcripts
required for the acquisition and assimilation of nitrogen. When
nitrogen is abundant, TORC1 activates the expression of the RP and
RiBi regulons while actively repressing the NCR-A and NCR-P regulons.
Conversely, when nitrogen levels are in growth-limiting
concentrations, TORC1 activity decreases, leading to reduced
activation of the RP and RiBi regulons and derepression of the NCR-A
and NCR-P regulons. In NCR-derepressing conditions, NCR transcripts,
including GAP1, MEP2, and PUT4, are the most abundant transcripts
(Supplemental Table S5). When a cell encounters a sudden increase in
environmental nitrogen, some highly expressed transcripts may be
targeted for accelerated degradation to increase the pool of free
ribosomes facilitating rapid translation of newly transcribed RiBi and
RP transcripts, thereby accelerating physiological remodeling of the
cell for rapid growth.  

\section{Materials and methods}

\subsection{Strains and culturing conditions}

We used the prototrophic haploid strain FY4 (MATa), which
is isogenic to the S288c reference strain, for all experiments. We
used minimal defined media for all experiments, using a common base
medium for nitrogen limitation, as described previously (Brauer et
al., 2008; Boer et al., 2010). The appropriate concentrations of
allantoin, glutamine, glutamate, urea, ammonium sulfate, proline, and
arginine were added from 100 mM stock. Batch culture experiments were
performed in 30$^{\circ}$C shaking incubators using 100-ml cultures. Continuous
culturing in chemostats using Sixfors bioreactors (Infors, Laurel, MD)
was performed as described (Brauer et al., 2008; Boer et al., 2010)
using a 300-ml working volume. Culture parameters were determined
using either a Klett colorimeter or a Coulter counter after
sonication. For perturbation studies, a single bolus of proline,
glutamine, or a mix of both was added to the chemostat to a final
concentration of 80 or 800 $\mu$M nitrogen.  

\subsection{RNA analysis} 

Cell samples for
mRNA analysis were preserved by rapid filtration and quick freezing
using liquid nitrogen. We isolated total RNA using hot acid–phenol
extraction and subsequently purified RNA samples using RNeasy columns.
We performed gene expression profiling using Agilent (Santa Clara, CA)
60-mer DNA microarrays and Cy3 and Cy5 incorporation as previously
described (Brauer et al., 2008). We used a common reference obtained
from a sample growing in an ammonium sulfate–limited chemostat at a
dilution rate of 0.12 hours$^1$ for all hybridization experiments and
hybridized labeled cRNA to Agilent Yeast DNA microarrays for 20 h at
65$^{\circ}$C. We washed arrays and scanned microarrays using an Agilent
two-color scanner and extracted hybridization signals using Agilent
Feature Extractor Software. Supplemental Table S6 gives the entire
data set of processed log2 ratios.  

\subsection{Pulse chase} 

Cells were grown in
600 ml of minimal medium containing 800 $\mu$M proline, 500 $\mu$M uracil, and
500 $\mu$M 4-thiouracil at 30$^{\circ}$C for 24 h. The culture was divided into two
300-ml cultures, and uracil was added to a final concentration of 2
mM. We acquired 20-ml samples after the chase using rapid filtration
and flash freezing in liquid nitrogen. At 13 min after starting the
chase, we added either glutamine to a final concentration of 400 $\mu$M or
an equal volume of water and acquired additional samples.  

After RNA
extraction, samples were mixed with an in vitro–transcribed thiolated
spike-in (BAC1200) at a ratio of 1 ng of spike-in to 25 $\mu$g of total
RNA and reacted with EZ-Link HPDP-Biotin (ThermoFisher Scientific,
Waltham, MA) at 2 mg/ml for 200 min. Reactions were cleaned up by
centrifugation and ethanol precipitation and then conjugated with 180
$\mu$l of streptavidin magnetic beads (M0253L; NEB, Ipswich, MA). Labeled
RNA was eluted using 5\% $\beta$-mercaptoethanol.  

Samples were reverse
transcribed with Moloney murine leukemia virus reverse transcriptase
(NEB) and random hexamer priming. We performed qPCR in technical
triplicate on a LightCycler 480 (Roche, Branchburg, NJ) using the
following primers: 5'-ACGGTATCAAGGGTTTGCCAAG-3' and
5'-GCATAAATGGCAGAGTTAC-3' for GAP1, 5'-TGGCGTACATGAATGTGTCTTCA-3' and
5'-GGTGATCCAACTCAAGATTC-3' for DIP5, and 5'-CTGGACGACTTCGACTACGG-3'
and 5'-ATCAGCCTTTCCTTTCGTCA-3' for the BAC1200 spike-in. Cp values
were calculated for each sample and the spike-in and log-linear
regression performed using the ratio of either GAP1 mRNA or DIP5 mRNA
to the spike-in in R.  

\subsection{mRNA decay estimation}

We estimated rates of
mRNA decay for all transcripts using high–temporal resolution data. We
used ratios (yt) of hybridization intensities for each transcript
obtained from two-color DNA microarrays cohybridized with a common
reference. Data were normalized to the initial data point (y0) and
then log-transformed. We modeled the degradation rate kdeg of each
gene: Formula where t is the sampling time in minutes. Transcript
half-lives were computed as ln(2)/kdeg. Accelerated degradation was
assessed by fitting the model Formula where ksteady-state deg is the
specific degradation rate for transcript i as reported in Neymotin et
al. (2014). For all linear modeling, we assessed statistical
significance of coefficients using a t statistic and determined
empirical p values by permuting data for each gene 1000 times. The
false discovery rate was determined using the qvalue package in R.
Data availability DNA microarray data are available through gene
expression omnibus (GEO) GSE57293.



\chapter{measuring destabilization and looking for regulators}

This
has been submitted for publication, as Miller, Brandt, Gresham 2018.
“BarSeq after FACS after FISH identifies mRNA decapping factors
associated with the swift repression of GAP1 mRNA upon a nitrogen
upshift” Darach Miller, Nathan Brandt, and David Gresham 

\section{Abstract}

Cellular responses to changing environmental conditions frequently
involve rapid reprogramming of the transcriptome, in part by
modulating mRNA synthesis. Altered mRNA degradation rates can
accelerate this mRNA regulation by acting to clear or stabilize extant
transcripts. Understanding the extent and mechanisms of
post-transcriptional regulation in different dynamic conditions will
allow us to determine if this regulatory mechanism is adaptive and
how. Budding yeast respond to an improvement in nitrogen-availability
by triggering a transcriptional reprogramming that functions to
upregulate ribosome biogenesis and repress alternative nitrogen-source
catabolism. Here, we measured mRNA stability across a nitrogen upshift
and found that 78 mRNA are subject to destabilization. This set is
enriched for factors of the Nitrogen Catabolite Repression (NCR)
regulon, secondary-active transporters, and factors of carbon
metabolism, suggesting that mRNA destabilization upon the upshift is
an uncharacterized mechanism contributing to NCR and other functional
changes. To explore the molecular basis of destabilization we focused
on a gene subject to NCR-control and a 3-fold increase in degradation
rate, GAP1. We developed a method to screen for the trans genetic
factors of this swift mRNA repression, combining branched-DNA mRNA
FISH, fluorescence-activated cell sorting, and low-input multiplexed
barcode sequencing to estimate GAP1 mRNA abundance for mutants in a
pooled prototrophic haploid yeast deletion collection. By using the
phenotype of defective GAP1 mRNA dynamics during the nitrogen upshift,
we identify that the Lsm1-7/Pat1 complex plays a role in GAP1
repression and that modulators of decapping activity also perturb the
GAP1 dynamics in combination with elements in the 5’ UTR. These
results identify candidates that may be responsible for transducing
the signal of a nitrogen upshift and suggests that different
translational status may be associated with altered stability upon the
nitrogen upshift.  

\section{Introduction}

Regulation of mRNA abundance in
response to environmental signals operates in concert with additional
layers of gene expression regulation, and is a centrally important
aspect of cellular responses to perturbations. Both synthesis and
degradation rates of mRNA determine the steady-state abundance of a
particular mRNA and modulate the kinetics with which regulation occurs
(Pérez-Ortín et al. 2013). In budding yeast, the rate of mRNA
degradation is affected by changes in many factors including
nutritional conditions  (Munchel et al. 2011; Jona et al. 2000),
cellular growth rate (García-Martínez et al. 2016), and environmental
stresses (Canadell et al. 2015). Regulated changes in mRNA degradation
rates during C. elegans oogenesis (West et al. 2017) and the early
early development of Drosophila (Alonso 2012) indicate that mRNA
degradation rate regulation fulfills an important mechanistic role in
gene expression regulation in diverse systems.  

Environmental shifts
trigger rapid reprogramming of the budding yeast transcriptome in
response to various stresses and nutrient additions (Gasch et al.
2000; Conway et al. 2012). mRNA degradation rate changes have been
characterized to play a role in responses to heat-shock, osmotic
stress, pH increases, and oxidative stress, sharing a similar program
of destabilization of mRNA coding for ribosomal-biogenesis gene
products and stabilization of stress-responsive mRNA (Canadell et al.
2015). Simultaneous increases in both synthesis and degradation rates
of some of these mRNA are suggested to serve to return the
transcriptome quickly to a new steady-state after effecting a
transient pulse of regulation (Shalem et al. 2008). Relieving nutrient
limitation with a glucose upshift has been shown to mediate both
stabilization of mRNA in the ribosomal protein subunit regulon (Yin et
al. 2003) and destabilization of gluconeogenic transcripts (de la Cruz
et al. 2002; Mercado et al. 1994). Destabilization of transcripts is
expected to reduce the corresponding protein products, but
transcriptional repression can have a delayed effect on reducing
protein levels compared to up-regulated genes (Lee et al. 2011). This
suggests that accelerated mRNA degradation may serve a different role.
Others have suggested that degradation can help recycle nucleotides
(Kresnowati et al. 2006) or that reprogramming the transcriptome would
help to reallocate the extant translational capacity of the cell to
enact a growth-optimal program (Kief and Warner 1981; Giordano et al.
2016; Shachrai et al. 2010). Identifying the genetic factors
responsible for the degradation would allow us to test if the
destabilization is adaptive, and if so to make progress in
understanding the mechanistic basis of this phenomenon. 

 Yeast cells
metabolize a wide variety of nitrogen sources, but preferentially
assimilate and metabolize specific nitrogen compounds. This is in part
controlled through transcriptional regulation known as “nitrogen
catabolite repression” (NCR)  (Magasanik and Kaiser 2002) of  mRNA
encoding a variety of transporters, metabolic enzymes, and regulatory
factors. Transcriptional synthesis repression is relieved in the
absence of a preferred nitrogen source or in the presence of
growth-limiting concentrations (in the low uM range) of any nitrogen
source, including preferred nitrogen sources (Airoldi et al. 2016;
Godard et al. 2007).  Expression of NCR target genes is mediated by
two activating GATA factors, Gln3p and Gat1p, and two additional
repressing GATA factors, Dal80p and Gzf3p that function to modulate
NCR expression levels. GAT1, GZF3, and DAL80 promoters contain GATAA
binding sites and thus transcriptional regulation of NCR targets
entails self-regulatory and cross-regulatory loops. When supplied with
a preferred nitrogen source such as glutamine, the NCR-activating
transcription factors Gat1p and Gln3p are excluded from the nucleus by
multiple mechanisms (Tate and Cooper 2013; Tate et al. 2017), but the
activity of products of genes subject to NCR are sometimes also
controlled after transcription occurs (Cooper and Sumrada 1983). For
example the General Amino-acid Permease Gap1p is known to be
post-translationally regulated (Stanbrough and Magasanik 1995), and is
rapidly inactivated upon a nitrogen upshift with adaptive consequences
(Merhi and Andre 2012; Risinger et al. 2006). Recently, we have
identified an additional level of regulation of NCR transcripts: cells
growing in NCR de-repressing conditions accelerated the degradation
rate of GAP1 and DIP5 mRNAs upon addition of glutamine (Airoldi et al.
2016). Thus, mRNA degradation control could be an uncharacterized
mechanism of immediately effecting NCR and other mRNA regulation upon
a glutamine upshift.  


Multiple pathways mediate the degradation of
mRNAs. The main pathway of mRNA degradation is deadenylation and
decapping prior to 5’ to 3’ exonucleolytic degradation by Xrn1p;
however, transcripts are also degraded 3’ to 5’ via the exosome, or
via activation of cotranslational quality control mechanisms (Parker
2012). Deadenylation of mRNAs by the Ccr4-Not complex allows the mRNA
to be bound by the Lsm1-7p/Pat1p complex, a heptameric ring comprising
the SM-like proteins Lsm2-7 and the cytoplasmic-specifying Lsm1
(Tharun et al. 2000; Sharif and Conti 2013), which then recruits
factors for decapping of the mRNA by Dcp2, followed by degradation by
Xrn1p. This pathway is rate-limited by the recruitment of the
decapping enzyme (Coller and Parker 2004), therefore Lsm1-7p, Pat1p,
and associated factors play a key role in regulating mRNA
degradation(Nissan et al. 2010). Regulation of this and other pathways
of degradation can alter the stability of individual mRNAs. In the
canonical example of Puf3p, this RNA-binding protein (RBP) recognizes
a unique sequence cis-element in 3’ UTRs (Olivas and Parker 2000) and
the effect of this association on mRNA degradation rates varies
depending on Puf3p phosphorylation status (Miller et al. 2014).
Upstream open reading frames (uORFs) can also trigger degradation of a
specific mRNA in cis via activation of mRNA quality control pathways
(Hinnebusch 2005). Recent studies indicate that transcript properties
associated with rates of translation affect mRNA degradation (Presnyak
et al. 2015; Neymotin et al. 2016) and previously an elegant mechanism
of competition between the decapping enzymes and translation
initiation processes has been described (????), demonstrating the
connection between mRNA translation and degradation. In addition to
RNA cis-elements, promoters have been shown to mark certain
RNA-protein (RNP) complexes to specify their post-transcriptional
regulation (Trcek et al. 2011; Braun et al. 2015; Haimovich et al.
2013; Mercado et al. 1994). This effector mechanisms may be controlled
by a variety of different signalling pathways including Snf1 (Young et
al. 2012; Braun et al. 2014) or TORC1 (Talarek et al. 2010). Thus, the
regulation of mRNA degradation rates entails numerous mechanisms that
collectively the tune stability of mRNAs in response to signalling
pathways.  

Here, we studied the global regulation of mRNA degradation
rates upon a nitrogen upshift using 4-thiouracil (4tU) labeling and
RNAseq. We find that a set of 78 mRNAs show clear evidence for
accelerated mRNA degradation, including many NCR transcripts as well
as components of carbon metabolism. To identify the mechanism
underlying accelerated mRNA degradation of a specific transcript,
GAP1, we developed a high-throughput genetic screen using single
molecule mRNA FISH (smFISH) as a marker for fluorescent activated cell
sorting (FACS), using barcode-sequencing (Bar-seq) to quantify DNA
barcodes in each bin, and modeling to estimate mRNA abundance for each
DNA barcode in the pool. We applied this to the barcoded yeast
deletion collection to screen for effects of each gene deletion on the
abundance of GAP1 mRNA in NCR de-repressing conditions of growth on
proline and the rapid repression of GAP1 mRNA 10 minutes after the
addition of glutamine, a repressive nitrogen source. We find that the
Lsm1-7p/Pat1p complex and decapping modifiers affect both GAP1 mRNA
steady-state expression and the accelerated degradation of GAP1 mRNA
upon a nitrogen upshift. This work expands our understanding of mRNA
stability regulation in remodeling the transcriptome during a relief
from growth-limitation and demonstrates a generalizable approach to
the study of genetic determinants of mRNA dynamics.  

\section{Results}
Transcriptional reprogramming precedes physiological remodeling
Cellular responses to environmental signals entail coordinated changes
in both gene expression and cellular physiology.  Previously, we
studied the steady-state and dynamic responses of Saccharomyces
cerevisiae (budding yeast) to environmental nitrogen (Airoldi et al.
2016), and found that the transcriptome is rapidly reprogrammed
following a nitrogen upshift in either a chemostat or batch culture.
To study physiological changes in response to an upshift, we measured
cell size and cell proliferation rates. A prototrophic haploid lab
strain (FY4, isogenic to S288c) grows with a 4.5 hour doubling time in
minimal media containing proline as a sole nitrogen source (Fig 1a).
Upon addition of 400uM glutamine the cells undergo a 2-hour lag period
during which growth rate remains unchanged (Fig 1a), but  the average
cell size continuously increases (21\% increase in mean volume)  (Fig
1b). The culture adopts a 2.1 hour doubling time following a lag of
about 2 hours as previously described (Carter et al. 1978), suggesting
that extensive physiological remodeling is required for faster growth
in response to a nitrogen upshift. By contrast, global gene expression
changes within three minutes of the upshift (Airoldi et al. 2016).
Thus, transcriptome remodeling precedes physiological remodeling and
initiation of a new rapid-growth state following a nitrogen upshift.
  

Figure 1. Kinetics of physiological and transcriptome remodeling
during a nitrogen upshift.  A nitrogen upshift was performed by adding
400uM glutamine to a culture of yeast cells growing in minimal media
containing 800uM proline as a sole nitrogen source. a) Culture density
before and after the upshift. Dotted lines denote linear regression of
the natural log of cell density against time before the upshift and 2
hours after the upshift. b) Cell size changes during the upshift.
Dotted lines denote the mean cell diameter before the upshift and  2
hours after the upshift. c) PCA analysis of global mRNA expression in
steady-state chemostats and during an upshift (Airoldi et al. 2016).
Steady-state nitrogen-limited chemostat cultures maintained at
different growth rates (colored circles) primarily vary along
principal component 2.  Expression following a nitrogen-upshift in
either a chemostat (squares) or batch culture (triangles) show similar
trajectories and primarily vary along principal component 1.  Fit grey
lines  illustrate the major trajectory of variation for the
steady-state and upshift experiments.  To evaluate concordance in
transcriptome remodeling between the two nitrogen upshift conditions,
and the extent to which they reflect changes in gene expression
observed during steady-state growth, we performed  principal component
analysis of global gene expression (Figure 1c). The first two
principal components, which account for almost half of the total
variation, clearly separate steady-state and nitrogen upshift
cultures.  Systematic changes in growth rate using steady-state
chemostat cultures primarily result in separation of gene expression
states along the second principal component.  By contrast, gene
expression states following a nitrogen upshift in both chemostat and
batch cultures follow similar trajectories, along the first principal
component.  This suggests that although a nitrogen upshift results in
a gene expression state that reflects an increase in cell growth rate
(Airoldi et al. 2016), the transcriptome is remodeled through a
distinct state.  In longer time-series experiments performed in
upshift experiments in chemostats, the gene expression trajectory
returns to the initial steady-state condition as the excess nitrogen
is depleted by consumption and dilution (Supplementary Figure 1).  To
investigate the functional reprogramming that distinguishes gene
expression in the upshift and steady-state conditions, we computed the
correlation of each transcript with the loadings on these first two
principal components and performed gene-set enrichment analysis
(Supplementary Table 1). Both steady-state and dynamic gene
expression trajectories increase with principal component 2, but they
diverge along principal component 2. Transcripts that are positively
correlated with principal component 1 and principal component 2 are
enriched for functions including ribosome biogenesis, nucleolus, and
tRNA processing and transcripts that are negatively correlated are
enriched for mRNAs associated with the vacuole, cell cortex, and
carbohydrate metabolism.  Transcripts that are positively correlated
with principal component 1 are enriched for  mitochondrial translation
and structural constituents of the ribosome, and negatively correlated
transcripts are enriched for endocytosis, plasma membrane,
sporulation, cell wall biogenesis, and responses stress responses.
This suggests that although the dynamic reprogramming of gene
expression in response to a nitrogen upshift reflects changes in
cellular processes that are also observed during systematic changes in
that steady-state growth rates, a nitrogen upshift results in a
distinct set of gene expression changes that likely reflect a distinct
physiological process of adapting to the new condition. Importantly,
the reprogramming is common to both the chemostat and the batch
upshift (Figure 1c). As batch cultures are a technically simpler
experimental system, we performed all subsequent experiments using
batch culture nitrogen upshifts.  Global analysis of mRNA stability
changes during the upshift Previously, we found that GAP1 and DIP5
mRNAs are destabilized in response to a nitrogen upshift (Airoldi et
al. 2016). We sought to determine if mRNA destabilization is specific
to NCR transporter mRNAs. Therefore, we combined 4-thiouracil (4tU)
labeling and RNA-seq using a label-chase design (Neymotin et al. 2014)
(Munchel et al. 2011). As 4tU labeling requires nucleotide transport,
which may be altered upon a nitrogen-upshift (Hein et al. 1995), we
designed experiments such that the chase was initiated prior to the
nitrogen upshift and analyzed data to quantify the change in stability
of each transcript upon the upshift (Figure 2a). We normalized data
using thiolated spike-ins by fitting a log-linear model to spike-in
counts across time (Supplementary Materials), which reduced noise and
increased our power to detect stability changes (Supplementary Table
2, Supplementary Table 3, Supplementary Table 4).  All model fits can
be explored using a Shiny application (see below).
  

Figure 2. Global mRNA stability change following a nitrogen upshift.
a) 4tU-labeled mRNA from each gene was measured over time, before and
after the addition of glutamine (nitrogen-upshift) or water (mock) at
the vertical dotted line. Linear regression models were fit to the
data with a pre-shift slope (solid line) and a post-shift change in
slope (dashed line). HTA1 is not destabilized, whereas mRNAs encoding
NCR-regulated transporters or pyruvate and trehalose metabolism
enzymes are destabilized. b) comparison between the pre-shift mRNA
degradation rate (y-axis) and the upshift mRNA degradation rate
(x-axis)  c)  comparison between changes in mRNA expression following
upshift (Airoldi et al. 2016) (y-axis) and the upshift mRNA
degradation rate (x-axis). Both plots share the same x-axis
Transcripts that are significantly destabilized are colored red, and
shown with red rug-marks in the marginal histogram for the x-axis at
top, and for each y-axis on the right . c)


To detect changes in mRNA degradation rates, we modeled
log-transformed data using linear regression and quantified the
significance of slope changes corrected for multiple hypothesis
testing ( Supplementary Table 5). Of 4,859 mRNAs that we measured we
identified 94 that increased in degradation rate and 38 that decreased
(FDR < 0.01 Storey et al. 2015). We used a model of nucleotide
labeling kinetics to assess the effect of an incomplete chase on our
experimental design ( Supplementary Materials), and found that
complete transcriptional inhibition alone could result in an apparent
17\% increase in the degradation rate (for a typical transcript by
assuming a median degradation rate prior to the upshift). Accordingly,
since transcript synthesis is bounded by zero, we choose to exclude
any apparent stabilization effects and focus only on destabilization
of at least a doubling (100\% increase) of degradation rates between
pre-shift and post-shift. This left 78 mRNA significantly destabilized
upon a nitrogen upshift. Given that many of these destabilized
transcripts are below the median degradation rate, we expect that the
possible contribution of synthesis rate changes to the measurement are
well below the median 17\% effect of instantaneous repression. This
conservative cutoff of at least a doubling in decay rates is a
tradeoff that appropriately restricts our calls of significant
destabilization to indicating a rapidly triggered destabilization of
the mRNA.  The median pre-shift half-life was 6.89 minutes and the
median half-life following the upshift was 6.4 minutes (Table 1)
suggesting that there is not a large global change in mRNA stability.
Indeed, the vast majority of transcripts (4,796 of 4,859) do not show
individual evidence for stability changes upon addition of glutamine
(e.g.  HTA1, Figure 2a). Transcripts that are destabilized following
the upshift tend to have slower decay rates in steady-state conditions
(Figure 2b). Global stability estimates are considerably lower than
previous estimates in rich medium (Munchel et al. 2011) (Neymotin et
al. 2014), which may reflect the different nutrient conditions or
methods used in our study. For 78 transcripts, we detect significant
destabilization upon the glutamine-upshift. These transcripts have a
median half-life of 9.46 minutes before the upshift and a median
half-life of 3.02 minutes following the upshift (a median 3.06-fold
increase in degradation rates). For example, mRNAs for the NCR
transporters GAP1, DAL5, and MEP2 (blue label, Figure 2a), the
pyruvate metabolism enzymes PYK2 and PYC1 (orange label), and
trehalose synthase subunits TPS1 and TPS2 (yellow label) are
destabilized upon the nitrogen upshift. Data and models for each gene
are available for inspection as an GUI interactive shiny object tool
(see http://shiny.bio.nyu.edu/users/dhm267/ or Supplementary Material
for instructions).  Table 1. Summaries of mRNA stability measurements
for  total and destabilized mRNA


	Median pre-shift
	

	Median post-shift
	

	Median change
	

	Specific degradation rate half-life (minutes) Specific degradation
rate half-life (minutes) Specific degradation rate All transcripts
0.100 6.92 0.110 6.32 0.00865 - Destabilized set 0.0732 9.46 0.229
3.02 0.158 - No change 0.101 6.89 0.108 6.40 0.00728 
	

To determine the specificity of mRNA destabilization we tested for
functional enrichment among the set of 78 most strongly destabilized
mRNAs. Destabilized transcripts are strongly enriched for NCR
transcripts (16 of  78, p < 10-11. Approximately half of the
destabilized transcripts are also annotated as “ESR-up” genes
(Supplementary Figure X), on the basis of  their rapid induction
during the environmental stress response (Gasch et al. 2000). The 78
genes that show accelerated mRNA degradation are also enriched (FDR <
0.05) for GO terms and KEGG pathways (Supplementary Table 7)
including glycolysis/gluconeogenesis (6), carbohydrate metabolic
process (24), trehalose-phosphatase activity (3), glycogen metabolic
process (7), pyruvate metabolic process (6), exopeptidase activity
(5), allantoin metabolic process (3), and secondary active
transmembrane transporter activity (8). This analysis shows that the
destabilization of mRNA upon a nitrogen upshift is a regulatory
mechanism enriched for, but not restricted to, NCR-regulated mRNA.
Destabilization also targets transcripts required for secondary-active
transporters and specific aspects of carbon metabolism, namely
glycogen, trehalose, and pyruvate metabolism.  To investigate the
extent to which mRNA stability changes contribute to transcriptome
reprogramming, we compared upshift degradation rates to mRNA
expression changes following the  upshift (as reported in (Airoldi et
al. 2016), Figure 2c). Changes in mRNA degradation rates and
expression change rates are anti-correlated (R2= -0.376), consistent
with stability changes impacting gene expression dynamics. However,
they are not entirely co-incident, as some destabilized transcripts do
not exhibit decreases in abundance (red points in Figure 2c and
Supplementary Figure 4), showing that expression changes do not
explain all changes in measured degradation rates. This analysis shows
that  increases in degradation rates play a large role in the rapid
reprogramming of the transcriptome upon a glutamine upshift, but that
in some cases cases they are counteracted by increases in mRNA
synthesis rates resulting in no significant change in abundance
(Shalem et al. 2008).  Functional coordination of mRNA stability
changes suggests  a possible role for cis-element regulation. We
analyzed UTRs and CDS for the presence of  sequence motifs or
enrichment for known motifs of RBPs. We tested for sequence and
structural motifs using several software tools (Materials and
Methods), and two definitions of UTR sequences (using a fixed 200bp
window upstream or downstream of the CDS or using the most distal
isoform ends defined by TIF-seq) (Pelechano et al. 2014). 3’ UTRs of
destabilized transcripts were depleted of Puf3p binding sites,
suggesting that this factor is not involved and regulates transcripts
distinct from this set (Supplementary Figure X). 5’ UTRs are enriched
for a GGGG motif, which we may be explained by convergence between
mRNA stability changes and transcript synthesis rate control as a
result of  Msn2/4, consistent with the overlap of membership with ESR
“up” genes (Supplementary Figure X). 5’ UTRs were enriched for
binding motifs reported for Hrp1p suggesting a role for this
nuclear-cytoplasmic shuttling  factor in affecting mRNA stability in
the cytosol (Supplementary Figure X). We also compared the CDS length
and fraction of optimal codons (Khong et al. 2017) in each feature and
found that on average, destabilized mRNAs are longer and contain more
optimal codons (Supplementary Figure X). Together, this analysis
suggests that the mechanism of destabilization may be related to cis
elements in the 5’ UTR and potentially to the translational status of
the mRNA.
  

Figure 3. Quantifying the kinetics of accelerated GAP1 mRNA
degradation using BFF.  a) Kinetics of accelerated GAP1 mRNA
degradation . A nitrogen-limited (batch proline) culture of yeast was
upshifted (glutamine addition), and GAP1 mRNA quantified using RT-qPCR
relative to an external spike-in mRNA standard. The dashed line
indicates a log-linear regression model fit to points after 2 minutes.
b) Flow cytometry of wild-type yeast in nitrogen-limited conditions
and following an upshift. The vertical grey lines indicate FACS gate
boundaries used for cell sorting. c) Representative cells  from each
bin sorted from wildtype cells. d) Quantification of the microscopy
data. Each black dot represents a single cell  The mean number of foci
per cell in  each bin  is displayed as a red point.  Developing a
genome-wide screen for trans-factors of GAP1 mRNA repression We sought
to identify trans-factors mediating accelerated mRNA degradation in
response to a nitrogen upshift. We selected GAP1 mRNA as
representative of transcript destabilization, as it is abundant in
nitrogen-limiting conditions and is rapidly cleared upon addition of
glutamine  (3.24-fold increase in degradation rate) (Figure 3a)
(Supplementary Table 5). Previous approaches to high-throughput
genetics of mRNA abundance have used either protein expression
reporters (Neklesa and Davis 2009) or extensive automation of nucleic
acid purification and qPCR (Worley et al. 2015). However, for our
purposes, we required a method that was suitable for direct
measurement of GAP1 mRNA to report on the dynamics of accelerated mRNA
degradation and amenable to replicated measurements across multiple
timepoints, using available facilities. Therefore, we developed a
single molecule mRNA FISH (smFISH) assay to quantify native GAP1
transcripts in single-cells, facilitating efficient screening of the
prototrophic yeast deletion collection (VanderSluis et al. 2014) using
fluorescence activated cell sorting (FACS) to separate mutants on the
basis of mRNA abundance. The abundance of each strain can then be
estimated using BARseq, a method for amplicon sequencing of the
15-22bp strain barcodes that identify each knockout (Smith et al.
2009; Robinson et al. 2013; Giaever and Nislow 2014). We employed this
approach, which is a variant of the Sort-seq experimental approach
(Peterman and Levine 2016), to estimate transcript abundance for each
mutant in the pool at two timepoints: before the upshift which GAP1
mRNA is fully induced (pre-shift) and 10 minutes after the upshift
(post-shift). This approach facilitates identification of mutants with
defects in mRNA regulation at both the transcriptional and
post-transcriptional level without altering GAP1 mRNA cis-elements
that may affect its regulation. Moreover, our design enables
identification of factors that regulate both the steady-state
abundance of GAP1 mRNA and its accelerated degradation following an
upshift.  Development of our screen required that GAP1 mRNA could be
accurately quantified in single cells using flow cytometry. We found
that  individually labeled probes tiled across GAP1 mRNA (Zenklusen et
al. 2008; Raj et al. 2008) were insufficiently bright to assay GAP1
mRNA expression using flow cytometry (data not shown), likely due to
the small cell size of nitrogen-limited cells and the low transcript
numbers in yeast cells compared to mammalian cells (Klemm et al.
2014). Therefore, we used branched DNA probes (Quantigene), which
serve to amplify the smFISH signal (Hanley et al. 2013). We developed
a fixation and permeabilization protocol (Supplementary Methods )
that enabled detection of the  distribution of  GAP1 mRNA in
steady-state nitrogen-limited conditions and its repression following
the  upshift (Figure 3b). In control experiments, we found that the
signal is eliminated in a GAP1 deletion or by omitting the  targeting
probe that confers specificity (Supplementary Figure 5, Figure 3b).
To validate flow cytometry  quantification we sorted a sample of cells
into quartiles and counted the number of fluorescent foci per cell
using microscopy (Figure 3c) . We found that increased flow cytometry
signal is associated with an increase in the number of foci in the
cells (Figure 3d, R2 = 0.607, p-value < 10-11 ).


SortSeq requires barcode sequencing to quantify the abundance of
genotypes in pooled assays. Previous methods using FACS to fractionate
the yeast deletion collection have used outgrowth of the sorted
fractions to generate sufficient material for PCR and sequencing using
Barseq (Sliva et al. 2016).  However, the fixation of cells precludes
outgrowth. We found that below approximately 106 templates, the
barcode PCR reaction using universal primers produced primer dimers
that outcompete the barcode PCR product. Therefore, we re-designed the
PCR reaction  (Robinson et al. 2013; Smith et al. 2009) to be robust
for very low sample inputs (Supplementary Materials). Our protocol
incorporates a 6-bp unique molecular identifier (UMI) into the first
round of extension to help correct for spurious quantification noise
from PCR, and uses 3’ phosphorylated DNA oligonucleotides and a
strand-displacing polymerase to block primer dimer formation and off
target amplification. We developed a custom bioinformatics pipeline to
take advantage of this new amplicon design, using pairwise alignment
for per-read quality-filtering and compatibility with variable barcode
length, and using the degenerate UMI barcodes to help account for PCR
duplicates.  (methods).  We refer to our experimental approach as BFF
(Bar-seq after FACS after FISH).  We used BFF to estimate GAP1 mRNA
abundance for every mutant in the haploid prototrophic deletion
collection (VanderSluis et al. 2014) in nitrogen-limiting conditions
and 10 minutes following the upshift. We performed biological
triplicates of the upshift and sampling, then processed the samples
independently through GAP1 mRNA FISH and FACS sorting into quartiles
of GAP1 mRNA expression (Figure 4a). These heterogeneous samples of
mutants were quantified using our above amplicon-sequencing protocol
in technical triplicate for every bin of sorted cells. We found that
UMIs approached saturation at a slower rate per input sample size than
expected for random sampling, consistent with a PCR amplification bias
towards repeated measurements ( Supplementary Figure 6) and adopted
the correction from (Fu et al. 2011) to account for possible UMI
collisions. After filtering counts (Supplementary Materials), we
calculated a pseudo-events metric that approximates a metric of mutant
cells sorted into each bin, and is the proportion of counts per mutant
multiplied by a scaling factor of how much of the library was sorted
into each bin. Using principal components analysis of these
pseudo-events, we find that the samples are separated primarily by
FACS bin within each condition and biological replicates are clustered
indicating that our approach reproducibly captures the variation of
GAP1 mRNA flow cytometry signal across the library   ( Supplementary
Figure 7).
  

Figure 4. Validation of BFF estimates of GAP1 mRNA  abundance.  a)
Flow cytometry analysis of GAP1 mRNA abundance in the prototrophic
deletion collection before and after the upshift.  The x-axis reports
expression in arbitrary units (a.u.) of fluorescence, vertical gray
lines denote the FACS gate boundaries, biological replicates are
indicated by color. b) Measurements for individual genes before and
following the upshift.  Black dashed lines indicate the log-normal
model fits.  Colors and axes as in a. c) Distribution of  mean
expression level for each mutant derived from maximum likelihood
fitting of a log-normal to pseudo-events for each mutant. The model
fits are drawn as black points connected by dashed lines as the
proportion of the fit model for that strain in each bin. cd) The Mean
GAP1 mRNA expression levels for individual mutants before and after
the upshift,   The distribution of all expression values for the pool
of mutants is shown as a reflected histogram for reference.
Estimating GAP1 mRNA abundance for individual  mutants We quantified
the distribution of GAP1 mRNA for each mutant by modeling
pseudoevents in each quartile, converted to proportions as a
log-normal distribution using likelihood maximization methods  (Figure
4b). From model fits we estimated the mean expression value for each
mutant and found that the distribution of means within each replicate
were similarly distributed (Figure 4c) and reflected the overall
distribution of flow cytometry signal (Figure 4a). To estimate GAP1
mRNA per strain, we used all replicate measurement to perform model
fitting and filtered models for sufficient measurements  (at least two
of three replicates in at least three of the four bins). We generated
expression distribution estimates for 3,230 strains, and used the mean
of each distribution as the estimate of GAP1 mRNA abundance for each
strain (Supplementary Table 13).  To validate our approach we
examined strains for which we expected to have a specific phenotype
and compared their mean expression level to the distribution of
expression for the entire population (Figure 4d). We find that the
wildtype genotype (his3KO, which is complemented by the spHis5 gene as
part of library construction) has an expression level that is
centrally located in the distribution both before and following the
upshift. The gap1KO genotype is a negative control, as no GAP1 mRNA
should be produced in the genotype, and we estimate that it is at the
extreme low end of the distribution before and following the upshift.
dal80KO, gzf3KO, and ure2KO are direct or indirect transcriptional
repressors of NCR, and we find that their GAP1 mRNA expression is
defective in repression with the dal80KO genotype having a stronger
defect in GAP1 repression. Counter-intuitively, a gat1KO, a
transcriptional activator of GAP1, appears to have higher steady-state
expression of GAP1 mRNA, but increased expression of GAP1 mRNA in a
gat1KO mutant has previously been reported (Scherens et al. 2006) and
is thought to result from the complex interplay of NCR transcription
factors on their own expression levels. Thus, analysis of a small
number of genotypes assayed using BFF validates our approach to pooled
analysis of mutant expression of GAP1 mRNA in steady-state conditions
and following an upshift.  To identify cellular processes that
regulate GAP1 mRNA abundance, we used gene-set enrichment analysis
(Supplementary Table 14). Mutants that exhibit high GAP1 mRNA
expression are enriched (FDR < 0.05) for sulfate assimilation.
Following the upshift we find mutants that maintain high GAP1 mRNA
expression are enrichment for negative regulation of gluconeogenesis
(Supplementary Figure X) and the Lsm1-7p/Pat1p complex (Figure 5a).
Mutants in the TORC1 signalling pathway were not enriched; however, we
find that a tco89KO mutant has  greatly increased GAP1 mRNA expression
before and after the shift (Supplementary Figure X), consistent with
the repressive role of TORC1 on the NCR regulon.To compare the
expression between the two timepoints for each mutant, we first
compared the difference between the log of the estimated means for
each strain by regressing the post-shift mean expression against the
pre-shift mean expression for each genotype (Supplementary Figure 8).
We used the residuals for each strain to identify mutants that clear
GAP1 mRNA with slower kinetics than expected on the basis of
steady-state expression.  We find that the Lsm1-7p/Pat1p complex is
strongly enriched for slower than expected GAP1 mRNA clearance
(Supplementary Table 14). Specifically the lsm1KO, lsm6KO, and pat1KO
strains are strongly impaired in the repression of GAP1 mRNA (Figure
5a) .
  

Figure 5. Disrupting the Lsm1-7p/Pat1p complex impairs accelerated
degradation of  GAP1 mRNA.  a) Distribution of fit GAP1 mRNA means for
mutants in the pool. Indicated by colored points and lines are the
means for labeled knockouts. b-d), GAP1 mRNA is quantified relative to
HTA1 mRNA before or 10 minutes after a glutamine upshift, in
biological triplicates. Lines are a log-linear regression fit.


To confirm the role of the Lsm1-7p/Pat1p  complex in clearing GAP1
mRNA during the nitrogen upshift we measured the rate of GAP1 mRNA
repression using qPCR for GAP1 mRNA normalized to the housekeeping
gene HTA1, which is not subject to destabilization upon the upshift
(Figure 2a). We also tested mutants that were not detected using BFF,
or were not suitable for modeling due to only being detected in the
highest GAP1 bins (e.g. xrn1KO Supplementary Figure X). We find that
the main 5’-3’ exonuclease xrn1KO and deadenylase complex (ccr4KO and
pop2KO) are impaired in GAP1 repression as expected for mutants in
this key mRNA degradation pathway (Supplementary Figure X, Figure
5b). We confirmed that the accelerated degradation of GAP1 mRNA is
impaired in a lsm1KO and lsm6KO (Figure 5c), but that this effect is
very slight (~30\% decrease in specific degradation rate, t-test p-val
< 0.03) which we address further in the discussion. We also tested
scd6KO and edc3KO, two modifiers of the decapping or processing-body
assembly functions associated with this complex, and found two
distinct phenotypes.  edc3KO has similar expression before the shift,
but is cleared much more slowly. scd6KO has a greatly reduced GAP1 BFF
signal and GAP1/HTA1 signal before the shift, failing to induce GAP1
mRNA to wild-type levels, but is still consistently slower in
repression of this residual amount. We also tested a tif4632KO, a
mutant in the eIF4G complex that is known to interact with Scd6p, and
found a similar phenotype. The phenotype from a deletion of an
initiation factor subunit suggested that perhaps the phenotype may be
specified in translation rate control on the mRNA, so we deleted
~100-150bp downstream of the stop codon (approximate 3’ UTR) or
upstream of the start codon (approximate 5’ UTR), and found that while
the 3’ UTR deletion did not have an effect the two 5’ UTR deletions
exhibited the same phenotype of reduced GAP1 mRNA before the upshift
and reduced rate of transcript clearance after (Figure 5d). These
measurements point towards altered mRNP composition of the
Lsm1-7p/Pat1p complex and associated decapping factors are associated
with defects in GAP1 mRNA repression upon a nitrogen upshift, and
importantly that the phenotype of the scd6KO, tif4632KO, or 5’ UTR
deletions preceed the addition of glutamine and suggest that these may
alter the priming of GAP1 to be susceptible to a destabilization
event.  As these factors are associated with processing-bodies, we
tested if microscopically-observable p-body dynamics may co-occur with
destabilization using Dcp2-GFP. We do not observe qualitative changes
in Dcp2-GFP distribution (data not shown, raw images in supplementary)
indicating that any differences in RNA and RNP interactions do not
result in a microscopically visible phenotype of processing-body foci,
as seen in other stresses. This is consistent with previous
investigations of amino-acid limitation stress (Hoyle et al. 2007) and
suggests that the defects in GAP1 mRNA clearance in the Lsm1-7p/Pat1p
complex and associated factors likely result from roles in decapping
or modulating translation of GAP1 mRNA.  Discussion The control of
mRNA stability allows cells to effect rapid transcriptome
reprogramming, especially in the clearance of un-necessary
transcripts. We refined our methods to facilitate a high-throughput
examination of the post-transcriptional regulation of the yeast
transcriptome in a growth up-shift environment. 4-thiouracil metabolic
labeling has been used many times to track mRNA stability in budding
yeast (Miller et al. 2011; Neymotin et al. 2014; Munchel et al. 2011)
and a pulse-chase design has been used to track stability of mRNA
during dynamic conditions (Braun et al. 2015; Munchel et al. 2011),
but we believe our modifications here will allow for improved
quantification of extant transcripts and interpretation of
measurements through explicit modeling of labelling dynamics to
account for some of the limitations (Pérez-Ortín et al. 2013). We
refer to this as a label-chase design because we are not tracking a
cohort of mRNA with a pulse-chase, but rather the whole steady-state
transcriptome. Additionally, continuing development of the label
purification biochemistry and incorporation of explicit per-transcript
efficiency terms will improve these measurements further (Chan et al.
2017).  We found that the yeast transcriptome is on average less
stable during nitrogen limitation. Given that this is the case in
comparison to label-incorporation and label-chase experimental
designs, we do not believe this is an artifact of the particular
label-chase design here.  This is different than what has been
observed upon entry into starvation of carbon-sources, so this may
suggest that the different nutrient limitation or degree of limitation
may play a role. Using these transcript stability estimates, we
compare them to transcriptome abundance dynamics during a nitrogen
upshift and find that although they have the expected relationship
more often than not ( anticorrelation, $R^2$=-0.376 ), we again see
that increases in degradation do not always result in rapid
repression. This contradiction has been observed for transcripts
up-regulated in stress conditions, and has been proposed as a
mechanism to effect a rapid rebalancing of the transcriptome after a
transient phase of reprogramming (Shalem et al. 2008). Thus it appears
that a similar phenomena occurs during the relief of stress. The
significant overlap of the genes destabilized during this upshift with
members of the ESR “up” regulon (Gasch et al. 2000) suggests that mRNA
stability is one mechanism that rapidly regulate these transcripts
during either improvement or worsening of environmental conditions.
The NCR regulon is rapidly repressed upon this nitrogen upshift, and
our measurements indicate that post-transcriptional regulation of
sixteen NCR transcripts accelerates this repression. However, this is
not limited to simple NCR but also targets transcripts functionally
enriched in carbon metabolism pathways, particularly pyruvate
metabolism. Our BFF assay found that mutants in negative regulation of
gluconeogenesis pathways were enriched in high GAP1 after the shift,
suggesting that disruptions to central carbon metabolism may have more
of a role in priming the metabolic state of the cell for rapid
clearance of GAP1.  Naively we had expected that the destabilizing
effect was mediated by an RBP binding the 3’ UTR of these
de-stabilized transcripts. Cis-element analysis eliminated known RBPs
from explaining the destabilization, and in particular demonstrated
that  Puf3p motifs de-enriched from the  destabilized set. These
destabilized transcripts were surprisingly enriched for a binding
motif of Hrp1p in the 5’ UTR. This essential component of mRNA
cleavage for polyadenylation in the nucleus  has been shown to shuttle
to the cytoplasm and bind to amino-acid metabolism mRNAs (Kim Guisbert
2005) and been shown to interact genetically to mediate
nonsense-mediated decay (NMD) of a PGK1 mRNA harboring an premature
stop-codonin the CDS (González et al. 2000) or a cis-element spanning
the 5’ UTR and first 92 coding bp of PPR1 mRNA (Kebaara 2003).
Elements in the 5’ UTR have also been demonstrated to destabilize GAL1
mRNA (Baumgartner et al. 2011) and SDH2 mRNA upon glucose addition,
perhaps due to the competition between translation initiation and
decapping mechanisms (de la Cruz et al. 2002). Interestingly, both
GAP1 and SDH2 share the feature of a second start codon downstream of
the canonical start (Neymotin et al. 2016). This, in light of recent
analyses further highlighting the contribution of translation dynamics
to mRNA stability (Cheng et al. 2017), suggests to us that the
mechanism of degradation may be modulated through dynamic changes in
translation initiation or elongation that trigger decapping of GAP1
and other mRNA. Additionally, promoter-mediated mRNA stability has
been demonstrated several times before in yeast, and in the
destabilization of mRNA upon a glucose upshift (Braun et al. 2015), so
these factors may be interacting with these sequences in the nucleus
to mark the mRNA for destabilization.Previous work from our group
mutated the start codon of GAP1 and found that this mutation reduced
steady-state mRNA abundances (Neymotin et al. 2016), so perhaps the
reduction of GAP1 mRNA in limiting conditions may reflect an altered
ribosome loading status. Future work interrogating this possible
interaction of translational activity with mRNA stability during
dynamic conditions could inform our understanding of the relationship
between the two in steady-state conditions.  To find the factors
driving this, we developed a trans-factor screen using mRNA FISH,
FACS, and sequencing. We believe this is the first time direct mRNA
abundance has been estimated using a SortSeq approach, although
sorting on indirect markers or using mRNA to enrich as subpopulation
has been demonstrated before (Klemm et al. 2014; Hanley et al. 2013).
This approach could be used with other barcoding mutagenesis
technologies, like transposon-sequencing or Cas9 mediated
perturbations, to systematically probe for the genetic basis of
transcript dynamic phenotypes. Additionally, the use of branched-DNA
mRNA FISH using the Affymetrix Quantigene technology (now Invitrogen
PrimeFlow RNA) or other methods (Rouhanifard et al. 2017) would allow
for mRNA estimation without requiring genetic manipulation, which
makes it suitable for use on natural variants in an extreme QTL
mapping application. While the cell wall of yeast makes optimization
crucial to this assay, future development of hybridization protocols
may improve accuracy and make the assay more robust (Richter et al.
2017; Wadsworth et al. 2017). The molecular methods designed should
permit accurate quantification of yeast deletion collection libraries
post-FACS sort with fixed samples, expanding the possibilities of
markers to fixed-cell flow cytometry assays.  We identify that mutants
in the Lsm1-7p/Pat1p complex have elevated GAP1 mRNA before and after
the shift, and have a defect in repressing GAP1 relative to HTA1 mRNA
upon the glutamine upshift. By qPCR normalized to a housekeeping gene
we confirm the expectation that this is not just a phenotype specific
to NCR mRNA, but that there is no steady-state phenotypic difference
in the relative quantification. There is a slight decrease in
repression rate. Our initial estimate using the BFF methodology may be
confounded, as technical reasons steered us towards using a quantile
binning strategy instead of using more bin resolution in the tails of
the distribution. While this gave us good design for estimating
deviance from the wild-type expression, this hindered our ability to
outright measure dynamics unless the phenotype was similar to or
crossed the wild-type value in one of the conditions (ex: scd6KO or
edc3KO). Given that the GAP1 mRNA is destabilized during this
transition we suspect that these core mRNA degradation factors are
directly involved.  Because factors associated with the Lsm1-7p/Pat1p
complex are also involved in processing-body formation we looked for
processing-body dynamics during the nitrogen upshift, but did not see
qualitative changes in the granularity of Dcp2-GFP signal. However, it
has been proposed that pre-existing mRNPs seed the formation of
processing-bodies, thus the phenotype may not be microscopic and would
require molecular assays to eliminate this possibility. Interestingly,
during cross-comparisons with recent datasets exploring mRNA
localization to RNP condensates we found that the set of destabilized
transcripts are on average longer in CDS and have an increased
codon-optimality, two factors that have been recently explored in
connection with stress-granule localization in yeast (Khong et al.
2017). Perhaps this phenotype is manifest at the molecular method, but
this possibility would require different molecular techniques to test
and exclude.  This study explores another example of mRNA
destabilization control in budding yeast. Future work contrasting this
process with the similar process in glucose upshifts may share
similarities or differences that would greatly inform our
understanding of mRNA stability specification. We also develop and
demonstrate a method to estimate mRNA abundance for every knock-out
mutant, in a high-throughput pooled approach. The methods here allow
for the use of fixed-cell flow-cytometry assays in pooled Sort-seq
assays on yeast, and would be useful to inform the development of
similar assays in other systems. Development of this approach to
estimating mRNA abundance on pooled mutants would enable the
combination of transcriptomics as a high-dimensional marker of
cellular signalling pathways with the use of transcript markers to
explore the genetics of these pathways.  Supplementary issues
Pulse-chase modeling ( I basically want to reprint the stuff in the
supplement here ) BFF rationale, methodology, and future directions (
I basically want to reprint the stuff in the supplement here, then
waste paper speculating )


\chapter{Investigations of physiological remodeling upon a nitrogen 
upshift}
\label{chapter:four}

This chapter describes lines of investigation that were not
pursued deeply, but may inform future investigations of the
physiological remodeling that occurs as yeast resumes rapid
growth.

The study of microbial physiology is a long standing area of
investigation, and with modern systems biology approaches this
question of how the physiological composition of microbes change in
order to accomplish the essential project of growth is still the
subject of advances both quantitative and conceptual 
\parencite{slator1918some,henrici1928morphologic,schaechter1958dependency,kjeldgaard1958transition,wehr1969macromolecular,waldron1977synthesis,carter1978protein,waldron1975effect,kief1981coordinate,scott2010interdependence,erickson2017global,kafri2016cost,metzl2017principles}.
In recent years, the study of changes in abundance of specific mRNA
factors in the budding yeast has characterized a phenomenon in which
approximately one quarter of the yeast transcriptome scales with
growth rate \parencite{brauer2008coordination,airoldi2009predicting}.
This phenomenon is characterized at the level of molecular species,
and thus can be compared to changes that occur in response to
stressors, summarized as the Environmental Stress Response 
\parencite{gasch2000genomic}. 
In addition, a shared signature of knockout mutants, commonly used to
probe gene function, is that associated with changes in cell-cycle
progression distribution due to growth rate changes
\parencite{o2014cell}.
Thus the appreciation of the systematic physiological changes that
occur in response to genetic perturbations holds light to many
biological problems, even if only to identify the domineering and 
confounding factor of growth-associated physiological changes.

%
%These, and other observations, described microbial growth generally as 
%possessing a kind of "interia" [@henrici1928morphologic], where the
%growth rate 
%of the old culture is maintained for a short period after a
%nutrient-shift.
%@sherman1924function studied changes in resistance in bacterial
%salt-stress 
%during this lag-phase to conclude that old cells would remodel their
%physiological state prior to initiating growth at the maximal rate
%possible of the new environment.
%@kjeldgaard1958transition, @wehr1969macromolecular
%waldron1977synthesis  found that following a nitrogen-source upshift,
%a yeast culture will continue its rate of accumulation of optical
%density
%for about 2 hours before increasing the rate to that appropriate to
%the new
%media.
%In contrast, the RNA accumulation appeared to lag only 10 minutes
%before
%accumulating at a rate temporarily faster than the new steady-state
%rate.
%This suggests that rRNA, thus ribosome content, 
%is a leading feature of this physioloical remodeling.
%A carbon-source upshift also results in massive regulation 
%that reprograms the yeast physiology and transcriptome (and other
%-omes) for 
%rapid growth \cite{kief1981coordinate}.
%It was originally observed that yeast paradoxically halts growth
%upon a glucose up-shift for about 60 minutes before resuming
%growth and protein-synthesis [@kief1981coordinate]. However,
%ribosomal RNA and protein were shown to still rapidly increase
%across the span of an hour, even while the rest of the cellular
%growth and division are halted. It is important to note that
%this study measured nascent relative to extant - that is, an
%increase in the ratio of nascent to extant can result from either
%accelerated synthesis or degradation of extant transcripts 
%[@kief1981coordinate].

%
%
%
\section{Changes in poly-adenylated transcript content per cell 
upon changes in growth rates}
%
%
%

This section describes work that contributed to a submitted article 
%that 
%has been submitted, rejected, and is currently undergoing revision 
%of the text to address a particular biological question with more
%focus.
%The article was 
titled:
\textit{"Growth Rate-Dependent Global Amplification of Gene Expression."}
Authorship of this article is: 
Niki Athanasiadou, Benjamin Neymotin, Nathan Brandt, 
\textbf{Darach Miller}, Daniel Tranchina, and David Gresham.
The \textit{biorxiv} draft is at \url{doi.org/10.1101/044735}

The writing and figures of this chapter are original to this document.

%
%
%
\subsection{Introduction}
%
%
%


We know that the total RNA content of a cell changes upon changes in
growth rates \parencite{waldron1975effect}.
We know that specific mRNA, each a small component of the cell's 
total RNA also change in relative abundance. A less-characterized 
question is if the whole mRNA transcriptome changes
and if this has a significant effect on the regulatory role
of absolute or relative changes in mRNA abundance.
Transcriptomic measurements are usually normalized to relative
measures, and is thus based (sometimes explicitly
\parencite{love2014moderated})
on the assumption that the total transcriptome does not change in 
abundance.  However, we now know of cases of where this 
assumption is violated \parencite{nie2012c}.

Spike-in normalized RNA sequencing can estimate absolute mRNA
abundance per cell, but has been criticized before for
being "too noisy" and instead computational methods of "removing
unwanted variation" were used \parencite{risso2014normalization}. 
Led by Rodoniki Athansidou, our group pursued a more thorough approach
to this design by normalizing RNA sequencing data using the ERCC
spike-in set \parencite{jiang2011synthetic}, 
using preliminary sequencing runs
to first determine the appropriate amount of spike-ins necessary for
accurate sequencing. Then, yeast were grown in systematically varied
nutrient limitations of growth, then RNA sequencing using a known
quantity of the exogenous spike-ins was used to normalize the
measurements to absolute mRNA per cell.

I sought to complement this work by orthogonally estimating the size 
of the whole yeast transcriptome. To do this, I adapted the screening 
strategy of \cite{amberg1992isolation} to flow cytometry. 
Essentially, this utilized a poly-deoxythymidine oligo singly labeled
with a fluorophore. This was hybridized in with the fixed and
permeabilized yeast cell, and the resulting fluorescence after washing
is taking to be a proxy for the number of hybridized poly-dT probes,
presumably hybridized to a poly-adenosine sequence, and thus mRNA.

Another motivation of this was to serve as a fixation-digestion
control for methods involving single-gene mRNA FISH.
We had patterns of mRNA FISH hybridization signal that appeared
bimodal \autoref{fig:gap1Delete}. 
This could be a technical issue of incomplete permeabilization due to
over-fixing, or a biological phenomenon.
To distinguish the two would take two-color FISH, with a positive
control \parencite{andersen2014genetic}.
Since nitrogen-limitation causes a severe restriction of the total
transcriptome content, we don't have an obvious pick for a uniformly
expressed positive control. 
However, most of the mRNA should be poly-adenylated, so FISH against
that sequence should be present in all cells, and in high-copy.
While I did not integrate this into the single gene mRNA FISH as an
internal control, I did use it to optimize fixation/permeabilization 
conditions.

%
%
%
\subsection{The assay design}
%
%
%

The assay uses a similar fixation permeabilization method as the
single-gene mRNA FISH assay, then an overnight hybridization using
dextran sulfate against a poly-dT probe, then flow cytometry.

The yeast cells are sampled via vacuum filtration onto nylon
filters, then the filters are quickly flash-frozen in liquid nitrogen.
These are resuspended in 0.75x PBS buffered 4\% PFA (from ampules from
EMS), the cells vortexed off the filter, then the filter discarded.
The cell suspension in the fixative is incubated for hours at RT to
complete fixation, with the assumption that rapid fixation halts
RNA metabolism in the cell and long-term fixation stabilizes the fixed
components into a configuration that can survive digestion and 
permeabilization. The fix is critically quenched using 2.5M glycine,
then collected by centrifugation and washed with PBS. The cells are
digested for one hour at 37C using lyticase and beta-mercaptoethanol
in 1.2M sorbitol buffered
by potassium phosphate at about 7.4 pH, with 20mM vanadyl
ribonucleoside complex to inhibit RNAses. 
This is washed and further permeabilized with 70\% ethanol overnight,
then is resuspened using hybridization buffer
(10\% dextran sulfate w/v, 2x SSC final, 100ug/ml ecoli tRNA, 
250mg/ml salmon sperm DNA) plus 100nM of a (dT)50+V oligo 5'-labeled
with with Alexa 488, as ordered from IDT. 
This is incubated for 14+ hours on a 37C roller drum, then washed with
2x SSC several times before resuspending in PBS and flowing through an
Accurri flow cytometer.
Poly(A) content signal was determined by the signal area on the
514/20nm detector.

To test this procedure, I used RNAseA treated cells as a negative
control.
\autoref{fig:ypdnlim} shows the RNAseA-treated controls for two
samples, where the treatment abrogates the signal for the vast
majority of the cells in the sample.

To optimize this design, I varied formamide from 0\% to 50\%, and
probe concentrations from 10nM to 1$\mu$M. I found that 100nM and 0\%
formamide saturated the signal of YPD-grown cells without largely
increasing the signal on the RNAseA-treated cells.
This assay takes approximately 4 hours of work spread over 3 days.
More detailed protocol is maintained by the Gresham laboratory.

%%%%% PUT A CONTROL FIG HERE.

%
%
%
\subsection{Nutrient limitation and transcriptome size}
%
%
%

Yeast growing in
YPD complete a division approximately every 1.5 hours (0.45 specific
growth rate), while 
proline-limited media (NLimPro) only supports division approximately
every 4.5 hours (.15 specific growth rate). 
Using this poly-dT FISH method, we see differences in the total 
poly-adenylated mRNA signal between the different media conditions
(\autoref{fig:ypdnlim}).
The distributions are significantly
different (KS test and Wilcoxon, p-value < $2.2 \times 10^{-16}$).
We know that fast growing cells (YPD) have more RNA per cell, so it
appears that part of this difference is contributed by a global
scaling of the mRNA content as well.
The fold-changes in the mean and median of the YPD-grown cells versus
the proline-limited cells were 3.34 and 3.68, respectively.

\afig{
  \includegraphics[width=.7\textwidth]{img/polya_ypdnlim_controls.png}
  }{
  Wild-type yeast grown in proline-limited media (left) or YPD rich
  media (right) were assayed in exponential growth for poly-A content.
  Included are RNAsed controls treated with RNAseA, to show negative
  samples.
  The plot is cropped from 0 to $10^5$ arbitrary units of polyA
  signal to show the center of the distributions.
  The distributions from different media have different means by 
  KS or Wilcoxon tests, with unreasonably small p-values.
  \label{fig:ypdnlim}
  }{Changes in whole cell polyA content in YPD or nitrogen-limitation.}

To investigate the dynamics of changes in poly-A abundance between 
different growth conditions, I grew cells in proline-limited media
overnight to reach a steady-state of growth, then collected samples
during a nitrogen-upshift.
I assayed the poly-A content of the cells using the above assay
(\autoref{fig:upshift}).
I found that the total poly-A content took about two hours to increase
to the new steady-state of a larger transcriptome, a similar timescale
as the changes in cell size and lag in population growth rate
(\autoref{fig:figure1a}).
The final steady-state differences were of a fold-change of 2.16 and
1.93 for the mean and median poly-A content, consistent with the
change between specific growth rates of 0.15 and 0.35 being lower
than the difference with YPD (0.45 specific growth rate). 

\afig{
  \includegraphics[width=\textwidth]{img/polya_upshift.png}
  }{
  Wild-type yeast were grown in proline-limited media, then glutamine
  was added at time 0 minutes. Samples were assayed for polyA content
  using the poly-dT assay.
  \label{fig:upshift}
  }{Changes in polyA content upon a nitrogen upshift.}

Previously, others in the lab (as described at the beginning of this
section) had used ERCC-normalized RNA sequencing
to assay the absolute abundance of mRNA in yeast grown at
systematically varied growth rates (0.12, 0.2, 0.3 specific growth 
rate) in chemostats. In a repeat experiment of this, I took samples 
from chemostats limited by nitrogen or carbon at these growth rates, 
and processed them to assay the distribution of poly-A content of the 
cells. \autoref{fig:nikis} shows the distributions and the
relationship between the distribution means and the estimates from
SPARQ (the spike-in normalized RNAseq method). We see that the poly-dT
method also captures the scaling of the whole yeast transcriptome
across different growth rates, and correlates well with the spike-in
normalized method.

\afig{
  \includegraphics[width=.48\textwidth]{img/polya_niki_box.png}
  \includegraphics[width=.48\textwidth]{img/polya_niki_summary.png}
  }{
  (Left) PolyA content was estimated for cultures grown in two nutrient
  limitations at three different dilution (growth) rates.
  (Right) Comparing these measurements to SPARQ (the spike-in 
  normalized RNAseq method) shows that two methods are well 
  correlated (Pearson's r=0.95, \texttt{cor.test} p-value = 0.003684,
  dashed-line shows linear regression through all points),
  although the poly-dT method remains uncalibrated.
  \label{fig:nikis}
  }{Measuring polyA content across systematically varied growth rates
    in chemostats, and comparison to a spike-in normalized RNA 
    sequencing method.}

\subsection{Conclusion and future directions}

This assay appears to detect changes in the scaling of the yeast
mRNA content between different growth rates. 
It is consistent with spike-in normalized RNAseq (random 
hexamer-primed) estimates of the total mRNA content.
As a flow cytometry 
assay this has the potential to be used as a marker for
high-throughput investigations of the genetics of transcriptome size
changes (or regulation), using methods as described in
\nameref{subsection:bff}. 
This method offers a conveniently high-throughput assay for total
transcriptome size, and as such is one more tool that microbial
physiologists can use to probe the functional changes that occur as
organisms systemically adapt to their environments and growth
programs.

However, more work remains to use the assay to reliably inform on
these changes without incorporating an orthogonal measure.
Changes in polyA tail length could hypothetically affect
hybridization, and a distribution shifting such that more of the
functional mRNA have a tail length less than minimum tail length
requisite for hybridization would produce a similar graded effect.
Hybridization of this probe to synthetic mRNA cross-linked to a nylon
substrate would allow quantitative testing of this in similar
conditions as the hybridization occurs, provided a method for
manufacturing accurately generated poly-A tail lengths exists.

Future investigations of mRNA content per cell will illuminate
the role or significance of total mRNA abundance 
versus relative mRNA abundance in gene regulation and physiological
adjustments to changing environments. Adjustment is apparent during a 
nitrogen upshift, what causes it, and is it adaptive?

\section{Screening for genes important for remodeling physiology for
growth}

\subsection{Introduction}



With changing physiology in response to growth rate changes, many
molecular and functional phenotypes change. One of these is the
resistance to stress. 
It has been long known that slow growing cells are more resistant to
stressors
\parencite{sherman1923physiological,elliott1993stress,lu2009slow}.
Yeast appears to have adapted to its ecological niche by adopting a
boom/bust, feast or famine approach to quickly
growing during favorable conditions at the expense of stress
resistance. 
Resistance to stress seems to offer "cross-protection", and the
anti-correlation of growth rate and stress resistance suggests that
the two processes might be opposed in mechanisms to
achieve these objectives.
The dimension of coordinated cellular growth may be a simple axis that
explains much of the variation in gene expression and phenotypic
differences in budding yeast
\parencite{brauer2008coordination,lu2009slow}.

One approach to identify the characteristics required for yeast to 
achieve a faster growth rate is to monitor the regulated changes that
occur upon the upshift. We could infer
that since the most logical response to a stress is to express this
adaptation, then the gene expression increasing upon a stress must be
adaptive \parencite{gould1979spandrels}. 
This has been demonstrated to be a false assumption, at least for the
case of heatshocks, as the genes whose expression increases do not 
overlap well with the
genes important for resistance \parencite{gibney2013yeast}.
The later functional genetic measurement is possible to do in 
high-throughput, as
the yeast community has access to a yeast deletion collection and
high-throughput means of assaying genetic effects on the quantitative
phenotype of continued existence. 
Thus, an assay of the functional consequences is a more direct
approach to understand these processes.

The nitrogen upshift enriches for %offers an obvious process to enrich for
differences in growth rate, by growth. 
Subtle effects can be magnified over time, for example a
1\% growth rate defect over 7 hours would be magnified to an abundance
change of at least 20\%. However, the phenotype I am interested in is
in the completion of remodeling for rapid growth, so I am most
interested in the duration of the lag between nitrogen addition and
increased growth. Thus, the compounding of growth rate effect does not 
apply. One approach would be to repeat the upshift many times on the
same batch of cells, but this greatly confounds the fitness between 
various growth stages and does not offer the reproducibility of cells
being in a particular physiological status --- nitrogen-limitation can
take hours to reach a steady-state of signalling
\parencite{tate2013five}, and the life history of an individual cell
could have physiological consequences.

After practicing the Feynmen method with a scientific advisor
(\nameref{section:acknow}), and given
the opportunity to work with a talented young scientist named Stephen
Nyarko, we decided to pursue this question by using the correlated
phenotypes of growth and susceptibility to stress.
The logic is that if we are interested in isolating mutants that are
defective in increasing their growth rate upon a nitrogen upshift, and
an increase in growth is associated with a susceptibility to stress,
then a somewhat-lethal stress should enrich for mutants defective in
susceptibility to the stress --- ie defective in rapidly increasing 
growth rate.
Upon further reading, we found that the group of Johan Theiveilen had
used a similar approach to isolate mutants defective for increasing
growth upon repletion of glucose, and had identified new critical
components of the PKA pathway, \textit{CYR1} and \textit{GPR1}
\parencite{van2000characterization}.
Thus encouraged, we intended to use the anti-correlation of growth
and stress resistance to isolate mutants defective in resuming growth
rapidly.

\subsection{Results}

We first determined if the heatshock resistance of the wild-type
FY4 yeast changed during a nitrogen upshift.
I grew cells in proline-limited media (approximately 4.5 hour doubling
time), then added glutamine to induce the nitrogen upshift.
For each sample, cells were subject to a 52C heatshock for 30 minutes
by the addition of pre-warmed media, or for negative control were
simply kept at room temperature. The processed samples were arrayed 
in a 96 well plate, then pinned onto YPD, grown at 30C for
approximately 40 hours, and imaged (\autoref{fig:dme180}).

\afig{
  \hfill
  \includegraphics[width=.2\textwidth]{img/160328exp180sample1.png}
  \includegraphics[width=.2\textwidth]{img/160328exp180sample2.png} 
  \includegraphics[width=.2\textwidth]{img/160328exp180sample3.png}
  \includegraphics[width=.2\textwidth]{img/160328exp180sample4.png}
  \\
  \begin{tikzpicture}[overlay
      ,font=\small,%font=\ttfamily
      ,inner sep=0pt,outer sep=0pt
      ,shift={(-8,0)}]
      ]
    \node[align=left] at (0,0) (o) {};
%
    \node[align=left] at ($(o)+(5,0)$) (dil) {Culture dilutions (5-fold)};
    \draw[->] (dil) to ($(dil)+(10,0)$);
%
    \node[align=left,anchor=east] at ($(o)+(2,3.00)$) {\tiny Heatshocked};
    \node[align=left,anchor=east] at ($(o)+(2,2.65)$) {\tiny Negative};
    \node[align=left,anchor=east] at ($(o)+(2,2.30)$) {\tiny Heatshocked};
    \node[align=left,anchor=east] at ($(o)+(2,1.95)$) {\tiny Negative};
    \node[align=left,anchor=east] at ($(o)+(2,1.60)$) {\tiny Heatshocked};
    \node[align=left,anchor=east] at ($(o)+(2,1.25)$) {\tiny Negative};
%
    \node[align=left] at ($(o)+(4,4.1)$) {Before upshift};
    \node[align=left] at ($(o)+(7.8,4.4)$) {61 minutes\\after upshift};
    \node[align=left] at ($(o)+(11.2,4.4)$) {132 minutes\\after upshift};
    \node[align=left] at ($(o)+(14.6,4.4)$) {204 minutes\\after upshift};
  \end{tikzpicture}
  }{
    Wild-type (FY4) cells were subject to nitrogen-limitation, then
    a nitrogen upshift with 400$\mu$M glutamine. Cultures were
    heatshocked at 52C for 30 minutes, or at roomtemperature 
    (negative). 5-fold dilutions were plated on YPD plates.
    \label{fig:dme180}
  }{Glutamine upshift causes a lost in heatshock-resistance.}

Thus, the glutamine upshift triggers a loss in resistance to the
heatshock. We then devised a screen, wherein a barcoded 
and pooled yeast deletion collection
is grown in conditions of nitrogen-limitation then upshifted.
Samples were taking before or 
after 120 minutes after the glutamine upshift,
heatshocked or not (negatives),
then outgrown to enrich for living mutants.
These libraries were sequenced using an amplicon-sequencing procedure
to quantify the mutants in the resulting library.

Direct measurement of mutant abundance is preferred, but we used
outgrowth of the heatshocked population, counting on the severe
selection of a heatshock to appropriately select.
We did this in six biological replicates in order to generate robust
signal.
These were extracted with standard Hoffman-Winston DNA preparations,
then amplified using the same primers and protocol as described in
\parencite{robinson2014design}. These were sequenced along with other
samples on an Illumina MiSeq run.

Barcode sequencing, like other molecule-counting applications of
sequencing like RNAseq, is presented to the researcher as a relative 
measurement in integer quantities. 
One of the first steps in reading this data in is to look at the
distribution of counts per strain barcode identified
\autoref{fig:sneDist}.
We see that our heatshock and outgrowth has a much more profound
distribution of effects.
For which genes is this significant?


Numerous statistical approaches
exist to normalize the data for accurate detection of differential
abundance. One flexible and robust method is using the \texttt{voom}
statistical pre-processing step with \texttt{limma}. 
This calculates
the expected noise contributed by low integer count observation, but
has the advantage of converting the measurement to a "counts per
million" relative metric for normalization. 
It also has useful visualizations for characterizing the 
distribution of signal across complex experimental designs.

One other observation in \autoref{fig:sneDist} is that the histograms
show a log-normal distribution of high counts, then a long tail
downwards. 
Then, there appears to be a low distribution of single digit counts
which enter the distribution from around zero \autoref{fig:sneDistSum}.
These are believed to occur from spurious amplicon products or
software misalignment counting barcodes that do not exist.
To characterize this further, I used \texttt{limma/voom} to generate
plots of variance against abundance for different thresholds of
cutoffs based on total counts across the entire library
\autoref{fig:vooming}.
I found that a threshold of 30 counts in total across the library
was sufficient to remove these effects.

\afig{
  \includegraphics[width=.6\textwidth]{img/sne_histogram.png}
  }{
  Histograms of counts, for each mutant in each sample. Three
  histograms show the occurrences of these observations for the
  library before the upshift (top), 2 hours after adding glutamine
  (middle), and after the heatshock and outgrowth (bottom). The wider
  spreading is a good indication of complex selection of large
  effects occurring in the library.  \label{fig:sneDist}
  }{Histogram of mutant counts, within each sample.}
\afig{
  \includegraphics[width=.6\textwidth]{img/sne_histogram_sum.png}
  }{
  Histograms of counts, for each mutant, summed across all samples
  in the three treatments: before the upshift (top), 2 hours after 
  adding glutamine (middle), and after the heatshock and outgrowth 
  (bottom). We see that most features are log normally distributed,
  but some appear to be noisy counts near zero, due to unknown
  factors. 
  \label{fig:sneDistSum}
  }{Histogram of mutant counts, summed across samples.}
\afig{
  \includegraphics[width=.7\textwidth]{img/sne_voomer.png}
  }{
  Each plot shows each gene average abundance (x-axis) against its
  residual variation (y-axis), with a line smoothing the relationship
  as expected by \texttt{limma} modeling. The threshold of minimum
  total counts per feature is shown for each plot in the grey bar.
  We see that thresholding above 30 counts \textbf{(right)} gives us the 
  expected relationship, while not thresholding \textbf{(left)}
  demonstrates how lowly abundance counts behave aberrantly with
  artificially reduced variance that may confound statistical
  analyses of barcode sequencing data (\texttt{limma} uses the model 
  fit as the line). 
  \label{fig:vooming}
  }{Diagnostic \texttt{limma/voom} plots show the effects of 
    low-count barcodes in confounding the noise model.}

I used this tool's
flexible general linear modeling interface to ask how our treatment
enriched for particular mutants.
We saw no significant effects from a glutamine upshift, 
confirming the intuition that this
is such a subtle effect of momentary fitness that it becomes hard to 
detect without amplification. 
Testing for the effect of changes in abundance based on a
glutamine treatment before the heatshock, we find that four deletion 
strains significantly
(multiple-hypothesis adjusted p-values < 0.05) increase in abundance 
specifically after glutamine treatment, and 41 are decreased in
abundance.

Of the four genes increased in abundance (suggesting a failure to
resume rapid growth), \textit{SLA1} and
\textit{CAP2} are involved in actin binding and dynamics.
\textit{SLA1} is involved in assembly of the cortical actin
cytoskeleton \parencite{holtzman1993synthetic}, 
while \textit{CAP2} is an actin barbed-end
capping protein that localizes to cortical actin patches
\parencite{amatruda1990disruption}. 
This suggests that these mutants specifically are involved in
remodeling the cortical exoskeleton in a way that makes cells more
susceptible to heatshock, or that these mutants are defective in
increases in stress-resistance associated with slow growth rates.
\textit{SXM1} over-expression rescues mutants defective in mRNA export 
from the nucleus \parencite{seedorf1997importin}, suggesting that it
may play a role in mRNA export itself and that mRNA export may
regulate some important downstream factor associated with increasing
growth.
\textit{MAE1} encodes a malate dehydrogenase. This reaction takes
malate, a citric-acid cycle metabolite, and converts it to pyruvate
\parencite{boles1998identification}.
Pyruvate is an essential substrate for the
biogenesis of the carbon structures of alanine, valine, and other 
amino-acids.
Carbon-skeletons of glutamine can enter the citric-acid cycle from a
point between the entry of pyruvate, and shunt from malate to pyruvate
via \textit{MAE1}.
Considering the enrichment of pyruvate metabolism mRNA identified as 
destabilized in the 4tU label-chase work in 
\autoref{subsection:stabilityChanges}, and that the same experiment 
showed either a stabilization or dramatic synthesis up-regulation of 
\textit{MAE1} mRNA upon the nitrogen upshift, one prediction might 
be that Mae1p provides a shunt by which yeast re-directs the excess of
carbon skeletons from glutamine deamination through the citric-acid
cycle to provide substrates for alanine biosynthesis. This may be
adaptive.

To explore this, I regenerated a \textit{mae1}$\Delta$ mutant using
a KanMX knockout cassette amplified from the yeast deletion
collection, confirming incorporation by PCR. I subjected this mutant
to a glutamine upshift, and saw an increase in the lag-phase duration
(\autoref{fig:mae1})
compared to wild-type (\autoref{fig:figure1a}). I sought to test if this
was specifically due to disruption of the malate to pyruvate shunt for
the effect of alanine metabolism, and so repeated the experiment but
added 200$\mu$M alanine, 200$\mu$M pyruvate, or water (mock) to the 
cell culture at the same time as glutamine.
I did not see a significant effect on the growth rate increase
(\autoref{fig:mae1}).
Thus, disruption of this gene may not result in slower upshift in
growth by virtue of blocking this metabolic pathway, but instead the
metabolic state of the cell before the upshift may not be well 
prepared to resume rapid growth.

\afig{
  \includegraphics[width=.7\textwidth]{img/dme231.png}
  \includegraphics[width=.7\textwidth]{img/dme242.png}
  \includegraphics[width=.7\textwidth]{img/dme244.png}
  }{
  A \textit{mae1}$\Delta$ strain was subject to a glutamine upshift.
  \textbf{(Top)} The mutant alone appears to show a slight defect in the lag 
  phase (approximately 3 hours compared to approximately 2 hours
  \autoref{fig:figure1a}).
  \textbf{(Middle and bottom)} The mutant had glutamine or glutamine and either
  alanine (middle) or pyruvate (bottom) added, with two cultures per
  treatment. Neither showed a significant effect in reducing lag phase
  or increasing growth rate.
  \label{fig:mae1}
  }{A \textit{mae1}$\Delta$ mutant is slower in a glutamine upshift,
    but this is not rescued by supplementation with alanine or
    pyruvate.}

\subsection{Conclusion}

We found mutants knocked out for several non-essential genes changed
their relative susceptibility upon heatshock treatment, suggesting
that their resistance does not decrease as much as wild-type upon the
re-addition of a nitrogen source with adding glutamine.
These mutants could be involved in either the increase in stress
resistance upon slow growth conditions, or the decrease in stress
resistance upon increase in growth. 
For the factors of the actin cytoskeleton, this points towards a
hypothesis that the increase in stress resistance results from
specifically the cortical actin network, and would be testable by
determining when these mutants are more or less resistant to stress
than the wild-type. For \textit{MAE1}, I found that there appears to
be a longer lag phase, but this is not rescued by addition of pyruvate
or alanine. This suggests that the deletion of this gene puts the cell
in a metabolic configuration less capable of rapidly increasing growth
rate upon glutamine addition.

\iffalse
%
%
%
\section{Apparent cell-cycle halt upon nitrogen-upshift}
%
%
%

Apparent cell-cycle halt upon
glutamine addition Previous work on cell cycle halt, basically just
alberghina, PKA and CLN1 This phenomenon has been previously seen.
Upshift ecoli, they get bigger.  This is thought to be because the
critical cell size threshold has been reset by growth signalling
pathways to a larger size. However, this result might argue that
instead it could be regulated by CLN1 transcript abundance.
Alberghina’s demonstrated that depends on Swi4?p, so there you go

Later, it has been shown that this halt in growth seems to occur
through a PKA-mediated repression of
CLN1$\cite{jiang1998??,$tokiwa1994??}



Experiment, results Conclusion ( anything that didn’t get into chapter
3 )
\fi

\chapter{Conclusion}
\label{chapter:five}

%The nitrogen-upshift in budding yeast is a model system for studying
%how diverse mechanisms of regulation converge on remodeling cellular
%physiology and reprogramming the molecular specifics to 
%carry out the feasting (or yeasting) of the \textit{Saccharomyces
%cerevisiae} feast-and-famine life cycle.

Upon repletion of nitrogen, \textit{Saccharomyces cerevisiae} 
resumes rapid growth.
This involves physiological remodeling as discussed in
\autoref{chapter:four}.
Rapid transcriptional reprogramming also occurs to repress mRNA of 
genes newly unneeded in the replete nitrogen condition, especially
NCR-regulated transporters as seen in \autoref{chapter:two} and
\autoref{chapter:three}.
In efforts to determine the genetic basis of the rapid clearance of
\textit{GAP1} mRNA in particular, I found that decapping modulators
play a role in these dynamics \autoref{chapter:three}. 
Here, I summarize these findings and speculate on future
directions that this work could contribute to.

\section{mRNA destabilization hastens functional reprogramming}

The five fastest decreasing mRNA in the transcriptome upon the
nitrogen upshift are NCR transporter mRNA 
(\autoref{fig:airoldi2016f6}).
Using principal components analysis, 
we find that the changes in mRNA abundances across the transcriptome
during the upshift are in some ways distinct from simply a switch
from a slow-growing gene expression state to a fast-growing gene
expression state (\autoref{fig:longTermPCA}).
Instead, there is an enrichment in promotion of RNA production and
processing, and a repression of cytoskeleton and membrane
organization (\autoref{subsec:pcaGoCorr}).

mRNA degradation rate control is known to play a role in other
environmental transitions, so I used a 4-thiouracil label-chase
experiment with RNAseq to characterize the stability and changes
of stability of individual mRNA in the yeast transcriptome 
(\autoref{fig:figure2a}).
To avoid problems of nitrogen-regulation of the uracil transporter
necessary for labeling changes, I used an interrupted chase
design.
To smooth technical noise inherent in the normalization of sequencing
data to low amounts of spike-ins, I used a normalization that models
the change of the whole yeast transcriptome.
This prevents particularly noisy measurements of spike-ins from 
being interpreted as changes in mRNA dynamics 
(\autoref{subsec:4tuNormalization}).
To address the potential confounding of synthesis rate changes with
measures of destabilization, I used modeling to characterize the
expected error and threshold our calls of significance on the
effect size (\autoref{subsec:4tuNormalization}).
These methods may assist the future application of the 4tU label-chase
approach to studying extant mRNA stability across dynamic conditions.

Using these methods, I estimated mRNA degradation rates for the
yeast transcriptome and found a median half-life of 6.92 minutes
(\autoref{tab:table1}).
This is less than previous estimates in rich media, suggesting that
mRNA may be less stable during these conditions of limitation of
nitrogen-quality in batch growth conditions. This may reflect 
an evolved preference for flexibility in nitrogen allocation over
the energy costs of turning over mRNA frequently.
Upon the upshift,
I found that degradation rates and changes in
mRNA abundance are, as expected, anti-correlated but not entirely
co-directional in their effect
(\autoref{fig:kkdComparison},\autoref{fig:comparisonDestabilized}).

I found that 78 mRNA are destabilized upon the upshift, with
enrichment for NCR mRNA as well as mRNA encoding factors of carbon
metabolism and vacuole components (\autoref{subsec:destabGoCorr}). 
Amongst these, \textit{GAP1} is subject to an
approximately three-fold increase in degradation rates. 
Thus, mRNA destabilization plays a role in the regulation of this
canonical NCR-regulated mRNA, in addition to the previously known
regulation at layers of mRNA synthesis and post-translational
control.

\section{Decapping is important for \textit{GAP1} clearance}

I aimed to determine the genetic factors of this swift clearance, and
so combined mRNA FISH with FACS and sequencing to estimate
\textit{GAP1} mRNA for mutants in a barcoded pool. Development of this
method required several technical advances in optimizing 
\textit{in situ} hybridization methods, developing a robust method 
for barcode sequencing in the face of seemingly inescapable dimer 
formation (\autoref{subsection:bff}), and implementing a modeling
analysis to estimate mutants distributions across bins of 
\textit{GAP1} abundance (\autoref{fig:figure4b}). 

Using this method, I determined that factors of the Lsm1-7p/Pat1p
complex were important for wild-type \textit{GAP1} expression
dynamics (\autoref{fig:figure5a}). Importantly, the modulators
\textit{EDC3} and \textit{SCD6} had defects that did not appear
to generally affect the whole transcriptome, unlike with defects in
\textit{LSM1} and \textit{LSM6} (\autoref{fig:figure5qpcr}).
Fortunately, more analyses of related genes had been completed
by Nathan Brandt, and a re-analysis of this data revealed that 
the Scd6p-interacting eIF4G2/Tif4632p had a similar defect.
Nathan Brandt had also characterized the effects of truncations of
sequence upstream of the \textit{GAP1} start codon or downstream
of the \textit{GAP1} stop codon, and re-analysis showed that the 5'
UTR truncations had a very similar phenotype as deletions of
\textit{SCD6} or \textit{TIF4632}.
These similarities suggest a connection between translation initiation
and mRNA stability during steady-state and dynamic conditions that
may inform our general understanding of the competition of these 
processes.

\section{Physiological changes that occur during nitrogen upshift}

Combining a fluorescent poly-dT probe assay
\parencite{amberg1992isolation} with flow cytometry, I was able to
orthogonally confirm the phenomenon of mRNA transcriptome scaling
in different nutrient environments (\autoref{fig:nikis}) and 
measure the dynamics of this transition (\autoref{fig:upshift}).
Given the compatibility of this assay with the SortSeq approach
described in \autoref{chapter:three}, this could be used to efficiently probe the 
genetic factors of transcriptome scaling dynamics in response to
changing environments. Understanding this process in yeast may
inform the study of how c-Myc signalling affects (or effects) 
this process \parencite{nie2012c}.

It is well known that
stress-resistance decreases with an increasing growth rate,
and we demonstrated this is true for a nitrogen-upshift as well
(\autoref{fig:dme180}).
We used this property to enrich a mutant library for defects in
increased susceptibility upon a nitrogen upshift, and found an
enrichment of two components that regulate the cortical actin 
cytoskeleton. Future studies of these, and the two other identified
mutants, may reveal growth-rate regulation of the cortical actin 
cytoskeleton involvement in yeast stress-resistance, and a potential
of Mae1p to connect nitrogen and carbon metabolism.

\section{Suggested future directions}

I would like to expand on these findings to suggest future 
directions of inquiry.

\subsection{All metabolism is connected}

We often reduce phenomena to simple models in order to
create a more powerful generality, but sometimes a
powerful categorization can be too far
\parencite{lazebnik2002can}.
In the label-chase experiment, I found that mRNA encoding factors of
carbon metabolism (namely pyruvate metabolism, and the isoenzymes 
\textit{PYK2} and \textit{HXK1}) were destabilized.
In the genetic screen (BFF), we found that mutants in negative
regulation of gluconeogenesis were enriched in mutants with high
\textit{GAP1} mRNA estimated after the shift
(\autoref{fig:gluco}), suggesting defects 
in repression. The core of nitrogen metabolism (glutamate and
glutamine) are intimately connected with the citric acid cycle via
alpha-ketoglutarate, and the roles of these metabolites or their
rates of inter-conversion have been proposed to play roles in
signalling changes in growth \parencite{fayyad2016yeast}.
Thus, continued study of the metabolic networks during perturbations
(like a glutamine upshift) may reveal an unappreciated cross-talk
amongst sub-networks that are currently considered as distinct. 
This may reveal contingent or adaptive regulation of diverse
signalling pathways from secondary metabolite read-outs that are
distal from the experimenter's intended input.
Here, glutamine is considered as a nitrogen-source but delivers to 
the cell two nitrogen atoms and a carbon skeleton that can be
delivered into the citric-acid cycle.
Thus, a systematic view of metabolism during this transitions may
reveal that the effects of a glutamine upshift may be more broad than
just nitrogen metabolism and regulation.

\subsection{Possible mechanisms of \textit{GAP1} clearance}

\textit{GAP1} mRNA is now an example of mRNA destabilized upon
a nutrient upshift. What effects this regulation?
I identified several candidate factors, here I speculate on the
implications of one group of factors.

Scd6p interaction with eIF4G1 has been primarily studied, 
but here we see a similar phenotype with eIF4G2 
(eIF4G1 was not tested). These are
homologs with very similar functionality
\parencite{clarkson2010functional}, 
although eIF4G1 is expressed higher
during “normal” laboratory growth conditions, ie rich media. 
eIF4G2 has been shown to be more highly expressed and localized to
processing-bodies during conditions of nutrient limitations, and
this is suggested to specify the storage of mRNA in anticipation of
resuming rapid growth \parencite{brengues2007accumulation}.
This initiation factor may help to globally (or specifically)
reduce the initiation frequency, thus reducing the fraction of
ribosomes consuming tRNA under conditions of low amino-acid,
and thus low charged-tRNA, availability.
This would help ensure an adequate
supply of charged-tRNAs for elongation of engaged ribosomes.
Differential regulation of 
translation initiation could explain the observation of sub-maximal
usage of extant ribosomes in initiation-limited regimes of cellular
growth \parencite{kafri2016cost,metzl2017principles}.
Continued studies of the role of Scd6p in specifying ribosome
loading in diverse environmental conditions could reveal a mechanism
to effect these adaptive changes in translation.
Additionally, a systematic study of the \textit{GAP1} 5' UTR could
reveal \textit{cis}-elements necessary for this regulation.

The translation and degradation of mRNA are intimately coupled
processes which compete for the same substrate, the mRNA. 
Very recently, imaging studies have challenged the 5'-3' closed loop
model of mRNA \parencite{adivarahan2017spatial}, and demonstrated
a strong effect of the translational status of ribosomal loading in 
separating the 5' and 3' ends of a single mRNA.
Degradation is mediated by connections between 3' and 5' ends, 
with the Lsm1-7p/Pat1p complex strongly promoting the decapping of
mRNA requisite for degradation in the main pathway of mRNA
degradation.
One explanation of the interaction between the degradation and
translation initiation machinery has been proposed as the competition
for the 5' m7G cap. 
The interaction of translation initiation factor eIFG2 and
decapping modulator Scd6p in affecting the degradation of
\textit{GAP1} suggests and interaction with ribosome loading status.
One explanation that connects these processes could be that the presence
of elongating ribosomes separates the 5' and 3' ends of the mRNA, 
similar to the cohesin-dependent loop extrusion model currently used 
by some to explain chromatin-organization. 
This would directly explain the connection between translation status
and degradation for individual mRNA as a simple function of 
3'-associated decapping factors stochastically interacting with 
the 5' end of the mRNA tether.

\subsection{An efficient method for estimating mRNA abundance in
barcoded mutant pools}

Much of what we know about mRNA degradation has relied on genetics,
but perturbing an exquisitely homeostatic system, like a living cell, 
can result in indirect effects as systems of regulatory feedback 
percolates the signal through the network. 
Thus in this work, the use of knockout mutants is the use of
an indeterminately perturbed system. 
Additionally, the laborious individual creation and maintenance of a 
collection of mutants can introduce potential artifacts of suppressor 
and passenger mutations \parencite{kwan2016rdna,markowitz2017reduced}.
The development of new methods for making mutant pools with 
high internal replication and minimal strain-handling selection
offers a new paradigm of yeast genetics that avoids or
minimizes some of these drawbacks \parencite{smith2016quantitative}.
The methods described here for doing SortSeq can be adapted
to any DNA-barcode-based sequencing assay of pooled mutants. 

With the integration of single-cell RNA sequencing with 
RNA-guided Cas9 genetic-editing 
\parencite{dixit2016perturb,hill2018design},
this approach of mRNA SortSeq may no longer be absolutely
necessary for doing high-throughput genetics for mRNA markers. 
However, this method works now with common lab reagents and equipment, 
and only requires quantifying the mutants across bins---
not quantifying all mRNA for all mutants.
For questions regarding a small number of transcripts,
the depth of replicates and timepoints possible with a SortSeq
design may be advantageous.
Additionally, the recalcitrance of budding yeast for 
single-cell molecular analysis is still technically challenging
\parencite{gasch2017single}.
Scaling this BFF assay up to more cells input and a greater investment
in FACS sorting \parencite{de2017deciphering} has the potential to 
estimate mRNA dynamics across multiple
timepoints for all barcoded mutants in a library.
However, single-cell RNA sequencing is advantaged in its global 
perspective and will presumably scale with methodological 
developments to increase cell throughput 
(thus replicating measures per genotype)
and sequencing throughput.
The efficiency of the pooled SortSeq approach could also be used to 
limit the search space of more precise but resource-intensive 
automated assays \parencite{worley2016genome}, 
and thus a hybrid approach could be used to
efficiently increase throughput of measured transcript dynamics. 
SortSeq for estimating mRNA abundance, or other steps of gene
expression, may be a useful orthogonal tool complementing
other approaches with its potential for scaling measurements across
timepoints, conditions, variants, and replicates.

\iffalse

\subsection{A role for pH in sensing nutrient upshifts?}

Addition of a preferred nitrogen source trigger a destabilization 
for several mRNA. 
Glucose upshifts result in a rapid but transient reduction in pH
\parencite{kresnowati2008quantitative}.
Hesso etc have shown that a similar thing could be happening.
Shifting pH can have strong biophysical effects, including
the aggregation of the poly-A binding protein pab1
Riback 2017
It could also be Gcn2 mediated (huesso)


Recent presentations at ICYGMB 2017 from the Andr\'{e} group have
pointed towards a role for proton-influx in triggering a pulse of
TORC1 activity, similar to that seen upon addition of a preferred
nitrogen source.

Thus, secondary-active proton symporters (like \textit{GAP1},
\textit{MEP2}, or ) could co-transport the signal

\parencite{kim2012need}

\fi


\SingleSpacing

\printbibliography[heading=bibintoc,title={References}]

\end{document}


