\chapter{Introduction}

Organisms adapt to
their environment by expressing different phenotypes as environments
change. We expect this to be an advantageous strategy, depending on
the various parameters of fitness advantages of each phenotype in each
environment balanced against the costs of innovating and maintaining
the machinery for adaptive differential gene expression
\citep{kussell2005phenotypic}. 
If different mechanisms have different properties of
efficacy and energetic costs, and if the phenotype to fitness
relationship varies, then different conditions may select
for the use of different mechanisms of achieving appropriately
regulated gene expression.
Thus, understanding the adaptive basis of the diversity of 
regulatory mechanisms may inform our understanding of selection 
on complex systems.

\section{Regulated gene expression} 

Expression of a protein-coding gene
product involves many complex steps, each with their own opportunities
for a variety of regulatory mechanisms. At the outset, DNA sequences
that encode genetic elements are transcribed into corresponding chains
of messenger RNA (mRNA). The rate of this transcription is helped by
factors that facilitate recruitment of RNA polymerase II (RNA Pol II)
and is hindered by factors that block this process by physical
occlusion or changes in the accessibility to the chromatin 
\citep{hahn2011transcriptional}. 
The translation of mRNA into a protein product by
ribosomes occurs at different rate for different genes and different
environments, regulated by a complex interplay between ribosomes and
associated translation factors, RNA binding proteins (RBPs), and
intrinsic factors of the mRNA like length or codon-usage 
\citep{dever2016mechanism}. 
For both the mRNA and its protein product, stability is also
important 
\textbf{(cite something about protein stability)} 
\citep{perez2013eukaryotic}. 
Additionally, localization or allosteric regulation can change
the activity of a gene product. Myriad factors contribute to the
expression of a gene product, and determining the functional adaptive
basis for particular regulatory mechanisms, if they are indeed
adaptive, would help us better understand the diversity of gene
regulatory mechanisms.

Functional transcriptome reprogramming during a
nitrogen upshift The budding yeast Saccharomyces cerevisiae is a
classically studied model system for many fields, including gene
expression regulation. In response to different nutrient
availabilities, yeast will change gene expression programs on the
whole transcriptome and adopt different rates of growth. The
particular nutrient environment is described in terms of the quality
and quantity of the nutrient provided. Quantity refers to just that,
the molar availability of the nutrient that the yeast can take up,
while quality is more of a empirical reference to how rapidly budding
yeast can biochemically incorporate the nutrient into their metabolism
--- ie rate of growth. One prediction from this understanding is that
altering the quantity of the nutrient availability to vary growth
rates in a range below which the quality limits growth rates will
elicit a common response between various nutrient limitations.
Indeed, studies systematically varying nutrient environments have
shown that about a quarter of the transcriptome is differentially
expressed at different steady-state nutrient-limited growth states,
regardless of nutrient used to limit growth 
\citep{brauer2008coordination,regenberg2006growth}. 
Statistical modeling of this process
determined that the molecular signature of this growth-rate signalling
could be capture in a small number of “calibrator” genes whose
expression was very well correlated with changes in growth rate or
perturbation of signalling pathways associated with this process, and
importantly also changed during dynamic transitions or upon
perturbation of growth signalling pathway PKA 
\citep{airoldi2009predicting}.
Dynamic transitions to better nutrient environments (nitrogen, carbon,
and phosphorus upshifts) shared a similar pattern 
\citep{conway2012glucose}, 
and the pattern of gene expression associated with increased
nutrient availability and growth rates is the opposite of the
Environmental Stress Response (ESR) --- a shared in co-regulation of
~600 mRNA across dynamic responses to various stressors 
\citep{gasch2000genomic}. 
Together, this shows how yeast has a common response that
largely corresponds to the suitability of the sensed sensed
environment. A better environment transduces to faster growth with
more growth associated mRNA and less stress response mRNA, and this
holds true in steady-state differences and dynamic transitions.  

One
classically studied transition between growth rates is the nitrogen
upshift. Yeast grows quickly when provided with nitrogen sources like
glutamine or ammonium sulfate, but can make use of various
less-preferred nitrogen sources like proline or urea by expressing
overlapping sets of specific and general nitrogen-source permeases
that concentrate these sources inside the cell for use.  

Various
nitrogen sources are then catabolized to eventually make glutamate and
glutamine, with an estimated ~85\% of macromolecular nitrogen coming
from the amino nitrogen in glutamate and the rest from the amide group
of glutamine 
\citep{magasanik2002nitrogen}. The addition of glutamine to
a nitrogen-limited culture, for example grown with only the
non-preferred proline as a nitrogen source, is called an upshift
because it is the change from a slow growing condition to one of rapid
growth. Through the use of a temperature-sensitive glutamine synthase
allele or treatment with methionine sulfoximine, it has been shown
that NCR is a response to intracellular glutamine availability
\textbf{(cites)}
. Upon an upshift, a regulatory phenomenon called nitrogen
catabolite repression (NCR) ensures that the set of NCR transporters,
metabolic enzymes, and regulatory factors are repressed.  

One layer of
the repression occurs at the level of transcript synthesis. Four of
the five GATA factors in yeast coordinate to control transcription of
NCR genes, with two factors (Gln3p and Gat1p) activating transcription
while two (Gzf3p and Dal80p) repress transcription 
\citep{hahn2011transcriptional,stanbrough1995transcriptional}. 
These factors are also subject to NCR
control to different extents, with the activators increasing the
expression of the repressive factors. This is thought to be an
adaptation to enable quick repression upon a nitrogen upshift, as may
be encountered when yeast is introduced to a new abundant nutrient
environment of grape or wort 
\textbf{(cite)}
. It has been long known that the
eukaryotic growth signalling pathway TORC1 largely regulates these
factors by controlling the activity of phosphatases and thus
localization of these transcription factors, via Ure2p for Gln3p 
\citep{beck1999tor} 
and unknown mechanisms for Gat1p
\textbf{(citez, also classic ure2 papz)}
. However, careful genetics over the last two decades in
Terrance Cooper’s laboratory has identified that the genetic
requirements for phenotypes differ in different environments, with
comparisons of nitrogen “starvation” (8+ hours) versus “limitation”
(<3 hours, or proline) showing that a large part of NCR regulation was
still unexplained by TORC1 signalling 
\citep{tate2013five}. By way
of a temperature-sensitive tRNA allele, they have since identified
that Gcn2p impinges in a parallel pathway through the 14-3-3 proteins
Bmh1/2 to promote the export of Gln3p and Gat1p 
\citep{tate2015gata,tate2015nitrogen,tate2017general}. 
Additionally, others have suggested that the
amino-acid permease Gap1p may directly signal to PKA, but more has to
be done 
(theivelen eh)
. Thus, multiple signalling pathways converge to
affect the import and export of NCR GATA factors to effect multiply
redundant layers of NCR transcript synthesis control.  

Gene product
regulation can also occur post-translationally. NCR has primarily
referred to the control of transcript synthesis rates, but it has been
long observed that upon addition of a preferred nitrogen source the
enzymatic and permease activities are repressed faster than can be
caused by a shut-off of synthesis 
\citep{cooper1983function}. A
classical NCR-regulated gene is the general amino-acid permease GAP1.
GAP1 mRNA is repressed much faster than the repression of the
protein-product 
\citep{stanbrough1995transcriptional}
, and we know that this
Gap1p shut-off is adaptive \citep{risinger2006activity}, 
perhaps due to an
excess of amino-acid transport causing ammonia toxicity 
\citep{hess2006ammonium}
or excess proton symport driving a depolarization against futile
Pma1p proton-export activity. This growth phenotype allowed the early
identification of mutants in this process, and this indicates that it
is mediated by a uniquitinyation mark that inactivates the permease
and leads to relocalization and degradation 
\textbf{(more risinger? (Risinger and Kaiser 2008) earlier? smething
from magasanik?)}
. Thus multiple layers redundantly repress
the NCR-regulated Gap1p.  

In Chapters 2 and 3, I show how mRNA
degradation also plays a role in this repression.  

\section{mRNA degradation and its regulation}

Even simply considering the regulation of mRNA
abundance, there are at least two processes that contribute --- that
of synthesis and degradation. We know much about transcript synthesis,
perhaps owing to the fact that virtually all events of mRNA synthesis
pass through a well-characterized reaction of synthesis by RNA Pol II,
capping and polyadenylation, and export into the cytoplasm. The
details may vary, but the common pathway is the same. Conversely, mRNA
degradation does have a main pathway that performs the bulk of mRNA
degradation but mRNA are also subject to divergent redundant pathways
that have been challenging to measure. Moreover, the rates of this
various process are subject to control in ways less well-understood.
While some similarity is thought to exist in how RBPs may recognize
cis-element sequences in RNA similar to how TFs recognize upstream
activating or repressing sequences in DNA, the single-stranded
nature[d] of mRNA complicates this process by blocking linear
cis-elements (Li et al. 2010). Additionally, these secondary
structures of RNA may be recognized as the cis-element, complicating
our approaches to recognize these patterns (Goodarzi et al. 2012).
Primary pathways of mRNA degradation The canonical protein-coding mRNA
is synthesized in the nucleus from a DNA template by RNA Pol II, and
is capped co-transcriptionally at the 5’ end with a m7G cap. As Pol II
transcribes sequence 3’ of the stop codon the cleavage factor complex
recognizes cis element binding motifs in the RNA to direct cleavage
and polyadenylation to specific sites in the mRNA. Upon successful
completion of this process, the nascent mRNA is exported to the
cytoplasm where it enters into the pool of translatable mRNA.
Typically, translation begins when initiation factors load ribosomal
subunits to scan the 5’ leader or untranslated region (UTR) for the
start codon where the process of coding sequence translation begins.
These initiation factors (eIF4F) bind the m7G cap to load ribosome
subunits (Dever et al. 2016), and thus most translation depends on the
cap ( with an exception demonstrated by internal ribosome entry sites
(Gilbert et al. 2007). The m7G cap is also critical for mRNA
stability. Xrn1p is a highly-processive combination of helicase and
exonucleolytic domains that as a single protein rapidly degrades
transcripts from a 5’ to 3’ end, recognizing unprotected 5’
phosphorylated ribonucleotides as substrates (Parker 2012). Thus, the
inverted linkage of the m7G escapes degradation.  

During rounds of
translation the poly-adenosine tail is shortened from about 65-90
adenosines to about 10 adenosines by a combination of the Pan2/3 and
Ccr4/Pop2 deadenylase complexes, with activity antagonized by the
poly-A binding protein Pab1p (Parker 2012; Decker and Parker 1993).
When the tail is thus shortened to ~10 adenosines, the Lsm1-7p/Pat1p
complex binds the remainder of the poly-A tail (Tharun et al. 2000).
This complex is a heptameric ring of the Lsm1-7 proteins with the
Lsm1p’s C-terminal domain elegantly spanning the center (Sharif and
Conti 2013) and the last eight residues projecting into this center
and critical for binding the shortened poly-A tail (Chowdhury et al.
2016). The critical function of this complex is to recruit and promote
activity of the decapping complex to the 5’ end of the mRNA, and in
cooperation with Pat1p (Chowdhury et al. 2014) the binding of this
complex to mRNA and to decapping factors is indeed correlated with
decapping of the mRNA (Chowdhury and Tharun 2009). Thus, the complex
maps the deadenylated status to the next step in mRNA degradation.  

A
cytoplasmic mRNA without a 5’ m7G cap is not long lived, by virtue of
Xrn1p, thus the recruitment and activation of the decapping complex is
thought to be the key regulatory step in rates of mRNA degradation
(Coller and Parker 2004). Dcp2p carries out the catalytic activity of
the holoenzyme but is promoted by the effects of Dcp1p, and in
comparing in vitro to functional in vivo assays of mutants it appears
that the catalytic rate of the enzyme is not the limiting step (Tharun
and Parker 1999). Rather it is re-modeling of the mRNP complex that
leads to association of the decapping enzyme complex with the 5’ cap,
and the rate of this process determines the activity of this
degradation pathway (Tharun and Parker 2001). This
decapping-enzyme-localization process is promoted by the Lsm1-7p/Pat1p
complex and inhibited by poly-A binding protein Pab1p (perhaps by
competition of Pab1p with Lsm1-7p/Pat1p for the poly-A tail) and eIF4E
(Coller and Parker 2004; Caponigro and Parker 1996). eIF4E, eIF4G, and
eIF4A comprise the cap-dependent translation-initiation factor eIF4F
(citez), which gives rise to an elegant model for the observed effect
of translation initiation inhibiting decapping by competition for the
5’ m7G cap (Huch and Nissan 2014), while the requirement of sufficient
poly-A tail for Pab1p to bind is consistent with the effect of
deadenylation in promoting deapping (Parker 2012). The Lsm1-7p/Pat1p
complex and Dcp2p/Dcp1p are physically associated (by
co-immunoprecipitation) with several factors that genetically modulate
the activity of remodeling step ---Dhh1p, Edc3p, and Scd6p (Nissan et
al. 2010).  

Dhh1p is a helicase that associates with polysomes (mRNA
with multiple ribosomes bound) and has been recently demonstrated to
be genetically required for the relationship between codon optimality
and mRNA stability (Radhakrishnan et al. 2016; Presnyak et al. 2015;
Sweet et al. 2012). It has been demonstrated to play a critical
genetic role in mapping codon-optimality and translation speed back to
changes in mRNA stability. Curiously, tethering Dhh1p to a 3’ UTR
using the MS2 system resulted in lower translation rate of an mRNA
despite causing more ribosomes to be associated with the mRNA (Sweet
et al. 2012), suggesting that Dhh1p resolves slowly translating
ribosomes by promoting decapping and mRNA degradation.  

Edc3p
physically associates with the decapping complex and stimulates its
activity (Nissan et al. 2010), but it has also been shown to be
important (Decker et al. 2007) but not critical (Huch et al. 2016; Rao
and Parker 2017) for the formation of processing-bodies. These are
microscopically visible foci of mRNA and degradation factors that form
in response to stress conditions but are assumed to be condensed from
mRNA-protein complexes that exist before stress (Sheth and Parker
2003; Lui et al. 2014; Rao and Parker 2017). Interestingly, the
canonical role of processing-bodies had been as a site of mRNA
degradation, but recent work with refined genetic tools demonstrated
that mRNA degradation takes place outside of these sites (Tutucci et
al. 2017). A mutant deleted of EDC3 and the C-terminal domain of the
essential LSM4 is deficient in processing-body formation with , and is
surprisingly this processing-body deficiency also correlates with a
deficiency in the stabilization of several mRNA upon osmotic stress
(Huch and Nissan 2017). Thus, the role of Edc3p may be to promote
interactions between these mRNPs (mRNA-protein complexes) in a manner
that affects how well mixed the degradation factors and targeted mRNAs
are.  

Scd6p inhibits the formation of the 48S pre-initation complex
(when the 40S subunit associates with eIF4E cap-dependent initiation
factors and begins to scan the 5’ UTR) via forming its own cap-binding
complex with eIF4G subunit eIF4G1 in an arginine-methylation-dependent
manner (Rajyaguru et al. 2012; Poornima et al. 2016). This is thought
to physically occlude the normal initiation complex from binding.
Scd6p also binds to several other factors in the Lsm1-7p/Pat1p
deadenylation-promoting complex (Nissan et al. 2010), and thus may
play an indirect role in preventing the pre-initiation complex from
binding and stabilizing the mRNA through translation.  

Other pathways
of mRNA degradation exist. If the poly-A tail is completely removed,
the cytoplasmic exosome complex can degrade the mRNA from 3’ to 5’.
This redundant mechanism allows a xrn1$\Delta$ mutant to grow, although
slowly, as the 5’ to 3’ pathway is throught to effect the bulk of mRNA
degradation (Parker 2012). The balance between the two may be because
of the enzymatic rate of digestion, but more likely because the
binding of the Lsm1-7 complex to the shortened poly-A tail protects
the mRNA from further deadenylation and thus represses this 3’ pathway
(cite Tharun on this).  

\section{Alternative pathways of mRNA degradation ---
quality control and other }

Three other pathways are known to act as a
co-translational layer of quality control, where errors detected by
abnormal translation processes result in destruction of the presumably
defective mRNA. Nonsense-Mediated Decay (NMD) is the canonical example
of this. A mutation, transcriptional error, alternative splicing
event, or abnormal translational event (like leaky scanning or a uORF,
explained later) can cause an mRNA to have a stop codon well before
the usual position, which is recognized for destruction by a
deadenylation-independent decapping and 5’->3’ decay (Muhlrad and
Parker 1994). How the aberrant nature of the misplaced stop codon is
detected is still a mystery, but NMD sensitivity is known to be
maximally active after Xaa have been translation and this effect
reduces to minimal activity towards the 3’ end of the transcript
(Losson and Lacroute 1979). Non-Stop Decay refers to the inverse of
NMD, no stop codon. Ribosomes that over-run into the poly-A tail
recruit degradation by the 3’ exonuclease (exo review). No-go Decay is
named after the phenomenon that triggers it, when ribosomes no go.
Difficult to translate sequences (lysine repeats) trigger the
endonucleolytic cleavage of the offending mRNA, which is then degraded
from the cut towards both ends (Doma and Parker 2006). Very recently,
it has been shown that it is likely the ribosome collisions that
promote the ribosome ubiquitination associated with the triggering of
No-Go decay (Simms et al. 2017), and other work has suggested that the
protein Asc1p may play a key role in this process (Ikeuchi and Inada
2016), perhaps via K63 ubiquitination (Saito et al. 2015). Together
these pathways surveil translating mRNAs for defects, but it is likely
that false positives in the recognition process also contribute to
their regulatory effects.  

Disruption of the NMD pathway is associated
with different expression of many transcripts. Recent genome-wide
analysis identifying ~900 mRNA upregulated upon deletion of any of
UPF1-3, and subsequent ribosome profiling found this targeting to be
associated with out-of-frame translation effects and non-optimal
codons (Celik et al. 2017). NMD has been implicated in the regulation
of ribosomal protein subunit pre-mRNA (Garre et al. 2013) in different
environmental conditions, has been shown to interact genetically with
Hrp1p and cis-elements spanning the start codon of PPR1 mRNA to target
this mRNA for degradation (Pierrat et al. 1993; Kebaara et al. 2003),
and may be triggered by upstream open reading frames (uORFs, discussed
later). These could be specific regulation, or aberrant probabilistic
activation due to the sensitivity of co-translational quality control
(Celik et al. 2017).  

The interaction of translation and mRNA
degradation Codon-optimality refers to the concept that certain codons
are translated by the ribosome more quickly than other codons. This is
thought to result in part from changes in tRNA abundance and in part
due to intrinsic differences in the decoding rates (Curran and Yarus
1989; Thomas et al. 1988), and often quantified using the tAI index
(dos Reis et al. 2004). The expectation that tRNA availability is
associated with increased rates of translation has been tested with
more recent ribosome footprint profiling experiments, and consistent
with this ribosomes tend to occupy optimal codons less often (Weinberg
et al. 2016).  

The functional relationship between codon-optimality
and mRNA degradation rate had been considered and rejected by a review
of single-transcript studies (Caponigro and Parker 1996). However,
with the advent of accurate genome-wide measurements of mRNA
degradation rates, we are able to explore the generality of this
principle in a relatively unbiased way. Several groups (Presnyak et
al. 2015; Neymotin et al. 2016; Harigaya and Parker 2016; Cheng et al.
2017) have found that poor codon-optimality and lower ribosome density
is associated with a higher degradation rate when considered on a
per-transcript basis. This can be explained through multiple models.
One model is that translation elongation rates are sensed, with slower
elongation accelerating the degradation of mRNA. Jeff Coller’s group
has worked extensively on Dhh1p, and found that it is genetically
required for the clear relationship between codon-optimality and mRNA
stability (Radhakrishnan et al. 2016; Presnyak et al. 2015). Although
the mechanism is at this point unclear, Dhh1p’s genetic association is
a fruitful hub to work out from.  

Alternatively, competition between
decapping enzymes and translation initiation factors for access to the
5’ m7G cap has long been proposed as a mechanism by which the two
processes interact (Schwartz and Parker 1999; Schwartz and Parker
2000). Karsten Weis’ group (Chan et al. 2017) reproduced the result
that slowing elongation with cycloheximide, sordarin, or 3AT treatment
slows mRNA degradation, but conversely inhibition of initiation with
hippuristanol or a dominant negative eIF4E increased degradation
rates. These measurements were made on the whole-transcriptome using
4-thiouracil and RNA sequencing, similar to RATEseq (Neymotin et al.
2014). The connection between the effect of elongation rates and
initiation rates could be explained by the known effect of slow
elongation rates inhibiting initiation events, as predicted (Shah et
al. 2013) and measured (Chu et al. 2014).  

Thus, much evidence points
to competition between translation initiation and 5’ to 3’ degradation
initiation at the cap as a major determinant of mRNA stability,
although the molecular work with Dhh1p suggests that events after
initiation still play a role. Other mRNA degradation pathways like NMD
or NGD during elongation (as discussed earlier) could also possibly
contribute to the effect.  

Regulation of mRNA degradation mRNA
degradation can be affected by various trans factors. While micro RNAs
are prolific in regulating mRNA in animals and plants, budding yeast
do not make use of this mechanism. Instead, in yeast mRNA degradation
appears to be determined by a combination of intrinsic properties like
length or codon-optimality, and trans factor RNA binding proteins
(RBPs) that can bind cis element sequences in the mRNA sequence to
effect changes in stability (cite? morris, cheng2017). The best
example of this is Puf3p, which binds motifs in mRNA with products
destined for mitochondrial function and degrades these in the
appropriate environment (Olivas 2000; Miller et al. 2013), perhaps by
mapping phosphorylation of Puf3p to association of these mRNA to
cytoplasmic granules (Lee and Tu 2015). Secondary structures may
complicate the recognition of linear cis elements, or be used as cis
elements in their own right (Li et al. 2010), for example Vts1p (Smaug
homolog) recognizes a small sequence motif in the context of the loop
of a stem-loop hairpin (She et al. 2017; Aviv et al. 2003).
Degradation rates can be affected by many mechanisms. Elements in
promoters (cis when in DNA but not part of the affected mRNA) can
“mark” transcripts for differential stability (Haimovich et al. 2013).
One of the most well known examples of this is Dbf2p loading onto SWI5
and CLB2 mRNA to effect destabilization upon mitosis (Trcek et al.
2011). In a direct example, the transcription activation domain of
Adr1p fused to a different DNA binding domain has been shown to be
sufficient to mark ADH2 mRNA for destabilization upon a glucose
upshift (Braun et al. 2015). Thus, RBPs may recognize sequence
elements in the mRNA or be loaded onto messenger ribonucleo-protein
complexes (RNPs) at synthesis (Gupta et al. 2016) to effect control of
mRNA stability in response to events in the cytoplasm.  

Non-RBP
mechanisms can also be used. Upstream open reading frames (uORFs) were
originally characterized using the phenotype of post-transcriptional
regulation of the Gcn2-regulated Gcn4p (Dever et al. 1992) in part
through quality control pathways (Ruiz-Echevarria and Peltz 1996).
Canonical and non-canonical start-codons can recruit initiation of
scanning ribosome subunits with a variety of effects on the
translation of the main coding sequence and the mRNA stability
(Spealman et al. 2017). Ribosomes may skip re-initiation at the
primary start codon to generate N-terminal diversity by initiating at
alternative start codons, or upon termination of a uORF very distant
from the 3’ end of the transcript trigger the NMD pathway to destroy
the mRNA (Dever et al. 2016). Additionally, modification of
nucleotides such as m6A can play a regulatory effect (citez).  

mRNA
localization within the cytoplasm may affect degradation by virtue of
regulating the accessibility of degradation factors. Processing-bodies
were originally described as cytoplasmic co-localized foci of 5’->3’
degradation factors that formed under stress induction conditions, and
on the basis of steady-state genetics and experiments with an MS2
aptamer-based live-imaging system, it was concluded that
processing-bodies are foci of active mRNA degradation (Sheth and
Parker 2003). These foci are usually studied by microscopy during
stresses, entry into stationary phase, and in the use of mutants in
degradation pathways, but recent advances in microscopy and nanoscopy
particle tracking have identified that these complexes are likely
condensations of previously-existing RNPs and depend on a network of
redundant interactions between mRNA 5’ to 3’ degradation protein
factors (Lui et al. 2014; Rao and Parker 2017). Additionally, recent
adjustments to the aforementioned MS2 aptamer system and explorations
during dynamic conditions point towards processing-bodies being sites
of degradation factor sequestration (Huch and Nissan 2017; Tutucci et
al. 2017). Interestingly, the formation of these processing-bodies are
halted upon cycloheximide treatment (Sheth and Parker 2003),
suggesting that translational status of the transcriptome and
processing-body composition may be related. In recent work, inhibition
of translation initiation was demonstrated to increase p-body
formation in correlation with increased degradation rates (Chan et al.
2017). Together, these observations indicate that processing-bodies
result from a complex balance of mRNA degradation initiation,
resolution, and mRNA degradation factor interactions with impacts on
the accessibility of degradation factors to mRNA targets of
degradation.  

The role of stability control in transcriptome
reprogramming The change in concentration of an mRNA ($R_t$) depends
on the rates of mature transcript synthesis ($k_s$) and mRNA
degradation ($k_d$). We assume that the cell volume is fixed, that
synthesis is a constant rate dependent on the unchanging concentration
of the DNA encoding the gene, and that degradation is a random first
order process of the mRNA interacting with a fixed and unsaturated
factor degradation. Thus, the change in mRNA over time is $$
\frac{dR_t}{dt} = k_s - R_t k_d$$ From this, the two rates determine
the steady-state equilibrium of $\frac{k_s}{k_d}$. Given a change in
these rates, a faster mRNA degradation rate will approach or relax to
the new equilibrium value quicker (Hargrove and Schmidt 1989), as the
doubling time (or half-life) of the mRNA is dependent on only the
degradation rate $\frac{log(2)}{k_d}$.  

While both synthesis and
degradation contribute to changes in abundance, changes in degradation
rates can cause the changes to occur more rapidly. If we expect that
the existence of a mechanism implies a selective pressure specifically
for it (Gould and Lewontin 1979), then we would expect that studying
an example of a transcript subject to both synthesis and degradation
regulation might reveal a balance of selection during steady-state and
dynamic conditions. 

\section{Stress conditions trigger rapid regulation of mRNA stability}

mRNA degradation rate changes have been characterized
to play a role in responses to heat-shock, osmotic stress, pH
increases, and oxidative stress, sharing a similar program of
destabilization of mRNA coding for ribosomal-biogenesis gene products
and stabilization of stress-responsive mRNA (Canadell et al. 2015;
Molina-Navarro et al. 2008; Shalem et al. 2011; Romero-Santacreu et
al. 2009; Molin et al. 2009; Castells-Roca et al. 2011; Miller et al.
2011; Garre et al. 2013). Simultaneous increases in both synthesis and
degradation rates of some of these mRNA are thought to serve to return
the transcriptome quickly to a new steady-state after effecting a
transient pulse of regulation (Shalem et al. 2008; Rabani et al.
2011), demonstrating a key functional role in stability control in
achieving a particular pattern of mRNA dynamics. Interestingly, these
stability changes appear to be a singular regulatory event
(Pérez-Ortín et al. 2013). Glucose deprivation stress also impacts
mRNA stability (Munchel et al. 2011). 

\subsection{Nutrient shifts also trigger mRNA stability changes }

In response to a carbon-source downshift
(glucose-grown cells resuspended in media with only galactose
available), functionally important regulatory changes in mRNA
stability occur (Munchel et al. 2011). Ribosome biogenesis associated
mRNAs are are destabilized, an effect that can be phenocopied by the
addition of rapamycin (inhibitor of the central growth signalling
TORC1 pathway) (Munchel et al. 2011). Conversely, a carbon source
upshift (galactose to glucose) triggers a destabilization of inducible
GAL genes, an effect that appeared to be restricted to the dynamic
condition as mRNA transgenically overexpressed in glucose media were
stable (Munchel et al. 2011).  

Global changes in transcription and
mRNA destabilization has been observed before (Jona et al. 2000), and
recently systematically measured to be correlated with changes in
growth rate (García-Martínez et al. 2016). The involvement of the
TORC1 pathway in this process has identified that its effect is
specified to differentially regulate the stability of certain
transcript sets (Albig and Decker 2001; Talarek et al. 2010).
Recently, a phosphoproteomics approach to studying signallng of the
AMPK homolog Snf1p during a carbon upshift identified a role in Xrn1p
phosphorylation in the specification by this factor (Braun et al.
2014). Thus, specific signalling pathways appear to effect large
changes in mRNA stability in response to different nutrient conditions
for growth. 

Relieving nutrient limitation with a glucose upshift has
been shown to mediate both stabilization of mRNA in the ribosomal
protein subunit regulon (Yin et al. 2003) and destabilization of
gluconeogenic transcripts (de la Cruz et al. 2002; Mercado et al.
1994; Scheffler et al. 1998; Lombardo et al. 1992). Mapping the
determinants of this effect has been met with mixed success. The
destabilization of SDH2 and GAL1 mRNA have been mapped to elements in
the 5’ UTR (Bennett et al. 2008) with destabilization of GAL1 being
associated with a growth advantage in switching carbon-source
environments (Baumgartner et al. 2011). JEN1 has been demonstrated to
be destabilized upon glucose addition, and this has been mapped to cis
elements of different transcription start sites which can act in trans
to regulate other co-expressed engineered alleles of JEN1 through
unknown mechanisms (Andrade et al. 2005) although subsequent work has
identified DHH1 as being a genetic factor of the destabilization (Mota
et al. 2014). Some transcripts respond at different levels of glucose
addition and differently in different genetically-perturbed metabolic
backgrounds (Yin et al. 2000), and disrupting signalling through the
PKA pathway affects destabilization of some mRNA but not others (Yin
et al. 2003). Thus, a systematic measurement of mRNA stability and a
broad determination of genetic factors of the transcript dynamics
would be useful for making progress at untangling the regulation of
mRNA stability in response to the increase in growth rate upon a
nutrient upshift.  

\section{What is the function of rapid txtome changes upon upshift?}

Up-regulation of mRNA abundance during an increase in growth
rate serves a clear functional purposes. The ribosomal protein (RP)
and ribosome biogenesis (RiBi) regulons are swiftly upregulated upon
repletion of nutrients to nutrient-limited cultures of yeast
(Jorgensen et al. 2004), and are well-correlated with growth rate in
both dynamic and steady-state conditions (does brauer or airoldi show
this?). Ribosomes are lower in slow growing conditions and need to be
upregulated upon resumption of rapid growth. (von der Haar estimates,
Tu, Hwa) (Estimated this is probably from 50k to about 200k.) The
relative allocation of gene expression resources in the cell is a
fundamentally important decision cells must make. Modeling of this
phenomenon across various conditions, mainly in E. coli as a model
system, as identified that a simple “pie-chart” model explains the
up-regulation of ribosome biogenesis relative to the rest of the
cellular investments well during steady-state.  

A recent study has
explored the theoretically best strategy during an increase in growth
rates, and found that a “bang-bang singular” strategy of complete
focus on generating gene expression machinery at the neglect of
metabolism machinery would be the optimal strategy for resuming growth
most rapidly (Giordano et al. 2016). With that, transcripts that are
stress responsive or important for metabolism in the old environment
are repressed upon a nutrient upshift.  

What does rapid transcriptional reprogramming achieve with respect to gene regulation?
In an integrative study of proteome and transcriptome dynamics from
the Gasch laboratory (Lee et al. 2011), the authors found that while
upregulation of mRNA did correlate with an increase in protein
abundance, the repression of mRNA did not correlate with a
downregulation of protein products on the timescales they measured.
This asymmetry makes sense, as with protein approximately 20 times
more stable than the mRNA they derive from (one of those reviews).
This suggests that accelerated mRNA degradation may serve a different
role. Others have suggested that degradation can help recycle
nucleotides (Kresnowati et al. 2006) or that reprogramming the
transcriptome would help to reallocate the extant translational
capacity of the cell to enact a growth-optimal program (Kief and
Warner 1981; Giordano et al. 2016; Shachrai et al. 2010). Identifying
the genetic factors responsible for the degradation would allow us to
test if the destabilization is adaptive, and if so to make progress in
understanding the mechanistic basis of this phenomenon.[e] 

\section{Measuring mRNA dynamics}

It is easier to measure the abundance of something than
it is to measure the change in abundance of something. While mRNA
abundance measurements for entire transcriptomes are now routine,
determining the rates that underlie this molecular phenotype has
lagged. Synthesis rate control has largely been assayed by techniques
like Genome Run On (GRO) sequencing (discussed below) to measure
transcription rates or measuring intron-exon ratios (pick one, NET
seq?) as a proxy for synthesis rates (Pérez-Ortín et al. 2013).
Degradation rate measurements have used a variety of methods, but are
now applied to the whole transcriptome with enough accuracy to enable
systematic modeling of the determinants of mRNA degradation rates
(Pérez-Ortín et al. 2013; Neymotin et al. 2016; Cheng et al. 2017).
Pioneering studies used pulse-chase experiments with radioactive
nucleotides to study turnover of the whole transcriptome, but were
unable to assay the process at the single-gene level (that one that
David sent me, and the one that Leon Chan sent me).  

To study
particular genes people have used promoters with inducible repression
characteristics to halt transcription. Transgenes like the
doxycycline-inducible TetOff (g something 1998?) use a heterologous
system. Researchers have also made use of the GAL1 promoter. Upon
addition of glucose, transcription of the GAL1 mRNA is immediately
halted. This property has been exploited to study mRNA stability in a
technically simple manner ( parker?, coller green ) and the promoter
has been adapted to higher throughput studies ( that bullshit short
half-life recent paper ). However, the last 100bp before the start
codon of GAL1 has been shown to be necessary for accelerated
degradation of this mRNA upon glucose addition (baumgarner, bennet,
hasty), challenging the interpretation of mRNA stability measured in
glucose-containing media with the GAL1 promoter fused directly
upstream of the start codon. While some researchers take care to limit
the glucose in the system (0.0?5\% in that bullshit paper) while using
heterologous binding sites, researchers have shown that glucose
concentrations of 0.0?1\% (56mM, check Ziv’s paper for making sure I’m
not screwing up a decimal place) have triggered instability of
gluconeogenic mRNA ( yin brown 2000 or 2003 ). Thus, while the GAL1
system is a convenient system for studying degradation intermediates (
parker and coller ), its use for studying the native stability of
different mRNA and in different environments is uncertain.  

A system
that does not rely on engineered cis-elements would avoid these issues
and scale to genome-wide assays, and thus two methods of
transcriptional inhibition were applied to study mRNA degradation
rates in landmark studies. A temperature sensitive rpb1-1 allele was
demonstrated to halt most Pol II transcription at non-permissive
temperatures, while the drugs thiolutin and 1,10-phenanthroline
inhibited polymerases including Pol II to mostly halt transcription.
These have been used widely, and are still used to this day. However,
it has been shown that used of thiolutin or 1,10-phenalanthroline
induces some heat-shock genes (Adams and Gross 1991), thiolutin
inhibits mRNA degradation in a dose-dependent way (Pelechano and
Pérez-Ortín 2008) (perhaps via inducing processing-body formation
(Huch et al. 2016)), and eliminating the essential RNA Pol II complex
from the nucleus has complex effects on the transcriptome dynamics (Yu
et al. 2016). While it may seem logical that studies of mRNA
associated with processes distinct from heat-shocks may be unaffected
by these, the complexity of the cell demonstrated itself in vital
controls run by (Mercado et al. 1994) which demonstrated that
gluconeogenic mRNA were subject to destabilization upon a
heat-shock[f]. Shutting off transcription has complex and difficult to
predict effects on transcript abundance as the cells die over the
course of the experiment.  

Orthogonal to these approaches is GRO-seq
(Genomic Run On)[g]. This method uses a sarkosyl treatment (and
salt?)[h] to fix RNA Pol II complexes onto genomic DNA by freezing
their elongation. Extraction and a defined in-vitro polymerase
extension by reversing the fixation before profiling the resulting
mRNA with microarrays or RNAseq allows for an estimate of the
instantaneous transcription rate status for each gene in a population
of cells. Interpretation of these numbers must be considered in the
context of the in-vitro environment of the elongation step, but this
method serves as an valuable orthogonal measure of transcript dynamics
--- and an instantaneous one.  

The development of 4-thiouracil
metabolic labeling of RNA (Dölken et al. 2008) has enabled a return to
the pulse-chase methodology in the development of genome-wide assays
of mRNA dynamics. Fundamentally, these assays work by changing the
labelling frequency of mRNA and tracking the dynamics as the labeled
mRNA abundance relaxes towards that new equilibrium. Below I review
the basis of these assays, then focus on their applications and where
the dissertation work is placed.  

If we consider mRNA abundance at a
certain time as being denoted as $M_t$, then I expect this number to
change as a zeroth order rate of synthesis per time ($k_s$) and a
first order rate of degradation per mRNA ($k_d$). While mRNA
degradation is a multi-step process (above) and more complex models
may identify nuances in the rates of progression through these
intermediates (Deneke et al. 2013), at steady-state the rate of
degradation of mRNA in a population should be well modeled by a single
first order rate ( ?can I find the cite for that? ).
$$\frac{dM_t}{dt} = k_s - M_t*k_d$$ Introducing a term $L$ that
denotes the fraction of newly synthesized mRNA that are labeled and
measured after purifying mRNA for the labeled mRNA (thus $M_t$ is just
labeled and captured mRNA), we can now model the changes as simply
$$\frac{dM_t}{dt} = L k_s - M_t*k_d$$ I introduce the superscript
notation of $L^o$ for the old labeling frequency and $L^n$ for the new
labeling frequency, and solving for the change of $M_t$ from some
steady-state equilibrium $L^o \frac{k_s^o}{k_d^o}$ to a new
equilibrium $L^n \frac{k_s^n}{k_d^n}$, and rearranging terms we get
$$M_t =  L^o \frac{k_s^o}{k_d^o} e^{-k_d^n t} + L^n \frac{k_s^n}{k_d^n} ( 1- e^{-k_d^n t} )$$ 
This matches well with our
intuition. On the right, the nascent transcripts are labeled at the
new rate and approach this new equilibrium controlled by the term $(
1- e^{-k_d^n t} )$, while on the left the extant transcripts approach
zero in an exponential decrease controlled by the term $e^{-k_d^n t}$.
These are both controlled by $k_d^n$, or the rate of mRNA degradation
after chasing the label. Thus, by measuring the transition between the
equilibrium we get the mRNA degradation rate, assuming that synthesis
rates do not change.  

Measuring specific rates with high confidence
requires a steady-state approximation. RATEseq is one method to do so,
using many timepoints to accurately model the approach of labeled mRNA
abundance to a new equilibrium (Neymotin et al. 2014). This
experimental design is theoretically the most accurate, although it
requires the assumption that the total mRNA (labeled + unlabeled) is
indeed at a steady-state abundance. Dynamic Transcriptome Analysis
(Miller et al. 2011) violates this assumption to explore changes in
degradation rates during 6 minute windows. While they sacrifice
high-confidence of an exact rate, the temporal resolution of stability
changes during osmotic stress has revealed an unprecedented dynamic
view of the regulation of mRNA dynamics during complex processes. This
approach requires that 4-thiouracil transport and incorporation into
nucleotide metabolism occurs during the course of the perturbation
experiment, but with the right measurements, normalization, and
integration with other datasets an accurate and dynamic picture of
transcriptome dynamics can be built. To assess mRNA stability changes
during dynamic processes, one can also label the transcriptome to
equilibrium and then chase out the label by adding an excess of
unlabeled nucleotides. This approach was used by researchers in the
Weis group to demonstrate changes in the stability of groups of mRNA
in response to environmental changes, namely shifts in carbon sources
and with rapamycin treatment inhibiting TORC1 (Munchel et al. 2011).
They found that the RP (ribosomal protein) regulon was destabilized
upon induction of nutrient starvation, demonstrating that mRNA
degradation is under tight regulation from nutrient-sensing pathways.
In Chapter 3 I demonstrate the use of a similar label-chase
experimental design, with with refined analysis to explore
single-transcript destabilization upon a nitrogen upshift.  

\section{Methods for determining the genetic basis of a transcript
dynamics phenotype}

mRNA is an intermediate in the expression of a protein product, and
has the key virtue of being much easier to measure than to measure
abundance of the protein product. This became especially true with the
advent of massively-parallelized DNA sequencers and the methods to
accurately convert transcriptomes to DNA libraries, ie RNAseq
(Shendure et al. 2017). For this reason, it is often used as a proxy
of gene expression at the protein level. Although the relationship is
strong when correcting for experimental noise (Csárdi et al. 2015),
the quantitative functional nature of this relationship within a
particular gene in different environments depends on the particular
gene in question (Franks et al. 2017). It is also clear that
transcriptomic and proteomic responses greatly vary in the timescales
of effect, with the transcriptome subject to rapid impulses of
changing abundance that may or may not result in longer term
regulation of the protein product (Cheng et al. 2016; Lee et al.
2011). Even then, protein abundance in a cell does not correspond
perfectly to its activity, be that regulated allosterically or by
localization.  

Given this disparity, what can we learn about adaptive
gene expression from mRNA abundance regulation? First, the expression
of a gene product requires mRNA, thus the binary expression of mRNA is
a predictor of the possibility of protein expression. Additionally,
cellular processes often impinge upon changes in mRNA abundance, be
they direct via regulation of abundance, activity, and localization of
activity of specific effectors or by indirect effects on common gene
expression machinery or cellular metabolism. In this way, a specific
perturbation of a signalling pathway is expected to broadcast to
changes in mRNA abundance. Quantification of the thousands of mRNA
that are expressed in a cell is a sensitively quantitative measurement
on thousands of dimensions, and can thus be used as a
relatively-unbiased indicator of cellular status with which to explore
the genetic requirements of particular signalling perturbations (that
mol sys bio paper). Thus, efficient methods to explore the genetic
basis of transcript dynamics upon a perturbation should help to
accelerate the study of cellular signalling pathways.  

Genetic screens
in yeast have been a powerful tool to narrow down the immense search
space of possibilities to a narrow set of hypotheses about a
biological process. Classically, these function by mapping some
phenotype of interest to a change in growth rate. For example, mutants
in transporters of a particular amino-acid can be isolated by feeding
the cells a toxic stereoisomer (like D-histidine). A more complex
method in Lee Hartwell’s classic screen for cell-cycle mutants used
the assay of growth at a low temperature and cessation of growth at a
high temperature to identify mutants in critically important pathways
(hartwell 1970), work that contributed to a 2001 Nobel Prize for
advancing our understanding of the cell cycle. However, this concept
becomes problematic when studying a molecular phenotype which is not
known to be adaptive. 

For example, gene regulation might not have a
clear phenotypic outcome or be subject to redundant layers of
regulation that mask the effect of a mutation. One solution is to
engineer a specific reporter into the expressed gene, such that
defects in gene expression can be assayed. It becomes more difficult
if the phenotype is a transient one, such that a reporter through
growth rate (perhaps a toxic peptide) does not have time to accumulate
the signal of growth. A fluorescent tag is one approach that bypasses
this requirement, as cells can be instantaneously assayed for the
level of GFP fluorescence through methods like flow cytometry. The GFP
can be fused to the protein of interest or simply placed downstream of
an appropriate reporter, such as the strategy employed by (Neklesa and
Davis 2009). They were able to use a DAL80 promoter upstream of a GFP
reporter to explore the genetic requirements for the NCR-regulated
expression of this promoter, discovering the SEACIT complex components
Npr2/3p upstream of TORC1. However, this method requires that the gene
expression phenotype be regulated at both the transcript synthesis
level and be relevant at the level of protein expression. Additionally
it requires that the GFP tag be a relevant and faithful reporter of
the protein abundance, a condition which is not always satisfied given
the stability of the GFP tag in the vacuole (find that cite).
Previous work to do genetics of transcript dynamics neklesa - sortseq,
reporters Mention problems of GFP stability worley - robotics exotic
methods, like COE, so antibodies scRNAseq I think tavazoie’s stuff
Fluorescent readout of mRNA degradation mechanisms at the protein
level. (Aviv et al. 2003) GAP1 mRNA degradation, which we identify as
being subject to accelerated mRNA degradation in Chapter 3 and 4,
occurs much faster than the repression of the protein-product. We also
know that Gap1p, the protein product of GAP1 mRNA, is subject to
de-activation and re-localization in response to a nitrogen upshift.
Thus, a functional assay of Gap1p is irrelevant to the dynamics of
GAP1 mRNA repression, and requires a novel method to screen for
genetic factors of this molecular phenotype.  

Ambitious work from the
Capaldi group developed a workflow using extensive automation to
perform qPCR assays for NSR1 mRNA abundance 19 minutes after induction
of an osmotic stress response (Worley et al. 2015). While accurate and
reproducible, the extensive automation and reagent usage to perform
qPCR on ~4700 mutant strains poses a financial and logistical
challenge to performing the assay, and to perform the assay in
different genetic backgrounds, in larger libraries, in replicates, or
in different timepoints.  

I composed several existing technologies to
develop a direct assay of transcript abundance in a high-throughput
pooled format. This is discussed in depth in Chapter 3. 

