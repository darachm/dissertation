\chapter{Introduction}
\label{chapter:one}

Organisms adapt to their environment 
by expressing different phenotypes 
as environments change. 
We expect this to be an advantageous strategy, 
depending on the fitness consequences of 
each phenotype in each environment 
balanced against the costs of innovating and maintaining
the machinery for adaptive differential gene expression
\parencite{kussell2005phenotypic}. 
One way to achieve this is through the regulation of gene expression.
If different mechanisms of gene expression act on different properties
of gene expression, or have different costs of energy or complexity, 
then we would expect that as the environment selects for the use of 
gene expression patterns that advantageously model the environment
\parencite{tagkopoulos2008predictive}, we may see the use of different
mechanisms used to achieve different regulatory demands.
Thus, understanding how different mechanisms can be used to
regulate one particular level of gene expression may inform our 
understanding of these mechanisms and how multiple levels of selection 
balance in causing an adaptive outcome.

Expression of a protein-coding gene
product involves many steps, 
each with a variety of regulatory mechanisms. At the outset, DNA sequences
that encode genetic elements are transcribed into corresponding 
messenger RNA (mRNA). The rate of this transcription is helped by
factors that facilitate recruitment of RNA polymerase II (PolII)
and is hindered by factors that block this process by physical
occlusion or changes in the accessibility to the chromatin 
\parencite{hahn2011transcriptional}. 
The translation of mRNA into a protein product by
ribosomes occurs at different rates for different genes and different
environments, and this approximately 15-fold variation 
\parencite{weinberg2016improved} is thought to be 
regulated by a complex interplay between ribosomes and
associated translation factors, RNA binding proteins (RBPs), and
intrinsic factors of the mRNA like length or codon-usage 
\parencite{dever2016mechanism}.
For both the mRNA and its protein product, stability is also
important \parencite{perez2013eukaryotic,mcmanus2015next}. 
Additionally, localization or allosteric regulation can change
the activity of a gene product. Myriad factors contribute to the
expression of a gene product, and determining the functional adaptive
basis for particular regulatory mechanisms, if they are indeed
adaptive, would help us better understand the diversity of gene
regulatory mechanisms.

The budding yeast \textit{Saccharomyces cerevisiae} is a
model system for many fields, including gene
expression regulation. In response to different nutrient 
availabilities yeast changes its rate of growth, and
this is accompanied by large changes in gene expression 
\parencite{brauer2008coordination,conway2012glucose}
and changes in physiology 
\parencite{waldron1977synthesis,carter1978protein}
including changed rate of proliferation \parencite{slator1918some}, 
cell size \parencite{jorgensen2004dynamic},
RNA content \parencite{waldron1975effect}, 
protein content \parencite{kief1981coordinate},
and resistance to stressors \parencite{elliott1993stress}.
This integration of diverse metabolic signals into a coordinated 
program of growth is thought take place with multiple hierarchies of 
control in response to different particular 
types of nutrients as well as the general availability of any 
nutrient source 
\parencite{winderickx2003feast,broach2012nutritional,cooper1982nitrogen}.
These signals are integrated into a similar systemic output by well conserved growth 
signalling pathways of eukaryotic cells, like TORC1, Snf1p/AMPK, 
or PKA \parencite{conrad2014nutrient,thevelein1994signal}.
Thus, the study of nutrient limitations to growth, especially with 
the aim of discerning particular molecular mechanisms and how these may overlap
\parencite{oliveira2015dynamic,oliveira2015inferring,tate2017general,stracka2014nitrogen,peli2015amino} 
would shed light on our understanding of how a eukaryotic cell 
regulates its most essential goal of growth 
in the face of diverse limitations.

The particular nutrient availability is described in terms of the 
quality and quantity of the source provided. Quantity refers to
the molar availability of the nutrient that the yeast can take up,
while quality is an empirical reference to how rapidly budding
yeast can biochemically incorporate the nutrient into their metabolism
and grow.
One prediction from this understanding is that
altering the quantity of the nutrient availability to vary growth
rates in a range below which the quality limits growth rates will
elicit a common response between various nutrient limitations.
Indeed, studies systematically varying nutrient environments have
shown that about a quarter of the transcriptome is differentially
expressed at different steady-state nutrient-limited growth states,
regardless of nutrient used to limit growth 
\parencite{brauer2008coordination,regenberg2006growth}, and that a
small subset of transcripts were regulated particular to the specific
limitation. 
Statistical modeling of this process
determined that the molecular signature of this growth-rate signalling
could be captured in a small number of “calibrator” genes whose
expression was very well correlated with changes in growth rate or
perturbation of signalling pathways associated with this process, and
importantly also changed during dynamic transitions or upon
perturbation of growth signalling pathway PKA 
\parencite{airoldi2009predicting}.
Dynamic transitions to better nutrient environments (nitrogen, carbon,
and phosphorus upshifts) shared a similar pattern 
\parencite{conway2012glucose}, 
and the pattern of gene expression associated with increased
nutrient availability and growth rates is the opposite of the
Environmental Stress Response (ESR) --- a shared in co-regulation of
$\sim$600 mRNA across dynamic responses to various stressors 
\parencite{gasch2000genomic}. 
Together, this shows how yeast has a common molecular response that
largely corresponds to the suitability of the sensed environment,
in addition to the physiological changes previously described.
A better environment translates to faster growth, with
more growth associated mRNA and less stress response mRNA, and this
holds true in different steady-states and in the dynamics of 
transitions between steady-states.


\section{Functional transcriptome reprogramming during a
nitrogen upshift}

One classically studied transition between growth rates is the 
nitrogen upshift. Yeast grows quickly when provided with nitrogen 
sources like glutamine or ammonium, but can make use of 
various nitrogen sources like proline or urea by 
expressing overlapping sets of specific and general nitrogen-source 
permeases that concentrate these sources inside the cell for use
\parencite{grenson1992amino}.
Various
nitrogen sources are then catabolized to eventually make glutamate and
glutamine, with an estimated $\sim$85\% of macromolecular nitrogen coming
from the amino nitrogen in glutamate and the rest from the side-chain
of glutamine 
\parencite{magasanik2002nitrogen}. The addition of glutamine to
a nitrogen-limited culture, for example grown with only the
non-preferred proline as a nitrogen source, is called an upshift
because it is the change from a slow growing condition to one of rapid
growth \parencite{kjeldgaard1958transition,waldron1977synthesis}. 
Upon an upshift, a regulatory phenomenon called nitrogen
catabolite repression (NCR) ensures that the set of transporters,
metabolic enzymes, and regulatory factors are repressed
\parencite{cooper1982nitrogen,cooper2002transmitting,magasanik2002nitrogen}. 
Through the use of a temperature-sensitive glutamine synthase
allele or treatment with methionine sulfoximine, it has been shown
that NCR appears to respond to intracellular glutamine availability 
\parencite{grenson1983inactivation,stracka2014nitrogen,crespo2002tor}.
However, recent work has specified that while the transient NCR
correlates with glutamine abundance, persistent NCR appears independent
\parencite{fayyad2016yeast}. Additionally, the use of a
\textit{gln1-37} mutant to prevent ammonium from glutamine conversion
found that ammonium addition triggered swift repression of
\textit{GAP1} and \textit{PUT4} despite not changing intracellular
glutamine levels \parencite{ter1998repression}. 
This suggests that the sensing might not be
simply glutamine abundance, but rather some sensor of flux through 
central nitrogen metabolism (like Gdh1p activity
\parencite{fayyad2016yeast}) or some other,
heretofore uncharacterized signalling mechanism.
Whatever the exact mechanism, the degree to which nitrogen sources
can support growth by providing substrates of central nitrogen
metabolism is sensed by yeast to appropriately and quantitatively
repress catabolic genes in conditions corresponding to the degree
to which they are unneeded for growth.

\vspace{1em}
\afig{
  \begin{tikzpicture}[scale=1.2
    ,every node/.style={scale=1.2}
    ,>={Latex[length=3mm,width=3mm]}
    ,textz/.style={font=\scriptsize,align=left}]
%    \node[textz] at (0,0) (glucose){glucose};
%    \node[textz] at (0,-1) (pyruvate){pyruvate};
%    \node[textz] at (-.5,-1.5) (acetyl CoA){};
    \node[textz] at (-2,-2.8) (oxaloacetate){};
    \node[textz] at (0.5,-3) (citrate){};
    \node[textz] at (1,-3) (akg){$\alpha$-ketoglutarate};
    \node[textz,right=1cm of akg] (glu) {glutamate};
    \node[textz,right=1cm of glu,fill=green!60] (gln) {glutamine};
    \node[textz,above right=1.5cm and -0.3cm of akg] (nh41) {NH4$^+$};
    \node[textz,above right=1.5cm and -0.3cm of glu] (nh42) {NH4$^+$};
    \node[textz,below left=1.5cm and 0.3cm of glu] (nh43) {NH4$^+$};
    \node[textz,below left=1.5cm and -2.0cm of gln] (akg2){$\alpha$-ketoglutarate};
    \node[textz,above=2cm of glu,fill=red!60] (pro) {proline};
    \node[textz,below=2cm of glu] (m1) {};
    \node[textz,below=2cm of gln] (m2) {};
%    \draw[->,shorten >= 2pt] (glucose) to (pyruvate);
%    \draw[->,shorten >= 2pt] (pyruvate) to (acetyl CoA);
%    \draw[->] (acetyl CoA) to[out=-70,in=150] (akg);
    \draw[->] (oxaloacetate) to[out=45,in=135] 
      %node[textz,pos=0.0,below,align=left] {citric-acid\\cycle} 
      (akg);
    \draw[->] (akg) to[out=-135,in=-45] 
      node[textz,pos=0.5,below,align=left] {citric-acid\\cycle}
      (oxaloacetate);
%
    \draw[->] (akg) to[out=45,in=135] (glu);
    \draw[->] (nh41) to[out=-90,in=135] (glu);
    \draw[->] (glu) to[out=-135,in=-45] (akg);
    \draw[->] (glu) to[out=-135,in=90] (nh43);
    \draw[->] (akg2) to[out=090,in=-45] (glu);
%
    \draw[->] (glu) to[out=45,in=135] (gln);
    \draw[->] (nh42) to[out=-90,in=135] (gln);
    \draw[->] (gln) to[out=-135,in=-45] (glu);
%
    \draw[->] (pro) to (glu);
%    \draw[->] (glu) to (m1);
%    \draw[->] (gln) to (m2);
  \end{tikzpicture}
  }{
    Glutamine (green) is a central metabolite, and is thus a preferred 
    nitrogen source for rapid growth. Proline (red) requires specific
    transport and metabolic enzymes to convert it to glutamate.
    Redrawn from \cite{magasanik2002nitrogen}.
  }{Core nitrogen metabolism in yeast.}

One layer of
the repression occurs at the level of transcript synthesis. Four of
the five GATA factors in yeast coordinate to control transcription of
NCR genes, with two factors (Gln3p and Gat1p) activating transcription
while two (Gzf3p and Dal80p) repress transcription 
\parencite{hahn2011transcriptional,stanbrough1995transcriptional,daugherty1993regulatory,scherens2006identification}. 
These factors are also subject to NCR
control to different extents, with the activators increasing the
expression of the repressive factors 
\parencite{cunningham2000nitrogen}. 
This is thought to be an
adaptation to enable quick repression upon a nitrogen upshift, as may
be encountered when yeast is introduced to a new abundant nutrient
environment of grape or wort. It has been long known that the
eukaryotic growth signalling pathway TORC1 largely regulates these
factors by controlling the activity of phosphatases and thus
localization of these transcription factors, via Ure2p for Gln3p 
\parencite{beck1999tor,cox2000saccharomyces} 
and unknown mechanisms for Gat1p
\parencite{georis2008tor}. 
However, the careful application of genetics has identified that the
requirements for phenotypes differ in different environments, with
comparisons of nitrogen “starvation” (8+ hours) versus “limitation”
(<3 hours, or proline) showing that about half of the
Gln3-localization regulation (resulting in transcriptional regulation
of NCR) was still unexplained by TORC1 signalling alone
\parencite{tate2013five}. By way
of a temperature-sensitive tRNA allele, researchers have since identified
that Gcn2p impinges in a parallel pathway through the 14-3-3 proteins
Bmh1/2 to promote the export of Gln3p and Gat1p 
\parencite{tate2015gata,tate2017general}. 
Additionally, others have suggested that the
amino-acid permease Gap1p may directly signal to PKA
\parencite{donaton2003gap1,van2009transport}.
Thus, multiple growth signalling pathways converge to
affect the import and export of NCR GATA factors to effect multiply
redundant layers of NCR transcript synthesis control.  

Gene product
regulation can also occur post-translationally. NCR has primarily
referred to the control of transcript synthesis rates, but it has been
long observed that upon addition of a preferred nitrogen source the
enzymatic and permease activities are repressed faster than can be
caused by a shut-off of synthesis 
\parencite{cooper1983function}. A
classical NCR-regulated gene is the general amino-acid permease GAP1.
GAP1 mRNA is repressed much faster than the repression of the
protein-product 
\parencite{stanbrough1995transcriptional}, and we know that this
Gap1p shut-off is adaptive \parencite{risinger2006activity}, 
perhaps due to an
excess of amino-acid transport causing ammonia toxicity 
\parencite{hess2006ammonium}
or excess proton symport driving a depolarization against futile
Pma1p proton-export activity. This growth phenotype allowed the early
identification of mutants in this process, and this indicates that it
is mediated by a uniquitinyation mark that inactivates the permease
and leads to relocalization and degradation 
\parencite{grenson1983inactivation,risinger2008different,merhi2012internal}. 
Thus multiple layers redundantly repress the NCR-regulated Gap1p.  

In Chapters 2 and 3, I show how mRNA
degradation also plays a role in this repression, inactivating some
NCR mRNA as well as mRNA associated with other metabolic processes.

\section{What is the function of rapid transcriptional repression
during an increase in growth rate?}

A landmark integrative study of proteome and transcriptome dynamics 
\parencite{lee2011dynamic} showed that
for most cases of mRNA repression upon osmotic stress, there was not
a correlated downregulation of protein products in the same timescale.
This asymmetry makes sense, with protein gene-products being 
approximately 30-50 times more stable than the mRNA intermediate 
\parencite{christiano2014global}. 
Given the assumption that adaptation implies function, what purpose
might this repression fulfill?

An increase in growth rate is associated with a rapid
up-regulation of the ribosomal protein (RP)
and ribosome biogenesis (RiBi) regulons
\parencite{griffioen1996ribosomal,jorgensen2004dynamic}. 
These regulons comprise the protein subunits and biogenesis factors
responsible for ribosome biogenesis, and their relative abudnance 
is well-correlated with 
growth rate in both dynamic and steady-state conditions 
\parencite{brauer2008coordination,airoldi2009predicting}.
While some of the more numerous macro-molecules in the cell, 
ribosomes are not infinite and the majority are likely engaged in 
peptide elongation
\parencite{shah2013rate,von2008quantitative,boehlke1975cellular},
and are less abundant 
\parencite{kief1981coordinate,powers1999regulation}
and with a lower rate of overall translation
\parencite{waldron1977evidence}
in slow-growth conditions.

The relative allocation of gene expression resources in the cell is a
fundamentally important decision cells must make, and modeling
of this phenomenon across various conditions in \textit{E. coli} has
led to a simple partitioning model in which the proteome can be
divided into functional sectors
\parencite{scott2010interdependence,scott2014emergence}.
While these simple models might explain the balance during
steady-state growth, how does the relative allocation of gene
expression machinery change during transitions?
In particular, what is the best approach to re-balancing this
allocation upon the resumption of rapid growth
\parencite{erickson2017global}?
%\parencite{metzl2017?}?

Recent work has explored this to identify that an optimal strategy
would be to focus gene expression machinery on expressing more gene 
expression machinery, at the neglect of an investment in metabolic
enzymes \parencite{giordano2016dynamical}. 
This phenomenon, a transient burst of ribosomal over production or a
"bang-bang singular" strategy appears to have been observed in
yeast before in upshifts 
\parencite{wehr1969macromolecular,griffioen1996ribosomal} 
and recently by our lab in nitrogen upshifts 
\parencite{airoldi2016steady}.
Thus, concordant repression of stress-response and
metabolism gene expression is theoretically expected to allow more
focus of gene expression machinery on this pulse of production.
There is evidence for this in \textit{E. coli}, where
\cite{shachrai2010cost} induced the expression of a fluorophore at
different growth stages to show that induction during lag phase 
had a significant impact on lag duration, while induction during
exponential phase did not have a significant effect on growth.
This is also true in yeast, where misexpression of transgenic 
fluorophores has a cost during this period of advantageous focus on 
producing gene expression machinery \parencite{kafri2016cost},
although this may be mediated by the maintenance of additional
stores of under-utilized ribosomes
\parencite{metzl2017principles,waldron1977evidence}.
Thus, the repression of newly unneeded mRNA in yeast may serve a 
role to reallocate the extant translational
capacity of the cell to enact a growth-optimal program 
\parencite{kief1981coordinate}. 
Others have suggested that swift repression may instead
help to recycle nucleotides \parencite{kresnowati2006transcriptome},
so identifying the genetic factors responsible for the repression 
would allow us to test if a particular regulatory event,
perhaps a destabilization of mRNA, is indeed adaptive and by which
mechanism.

\section{mRNA degradation and its regulation}

Even when considering only the regulation of mRNA abundance, 
there are at least two processes that contribute --- that
of synthesis and degradation. We know much about transcript synthesis,
perhaps owing to the fact that virtually all events of mRNA synthesis
pass through a well-characterized reaction of synthesis by RNA Pol II,
capping and polyadenylation, and export into the cytoplasm. The
details may vary, but the common pathway is the same. mRNA
degradation does have a main pathway that performs the bulk of mRNA
degradation, but mRNA are also subject to divergent redundant pathways
that have been challenging to measure. Moreover, the rates of these
various processes are subject to control in ways less well-understood.
While some similarity is thought to exist in how RBPs may recognize
\textit{cis}-element sequences in RNA similar to how TFs recognize 
upstream
activating or repressing sequences in DNA, the single-stranded
nature of mRNA complicates this process with that ability to form
diverse secondary structures that can block linear 
\textit{cis}-elements \parencite{li2010predicting}.
Additionally, these secondary
structures of RNA may be recognized as the \textit{cis}-element
\parencite{aviv2003rna,she2017comprehensive}, 
complicating our approaches to recognize these patterns
\parencite{goodarzi2012systematic}.

\subsection{Primary 5' to 3' pathway of mRNA degradation}

The canonical protein-coding mRNA
is synthesized in the nucleus from a DNA template by RNA PolII, and
is capped co-transcriptionally at the 5' end with a m7G cap. As PolII
transcribes sequence 3' of the stop codon the cleavage factor complex
(of which Hrp1p is a sub-unit \parencite{chen1998specific})
recognizes cis element binding motifs in the RNA to direct cleavage
and polyadenylation to specific sites in the mRNA. Upon successful
completion of this process, the nascent mRNA is exported to the
cytoplasm where it enters into the pool of translatable mRNA.
Typically, translation begins when initiation factors load ribosomal
subunits to scan the 5' leader or untranslated region (UTR) for the
start codon, where the process of coding sequence translation begins.
\afig{
  \begin{tikzpicture}[scale=1.0
    ,every node/.style={scale=1.0}
    ,>={Latex[length=3mm,width=3mm]}
%    ,snakey/.style={decorate,decoration={snake,segment length=0.75cm}}
    ,textz/.style={ font=\scriptsize,align=left }
    ]
%
    \draw[very thick] (-6,2) -- node[textz,above,pos=0.9] (dna) {genome} (6,2);
%    \draw[thick,color=red] (-2,-1) .. controls (-1.75,-1.2) ..  (-1.5,-1);
    \draw[thick,color=red] (-1.5,1) .. controls (-1.25,1.2) ..  (-1,1);
    \draw[thick,color=red] (-1,1) .. controls (-.75,0.8) and (-.75,1.2) .. (-0.5,1.5);
    \draw[thick,color=red] (-0.5,1.5) .. controls (0,2) .. (1,2);
    \node[anchor=east,font=\tiny] at (-1.5,1.0) {m7G};
%
    \node[textz,anchor=west] at (0,1.5) {transcription};
%
    \draw[thick,->] (0,1.00) -- (0,-1.75);
    \node[textz,anchor=west] at (0,0) {cleavage, polyadenylation,\\export to cytoplasm};
% First translating one
    \draw[thick,color=red] (-2,-2.0) -- (1,-2.0);
    \node[anchor=east,font=\tiny] at (-2.0,-2.0) {m7G};
    \node[anchor=west,font=\tiny] at (1,-2.0) {AAAAAAAAAAAAAAAAA};
    \node[circle,fill=yellow!80!black,textz] at (2,-1.5) {Pab1p};
    \node[circle,fill=yellow!80!black,textz] at (3.5,-1.5) {Pab1p};
    \node[anchor=east,font=\tiny,circle,fill=yellow] at
      (-2.8,-2.0) (eif4f1) {eIF4F};
    \draw[fill=blue,color=blue!50] (-1.4,-2.050) ellipse (0.16cm and 0.08cm);
    \draw[fill=blue,color=blue!50] (-1.4,-1.900) ellipse (0.16cm and 0.16cm);
    \draw[->] (eif4f1) to[out=45,in=135] node[above,pos=0.5,textz]
      {loads ribosome} (-1.6,-1.8);
% Second one
    \draw[thick,color=red] (-2,-4.0) -- (1,-4.0);
    \node[anchor=east,font=\tiny] at (-2.0,-4.0) {m7G};
    \node[anchor=west,font=\tiny] at (1,-4.0) {AAAAAAAAAAAA};
    \node[circle,fill=yellow!80!black,textz] at (2,-3.5) {Pab1p};
    \node[anchor=east,font=\tiny,circle,fill=yellow] at
      (-2.8,-4.0) (eif4f2) {eIF4F};
%
    \draw[thick,->] (0,-2.50) -- (0,-3.5);
    \node[textz,anchor=east] at (0,-3.0) {continued translation};
    \draw[fill=blue,color=blue!50] (0.4,-4.050) ellipse (0.16cm and 0.08cm);
    \draw[fill=blue,color=blue!50] (0.4,-3.900) ellipse (0.16cm and 0.16cm);
    \draw[fill=blue,color=blue!50] (-0.8,-4.050) ellipse (0.16cm and 0.08cm);
    \draw[fill=blue,color=blue!50] (-0.8,-3.900) ellipse (0.16cm and 0.16cm);
% NMD
    \draw[thick,->] (0,-0.50) .. controls (5,-0.5) and (6.5,-1.0) .. 
      node[above,pos=0.7,textz,sloped] {error in transcription} (6.5,-2.5);
%
    \draw[thick,color=red] (5,-3.0) -- (8,-3.0);
    \node[anchor=east,font=\tiny] at (5.0,-3.0) {m7G};
    \node[anchor=west,font=\tiny] at (8,-3.0) {AAAAAA};
    \draw[fill=blue!50,color=blue!50] (5.8,-3.050) ellipse (0.16cm and 0.08cm);
    \draw[fill=blue!50,color=blue!50] (5.8,-2.900) ellipse (0.16cm and 0.16cm);
    \node[fill=red,textz,circle] at (6.5,-3.000) (stop) 
      {\scriptsize STOP};
    \draw[-|] (stop) to[out=-90,in=-90] 
      node[midway,below,textz] {decapping} ($(stop)+(-2,-0.3)$);
% Final mRNA for deg
    \draw[thick,color=red] (-2,-6.0) -- (1,-6.0);
    \node[anchor=east,font=\tiny] at (-2.0,-6.0) {m7G};
    \node[anchor=west,font=\tiny] at (1,-6.0) {AAAAAA};
    \node[anchor=east,font=\tiny,circle,fill=yellow] at
      (-2.8,-6.0) (eif4f2) {eIF4F};
    \draw[fill=blue,color=blue!50] (-0.4,-6.050) ellipse (0.16cm and 0.08cm);
    \draw[fill=blue,color=blue!50] (-0.4,-5.900) ellipse (0.16cm and 0.16cm);
%
    \draw[thick,->] (0,-4.50) -- (0,-5.5);
    \node[textz,anchor=east] at (0,-5.0) {continued translation};
%
    \draw[thick,->] (6,-5.0) to[out=-90,in=90] (6,-8.0);
%
    \draw[thick,->] (0,-6.5) to[out=-90,in=90] (0,-8.0);
%
    \coordinate (decap) at (-1,-10.1);
    \draw[color=red,thick] ($(decap)+(210:1.0)$) arc (210:-30:1.5);
    \node[textz,rotate=140,anchor=east] at 
      ($(decap)+(210:1.3)$) (cap) {m7G};
    \node[textz,rotate=40,anchor=east] at 
      ($(decap)+(-30:1.5)$) {AAAAA};
    \node[circle,fill=green!80,textz] at 
      ($(decap)+(-40:1.5)$) (lsm) {Lsm1-7p};
    \node[circle,fill=white!80!black,textz
      ,below right=0cm and 0cm of lsm] (pat) {Pat1p};
    \node[circle,fill=white!80!black,textz
      ,below=0cm of lsm] (dcp) {Dcp2p};
    \node[circle,fill=white!80!black,textz
      ,below right=0cm and 0cm of dcp] {Edc3p};
    \node[circle,fill=white!80!black,textz
      ,below right=0cm and 0cm of pat] {Scd6p};
%
    \draw[-|] (dcp) to[out=-100,in=-170] 
      node[pos=0.5,below,textz,sloped] {decapping} 
      ($(decap)+(210:1.7)$);
%
    \coordinate (deg) at (5,-9.0);
    \draw[thick,->,bend left] ($(decap)+(2,2)$) to ($(deg)+(-0.5,0.5)$);
%
    \draw[thick,color=red] (deg) to ($(deg)+(3,0)$);
    \node[textz,anchor=west] at ($(deg)+(3,0)$) {AAAAA};
    \node[circle,fill=green!80,textz] at 
      ($(deg)+(3.5,0)$) {Lsm1-7p};
    \node[circle,fill=red!50,textz] at 
      ($(deg)+(0,0)$) {Xrn1p};
%
    \draw[thick,color=red] ($(deg)+(1,-2)$) to ($(deg)+(3,-2)$);
    \node[textz,anchor=west] at ($(deg)+(3,-2)$) {AAAAA};
    \node[circle,fill=green!80,textz] at 
      ($(deg)+(3.5,-2)$) {Lsm1-7p};
    \node[circle,fill=red!50,textz] at 
      ($(deg)+(1,-2)$) {Xrn1p};
%
    \draw[thick,color=red] ($(deg)+(2,-4)$) to ($(deg)+(3,-4)$);
    \node[textz,anchor=west] at ($(deg)+(3,-4)$) {AAAAA};
    \node[circle,fill=green!80,textz] at 
      ($(deg)+(3.5,-4)$) {Lsm1-7p};
    \node[circle,fill=red!50,textz] at 
      ($(deg)+(2,-4)$) {Xrn1p};
%
    \node[textz,rotate=-60] at ($(deg)+(-0.5,-2)$) {5' to 3' degradation};
%
  \end{tikzpicture}
  }{
    Newly transcribed and poly-adenylated mRNA are progressively
    deadenylated until the Lsm1-7p/Pat1p complex binds to recruit
    decapping factors and Dcp2p.
    Co-transcriptional proof-reading surveils the transcript for
    aberrant translation dynamics.
    Nonsense-mediated decay, for example, triggers decapping.
    A decapped mRNA is rapidly degraded from 5' to 3' by Xrn1p.
  }
  {Diagram of canonical deadenylation-dependent 5' to 3' mRNA degradation}
These initiation factors (eIF4F) bind the m7G cap to load ribosome
subunits \parencite{dever2016mechanism}, and thus most translation depends on the
cap (with exceptions demonstrated by internal ribosome entry sites
\parencite{gilbert2007cap}).
The m7G cap is also critical for mRNA
stability. Xrn1p is a highly-processive combination of helicase and
exonucleolytic domains that alone can rapidly degrade
transcripts from a 5' to 3' end, recognizing unprotected 5'
phosphorylated ribonucleotides as substrates \parencite{parker2012rna}. 
Thus, the inverted linkage of the m7G escapes degradation.  


During rounds of
translation the poly-adenosine tail is shortened from about 65-90
adenosines to about 10 adenosines by a combination of the Pan2/3 and
Ccr4/Pop2 deadenylase complexes, with activity antagonized by the
poly-A binding protein Pab1p
\parencite{parker2012rna,decker1993turnover}.
When the tail is thus shortened, the Lsm1-7p/Pat1p
complex binds the remainder of the poly-A tail
\parencite{tharun2000yeast}.
This complex is a heptameric ring of the Lsm1-7 proteins with the
Lsm1p's C-terminal domain elegantly spanning the center 
\parencite{sharif2013architecture},
and the last eight residues projecting into this center
and critical for binding the shortened poly-A tail 
\parencite{chowdhury2016mutagenic}.
The critical function of this complex is to recruit and promote
activity 
of the decapping complex to the 5' end of the mRNA, and in
cooperation with Pat1p \parencite{chowdhury2014pat1}
the binding of this
complex to mRNA and to decapping factors is indeed correlated with
decapping of the mRNA 
\parencite{chowdhury2009activation}.
Thus, the complex
maps the deadenylated status to the next step in mRNA degradation.  

A
cytoplasmic mRNA without a 5' m7G cap is not long lived, by virtue of
Xrn1p, thus the recruitment and activation of the decapping complex is
thought to be the key regulatory step in rates of mRNA degradation
\parencite{coller2004eukaryotic}.
Dcp2p carries out the catalytic activity of
the holoenzyme but is promoted by the effects of Dcp1p, and in
comparing \textit{in vitro} to functional \textit{in vivo} assays 
of mutants it appears
that the catalytic rate of the enzyme is not the limiting step 
\parencite{tharun1999analysis}. 
Rather it is re-modeling of the mRNP (mRNA-protein) complex that
leads to association of the decapping enzyme complex with the 5' cap,
and the rate of this process determines the activity of this
degradation pathway 
\parencite{tharun2001targeting}.
This
decapping-enzyme-localization process is promoted by the Lsm1-7p/Pat1p
complex and inhibited by poly-A binding protein Pab1p (perhaps by
competition of Pab1p with Lsm1-7p/Pat1p for the poly-A tail) and eIF4E
\parencite{coller2004eukaryotic,caponigro1996mechanisms}. 
eIF4E and eIF4G
compose the cap-dependent translation-initiation factor eIF4F
\parencite{dever2016mechanism}, 
which gives rise to an elegant model of competition for the 5' m7G cap
to explain the observation that
translation initiation inhibits decapping
\parencite{huch2014interrelations}.
The requirement of sufficient
poly-A tail for Pab1p to bind is consistent with the effect of
deadenylation in promoting decapping (Parker 2012), and thus a model
has emerged wherein deadenylation promotes the association of the 
Lsm1-7p/Pat1p complex to the shortened poly-A tail, then re-arranges
to associate this 3' end of the mRNA molecule with the 5' end,
resulting in an interaction that recruits and triggers decapping
of the mRNA. 
The Lsm1-7p/Pat1p
complex and Dcp2p/Dcp1p are physically associated (by
co-immunoprecipitation) with several factors that genetically modulate
the activity of remodeling step ---Dhh1p, Edc3p, and Scd6p 
\parencite{nissan2010decapping}.

Dhh1p is a helicase that associates with polysomes (mRNA
with multiple ribosomes bound) and has been recently demonstrated to
be genetically required for the relationship between codon optimality
and mRNA stability 
\parencite{sweet2012dead,radhakrishnan2016dead,presnyak2015codon}. 
Curiously, tethering Dhh1p to a 3' UTR
using the MS2 system resulted in lower translation rate of an mRNA
despite causing more ribosomes to be associated with the mRNA
\parencite{sweet2012dead}, suggesting that Dhh1p resolves slowly 
translating ribosomes by promoting decapping and mRNA degradation.  
However, physically tethering Dhh1p may affect its role if
topology of interactions is important or movement along the mRNA
is important for its function.
Dhh1p also appears to play a role in promoting the association of
mRNPs into processing bodies.

Edc3p
physically associates with the decapping complex and stimulates its
activity \parencite{nissan2010decapping}, 
but it has also been shown to be
important \parencite{decker2007edc3p,huch2016decapping} but not critical 
\parencite{rao2017numerous}
for the formation of processing-bodies. 
These are
microscopically visible foci of mRNA and degradation factors that form
in response to stress conditions but are assumed to be condensed from
mRNA-protein complexes that exist before stress 
\parencite{sheth2003decapping,lui2014granules,rao2017numerous}. 
A mutant deleted of \textit{EDC3} and the C-terminal domain of the
essential \textit{LSM4} is deficient in processing-body formation, 
and surprisingly this processing-body deficiency also correlates with a
deficiency in the stabilization of several mRNA upon osmotic stress
\parencite{huch2017mrna}. 
Edc3p plays a role in promoting the association of mRNPs
together into processing-bodies, and perhaps its effect on 
decapping is by virtue of co-localizing degradation factors in 
processing bodies.



Scd6p inhibits the formation of the 48S pre-initation complex
(when the 40S subunit associates with eIF4E cap-dependent initiation
factors and begins to scan the 5' UTR) via forming its own cap-binding
complex with eIF4G subunit eIF4G1 in an arginine-methylation-dependent
manner \parencite{rajyaguru2012scd6,poornima2016arginine}. 
This is thought
to physically occlude the normal initiation complex from binding.
Scd6p also binds to several other factors in the Lsm1-7p/Pat1p
deadenylation-promoting complex \parencite{nissan2010decapping}, 
and thus may play an indirect role in preventing the pre-initiation 
complex from binding and stabilizing the mRNA through translation.  

\subsection{Alternative pathways of mRNA degradation ---
3' to 5' and quality control}

Other pathways
of mRNA degradation exist. If the poly-A tail is completely removed,
the cytoplasmic exosome complex can degrade the mRNA from 3' to 5'.
This redundant mechanism allows a \textit{xrn1}$\Delta$ mutant to grow, although
slowly, as the 5' to 3' pathway is throught to effect the bulk of mRNA
degradation \parencite{parker2012rna}. 
The balance between the two may be because
of the enzymatic rate of digestion, but more likely because the
binding of the Lsm1-7p complex to the shortened poly-A tail protects
the mRNA from further deadenylation and thus inhibit this 3' pathway
\parencite{tharun2009lsm1}.

Three other pathways are known to act as a
co-translational layer of quality control, where errors detected by
abnormal translation processes result in destruction of the presumably
defective mRNA. Nonsense-Mediated Decay (NMD) is the canonical example
of this. A mutation, transcriptional error, alternative splicing
event, or abnormal translational event (like leaky scanning or a uORF,
explained later) can cause an mRNA to have a stop codon well before
the usual position, which is recognized for destruction by a
deadenylation-independent decapping and 5' to 3' decay 
\parencite{muhlrad1994premature}. 
How the aberrant nature of the misplaced stop codon is
detected is still a mystery, but NMD sensitivity is known to be
more active at the 5' end of the coding sequence (after translating
a sufficient stretch of amino-acids), with activity reducing
towards the 3' end of the transcript
\parencite{losson1979interference}. 
Non-Stop Decay refers to the inverse of
NMD, no stop codon. Ribosomes that over-run into the poly-A tail
recruit degradation by the 3' exonuclease 
\parencite{schmid2008exosome}. No-Go Decay is
named for the phenomenon that triggers it, when ribosomes "no go"
(encounter a difficult to elongate sequence).
Difficult to translate sequences (such as arginine or lysine repeats) 
trigger the
endonucleolytic cleavage of the offending mRNA, which is then degraded
from the cut site towards both ends 
\parencite{doma2006endonucleolytic}. While the molecular mechanisms of
this process have been not been comprehensively defined, 
it has been shown that it is likely the ribosome collisions that
promote the ribosome ubiquitination associated with the triggering of
No-Go decay \parencite{simms2017ribosome}, and
other work has suggested that the
protein Asc1p \parencite{ikeuchi2016ribosome} 
or K63 ubiquitination \parencite{saito2015inhibiting}
may play a role.
Thus a model for ribosomal subunits sensing ribosome collisions 
and activating this quality control pathway through ubiquitination
of ribosome-associated factors offers an elegant mechanism to sense
locally-stalled ribsomes, although this idea is still beginning to be
explored.

%@shoemaker2012translation

Together
these pathways surveil translating mRNAs for defects, but it is likely
that false positives in the recognition process also contribute to
their regulatory effects.  
Disruption of the NMD pathway is associated
with different expression of many transcripts. Recent genome-wide
analysis identifying $\sim$900 mRNA upregulated upon deletion of any of
UPF1-3, and subsequent ribosome profiling found this targeting to be
associated with out-of-frame translation effects and non-optimal
codons \parencite{celik2017high}.
NMD has been implicated in the regulation
of ribosomal subunit protein pre-mRNA 
\parencite{garre2013nonsense}
in different
environmental conditions, has been shown to interact genetically with
Hrp1p and cis-elements spanning the start codon of PPR1 mRNA to target
this mRNA for degradation 
\parencite{pierrat19935,kebaara2003upf},
and may be triggered by upstream open reading frames (uORFs, discussed
later). These could be specific regulatory events, 
or aberrant probabilistic
activation due to the sensitivity of co-translational quality control
\parencite{celik2017high}.

% Chen 1998 hrp1 was figured using adh2 !!!!

\subsection{The interaction of translation and mRNA
degradation}

Codon-optimality refers to the concept that certain codons
are translated by the ribosome more quickly than other codons. This is
thought to result in part from changes in tRNA abundance and in part
due to intrinsic differences in the decoding rates 
\parencite{curran1989rates,thomas1988codon}, and is often quantified
using the tRNA adaptation index (tAI)
\parencite{reis2004solving}. The expectation that tRNA availability is
associated with increased rates of translation has been tested with
more recent ribosome footprint profiling experiments, and consistent
with this ribosomes tend to occupy optimal codons less often 
\parencite{weinberg2016improved}.  

The functional relationship between codon-optimality
and mRNA degradation rate had been considered and rejected by a review
of single-transcript studies 
\parencite{caponigro1996mechanisms}. However,
with the advent of accurate genome-wide measurements of mRNA
degradation rates, we are able to explore the generality of this
principle in a relatively unbiased way. Several groups 
\parencite{presnyak2015codon,neymotin2016multiple,harigaya2016codon,cheng2017cis}
have found that poor codon-optimality and lower ribosome density
is associated with a higher degradation rate when considered on a
per-transcript basis. This can be explained through multiple models.
One model is that translation elongation rates are sensed, with slower
elongation accelerating the degradation of mRNA. Jeff Coller's group
has worked extensively on Dhh1p, and found that it is genetically
required for the clear relationship between codon-optimality and mRNA
stability 
\parencite{presnyak2015codon,radhakrishnan2016dead}. Although
the mechanism is at this point unclear, Dhh1p's genetic association 
is an important link from which to start.

Alternatively, competition between
decapping enzymes and translation initiation factors for access to the
5' m7G cap has long been proposed as a mechanism by which the two
processes interact 
\parencite{schwartz1999mutations,schwartz2000mrna}. Karsten Weis' group 
\parencite{chan2017non} reproduced the result
that slowing elongation with cycloheximide, sordarin, or 3AT treatment
slows mRNA degradation, but conversely inhibition of initiation with
hippuristanol or a dominant negative eIF4E increased degradation
rates. These measurements were made on the whole-transcriptome using
4-thiouracil and RNA sequencing, similar to RATEseq 
\parencite{neymotin2014determination}. 
The connection between the effect of elongation rates and
initiation rates could also be explained by the effect of slow
elongation rates inhibiting initiation events, as predicted 
\parencite{shah2013rate} and measured \parencite{chu2014translation}.

Thus, much evidence points
to competition between translation initiation and 5' to 3' degradation
initiation at the cap as a major determinant of mRNA stability,
although the molecular work with Dhh1p suggests that events after
initiation still play a role. Other mRNA degradation pathways like NMD
or NGD during elongation (as discussed earlier) could also possibly
contribute to the effect.  

\subsection{ Regulation of mRNA degradation }

mRNA
degradation can be affected by various \textit{trans} factors. While micro RNAs
are prolific in regulating mRNA in animals and plants, budding yeast
do not make use of this mechanism. Instead, in yeast mRNA degradation
appears to be determined by a combination of intrinsic properties like
length or codon-optimality, and trans factor RNA binding proteins
(RBPs) that can bind \textit{cis} element sequences in the mRNA sequence to
effect changes in stability 
\parencite{li2010predicting}. The best
example of this is Puf3p, which binds motifs in mRNA with products
destined for mitochondrial function and degrades these in the
appropriate environment 
\parencite{olivas2000puf3,miller2013carbon}, perhaps by
mapping phosphorylation of Puf3p to association of these mRNA to
cytoplasmic granules 
\parencite{lee2015glucose}. Secondary structures may
complicate the recognition of linear \textit{cis} elements, or be used
as \textit{cis}
elements in their own right \parencite{li2010predicting},
for example Vts1p (Smaug homolog) recognizes a small sequence 
motif in the context of the loop of a stem-loop hairpin 
\parencite{she2017comprehensive,aviv2003rna}.
Degradation rates can be affected by many mechanisms. Elements in
promoters (\textit{cis} when in DNA but not part of the affected mRNA) can
“mark” transcripts for differential stability 
\parencite{haimovich2013gene}.
One of the most well known examples of this is Dbf2p loading onto 
\textit{SWI5}
and \textit{CLB2} mRNA to effect destabilization upon mitosis 
\parencite{trcek2011single}. 
In a direct example, the transcription activation domain of
Adr1p fused to a different DNA binding domain has been shown to be
sufficient to mark \textit{ADH2} mRNA for destabilization upon a glucose
upshift \parencite{braun2016snf1}.
Thus, RBPs may recognize sequence
elements in the mRNA or be loaded onto messenger ribonucleo-protein
complexes (RNPs) at synthesis 
\parencite{gupta2016translational}
to effect control of
mRNA stability in response to events in the cytoplasm.  

Non-RBP
mechanisms can also be used. Upstream open reading frames (uORFs) were
originally characterized using the phenotype of post-transcriptional
regulation of the Gcn2-regulated Gcn4p 
\parencite{dever1992phosphorylation}
in part through quality control pathways 
\parencite{ruiz1996utilizing}.
Canonical and non-canonical start-codons can recruit initiation of
scanning ribosome subunits with a variety of effects on the
translation of the main coding sequence and the mRNA stability
\parencite{spealman2017conserved}. Ribosomes may skip re-initiation at the
primary start codon to generate N-terminal diversity by initiating at
alternative start codons, or upon termination of a uORF very distant
from the 3' end of the transcript trigger the NMD pathway to destroy
the mRNA \parencite{dever2016mechanism}. 

While the primary-sequence of the mRNA is often thought to be the
primary source of \textit{cis}-elements,
mRNA can be modified in a variety of ways.
The most extensively studied modification so far is m6A methylation of
adenines, with demonstrated consequences for localization and
stability \parencite{gilbert2016messenger}.
The role of these, or other modifications, is still being
investigated.
The critical nature of the 5' m7G cap suggests that the
discovery of other capping structures, such as recent identification
of NAD+ capped mRNA in yeast \parencite{walters2017identification},
could provide another type of RNA modifications loading during
transcription to affect processes of translation initiation and 
transcript degradation.
Although at a low percentage, these modifications suggest that non-m7G
caps could contribute to a sizable fraction of euarkyotic mRNA.
Application of improvements in biochemical assays of modifications as
well as long-read sequencing technology (both PacBio and 
Nanopore) may yield a new and informative perspective on the single
molecule extent of base modifications and their impacts on the
progression of mRNA through various degradation intermediates.

mRNA
localization within the cytoplasm may affect degradation by virtue of
regulating the accessibility of degradation factors. Processing-bodies
were originally described as cytoplasmic co-localized foci of 5' to 3'
degradation factors that formed under stress induction conditions, and
on the basis of steady-state genetics and experiments with an MS2
aptamer-based live-imaging system, it was concluded that
processing-bodies are foci of active mRNA degradation 
\parencite{sheth2003decapping}.
These foci are usually studied by microscopy during
stresses, entry into stationary phase, and in the use of mutants in
degradation pathways, but recent advances in microscopy and nanoscopy
particle tracking have identified that these complexes are likely
condensations of previously-existing RNPs and depend on a network of
redundant interactions between mRNA 5' to 3' degradation protein
factors 
\parencite{lui2014granules,rao2017numerous}.
Additionally, recent
adjustments to the aforementioned MS2 aptamer system and explorations
during dynamic conditions point towards processing-bodies being sites
of degradation factor sequestration 
\parencite{huch2017mrna,tutucci2017improved}.
Interestingly, the formation of these processing-bodies are
halted upon cycloheximide treatment 
\parencite{sheth2003decapping},
suggesting that translational status of the transcriptome and
processing-body composition may be related. In recent work, inhibition
of translation initiation was demonstrated to increase p-body
formation in correlation with increased degradation rates 
\parencite{chan2017non}. 
Together, these observations indicate that processing-bodies
result from a complex balance of mRNA degradation initiation,
resolution, and mRNA degradation factor interactions with impacts on
the accessibility of degradation factors to mRNA targets of
degradation.  

\section{The role of stability control in transcriptome
reprogramming}

The change in concentration of an mRNA ($R_t$) depends
on the rates of mature transcript synthesis ($k_s$) and mRNA
degradation ($k_d$). We assume that
synthesis is a constant rate dependent on the unchanging concentration
of the DNA encoding the gene, and that degradation is a first
order process of the mRNA interacting with a fixed and unsaturated
factor degradation. We also disregard dilution from cell volume
changes because this is approximately 50 times slower than the average
rates of mRNA degradation and changes will thus not play a large role
in our measurements of mRNA dynamics on this timescale.  
Thus, the change in mRNA over time is $$
\frac{dR_t}{dt} = k_s - R_t k_d$$ From this, the two rates determine
the steady-state equilibrium of $\frac{k_s}{k_d}$. 
Given a singular regulatory event,
the doubling time (or half-life) of the mRNA is dependent on only the
degradation rate $\frac{log(2)}{k_d}$ and thus 
a faster mRNA degradation rate will approach or relax to
the new equilibrium value quicker \parencite{hargrove1989role}.

While both synthesis and
degradation contribute to changes in abundance, changes in degradation
rates can cause the changes to occur more rapidly. If we expect that
the existence of a mechanism implies a selective pressure specifically
for it 
\parencite{gould1979spandrels},
then we would expect that studying
an example of a transcript subject to both synthesis and degradation
regulation might reveal a balance of selection across steady-state and
dynamic conditions. 

\subsection{Stress conditions trigger rapid regulation of mRNA stability}

mRNA degradation rate changes have been characterized
to play a role in responses to heat-shock, osmotic stress, pH
increases, and oxidative stress, sharing a similar program of
destabilization of mRNA coding for ribosomal-biogenesis gene products
and stabilization of stress-responsive mRNA 
\parencite{canadell2015impact,molina2008comprehensive,shalem2011transcriptome,romero2009specific,molin2009mrna,castells2011heat,miller2011dynamic,garre2013nonsense}.
Simultaneous increases in both synthesis and
degradation rates of some of these mRNA are thought to serve to return
the transcriptome quickly to a new steady-state after effecting a
transient pulse of regulation 
\parencite{shalem2008transient,rabani2011metabolic},
demonstrating a key functional role in stability control in
achieving a particular pattern of mRNA dynamics. Interestingly, these
stability changes seem to usually be a singular regulatory event
\parencite{perez2013eukaryotic}, suggesting that the mechanism is
coordinated in its effect across the entire transcriptome during
the first response to the stress.

\subsection{Nutrient shifts also trigger mRNA stability changes }

In response to a carbon-source downshift
(glucose-grown cells resuspended in media with only galactose
available), functionally important regulatory changes in mRNA
stability occur 
\parencite{munchel2011dynamic}. Ribosome biogenesis associated
mRNAs are destabilized, an effect that can be phenocopied by the
addition of rapamycin (inhibitor of the central growth signalling
TORC1 pathway). Conversely, a carbon source
upshift (galactose to glucose) triggers a destabilization of inducible
GAL genes, an effect that appeared to be restricted to the dynamic
condition as mRNA transgenically overexpressed in glucose media were
stable \parencite{munchel2011dynamic}.

Global changes in transcription and
mRNA destabilization have been observed before 
\parencite{jona2000glucose}, and
recently systematically measured to be correlated with changes in
growth rate 
\parencite{garcia2016growth}. The involvement of the
TORC1 pathway in this process has identified that its effect is
specified to differentially regulate the stability of certain
transcript sets 
\parencite{albig2001target,talarek2010initiation}.
Recently, a phosphoproteomics approach to studying signallng of the
AMPK homolog Snf1p during a carbon upshift identified a role in Xrn1p
phosphorylation in the specification by this factor 
\parencite{braun2014phosphoproteomic}. 
Thus, specific signalling pathways appear to effect large
changes in mRNA stability in response to different nutrient conditions
for growth. 

Relieving nutrient limitation with a glucose upshift has
been shown to mediate both stabilization of mRNA in the ribosomal
protein subunit regulon 
\parencite{yin2003glucose}
and destabilization of
gluconeogenic transcripts 
\parencite{de2002role,mercado1994levels,scheffler1998control,lombardo1992control}. 
Mapping the
determinants of this effect has been met with mixed success. The
destabilization of \textit{SDH2} and \textit{GAL1} mRNA has been 
mapped to elements in the 5' UTR 
\parencite{scheffler1998control,bennett2008metabolic}
with destabilization of GAL1 being
associated with a growth advantage in switching carbon-source
environments \parencite{baumgartner2011antagonistic}. 
\textit{JEN1} encodes a lactate/pyruvate transporter that is
destabilized in rich carbon-sources like glucose, and this has been
determined to result from transcription from a downstream
transcription start-state destabilizing the larger isoform, through
unknown mechanisms \parencite{andrade2005multiple}.
Subsequent work has identified \textit{DHH1} as being a genetic 
factor of the \textit{JEN1} destabilization \parencite{mota2014role}. 
Some transcripts respond at different levels of glucose
addition 
\parencite{yin2000differential}, and disrupting signalling through the
PKA pathway affects destabilization of some mRNA but not others 
\parencite{yin2003glucose}. 
Thus, a systematic measurement of mRNA stability and a
broad determination of genetic factors of the transcript dynamics
would be useful for making progress at untangling the regulation of
mRNA stability in response to the increase in growth rate upon a
nutrient upshift.  

\section{Measuring mRNA dynamics}

%It is easier to measure the abundance of a mRNA than it is to 
%measure the change in abundance of a mRNA. 
While mRNA
abundance measurements for entire transcriptomes are now routine,
determining the rates that underlie this molecular phenotype has
lagged. Synthesis rate control has largely been assayed by techniques
like Genome Run On (GRO) sequencing (discussed below) to measure
transcription rates or measuring intron-exon ratios 
\parencite{gray2014snapshot} as a proxy for synthesis rates 
\parencite{perez2013eukaryotic}.
Degradation rate measurements have used a variety of methods, but are
now applied to the whole transcriptome with enough accuracy to enable
systematic modeling of the determinants of mRNA degradation rates
\parencite{perez2013eukaryotic,neymotin2016multiple,cheng2017cis}.

Pioneering studies used pulse-chase experiments with radioactive
nucleotides to study turnover of the whole transcriptome 
\parencite{petersen1976half}.
Single gene measurements allowed the characterization of individual
gene rates of turnover, but required the use of other methods for
tracking the dynamics.
One approach is to use promoters with inducible repression
characteristics to halt transcription, for example transgenic
repressors like the doxycycline-inducible Tet-Off 
\parencite{gari1997set}.
Researchers have also made use of the native \textit{GAL1} promoter. 
Upon addition of glucose, transcription of the \textit{GAL1} mRNA is 
immediately halted. This property has been exploited to study mRNA 
stability in a technically simple manner and has formed the basis of 
much of what we understand about the pathways of mRNA degradation
\parencite{parker2012rna,coller2004eukaryotic}.
While the \textit{GAL1} system is a convenient system for studying 
degradation intermediates, its demonstrated destabilization upon
glucose addition makes uncertain its use for studying the native stability 
of different mRNA in different environments.  

A system
that does not rely on engineered \textit{cis}-elements would avoid 
these issues and scale to genome-wide assays, and thus two methods of
transcriptional inhibition were applied to study mRNA degradation
rates in landmark studies. A temperature sensitive \textit{rpb1}-1 
allele was demonstrated to halt most PolII transcription at 
non-permissive temperatures, while the drugs thiolutin and 
1,10-phenanthroline
inhibited polymerases including PolII to mostly halt transcription.
These have been used widely, and are still used to this day. However,
it has been shown that use of thiolutin or 1,10-phenalanthroline
induces some heat-shock genes \parencite{adams1991yeast}, thiolutin
inhibits mRNA degradation in a dose-dependent way 
\parencite{pelechano2008transcriptional}
(perhaps via inducing processing-body formation,
\cite{huch2016decapping}), 
and eliminating the essential RNA PolII complex
from the nucleus has complex effects on the transcriptome dynamics 
\parencite{yu2016rna}. While it may seem logical that studies of mRNA
associated with processes distinct from heat-shocks may be unaffected
by these, the complexity of the cell demonstrated itself in vital
controls run in \cite{mercado1994levels} which demonstrated that
gluconeogenic mRNA were subject to destabilization upon a
heat-shock. Thus in several examples we see that shutting off 
transcription has complex and difficult to
predict effects on transcript abundance as the cells die.

Orthogonal to these approaches is Genomic Run On approaches
\parencite{garcia2004genomic}, including microarray or sequencing
based assays \parencite{pelechano2010complete}. 
This method uses a cold sarkosyl treatment to fix RNA PolII 
complexes onto genomic DNA by freezing their elongation, then
extraction and a defined in-vitro polymerase extension and profiling 
the resulting mRNA with microarrays or RNAseq allows for an estimate 
of the instantaneous transcription rate status for each gene in a 
population of cells. Interpretation of these numbers must be 
considered in the context of the in-vitro environment of the 
elongation step, but this method serves as an valuable orthogonal 
measure of transcript dynamics --- and an instantaneous one.  

The development of 4-thiouracil metabolic labeling of RNA 
\parencite{dolken2008high}
has enabled a return to the pulse-chase methodology in the 
development of genome-wide assays of mRNA dynamics. 
A thiol-containing analog of uracil, 4-thiouracil is readily
incorporated by yeast into their nucleotide metabolism and thus into
the mRNA. The label does not perturb growth at low concentrations 
(< 50$\mu$M) \parencite{burger20134}, and supports normal growth.
Fundamentally, these assays work by changing the labelling frequency 
of mRNA at a time and tracking the dynamics as the labeled
mRNA abundance relaxes towards a new equilibrium of labeling. 
Below I review the basic model of these assays, 
then focus on their applications and where
the dissertation work is placed.  

If we consider mRNA abundance at a
certain time as being denoted as $M_t$, then I expect this number to
change as a zeroth order rate of synthesis per time ($k_s$) and a
first order rate of degradation per mRNA ($k_d$). While mRNA
degradation is a multi-step process (above) and more complex models
may identify nuances in the rates of progression through these
intermediates \parencite{deneke2013complex}, at steady-state the rate of
degradation of mRNA in a population should be well modeled by a single
first order rate \parencite{thattai2016universal}.
$$\frac{dM_t}{dt} = k_s - M_t*k_d$$ Introducing a term $L$ that
denotes the fraction of newly synthesized mRNA that are labeled and
measured after purifying mRNA for the labeled mRNA (thus $M_t$ is just
labeled and captured mRNA), we can now model the changes as simply
$$\frac{dM_t}{dt} = L k_s - M_t*k_d$$ I introduce the superscript
notation of $L^o$ for the old labeling frequency and $L^n$ for the new
labeling frequency, and solving for the change of $M_t$ from some
steady-state equilibrium $L^o \frac{k_s^o}{k_d^o}$ to a new
equilibrium $L^n \frac{k_s^n}{k_d^n}$, and rearranging terms we get
$$M_t =  L^o \frac{k_s^o}{k_d^o} e^{-k_d^n t} + L^n \frac{k_s^n}{k_d^n} ( 1- e^{-k_d^n t} )$$ 
This matches well with our
intuition. On the right, the nascent transcripts are labeled at the
new rate and approach this new equilibrium controlled by the term $(
1- e^{-k_d^n t} )$, while on the left the extant transcripts approach
zero in an exponential decrease controlled by the term $e^{-k_d^n t}$.
These are both controlled by $k_d^n$, or the rate of mRNA degradation
after chasing the label. Thus, by measuring the transition between the
equilibrium we get the mRNA degradation rate.

Measuring specific rates with high confidence
requires a steady-state approximation. RATEseq is one method to do so,
using many timepoints to accurately model the approach of labeled mRNA
abundance to a new equilibrium 
\parencite{neymotin2014determination}. This
experimental design is theoretically the most accurate, although it
requires the assumption that the total mRNA (labeled + unlabeled) is
indeed at a steady-state abundance. Dynamic Transcriptome Analysis
\parencite{miller2011dynamic}
violates this assumption to explore changes in
degradation rates during 6 minute windows. While the method sacrifices
high-confidence of an exact rate, the temporal resolution of stability
changes during osmotic stress has revealed an unprecedented dynamic
view of the regulation of mRNA dynamics during complex processes. This
approach requires that 4-thiouracil transport and incorporation into
nucleotide metabolism occurs during the course of the perturbation
experiment, but with the right measurements, normalization, and
integration with other datasets an accurate and dynamic picture of
transcriptome dynamics can be built. To assess mRNA stability changes
during dynamic processes, one can also label the transcriptome to
equilibrium and then chase out the label by adding an excess of
unlabeled nucleotides. This approach was used by researchers in the
Weis group to demonstrate changes in the stability of groups of mRNA
in response to environmental changes, namely shifts in carbon sources
and with rapamycin treatment inhibiting TORC1
\parencite{munchel2011dynamic}

In Chapter 3 I demonstrate the use of a similar 4-thiouracil
label-chase experimental design, with refined analysis to 
explore per-transcript destabilization upon a nitrogen upshift.  

\section{Methods for determining the genetic basis of a transcript
dynamics phenotype}

mRNA is an intermediate in the expression of a protein product, and
has the key virtue of being easy to measure in bulk.
This has become especially true with the
advent of massively-parallelized DNA sequencers and the methods to
accurately convert transcriptomes to DNA libraries (RNAseq)
\parencite{shendure2017dna}. For this reason, it is often used as a proxy
of gene expression at the protein level. Although the relationship is
strong when correcting for experimental noise 
\parencite{csardi2015accounting},
the quantitative functional nature of this relationship within a
particular gene in different environments depends on the particular
gene in question 
\parencite{franks2017post}. It is also clear that
transcriptomic and proteomic responses greatly vary in the timescales
of effect, with the transcriptome subject to rapid impulses of
changing abundance that may or may not result in longer term
regulation of the protein product 
\parencite{cheng2016differential,lee2011dynamic}. 
Even then, protein abundance in a cell does not correspond
perfectly to its activity, be that regulated allosterically or by
localization.  

Given this disparity, what can we learn about adaptive
gene expression from mRNA abundance regulation? First, the expression
of a gene product requires mRNA, thus the binary expression of mRNA is
a predictor of the possibility of protein expression. Additionally,
cellular processes often impinge upon changes in mRNA abundance, be
they direct via regulation of abundance, activity, and localization of
activity of specific effectors or by indirect effects on common gene
expression machinery or cellular metabolism. In this way, a specific
perturbation of a signalling pathway is expected to broadcast to
changes in mRNA abundance. Quantification of the thousands of mRNA
that are expressed in a cell is a sensitively quantitative measurement
on thousands of dimensions, and can thus be used as a
relatively-unbiased indicator of cellular status with which to explore
the genetic requirements of particular signalling perturbations 
\parencite{gapp2016parallel}. Thus, efficient methods to explore the genetic
basis of transcript dynamics upon a perturbation should help to
accelerate the study of cellular signalling pathways.  

Genetic screens
in yeast have been a powerful tool to narrow down the immense search
space of possibilities to a narrow set of hypotheses about a
biological process. Classically, these function by mapping some
phenotype of interest to a change in growth rate. For example, mutants
in transporters of a particular amino-acid can be isolated by feeding
the cells a toxic stereoisomer (like D-histidine). A more complex
method in Lee Hartwell's classic screen for cell-cycle mutants used
the assay of growth at a low temperature and cessation of growth at a
high temperature to identify mutants in critically important pathways
\parencite{hartwell1970genetic}, work that contributed to a 2001 Nobel Prize for
advancing our understanding of the cell cycle. However, this concept
becomes problematic when studying a molecular phenotype which is not
known to be adaptive, and thus is not known to be selected in this
process of accelerated mRNA degradation.

For example, gene regulation might not have a
clear phenotypic outcome, or could be subject to redundant layers of
regulation that mask the effect of a mutation. One solution is to
engineer a specific reporter into the expressed gene, such that
defects in gene expression can be assayed. It becomes more difficult
if the phenotype is a transient one, such that a reporter through
growth rate (perhaps a toxic peptide) does not have time to accumulate
the signal of growth. A fluorescent tag is one approach that is more
direct, as cells can be instantaneously assayed for the
level of GFP fluorescence at that moment via flow cytometry. The GFP
can be fused to the protein of interest or simply placed downstream of
an appropriate reporter, such as the strategy employed by 
\cite{neklesa2009genome}. 
These researchers were able to use a DAL80 promoter upstream of a GFP
reporter to explore the genetic requirements for the NCR-regulated
expression of this promoter, discovering the SEACIT complex components
Npr2/3p upstream of TORC1. 
This approach is compatible with a pooled assay using
barcode-sequencing technologies, also known as SortSeq
\parencite{kinney2010using,oikonomou2014systematic,peterman2016sort,de2017deciphering}.
While appropriate for studying steady-state processes, this approach
often uses fluorophores that require that the
expression phenotype be relevant at the level of protein expression,
and that the GFP tag be a relevant and faithful reporter of
the protein abundance --- a condition which is not always satisfied 
given the stability of the cleaved GFP tag in the vacuole
\parencite{conibear2002studying}.

\textit{GAP1} mRNA degradation, which we identify as
being subject to accelerated mRNA degradation in Chapter 2 and 3,
occurs much faster than the repression of the protein-product 
\parencite{hein1997ac}. We also
know that Gap1p, the protein product of \textit{GAP1} mRNA, is subject to
de-activation and re-localization in response to a nitrogen upshift.
Thus, a functional assay of Gap1p is irrelevant to the dynamics of
\textit{GAP1} mRNA repression, and requires a novel method to screen for
genetic factors of this molecular phenotype.  

Ambitious work from the
Capaldi group developed a workflow using extensive automation to
perform qPCR assays for NSR1 mRNA abundance 19 minutes after induction
of an osmotic stress response 
\parencite{worley2016genome}. While accurate and
reproducible, the extensive automation and reagent usage to perform
qPCR on $\sim$4700 mutant strains poses a financial and logistical
challenge. To perform the assay in
different genetic backgrounds, in larger libraries, in replicates, or
in different timepoints requires a pooled approach that can scale
genome-wide without greatly increasing costs. 
To do this, I adapted a mRNA
FISH assay to use in budding yeast.
While mRNA has been observed by flow cytometry before
(\cite{yu1992sensitive} described its use on $\sim$1800 copies of
histone mRNA), the use of branched DNA probe sets has made sorting
on low copies of mRNA possible \parencite{hanley2013detection}. 
In Chapter 3 I discuss in depth the combination of this technology
with the aforementioned SortSeq barcode sequencing and modeling
approaches to directly estimate transcript abundance in a 
high-throughput pooled format, and without genetic modifications.

